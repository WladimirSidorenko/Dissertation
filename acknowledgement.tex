% FILE: acknowledgement.tex  Version 2.1
% AUTHOR:
% Universit�t Duisburg-Essen, Standort Duisburg
% AG Prof. Dr. G�nter T�rner
% Verena Gondek, Andy Braune, Henning Kerstan
% Fachbereich Mathematik
% Lotharstr. 65., 47057 Duisburg
% entstanden im Rahmen des DFG-Projektes DissOnlineTutor
% in Zusammenarbeit mit der
% Humboldt-Universitaet zu Berlin
% AG Elektronisches Publizieren
% Joanna Rycko
% und der
% DNB - Deutsche Nationalbibliothek

\textbf{Acknowledgment.}

At the beginning of this thesis, I would like to say a really big
thank you to all the people who supported me throughout these
difficult but immensely enjoyable years of work on my dissertation.
It were these people who always were there, when I needed them.  It
were these people who taught, advised, and helped me with my research.
And it were they who, I believe, made me stronger, more clever, and
perhaps also more confident in my knowledge.  I am heavily indebted to
all of them and I am absolutely convinced that this work had never
come into existence if I did not have such a great working environment
as I was lucky to have it in Potsdam.

It would certainly take me another dissertation to express all the
great debt of gratitude that I owe my colleagues, teachers, friends,
advisors, and family, but I am afraid they would not like to wait that
long for me to finish it.  So, I should rather constrain myself in
this acknowledgment and try to compress all the infinite gratefulness
that I have for all of them in just few yet still essential words.

First of all, I certainly would like to thank my advisor, Manfred
Stede, who was much more a ``Doktorvater'' than just a supervisor for
me.  I thank Manfred for his courage in employing me, for the numerous
hours that he invested in our discussions, and for his valuable hints,
suggestions, and sound criticism.  Manfred's experience and wisdom
laid the foundation of the success of this work.  And his patience
allowed me to bring this work to the end.  Manfred introduced me to
the amazing world of science, a world which does not require a visa
but which does require a sober mind and a considerable load of
knowledge.  And I sincerely hope that he will open the doors to this
unbelievable world to many other Ph.D. students.

Of course, having a good ``Doktorvater'' is an invaluable precondition
for any finished dissertation, but another prerequisite which plays an
equally important role is certainly one's own family at home.  And I
have to admit that I had a great luck with both my families: the
traditional and the ``Doktor'' one.  My wife and my parents were the
people who, I believe, suffered most from my absence and the lack of
spare timen and I also convinced that these people would be happiest
it if I defend this thesis successfully.  And in this acknowledgment,
I would simultaneously like to apologize for my probably neglected
family duties and to promise that I will make up for them in future.

My time at the University of Potsdam certainly had not been that
enjoyable if it were not my colleagues: Tatjana Scheffler, Arne
Neumann, Yulia Grishina, Andreas Peldszus, and Jonathan Sonntag.  I
thank Tatjana for her help in the cooperation project ``Discourse
Analysis of Social Media''; I thank Arne for our talks, cultural
excursions, and his nice attitude to the people.  I am also much
indebted to Yulia who was by far the most reliable annotator in many
of our conducted annotation experiments.

I would also like to thank Konstantina Garoufi, Nikos Engonopoulos,
Mart\'{i}n Villalba, and \v{Z}eljko Agi\'{c} for our interesting talks
and our enjoyable common lunches.
