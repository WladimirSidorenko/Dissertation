% FILE: chapter-sentiment.tex  Version 0.01
% AUTHOR: Uladzimir Sidarenka

% This is a modified version of the file main.tex developed by the
% University Duisburg-Essen, Duisburg, AG Prof. Dr. G�nter T�rner
% Verena Gondek, Andy Braune, Henning Kerstan Fachbereich Mathematik
% Lotharstr. 65., 47057 Duisburg entstanden im Rahmen des
% DFG-Projektes DissOnlineTutor in Zusammenarbeit mit der
% Humboldt-Universitaet zu Berlin AG Elektronisches Publizieren Joanna
% Rycko und der DNB - Deutsche Nationalbibliothek

\chapter{Sentiment}

\section{Introduction to Sentiment Analysis}
\subsection{Definitions and Related Work}
\subsubsection{Granularity Levels of Sentiment Analysis}
\subsubsection{Rule-based and Machine Learning Approaches to the Analysis of Sentiments}
\subsection{Sentiment Analysis on Twitter}

\section{Sentiment Corpus}
\subsection{Selection Criteria}
\subsubsection{Topic Selection}
\subsubsection{Formal Selection}

\subsection{Annotation Scheme}
\subsection{Statistics and Preliminary Results}
\begin{table}[h]
  \centering\small
  \caption[Sentiment corpus statistics]{Statistics on the annotated
    sentiment corpus.\\ POL = corpus part with discussions about
    general politic topics; FE = corpus part describing the federal
    election 2013; PE = corpus part with discussions about the Pope
    election 2013; GEN = part of the corpus containing tweets with no
    particular topic}
  \begin{tabular}{|>{\centering}p{0.15\textwidth}|*{4}{>{\centering}p{\oosixthClmnWidth}|}
      >{\centering\bfseries}p{\oosixthClmnWidth}|*{4}{>{\centering}p{\oosixthClmnWidth}|}
      >{\centering\bfseries}p{\oosixthClmnWidth}|}
    \hline

    \multirow{2}{*}{\parbox{0.13\textwidth}{\centering Markable Type}}
    & \multicolumn{5}{>{\centering}p{7\oosixthClmnWidth}|}{Annotator
      1} &
    \multicolumn{5}{>{\centering}p{7\oosixthClmnWidth}|}{Annotator
      2}\tabularnewline\cline{2-11}

    & POL & FE & PE & GEN & Total & POL & FE & PE & GEN &
    Total\tabularnewline\hline

    Sentiment & 212 & 222 & 163 & 131 & 728 & 317 & 335 & 314 & 305 & 1271
    \tabularnewline\hline

    Source & 101 & 119 & 68 & 73 & 361 & 114 & 109 & 94 & 85 & 402
    \tabularnewline\hline

    Target & 229 & 279 & 184 & 151 & 843 & 342 & 369 & 328 & 324 & 1363
    \tabularnewline\hline

    Emotional Expression & 727 & 689 & 581 & 811 & 2808 & 662 & 669 & 671 & 768 & 2770
    \tabularnewline\hline

    Intensifier & 16 & 32 & 14 & 44 & 106 & 31 & 35 & 31 & 58 & 155
    \tabularnewline\hline

    Diminisher & 2 & 4 & 3 & 2 & 11 & 2 & 9 & 4 & 2 & 17
    \tabularnewline\hline

    Negation & 18 & 15 & 23 & 14 & 70 & 33 & 33 & 31 & 23 & 120
    \tabularnewline\hline
  \end{tabular}
  \label{table:sentiment-agreement-topics}
\end{table}

\subsection{Inter-Annotator Agreement}
\subsubsection{Inter-Annotator Agreement for Topics}
\begin{table}[h]
  \centering\small
  \caption[Inter-annotator agreement for the sentiment corpus across
    topics]{Inter-annotator agreement for the sentiment corpus across
    topics.\\ POL = corpus part with discussions about general politic
    topics; FE = corpus part describing the federal election 2013; PE
    = corpus part with discussions about the Pope election 2013; GEN =
    part of the corpus containing tweets with no particular topic}
  \begin{tabular}{|>{\centering}p{0.15\textwidth}|*{4}{>{\centering}p{\oosixthClmnWidth}|}
      >{\centering\bfseries}p{\oosixthClmnWidth}|*{4}{>{\centering}p{\oosixthClmnWidth}|}
      >{\centering\bfseries}p{\oosixthClmnWidth}|}
    \hline

    \multirow{2}{*}{\parbox{0.13\textwidth}{\centering Markable Type}}
    &
    \multicolumn{5}{>{\centering}p{7\oosixthClmnWidth}|}{$\kappa$-Agreement
      for Binary Overlap} &
    \multicolumn{5}{>{\centering}p{7\oosixthClmnWidth}|}{$\kappa$-Agreement
      for Proportional Overlap}\tabularnewline\cline{2-11}

    & POL & FE & PE & GEN & Total & POL & FE & PE & GEN &
    Total\tabularnewline\hline

    Sentiment & 0.35 & 0.35 & 0.45 & 0.41 & 0.39 & 0.27 & 0.29 & 0.36 & 0.34 & 0.32
    \tabularnewline\hline

    Source & 0.39 & 0.27 & 0.41 & 0.41 & 0.37 & 0.38 & 0.28 & 0.4 & 0.4 & 0.36
    \tabularnewline\hline

    Target & 0.32 & 0.38 & 0.4 & 0.39 & 0.38 & 0.26 & 0.28 & 0.31 & 0.32 & 0.3
    \tabularnewline\hline

    Emotional Expression & 0.64 & 0.57 & 0.68 & 0.66 & 0.64 & 0.6 & 0.54 & 0.65 & 0.63 & 0.61
    \tabularnewline\hline

    Intensifier & 0.46 & 0.48 & 0.21 & 0.62 & 0.52 & 0.46 & 0.48 & 0.21 & 0.6 & 0.51
    \tabularnewline\hline

    Diminisher & 0.67 & 0.44 & 0.0 & 0.4 & 0.37 & 0.67 & 0.44 & 0.0 & 0.4 & 0.37
    \tabularnewline\hline

    Negation & 0.44 & 0.1 & 0.36 & 0.21 & 0.28 & 0.44 & 0.1 & 0.36 & 0.21 & 0.28
    \tabularnewline\hline
  \end{tabular}
  \label{table:sentiment-agreement-topics}
\end{table}

\subsubsection{Inter-Annotator Agreement for Formal Criteria}

\subsection{Analysis of Annotator Mistakes}
\subsection{Related Work}
\subsection{Conclusions}

\section{Identification of Lexical Polarity}
\subsection{Ontology-based Polarity Identification}
\subsection{Corpus-based Polarity Identification}
\subsection{Machine Learning Approaches to Polarity Identification}
\subsection{Hybrid Methods for Polarity Identification}
\subsection{Related Work}
\subsection{Conclusions}

\section{Identification of Targets, Sources, and Sentiments}
\subsection{Rule-based Approaches to Sentiment Tagging}
\subsection{Machine Learning Approaches to Sentiment Tagging}
\subsubsection{Compared Systems}
\subsubsection{Features}
\subsubsection{Ablation Tests}
\subsubsection{Evaluation}
\subsection{Related Work}
\subsection{Conclusions}

\section{Identification of Polarity and Intensity of Sentiments}
\subsection{Compositional Methods for Identification of Polarity}
\subsection{Compositional Methods for Identification of Intensity}
\subsection{Related Work}
\subsection{Conclusions}
