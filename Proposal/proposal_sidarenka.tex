\documentclass[a4paper,11pt]{article}
%%%%%%%%%%%%%%%%%%%%%%%%%%%%%%%%%%%%%%%%%%%%%%%%%%%%%%%%%%%%%%%%%%
\usepackage{paralist}
\usepackage{amsthm}

%%%%%%%%%%%%%%%%%%%%%%%%%%%%%%%%%%%%%%%%%%%%%%%%%%%%%%%%%%%%%%%%%%
\renewcommand{\labelenumii}{\theenumi.\arabic{enumii} }
\theoremstyle{definition}
\newtheorem{exmp}{Example}[section]

%%%%%%%%%%%%%%%%%%%%%%%%%%%%%%%%%%%%%%%%%%%%%%%%%%%%%%%%%%%%%%%%%%
\title{Research Proposal for PhD-Thesis}
\author{Uladzimir Sidarenka}
\date{\today}

%%%%%%%%%%%%%%%%%%%%%%%%%%%%%%%%%%%%%%%%%%%%%%%%%%%%%%%%%%%%%%%%%%
\begin{document}
\maketitle

%% zwischen Kap 1 und 2 w�re wahrscheinlich eines der Art "Background:
%% Microblogging / Twitter" nicht schlecht.

%% Ich w�rde empfehlen jetzt f�r jede subsection einen l�ngeren Absatz
%% einzuf�gen (viertel / halbe Seite), der die Zielsetzung dieser
%% subsection angibt: was soll hier passieren, was sind die
%% Arbeitsschritte f�r Dich, ggf.: wie ist die Relation dieser subsection
%% zu den bisherigen Abschnitten. So dass man sich insgesamt ein Bild von
%% der gesamten Dissertation machen kann, auch wenn man bisher nichts von
%% ihr geh�rt hat. Je konkreter das wird, umso besser - zum Beispiel
%% kannst Du vielleicht in (4) schon mal kurz skizzieren, was denn
%% discourse-level sentiment von dem bisher besprochenen sentence-level
%% sentiment unterscheidet.  In (2) kannst Du schon mal aufschreiben, was
%% Deine Datengrundlage ist (Zusammensetzung des Korpus).  Auch das
%% Einarbeiten konkreter Beispiele w�re nicht schlecht, zB f�r die
%% Sentimentannotation.

%% Das Ziel des ganzen Dokuments w�re: Stell Dir vor, Du machst eine
%% Inhaltsangabe f�r einen Computerlinguisten, der das Vokabular
%% beherrscht, aber Dein Projekt nicht kennt. Wenn er das proposal
%% gelesen hat, sollte er einen klaren Eindruck davon bekommen, was Du
%% machst.

%% (Optional kannst Du bei den subsections auch angeben, ob Teile der
%% Arbeit bereits erledigt sind.)

%% Meinst Du das l�sst sich in erster Version bis Anfang der Woche
%% machen? Dann k�nnen wir ja zusammen noch etwas weiter dran basteln.

\section{Introduction}

The explosive expansion of online communication means (OCM) and social
networks in recent decades has lead to a change in the language in which
people communicate with each other via these means.  While communication
partners still are striving to express their opinions, attitudes, thoughts,
and moods in their full generality, they often do not have reach enough
possibilities for doing that in OCM as they would have them in everyday
speech.  One, for example, cannot show her vein via facial expression or voice
tonality in Twitter, neither there is wide room for kinesics in Facebook
posts.  As a consequence, new language forms and meanings started appearing in
online talks -- smileys replaced users' faces, exclamation marks replaced the
voice etc.

This work is supposed to be a bottom-up study of ways and means how people
express their polar attitudes or sentiments in online communication.  The
object of this study are users' discourses on Twitter.  The aim of the work is
to detect and determine how polar attitudes of people are expressed in these
discourses and what and first of all how different levels of the language
interact with each other when people express their opinions.

We begin with the lowest level of natural language processing and investigate
on the basis of character and word segments how widely the language of OCM
deviates from the standard language form.  In section
\ref{section-preprocessing} we subsequently propose a text normalization
procedure for mitigating the most spurious effects of casual writings which
were found during the analysis.  After a short digression in section
\ref{section-nlp-pipeline} in which intermediate NLP modules are described, we
change the emphasis of our work on the way how polar attitudes of people are
expressed on sentential and subsentential level and how these forms of
sentiments can be efficiently processed automatically.  These are the main
topics of section \ref{section-sentsentiment} in our thesis.  Since the
expressions of polar opinions however do not solely confine themselves to the
sentence level only, we take one step further and explore how relations
between different discourse segements can help us detect more polar
expressions and how already detected sentiments can be enriched with their
missing sources and targets by utilizing discourse analysis.

\section{Corpus}

At the very beginning our survey, we start with the preparation of a corpus in
which information relevant to sentiment extraction is annotated.  The aim of
this annotation is:
\begin{itemize}
  \item to determine whether polar judgements of users can be reliably
    annotated by people thus proving the amenability of this task to automatic
    computer analysis (this hypothesis has already been proven by Wiebe
    \cite{TWilsonDiss08} for English newspaper texts, our goal is to confirm
    this notion for German social media discourses);

  \item to evaluate existing systems and resources used for extracting
    German sentiments;

  \item to detect by which grammatical and rhetorical structures polar
    opinions are expressed;

  \item to find out which lexical, grammatical, and rhetorical clues
    might help us detect sentiments automatically.
\end{itemize}

For the above goals, we decided to annotate the following lexical
clues in the corpus:
\begin{itemize}
  \item \textit{emotional expressions} (lexical items which inherently possess
    some polar connotation in their meaning, e.g. ``\textbf{nice}'',
    ``\textbf{disgusting}'');
  \item \textit{intensifiers} and \textit{diminishers} (lexical items which
    may shift or vice versa reduce primary polar meaning of emotional
    expressions, e.g. ``\textbf{very} polite'', ``\textbf{hardly} pleasant'');
  \item \textit{negations} (lexical or grammatical constructions which on
    their own can turn the polarity of an expression to its complete opposite,
    e.g. ``\textbf{not} good'' meaning ``bad'' or ``\textbf{without} fear''
    meaning ``brave'').
\end{itemize}
The selection of these attributes was inspired by the annotation scheme used
for Layer 2 of the multi-layer sentiment corpus MLSA (cf. \cite{MLSA}) and
should primarily serve for evaluation of existing sentiment lexica, as well as
training basis for further machine learning based sentiment analysis.

To train and test the performance of the actual sentiment extraction system,
we have also annotated spans of the text encompassing the following elements
of polar opinions:
\begin{itemize}
  \item the actual \textit{sentiment}s which we regarded as text spans
    comprising all the relevant information pertaining to the expression of
    one's subjective attitude;
  \item sentiment \textit{target}s which considered as actual topics or
    subjects of particular polar judgements;
  \item sentiment \textit{source}s -- which represented immediate authors of
    the respective opinions.
\end{itemize}

In order to keep the corpus as balanced as possible but at the same
time to capture the maximal number of sentimental expressions, we have
assembled it from three equal parts:
\begin{enumerate}
\item the first part was represented by sentences which had at least
  one of their words present in the sentiment lexicon SentiWS
  (cf. \cite{SentiWS});

\item the second part consisted of sentences which did not have any of
  the lexicon words but contained some emoticon or multiple
  repetitions of exclamation marks;

\item finally, the conluding part was randomly sampled from the rest
  of our larger data collection in order to see how often sentiments
  are expressed in messages not complying with the former two
  criteria.
\end{enumerate}

All of the above three parts were drawn in equal shares consisting of 111
messages each from four different data collections, namely:
\begin{enumerate}
  \item a corpus of generic users' tweets gathered in April 2013;
  \item a corpus of users' tweets pertaining to the election of the
    Pope Francis;
  \item a corpus of users' tweets pertaining to general political
    tweets gathered in spring 2013;
  \item a corpus of users' tweets pertaining to the federal election
    in Germany in autumn 2013.
\end{enumerate}
The total size of the corpus run up to $3 * 4 * 111 = 1332$ messages.

This corpus has been annotated by ... annotators.  In a conluding step, we
also plan to measure the inter-annotator agreement and to see which annotation
elements have presented most difficulties to the annotators.

\textbf{Open Questions:} Maybe it would make sense to add discourse level
annotation already to this corpus?  We than would have all the annotation and
corpus questions solved in one place.


\section{Preprocessing}\label{section-preprocessing}

In order to ease and improve further NLP preprocessing with established NLP
applications, we decided to perform some normalization steps on the text prior
to doing its analysis.  The necessity of such normalization was motivated by
the fact that many of the existing natural language toolkits had been created
before online communication was widely spread and therefore were not prepared
for many phenomena which OCM discourses brought along with them.

Before devising the normalization procedure, a study of unknown words found in
Twitter messages was conducted.  We have regarded as unknown tokens which were
classified as out-of-vocabulary (OOV) by either \texttt{TreeTagger}
\cite{Schmid} or the freely available spell checking program
\texttt{hunspell}.  Our study showed that approximately 20~\% of all input
tokens on average were considered as OOV by both of the tools.  We subdivided
the detected OOV tokens into three major classes:
\begin{enumerate}
\item Limitations of machine readabale dictionaries (i.e. valid in-vocabulary
  terms which had erroneously been omitted from corresponfing tools'
  dictionaries);
\item Stylistic specifics of text genre (i.e. words which did not belong to
  the standard language but were perfectly valid terms in the domain of web
  discourse);\label{item-style}
\item Deviating spellings (i.e. words whose spelling either accidentally or on
  purpose differred from the official spelling norm).\label{item-spell}
\end{enumerate}
Of these three classes only the latter two were considered to be subject to
normalization.

A more thorough analysis of the detected OOV items also revealed the fact that
class \ref{item-style} and a half of the items belonging to class
\ref{item-spell} could easily be tackled by a set of transformation rules
(cf. \cite{Sidorenko}).  The remaining part of class \ref{item-spell} would
however require the use of probabilistic methods.

\section{Natural Language Processing Pipeline}\label{section-nlp-pipeline}
Along with normalization we simultaneously perform part-of-speech tagging
using the \texttt{TreeTagger} program by Helmut Schmid \cite{Schmid}.  Several
recent studies including \cite{Neunerdt} and \cite{Rehbein} have proved that
this tagger showed a superior quality for tagging social media texts in
comparison with other available taggers.

As dependency parser Bernd Bohnet's Mate Parser \cite{Bohnet} is used since
this parser also has shown better results in comparative studies
\cite{Sennrich}.

\textbf{Open Questions:} Should we add other modules with their description,
e.g. morphologic analyzer for splitting compounds, named entity recognition
etc.?

\section{Sentence-Level Sentiment Analysis}\label{section-sentsentiment}
In this chapter we describe the main module for extracting sentiment relations
from text.  The goal of this module is to extract a triple of elements $(s_i,
t_i, p_i)$ in which $s_i$ represents the source of a polar opinion, $t_i$
stands for its target, and $p_i$ is the respective polarity of the attitude.
Such triple representation is motivated by the traditional Liu's sentiment
quintuple \cite{Liu}. We however removed the elements \textit{aspect of an
  entity} and \textit{time of opinion's expression} from it.

We pursue to extract such triples by using a probabilistic graphical model
formalism, namely Markov Logic Network (MLN).  Our main tool for this is the
\texttt{Alchemy} framework developed by Domingos et al. \cite{Domingos}.  The
strength of this model is, on the one hand, the possibility to impose strong
constraints on the desired output and, on the other hand, the potential to
represent actual causality effects found in text in an easy and native way,
namely by means of first-order-logic predicates.  One of such predicates, for
example, could enforce an absolute contraint that a given word if it is
assumed to be sentiment's target should obligatory have either positive or
negative polarity.  Another logic rule could state that the polarity of a
presumable target should correlate with the prior polarity of its immediate
syntactic dependents, it could however be switched by negations and so on.

The task of the MLN is then to compute a joint probability distribution over
all possible variants on the basis of given training examples so that this
joint probability should obligatory (or at least as much as possible) satisfy
provided predicates and constraints.  As basic features for these constraints
we use the prior opinion polarity of words as they were learned from our
training corpus and/or existing polarity lexica.  Dependency grammar
connectors would allow us to track how syntactical relations could change this
polarity in context.  This polarity, for example, could be increased/decreased
by intesifiers or diminishers or turned to the opposite by the negations.  It
would also be possible to learn from corpus, in which case given word could be
considered as sentiment's target if, for instance, we know that some polar
words are located in its immediate vicinity in the syntactic tree.

The limitedness of this approach however is a relatively local scope of the
features.  Frankly speaking, one only could access setimental clues which are
present in the immediate respective clause.  Nevertheless, it could allow us
to capture such easy cases as specified in Examples \ref{exmpl-1} and
\ref{exmpl-2}.
\begin{exmp}\label{exmpl-1}
Westerwelle ist ein schlechter Politiker.  Er soll jetzt blo\ss{} nicht denken das
er f\"ur das Amt des Bundeskanzlers geeignet ist.
\end{exmp}

\begin{exmp}\label{exmpl-2}
Ich hab vieles nicht an der jetzt am Boden liegenden FDP gemocht, ...
\end{exmp}

It should however be noted that the confinement to a single clause could be a
significant drawback for sentiment system.  The respective sources and targets
of a given sentiment relation could be present outside of the local context
and embedded somewhere earlier in the discussion.  Moreover sentiment's
polarity could only become vivid if two or more rhetorical units were
considered.  This discourse augmentation should be the topic of Section
\ref{section-discsentiment}.


\section{Discourse-Level Sentiment Analysis}\label{section-discsentiment}
The full version of Example \ref{exmpl-2} in fact looks as follows:
\begin{exmp}\label{exmpl-3}
Ich hab vieles nicht an der jetzt am Boden liegenden FDP gemocht, den Guido z.B.
\end{exmp}
The addition ``den Guido z.B.'' on its own does not convey a sentiment on its
own if taken outside os the context.  If we however consider the fact that
this was an explanatory supplement to an immediately preceding sentiment
relation, we could draw the conclusion that ``Guido'' in fact is another
sentiment with the same polarity as its preceding counterpart.

Further examples of discourse relevance for opinion mining are contrastive and
concessive relations between different rhetorical units, for instance:
\begin{exmp}\label{exmpl-4}
Sie ist eine gute S\"angerin, aber sie kann sich auf der Szene nicht bewegen.
\end{exmp}
The sole fact that someone cannot move on the stage does not imply that it
necessarily should be a polar attitude.  One could speak, for example, about a
scrub which cannot move by itself.  This however would be just a statement of
an objectively given fact.  In Example \ref{exmpl-4} we however can see, that
the preceding rhetorical unit was an expression of positive attitude,
therefore its contradiction should rather be a negative opinion.

This is the first of the two main goals of this section.  We need to
investigate which and how many sentiment relations can be captured by
considering discourse relations.  For detecting such relations, we are
planning to use a rhetorical structure parser which could segment text and
give us the rhetirical relations between different segments.  Such relations
and belonging of words to particular rhetorical units could be incorporated in
our training set and used as additional features in the inference module.

Another part of this chapter is supposed to be a study in how many cases
sources and targets of sentiment relations are missing from given immediate
rhetorical units and what would be the best way to detect them in preceding
rhetorical segments.

\textbf{Open Questions:} We obviously need an RST parser, but is there one for
German? Or could one of the existing parsers for English be rapidly adjusted
-- do we have time for it?

\bibliographystyle{plain} \bibliography{bibliography}
\end{document}
