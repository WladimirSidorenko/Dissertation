% FILE: appendixA.tex  Version 2.1
% AUTHOR:
% Universit�t Duisburg-Essen, Standort Duisburg
% AG Prof. Dr. G�nter T�rner
% Verena Gondek, Andy Braune, Henning Kerstan
% Fachbereich Mathematik
% Lotharstr. 65., 47057 Duisburg
% entstanden im Rahmen des DFG-Projektes DissOnlineTutor
% in Zusammenarbeit mit der
% Humboldt-Universitaet zu Berlin
% AG Elektronisches Publizieren
% Joanna Rycko
% und der
% DNB - Deutsche Nationalbibliothek

\chapter{CRF Training and Inference}\label{chap:apdx:crf-inference}

%%%%%%%%%%%%%%%%%%%%%%%%%%%%%%%%%%%%%%%%%%%%%%%%%%%%%%%%%%%%%%%%%%

\tocless\section{Training}

Traditionally, the main objective of CRF's training consists in
finding such feature parameters $\bm{\Theta}$ that maximize the
log-likelihood of a training set
$\mathcal{D}=\{(\bm{x}[m],\bm{y}[m])\}_{m=1}^M$ (where $M$ represents
the total number of training examples, $\bm{x}[m]$ denotes the feature
vector of the $m$-th example, and $\bm{y}[m]$ stands for the vector of
its gold labels):
\begin{equation}\label{eqn:llhood}\scriptstyle
  \begin{split}
    \bm{\Theta} = \argmax_{\bm{\Theta}}
    \sum\limits_{m=1}^{M}\ell_{\bm{Y}|\bm{X}} = \argmax_{\bm{\Theta}}
    \sum\limits_{m=1}^{M}\ln{p_{\bm{\Theta}}(\bm{y}[m]|\bm{x}[m])}.
  \end{split}
\end{equation}
The conditional probability of gold labels for the $m$-th training
instance ($p_{\bm{\Theta}}(\bm{y}[m]|\bm{x}[m])$) is typically
computed as:
\begin{equation*}\label{eqn:prob}\scriptstyle
  \begin{split}
    p_{\bm{\Theta}}(\bm{y}[m]|\bm{x}[m]) =
    \frac{\exp\Big(\sum\limits_{i=1}^n\sum\limits_{k}\theta_{k}\mathit{f}_{k}(\bm{y}[m],
      \bm{x}, i)\Big)}{Z_{\bm{x}}},
  \end{split}
\end{equation*}
where $n$ represents the total length of that instance (in the case of
sentences, $n$ is usually the number of tokens); $\theta_k$ and
$\mathit{f}_k$ are the weight and the value of the $k$-th feature; and
$Z_{\bm{x}}$ is a normalization factor, which is estimated over all
possible features $f$ and label assignments $\mathcal{Y}^n$:
\begin{equation*}\small
  \begin{split}
    Z_{\bm{x}} \defeq \sum\limits_{\bm{y}' \in
      \mathcal{Y}^n}\exp\Big(\sum\limits_{i=1}^n\sum\limits_{k}\theta_{k}\mathit{f}_{k}(\bm{y}',
    \bm{x}[m], i)\Big).
  \end{split}
\end{equation*}
The partial derivatives of feature weights, which are needed for
optimizing the log-likelihood in Equation~\ref{eqn:llhood}, are known
to be equal to the difference between the empirical and model's
expectation of features over the whole training corpus:
\begin{equation}\label{eqn:pderiv}\small
  \begin{split}
    \frac{\partial}{\partial{\theta_k}}\ell_{\bm{Y}|\bm{X}} = &
    \sum\limits_{m=1}^M\sum\limits_{i=1}^n\big(\mathit{f}_k(\bm{y}[m],
    \bm{x}[m], i) - \bm{E}_{\bm{\Theta}}[\mathit{f}_k(\mathcal{Y}^n, \bm{x}, i)] \big)
  \end{split}
\end{equation}

\tocless\subsection{Linear-chain CRFs}

In the case of first-order linear-chain CRFs, these derivatives are
usually estimated with the help of the forward-backward algorithm (a
specific case of the belief-propagation method~[\citeauthor{Pearl:82},
  \citeyear{Pearl:82}]), which can be briefly described as follows.

For each position $i$ of training instance $\bm{x}[m]$ and for each
label $y$ of tagset $\mathcal{Y}$, one first computes the forward
score $\alpha[y][i]$ as:
\begin{equation}\label{eqn:folc-alpha}\small
  \begin{split}
    \alpha[y][i] =& \sum\limits_{y' \in\mathcal{Y}}\alpha[y'][i-1]t(y',y,i-1,i)s(y,i)
  \end{split}
\end{equation}
where $t(y',y,i-1,i)$ is the exponentiated sum of transition features
$\mathit{f}_t(y', y, \bm{x}, i-1, i)$ (which denote the transition
from label $y'$ at position $i-1$ to label $y$ at position $i$)
multiplied with their respective weights $\theta_t$:
\begin{equation*}\label{eqn:sm-trans-score}\small
  \begin{split}
    t(y',y,i-1,i)\defeq\exp\Big(\sum\limits_{t}\theta_t\mathit{f}_t(y',y,\bm{x},i-1,i)\Big).
  \end{split}
\end{equation*}
Similarly, $s(y, i)$ denotes the exponent of the sum of state features
$f_s$ times their weights $\theta_s$:
\begin{equation*}\label{eqn:sm-trans-score}\small
  \begin{split}
    s(y,i)\defeq\exp\Big(\sum\limits_{s}\theta_s\mathit{f}_s(y, \bm{x},
    i)\Big).
  \end{split}
\end{equation*}

The normalizing factor $Z_{\bm{x}}$ is then easily estimated as the
sum of all values from the last column of matrix $\alpha$:
\begin{equation*}\small
  Z_{\bm{x}} = \sum\limits_{y \in \mathcal{Y}}\alpha[y][n].
\end{equation*}

After estimating the forward scores, we can compute the backward
scores $\beta$ by applying the same procedure in reverse---from right
to left:
\begin{equation}\label{eqn:folc-beta}\small
  \begin{split}
    \beta[y][i] =& \sum\limits_{y' \in
      \mathcal{Y}}\beta[y'][i+1]t(y,y',i,i+1)s(y',i+1).%
  \end{split}
\end{equation}

The marginal probabilities $p_m$ of state and transition features are
then estimated as:
\begin{equation*}\label{eqn:folc-tmarginal}\small
  \begin{split}
    p_m(\mathit{f}_s(y, \bm{x},i)) =&\frac{1}{Z_{\bm{x}}}%
    \alpha[y][i]\beta[y][i];\\%
    p_m(\mathit{f}_t(y', y, \bm{x}, i-1, i)) &=\frac{1}{Z_{\bm{x}}}%
    \alpha[y'][i-1] \beta[y][i] s(y,i).
  \end{split}
\end{equation*}

Knowing these probabilities, we easily obtain the gradient of feature
weights using Equation~\ref{eqn:pderiv}.

\tocless\subsection{Semi-Markov CRFs} In contrast to linear-chain
CRFs, semi-Markov conditional random fields do not model transitions
between identical labels (\eg{} \textsc{Source} $\rightarrow$
\textsc{Source}), but instead try to partition the input into
contiguous spans of identical tags and infer the most likely label
assignment for these spans.

In order to do so, the model first determines the maximum possible
length ($L$) of a segment with identical labels that exists in the
training set.  The forward and backward scores are then calculated as:
\begin{equation}\label{eqn:sm-alpha}\small
  \begin{split}
    \alpha[y][i] =&
    \sum\limits_{d=0}^{L-1}\sum\limits_{\{y'\in\mathcal{Y}|y'\ne{}y\}}%
    \alpha[y'][i-d-1]\\%
    &\times t(y',y,i-d-1,i-d)s(y,[i-d,i]);
  \end{split}
\end{equation}

\begin{equation}\label{eqn:sm-beta}\small
  \begin{split}
    \beta[y][i] =& \sum\limits_{d=0}^{L-1}%
    \sum\limits_{\{y'\in\mathcal{Y}|y'\ne{}y\}}\beta[y'][i+d+1]\\%
    &\times t(y,y',\bm{x},i+d,i+d+1)s(y,[i,i+d]);
  \end{split}
\end{equation}
where $s(y,[i,i+d])$ is the exponentiated sum of all state features
$s(y, j)$ that are activated on the interval $[i,\dots,i+d]$.
%% (it could, for example, be the same binary state feature as in the
%% previous section but it would be activated multiple times if
%% $y=$\textsc{Target} and the span $[x_{i-d}, \dots, x_i]$ contains
%% multiple capitalized words).

The marginal probabilities of state and transition features are then
computed as:
\begin{equation*}\label{eqn:sm-marginal}\small
  \begin{split}
    p_m(&\mathit{f}_s(y,\bm{x},[i-d,i])) =\frac{1}{Z_{\bm{x}}}%
    s(y,[i-d,i])\\%
    &\times\sum\limits_{\{y'\in\mathcal{Y}|y'\ne{}y\}}%
    \alpha[i-d-1][y']t(y',y,i-d-1,i-d)\\%
    &\times\sum\limits_{\{y''\in\mathcal{Y}|y''\ne{}y\}}%
    \beta[i+1][y'']t(y,y'',i,i+1);\\%
  \end{split}
\end{equation*}
and
\begin{equation*}\small
  \begin{split}
    p_m(\mathit{f}_t(y',y,\bm{x},[i-d,i]))&=\frac{1}{Z_{\bm{x}}}\alpha[y'][i-1]\beta[y][i]\\%
    &\times{}t(y',y,i-1,i).%
  \end{split}
\end{equation*}

%% Assuming that the time needed to compute each feature is constant, the
%% asymptotic running time of this algorithm is $\mathcal{O}(n) =
%% nL|\mathcal{Y}|^2$ where $|\mathcal{Y}|$ is the size of the underlying
%% tagset.

\tocless\subsection{Higher-order CRFs}

In contrast to first-order models, which only consider the scores of
one immediate label to the left when computing the $\alpha$ values or
one immediate label to the right when estimating the $\beta$ scores,
higher-order CRFs keep separate track of each \emph{sequence of
  labels} that might precede or follow the currently analyzed token.

In particular, instead of simply computing the scores for each tagset
label $y \in \mathcal{Y}$ at each sentence position $i$, higher-order
conditional random fields estimate these values for complete sequence
of tags $y_{1}, \ldots, y_{d}$, where $d$ is the order of the model.

This extension is possible for both linear-chain- and semi-Markov
CRFs, and the way of estimating forward and backward scores as well as
computing marginal probabilities $p_m$ is almost identical to the
respective original implementations.  The only differences in the
higher-order case are that
\begin{itemize}
  \item it now becomes possible to use state and transition features
    that are associated with label chains up to length $d$ and not
    only single tags (\eg{} instead of using a state feature which
    tells that verbs starting with ``mis'' are likely to be
    \texttt{sentiment}s, one can refine the feature function and say
    that such words very probably represent \texttt{sentiment}s
    preceded by \texttt{source}s, \ie{} are associated with the label
    sequence <\texttt{source}, \texttt{sentiment}>);
  \item secondly, when estimating the scores $\alpha[y_{1}, \ldots,
    y_{d}][i]$ and $\beta[y_{1}, \ldots, y_{d}][i]$, one does not
    simply iterate over all cells of the previous or next column of
    the corresponding matrix, but only considers those preceding or
    following states that allow the label sequence $y_{1}, \ldots,
    y_{d}$ at the $i$-th position.  That is,
    Equations~\ref{eqn:folc-alpha} and~\ref{eqn:folc-beta} become:
\begin{equation*}
  \begin{split}
    \alpha[y_{1}, \ldots, y_{d}][i] =& \sum\limits_{\{y'_{1},\ldots,y'_{d}\in\mathcal{Y}^d|y'_{2},\ldots,y'_{d}=y_{1},\ldots,y_{d-1}\}}\alpha[y'_{1},\ldots,y'_{d}][i-1]\\
    &\times t\left(\left(y'_{1},\ldots,y'_{d}\right),\left(y_{1},\ldots,y_{d}\right),i-1,i\right)s\left((y_{1},\ldots,y_{d}),i\right)
  \end{split}
\end{equation*}
and
\begin{equation*}
  \begin{split}
    \beta[y_{1},\ldots,y_{d}][i] =& \sum\limits_{\{y'_{1},\ldots,y'_{d}\in\mathcal{Y}^d|y_{2},\ldots,y_{d} = y'_{1},\ldots,y'_{d-1}\}}\beta[y'_{1},\ldots,y'_{d}][i+1]\\
    &\times t\left(\left(y_{1},\ldots,y_{d}\right),\left(y'_{1},\ldots,y'_{d}\right),i,i+1\right)s\left((y'_{1},\ldots,y'_{d}),i+1\right),
  \end{split}
\end{equation*}
respectively.  The same change also applies to
Equations~\ref{eqn:sm-alpha} and~\ref{eqn:sm-beta} in the case of
semi-Markov models.
\end{itemize}

%% It should, however, be noted that the number of label sequences (which
%% is the first dimension of matrices $\alpha$ and $\beta$) increases
%% exponentially with the order of the model and can achieve
%% $\sum_{i=1}^d\left\vert\mathcal{Y}\right\vert^i$ for the linear-chain
%% CRF and
%% $\sum_{i=1}^d{\left(\left\vert\mathcal{Y}\right\vert{}-1\right)}^{i}$
%% for the semi-Markov model.\footnote{We sum over all powers from 1 to
%%   $d$ because higher-order models usually subsume all lower-order
%%   dependencies.}

%% In order to alleviate this exponential growth, we applied the
%% heutistical algorithm of \citet{Nguyen:14} which we can briefly
%% describe as follows: Let $\mathcal{Z} = \{z^1,\ldots,z^M\}$ denote the
%% set of all label patterns up to length $d$ that were observed in the
%% training set and let $\mathcal{A}$ denote the union of all tagset
%% labels $\{y| y \in \mathcal{Y}\}$ and proper prefixes of the label
%% patterns: $\mathcal{A} =
%% \mathcal{Y}\cup\{z^{i}_{1:k}|1\leq{}k<\vert{}z^{i}\vert{},1\leq{}i\leq{}M\}$.
%% Then, forward scores of the label patterns $\mathcal{A}$ for
%% semi-Markov models can be computed using the following formula:
%% \begin{equation*}\label{eqn:ho-alpha}\small
%%   \begin{split}
%%     \alpha[a][i] =&
%%     \sum\limits_{d=0}^{L-1}\sum\limits_{\{a'\in\mathcal{A}|a\leq{}^{s}_{\mathcal{A}}a'y\}}%
%%     \alpha[a'][i-d-1]\\%
%%     &\times t(a',a,i-d-1,i-d)s(a,[i-d,i]);\\%
%%   \end{split}
%% \end{equation*}
%% where $a\leq{}^{s}_{\mathcal{A}}a'y$ means that $a$ is the longest
%% possible suffix in $\mathcal{A}$ for the concatenation of the prefix
%% $a'$ and the label $y$.

%% The set of backward states $\mathcal{B}$ is then constructed by
%% concatenating every prefix $a \in \mathcal{A}$ with every tagset label
%% $y \in \mathcal{Y}$: $\mathcal{B} = \mathcal{A}\mathcal{Y}$.  The
%% backward scores $\beta$ are then computed for every state
%% $b\in\mathcal{B}$ in a similar way as the forward states $\mathcal{A}$
%% using the longest suffix relation:
%% \begin{equation*}\label{eqn:ho-alpha}\small
%%   \begin{split}
%%     \beta[b][i] =& \sum\limits_{d=0}^{L-1}%
%%     \sum\limits_{\{b'\in\mathcal{B}|b'\leq{}^{s}_{\mathcal{S}}by\}}\beta[b'][i+d+1]\\%
%%     &\times t(b,b',\bm{x},i+d,i+d+1)s(b,[i,i+d]);
%%   \end{split}
%% \end{equation*}

%% A detailed description of the computation of marginal probabilities
%% can be found in \citet{Nguyen:14}.

\tocless\subsection{Tree-structured CRFs}
%% As difficult as it might seem to be, the forward-backward algorithm is
%% just a simplified version of a more powerful algorithm that can be
%% applied to any acyclic graph structures and which is commonly referred
%% to as belief-propagation \cite{Pearl:82}.

The main difference between applying the belief-propagation algorithm
to trees instead of linear chains is that the inference flow happens
in a ``vertical'' way---from tree's leaves to its root and vice
versa---whereas in the standard forward-backward setting, we typically
compute the scores ``horizontally''---from the left-most word of a
sequence to the right-most one and then in the opposite direction.

More precisely, the $\alpha$ and $\beta$ scores for trees are
estimated as:
\begin{equation*}\label{eqn:tree-alpha-beta}\small
  \begin{split}
    \alpha[y][p] =& \prod\limits_{c \in \text{children}(p)}\left(%
    \sum\limits_{y' \in \mathcal{Y}} \alpha[y'][c] t(y',y,\bm{x},c, p)\right)%
    s(y, p);\\%
    \beta[y][c] =& \sum \limits_{y' \in \mathcal{Y}} %
    \frac{\alpha[y'][p]\beta[y'][p]}{\alpha_{c\rightarrow{}p}}%
    t(y, y', \bm{x},c,p);
  \end{split}
\end{equation*}
where $p$ is the index of the parent node of token $c$, and
$\alpha_{c\rightarrow{}p}$ is the part of the $\alpha$ score of token
$p$ that has been previously propagated to it from its child $c$:
\begin{equation*}\small
  \begin{split}
 \alpha_{c \rightarrow{} p} &\defeq \sum\limits_{y'' \in \mathcal{Y}} \alpha[y''][c] t(y'',y,\bm{x},c,p)
  \end{split}
\end{equation*}

The normalizing factor $Z_{\bm{x}}$ and marginal probabilities of
state features are calculated in the same way as for the linear-chain
models with the only difference that the partition factor $Z$ is
computed as the sum of the $\alpha$-scores of the root word $r$ and
not of the last word $n$ of the instance.

The marginal probabilities of transition features are computed using
the following equation:
\begin{equation*}\label{eqn:tree-tmarginal}\small
  \begin{split}
    p_m(\mathit{f}_t(y', y, \bm{x},c,p)) =&%
    \frac{\alpha[y'][c]t(y', y, c, p)\alpha[y][p]\beta[y][p]}{\alpha_{c\rightarrow{}p}Z_{\bm{x}}}.
  \end{split}
\end{equation*}

\tocless\section{Inference}

Once model parameters have been learned, one applies the optimized
model to new, unseen instances in order to predict their most probable
labels---a task which is commonly refered to as \emph{inference}.

For CRFs, inference actually boils down to computing the matrix
$\alpha$ with the following minor modifications:
\begin{itemize}
\item First of all, instead of taking the sum over all previous labels
  $y'\in\mathcal{Y}$ when computing $\alpha[y][i]$, one only estimates
  the maximum possible score for that cell that is possible
  w.r.t.\ the probabilities of its preceding labels;

\item Second, apart from storing the maximum (unnormalized)
  probability of label $y$ at the $i$-th position, one also stores the
  label at the previous position ($i-1$) that has lead to the maximum
  value of $\alpha[y][i]$.
\end{itemize}
In other words, in the case of linear-chain CRFs, we transform
Equation~\ref{eqn:folc-alpha} into:
\begin{equation}\label{eqn:folc-viterbi}\small
  \begin{split}
    \alpha[y][i] =& <\max_{y' \in\mathcal{Y}}\left(a(y',y,i-1,i)\right),
    \argmax_{y' \in\mathcal{Y}}\left(a(y',y,i-1,i)\right)>
  \end{split}
\end{equation}
where
\begin{equation*}\small
  \begin{split}
    a(y',y,i-1,i) \defeq \alpha[y'][i-1][0]t(y',y,i-1,i)s(y,i).
  \end{split}
\end{equation*}

We similarly modify the $\alpha$-computation in the semi-Markov case,
but this time, apart from remembering the highest possible probability
of the $y$-th label at the $i$-th position and its most likely
predecessor, we also need to store the most probable length of the tag
span $y$, \ie{}:
\begin{equation*}\small
  \begin{split}
    \alpha[y][i] =& <\max_{y' \in\mathcal{Y}, d \in [1,\ldots,L]}\left(a(y',y,d,i)\right),
    \argmax_{y' \in\mathcal{Y}, d \in [1,\ldots,L]}\left(a(y',y,d,i)\right)>
  \end{split}
\end{equation*}
with $a$ now defined as:
\begin{equation*}\small
  \begin{split}
    a(y',y,d,i) \defeq \alpha[y'][i-d-1][0]t(y',y,i-d-1,i-d)s(y,[i-d,i]).
  \end{split}
\end{equation*}

Finally, in the case of tree-structured CRFs, we could have basically
completely re-used the formula from Equation~\ref{eqn:folc-viterbi} if
each tree node only had one child.  But since, most of the time, this
is rarely the case, we need to circumvent the need for storing
multiple child labels in a single $\alpha$ cell because this
significantly slows down the inference due to additional memory
allocation on the fly.  The way we do that is by applying the
following trick: instead of storing in each cell $\alpha[y][i]$ the
maximum possible score for the $y$-th tag at the $i$-th position and
the labels of its children that have lead to this score, we store the
score and the most likely tag of the $i$-th node that yielded the
maximum possible value for the $y$-th tag at the parent position,
\ie{}:
\begin{equation}\small
  \begin{split}
    \alpha[y][c] =& <\max_{y' \in\mathcal{Y}}\left(a(y',y,c,p)\right),
    \argmax_{y' \in\mathcal{Y}}\left(a(y',y,c,p)\right)>
  \end{split}
\end{equation}
where
\begin{equation*}\small
  \begin{split}
    a(y',y,i-1,i) \defeq \alpha[y'][c][0]t(y',y,c,p)s(y,p).
  \end{split}
\end{equation*}
with $c$ denoting the index of the child, and $p$ standing for the
index of the parent.

After computing the scores for the final node, we scan the last (root)
column of the $\alpha$ matrix for the maximum value and trace back the
complete sequence of labels that has yielded this score---a procedure
which is commonly known as the Viterbi algorithm.
