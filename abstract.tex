% FILE: abstract.tex  Version 2.1
% AUTHOR:
% Universit�t Duisburg-Essen, Standort Duisburg
% AG Prof. Dr. G�nter T�rner
% Verena Gondek, Andy Braune, Henning Kerstan
% Fachbereich Mathematik
% Lotharstr. 65., 47057 Duisburg
% entstanden im Rahmen des DFG-Projektes DissOnlineTutor
% in Zusammenarbeit mit der
% Humboldt-Universitaet zu Berlin
% AG Elektronisches Publizieren
% Joanna Rycko
% und der
% DNB - Deutsche Nationalbibliothek
%% \begin{abstract}
%% Here is the english abstract.
%% \end{abstract}
%-englische-Zusammenfassung---------------------------------------
\selectlanguage{english}
\section*{Abstract}
\addcontentsline{toc}{section}{Abstract}

The growing popularity of online communication platforms along with the
increasing accessibility of their data to the research community made
Internet-based conversations (IBC) an attractive research object for many
scientific disciplines, including economics, political science, sociology etc.
But dealing with the abundance of unstructured text data in manual way, as it
used to be the case in several human sciences for many years, is both
cumbersome and inefficient.  Therefore, if we want to understand the way the
Internet-aware society works, we urgently need new effecient automatic natural
language processing (NLP) tools which are able to deal even with such
``unconventional'' linguistic variations as the Web language.

Unfortunately, until the last few years, online conversations have not
attracted as much attention of the scientific NLP community as they probably
deserved.  This concerns both the number of theoretical works dealing with IBC
and the number of practical applications adapted to the peculiarities of the
online language.  This situation is even more aggravated for languages other
than English.  In our thesis, we aim to fill the existing theoretical and
practical gaps in the research of German Web language.

From the theoretical point of view, we should investigate which linguistic
phenomena make the Web language different from its canonical variant.  We also
suggest a linguistically motivated classification of the detected IBC
phenomena into classes and conduct a quantitative analysis of those classes on
real corpus data.  By this means, we try to determine which linguistic IBC
aspects are most productive on the German Web nowadays and which of them have
the most negative effect on the quality of automatic text processing
applications.

From the practical perspective, we evaluate the performance of existing NLP
programs and techiques on online messages.  Then, we investigate possible ways
to improve that performance and to make NLP technologies more robust and more
suitable for the new form of language.

The object of our research are user discussions on the
Twitter\footnote{\url{https://twitter.com}} platform.  Since these discussions
differ from other forms of conversations in many ways and on all levels of
language, we also conduct our analysis on several NLP tasks with different
levels of analysis granularity.  The first level -- which we
call \emph{subsentential} -- deals with low-level preprocessing and
segmentation issues for words and sentences.  On the next level of analysis --
which we refer to as \emph{sentential} -- we concentrate on the opinion
extraction task also known as sentiment analysis.  Finally, in Chapter \ref{}
we turn our attention to the problem of discourse analysis, i.e. the problem
of detecting discourse units and relations between them in text.  We call this
level \emph{suprasentential}.

We perform our analysis on three different but related analysis subtasks:
\begin{inparaenum}[\itshape i)\upshape]
\item low-level pre-processing and segmentation,
\item sentiment extraction, and
\item discourse analysis.
\end{inparaenum}
For each of these tasks, we estimate the performance of the exisiting NLP
solutions.  We then make a linguistic analysis of the errors brought along by
the style and then propose our ways for avoiding those errors.

We first begin with the \emph{subsentential} level of automatic analysis.  At
this level, we investigate how difficult the automatic segmentation of words
and sentences is for tweets.  After choosing the best tokenization and
sentence splitting technique, we then preliminary estimate the performance of
existing PoS-taggers and syntactic parsers that exist for German.  Since
Twitter messages are known to contain many nontraditional words or words with
deviating spelling, we next make an attempt to

On the first level of analysis -- which we call \emph{subsentential} -- we
investigate how well existing NLP-applications, such as sentence splitters,
tokenizers, PoS-taggers, and syntactic parsers, perform on tweets and what
difficulties may be encountered by these tools in short messages which usually
do not occur in other text genres.  In order to assess the unconventionality
of Twitter posts quantitatively, we estimate the rate of out-of-vocabulary
(OOV) tokens in users' messages, classify found OOV's, and suggest several
techniques of how to deal with such phenomena in order to mitigate their
negative effect on the performance of NLP-applications.

We then turn to the \emph{sentential} level of automatic text analysis on the
example of the sentiment extraction task.  Here, we first present a novel
comprehensive corpus of German tweets manually annotated with fine-grained
sentiment relations.  From the psychological and linguistic perspectives, we
first show how difficult the task of sentiment annotation is for humans, what
language phenomena are typically used to express evaluative opinions, and by
which linguistic means the polarity and intensity of evaluations can be
affected.  With these observations in mind, we then demonstrate how well the
sentiment classification task can be done automatically using machine learning
and lexicon-based techniques.  In our experiments, we consder three related
subtasks:
\begin{inparaenum}[\itshape a)\upshape]
 \item automatic classification of prior polarity of words,
 \item machine learning-based prediction of fine-grained sentiment
 constituents (i.e. opinion holders, targets, and evaluation spans), and,
 finally,
 \item automatic detection of sentiment intensity and polarity.
\end{inparaenum}

In an attempt to improve the performance of our automatic sentiment extraction
system, we focus our attention on the \emph{suprasentential}
or \emph{discourse} level of natural language analysis and show how difficult
the automatic recognition of discourse segments and relations between those
segments is for Twitter.  In this part of the work, we adapt the traditional
definition of the rhetorical structure theory (\shortcite{Mann-88}) to the
peculiarities of tweets and extend this formalism by adding dialogue relations
for definining relations between tweets.


In the concluding step, we demonstrate how different levels of analysis might
beefit from each other by showing the impact of the improvements in low-level
processing and incorporation of discourse information on the performance of
the sentiment extraction module.
