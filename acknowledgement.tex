% FILE: acknowledgement.tex  Version 2.1
% AUTHOR:
% Universit�t Duisburg-Essen, Standort Duisburg
% AG Prof. Dr. G�nter T�rner
% Verena Gondek, Andy Braune, Henning Kerstan
% Fachbereich Mathematik
% Lotharstr. 65., 47057 Duisburg
% entstanden im Rahmen des DFG-Projektes DissOnlineTutor
% in Zusammenarbeit mit der
% Humboldt-Universitaet zu Berlin
% AG Elektronisches Publizieren
% Joanna Rycko
% und der
% DNB - Deutsche Nationalbibliothek

\textbf{Acknowledgment.}

At the beginning of this thesis, I would like to say a really big thank you to
all people who supported me throughout these difficult but immensely enjoyable
years of work on my dissertation; to people who always were there, when I
needed them; to people who taught, advised, and helped me with my research and
who, I believe, made me stronger, perhaps more clever, and probably also more
confident in my knowledge.  These people are my colleagues, my friends, my
teachers, but, first of all, my family and my advisor.

I am heavily indebted to all my supporters and I am absolutely convinced that
this work had never come into existence without them.  I am also fairly sure
that it would certainly take me another dissertation to express the whole debt
of gratitude that I owe each of my helpers, but I am afraid they would not
like to wait for me that long to finish it.  Therefore, as hard as it is, I
have to keep myself short in this acknowledgment and try to describe all the
infinite gratefulness that I feel with just few but nevertheless utterly
sincere words.

First and foremost, I, of course, would like to thank my advisor, Professor
Manfred Stede, who was much more a ``Doktorvater'' than just a supervisor for
me.  I thank Manfred for his courage in employing me, for the numerous hours
that he invested in our discussions, and for his valuable hints, suggestions,
and criticism.  His experience and wisdom laid the foundation for the success
of this work.  And his patience allowed me to bring this thesis to the end.  I
am very thankful to Manfred that he introduced me to the amazing world of
science, a world which does not require a visa but which does require a sober
mind and a considerable load of knowledge.  I sincerely hope that he will open
the doors to this unbelievably beautiful world for many other Ph.D. students
that he will certainly have in the future.

It is a matter of course that having a good ``Doktorvater'' is an obligatory
precondition for any completed dissertation, but another prerequisite which
plays an equally important role is still one's own family at home.  And I have
to admit that I was extremely lucky with both my families: the scientific and
the traditional one.  My wife and my parents have always given me incredible
support.  But, unfortunately, they also were those who, I think, suffered most
from my absence and lack of my spare time.  So, taking this rare opportunity
to speak to them from these pages, I would like to simultaneously apologize
for my possibly neglected family duties and to promise that I will make up for
it twofold.

Of course, my time at the University of Potsdam had never been that enjoyable
if I did not have such great colleagues as Tatjana Scheffler, Stephan Oepen,
Arne Neumann, Yulia Grishina, Andreas Peldszus, and Jonathan Sonntag.  I thank
Tatjana for her help with our research project ``Discourse Analysis of Social
Media'' and for her valuable advices and corrections of my papers.  I thank
Arne for our talks, cultural excursions, and his nice attitude to people.  And
I am, of course, much indebted to Yulia who was the most reliable annotator in
our numerous conducted annotation experiments.

I am also grateful to our colleagues from the chair of Theoretical
Computational Linguistics -- Alexander Koller and Thomas Hanneforth.  The
unsurpassable level of Alexander's professional competence and the consummate
quality of his work set an example for me that I eagerly tried to follow.  And
even if my attempt is still far away from completion, it would have been even
less successful if I did not take the unique opportunity to visit Thomas
Henneforth's excellent courses in automata theory and C++.  Moreover, I would
like to say a particularly big thank you to Thomas for the priceless research
hints that he gave me in our private discussions.

With greatest gratitude, I should also mention Konstantina Garoufi, Nikos
Engonopoulos, Mart\'{i}n Villalba, and \v{Z}eljko Agi\'{c}.  It has always
been a pleasure for me to listen to them in our talks and to enjoy their
company during our lunches.  I wish them great success in their academic and
industrial careers and I am also already looking forward to seeing them on
conferences, workshops, and meetings.

There are, certainly, much more people whom I would like to mention and who
made a significant contribution to the progress of this work, but, as I said,
I have to keep myself short.  So, if I failed to mention somebody on these
pages
