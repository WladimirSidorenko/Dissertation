% FILE: main.tex  Version 2.1
% AUTHOR:
% Universit�t Duisburg-Essen, Standort Duisburg
% AG Prof. Dr. G�nter T�rner
% Verena Gondek, Andy Braune, Henning Kerstan
% Fachbereich Mathematik
% Lotharstr. 65., 47057 Duisburg
% entstanden im Rahmen des DFG-Projektes DissOnlineTutor
% in Zusammenarbeit mit der
% Humboldt-Universitaet zu Berlin
% AG Elektronisches Publizieren
% Joanna Rycko
% und der
% DNB - Deutsche Nationalbibliothek

\chapter{Discourse-Augmented Sentiment Analysis}
Somasundaran et al., 2009

%% \section{Introduction to Discourse Analysis}
%% \subsection{Definitions and Related Work}
%% \subsection{Peculiarities of Discourse Analysis on Twitter}

%% \section{Corpus of Discourse Relations}
%% \subsection{Annotation Scheme}
%% \subsection{Statistics and Preliminary Results}
%% \subsection{Inter-Annotator Agreement}
%% \subsection{Related Work}
%% \subsection{Conclusions}

\section{Correlation between Discourse Relations and Sentiment}
\subsection{Discourse Influence on Sentiments}
\subsection{Sentiment Influence on Discourse}
\subsection{Related Work}
\subsection{Conclusions}

\section{Discourse-augmented Sentiment Analysis}
\subsection{Discourse-related Features}
\subsection{Analysis of Results}
\subsection{Related Work}
\subsection{Conclusions}
