% FILE: sentiment_fgsa.tex  Version 0.0.1
% AUTHOR: Uladzimir Sidarenka

% This is a modified version of the file main.tex developed by the
% University Duisburg-Essen, Duisburg, AG Prof. Dr. G�nter T�rner
% Verena Gondek, Andy Braune, Henning Kerstan Fachbereich Mathematik
% Lotharstr. 65., 47057 Duisburg entstanden im Rahmen des
% DFG-Projektes DissOnlineTutor in Zusammenarbeit mit der
% Humboldt-Universitaet zu Berlin AG Elektronisches Publizieren Joanna
% Rycko und der DNB - Deutsche Nationalbibliothek

\section{Fine-Grained Sentiment Analysis}\label{sec:snt:fgsa}

The main goal of fine-grained sentiment analysis (FGSA) is
traditionally defined as the identification of subjective evaluative
opinions (\emph{sentiments}), the holders of these opinions
(\emph{sources}), and their respectively evaluated objects
(\emph{targets}) in text.  Since an accurate automatic prediction of
these elements would enable us to track public attitude to literally
any object (e.g., a product, a service, or a political decision), FGSA
is commonly considered to be one of the most attractive, necessary,
but, unfortunately, also challenging goals in computational
linguistics.

Researchers usually consider this objective as a sequence labeling
(SL) task and address it with either of the two popular SL techniques:
conditional random fields or recurrent neural networks.  The former
approach represents a discriminative probabilistic graphical model
which relies on hand-crafted features; the latter framework utilizes a
recursive computational loop which can learn feature representations
completely automatically.  In this section, we are going to evaluate
each of these methods in detail in order to find out which of these
algorithms is better suited for the domain of German Twitter.
However, before we proceed with our evaluation, we should first make a
short linguistic digression and briefly discuss the definition of
textual spans, to which these approaches should assign their labels,
and the evaluation metrics, with which we should estimate the quality
of this assignment.

\subsection{Definition of Sentiment, Target, and Source Spans}
Despite some notable advances and an ongoing active research on
fine-grained opinion extraction, the crucial task of defining the
exact boundaries of sentiment spans and the spans of their respective
targets and sources has not been addressed in the literature with the
due attention yet.  Researchers typically overlook this problem,
leaving its solution to the discretion of their annotators
\cite[cf.][]{Wiebe:05,Klinger:13}.

In contrast to this, instead of relying on intuitive decisions of our
coders, we explicitly defined a rule for determining opinions'
boundaries by telling the experts to assign the \texttt{sentiment} tag
to ``\emph{minimal complete syntactic or discourse-level units that
  included both the target of an opinion and its actual evaluation}''.

In other words, during the annotation, our coders first had to
identify evaluated objects (targets) in text, then find the respective
evaluative expressions of these objects (usually but not necessarily
polar terms), and, finally, determine the smallest syntactic
components (noun or verb phrases) or discourse units (clauses or
sentences) where both of these entities appeared together.  An
annotation example labeled in compliance with this rule is provided
below:
\begin{example}\label{snt:fgsa:exmp:sent-anno1}
  \upshape\sentiment{Der neue Papst gilt als
    bescheidener, zur\"uckgenommener Typ.}\\[0.8em]
  \noindent\sentiment{The new Pope is believed to be a sober, modest
    man.}
\end{example}
\noindent In this sentence, our experts had to label the complete
clause as a sentiment, since this unit was the minimal syntactic
constituent which included both the object of an evaluation---``der
neue Papst'' (\textit{the new pope})---and the evaluation
itself---``bescheidener, zur\"uckgenommener Typ'' (\textit{a sober,
  modest man}).

We applied the same principles of minimality and completeness to the
annotation of targets and sources, requiring the main components of
these elements (typically nouns or verbs) to be labeled along with all
their syntactic dependants.  Accordingly, the correct annotation of
the target in the previous example had to look as follows:
\begin{example}\label{snt:fgsa:exmp:sent-anno2}
  \upshape\sentiment{\target{Der neue Papst} gilt als
    bescheidener, zur\"uckgenommener Typ.}\\[0.8em]
  \noindent\sentiment{\target{The new Pope} is believed to be a sober,
    modest man.}
\end{example}
\noindent with the \texttt{target} tag encompassing the whole noun
phrase---``der neue Papst'' (\textit{the new pope})---not only its
main noun.

Similarly, source elements had to cover complete syntactic structures
as shown in Example~\ref{snt:fgsa:exmp:src-anno1}:
\begin{example}\label{snt:fgsa:exmp:src-anno1}
  \upshape\sentiment{Die Homosexuellenehe war f\"ur \source{den Kardinal, der jetzt Papst ist,} eine Zerst\"orung von Gottes Plan}\\[0.8em]
  \noindent\sentiment{For \source{the cardinal, who is the Pope now,}
    the same-sex marriage was a destruction of God's plan.}
\end{example}
\noindent This time, again, the whole noun phrase including its
dependent attributive clause---``den Kardinal, der jetzt Papst ist,''
(\textit{the cardinal, who is the Pope now,})---had to be labeled with
the \texttt{source} tag because this constituent was the only
\emph{minimal complete} syntactic node which encompassed both the
immediate holder of the opinion---``Kardinal'' \textit{cardinal}---and
its grammatical dependants without including any of its parental
elements.

\subsection{Evaluation Metrics}
A natural question which arises after defining the span boundaries for
human coders is that of the best way to compare these spans with
automatically assigned labels.  One possibility to estimate the
quality of such automatic assignment is to compute the precision,
recall, and \F{}-scores using either the binary overlap or exact match
metric \cite{Choi:06,Breck:07} .  The former method considers an
automatically labeled span as correct if it has at least one token in
common with a labeled entity from the gold annotation.  The latter
metric only regards as true positives automatic spans which have
absolutely identical boundaries with the gold labels.  Unfortunately,
both of these approaches are somewhat problematic: While the binary
overlap might be overly optimistic, always assigning perfect scores to
automatic spans which cover the whole sentence; the method of exact
match might, vice versa, be too drastic, considering the whole
assignment as false if only one (possibly irrelevant) token is
missing.

Instead of relying on these measures, we opted for the ``golden mean''
solution to this problem which was proposed by \citet{Johansson:10}.
In their work, the authors introduced another metric for estimating
the quality of an automatic assignment, in which they penalized the
predicted spans proportionally to the number of tokens whose labels
were different from the gold annotation.  More precisely, given a set
of manually annotated gold entities $\mathcal{S}$ and automatically
tagged spans $\widehat{\mathcal{S}}$, they estimated the precision of
the automatic assignment as:
\begin{equation*}
  P(\mathcal{S}, \widehat{\mathcal{S}}) = \frac{C(\mathcal{S}, \widehat{\mathcal{S}})}{|\widehat{\mathcal{S}}|},
\end{equation*}
where the \emph{span coverage} metric
$C(\mathcal{S},\widehat{\mathcal{S}})$ was computed as the number of
overlapping tokens across all pairs of manually ($s_i$) and
automatically ($s_j$) annotated entities: $C(\mathcal{S},
\widehat{\mathcal{S}}) = \sum_{s_i \in \mathcal{S}}\sum_{s_j \in
  \widehat{\mathcal{S}}}c(s_i, s_j)$; and $|\widehat{\mathcal{S}}|$
meant the total number of tokens automatically labeled with the given
tag.  Similarly, the recall of such assignment was estimated as:
\begin{equation*}
  R(\mathcal{S}, \widehat{\mathcal{S}}) = \frac{C(\mathcal{S}, \widehat{\mathcal{S}})}{|\mathcal{S}|},
\end{equation*}
and the \F{}-measure was normally computed as the harmonic mean of the
precision and recall scores:
\begin{equation*}
  F_1 = 2\times\frac{P \times R}{P + R}.
\end{equation*}
Since this proportional estimation adequately accommodates both
extrema of an automatic assignment---too long and too short
spans---and also penalizes for erroneous and spurious labels, we will
rely on it throughout our subsequent experiments.

\subsection{Fine-Grained Sentiment Analysis Using Conditional Random
  Fields}
The first automatic approach that we are going to evaluate in this
section is that of the conditional random fields (CRFs).  First
introduced by \citet{Lafferty:01}, CRFs rapidly grew in popularity,
turning into one of the most commonly used probabilistic frameworks,
which dominated the NLP field for more than a decade.  The main resons
for such huge success were:
\begin{enumerate}[1)]
\item the \emph{structural nature} of CRFs, which, in contrast to
  single-entity classifiers such as logistic regression or SVM, make
  their predictions over a sequence of covariates, trying to find the
  most likely label assignment for the whole sequence not only its
  individual elements;
\item the \emph{discriminative power} of this framework, which, in
  contrast to generative probabilistic models such as HMM
  \cite{Rabiner:86}, optimizes the conditional probability
  $P(\boldsymbol{Y}|\boldsymbol{X})$ instead of maximizing the joint
  distribution $P(\boldsymbol{X},\boldsymbol{Y})$ and consequently can
  efficiently deal with overlapping and correlated features;
\begin{example}[Overlapping and Correlated Features]
  In order to demonstrate the different effects of correlated and
  overlapping features on generative and discriminative models, let us
  go through an example where we need to predict whether a tweet
  mentioning ``Merkel'' and ``Steinmeier'' is about the Christian
  Democratic Union (\texttt{CDU}) or Social Democratic Party of
  Germany (\texttt{SPD}).  As features for this task, we will use
  lexical unigrams appearing in the training data.  Assuming that our
  training set consists of three messages mentioning ``Merkel'' and
  one microblog mentioning ``Steinmeier'' which are labeled as
  \texttt{CDU} plus one tweets mentioning ``Merkel'' and three posts
  mentioning ``Steinmeier'' which are annotated as \texttt{SPD}, the
  na\"{i}ve Bayes model will estimate the probability of the two
  competing classes as:
  \begin{align*}
    P(\mathbf{x}, CDU) =& P(\textrm{Merkel},\textrm{Steinmeier}|CDU)\times P(CDU)\\
    =& P(\textrm{Merkel}|CDU)\times P(\textrm{Steinmeier}|CDU) \times P(CDU)\\
    =&\frac{3}{4}\times\frac{1}{4}\times\frac{4}{8}\approx 0.0938\\
    P(\mathbf{x}, SPD) =& P(\textrm{Merkel},\textrm{Steinmeier}|SPD)\times P(SPD)\\
    =& P(\textrm{Merkel}|SPD)\times P(\textrm{Steinmeier}|SPD) \times P(SPD)\\
    =&\frac{1}{4}\times\frac{3}{4}\times\frac{4}{8}\approx 0.0938.\\
  \end{align*}
  After normalizing these probabilities, we will get equal 50\%
  chances for each of the parties, which is fair regarding the token
  distribution in our corpus.  However, if we replace ``Merkel'' with
  ``von der Leyen'' in the training data and test example and rerun
  this experiment, the probability will get significantly skewed:
  \begin{align*}
    P(\mathbf{x}, CDU) =& P(\textrm{von},\textrm{der},\textrm{Leyen},\textrm{Steinmeier}|CDU)\times P(CDU)\\
    =& P(\textrm{von}|CDU)\times P(\textrm{der}|CDU)\times P(\textrm{Leyen}|CDU)\\
    &\times P(\textrm{Steinmeier}|CDU) \times P(CDU)\\
    =&\frac{3}{4}\times\frac{3}{4}\times\frac{3}{4}\times\frac{1}{4}\times\frac{4}{8}\approx 0.0527\\
    P(\mathbf{x}, SPD) =& P(\textrm{von},\textrm{der},\textrm{Leyen},\textrm{Steinmeier}|SPD)\times P(SPD)\\
    =& P(\textrm{von}|SPD)\times P(\textrm{der}|SPD)\times P(\textrm{Leyen}|SPD)\\
    &\times P(\textrm{Steinmeier}|SPD) \times P(SPD)\\
    =&\frac{1}{4}\times\frac{1}{4}\times\frac{1}{4}\times\frac{3}{4}\times\frac{4}{8}\approx 0.0059,\\
  \end{align*}
  which, after normalization, would result in 90\% chances for
  \texttt{CDU} and a 10\% score for \texttt{SPD}, even though we only
  changed the name of the politician.

  A different situation is observed for discriminative models such as
  maximum entropy: Instead of optimizing the joint distribution
  $P(\mathbf{x}, y)$ as it is done by generative frameworks,
  discriminative classifiers seek to optimize the conditional
  likelihood $P(y|\mathbf{x})$ by maximizing the probability of the
  training set $\sum_{i=1}^N\log P(y_i|\mathbf{x}_i,
  \mathbf{w})$, where $N$ is the number of training instances,
  $y_i$ is the gold label for the $i$-th instance (e.g., 1 for
  \texttt{CDU} and 0 for \texttt{SPD}), $\mathbf{x}_i$ stands for the
  feature vector of this instance, and $\mathbf{w}$ represent
  the learned coefficients of these feature.  The probability $P$ is
  typically normally computed using the sigmoid function $\frac{1}{1 +
    e^{-(\mathbf{x}_i, \mathbf{w})}}$.  Working through the
  partial derivatives of this probability, we will arrive at the
  solution $w_1 \approx 0.5052$ for the feature ``Merkel'' and
  $w_2 \approx -0.5052$ for the feature ``Steinmeier'' for the
  first example, which would again result in equal 50\% chances for
  both classes.  In the second example, however, all three features
  ``von'', ``der'', and ``Leyen'' will get an equal weight of $\approx
  0.2936$, and the ``Steinmeier'' feature will receive a coefficient
  of $\approx -0.4364$, which would result in 60\% probability for the
  test message being about the CDU and 40\% that the tweet is about
  the SPD.  Even though this still means a slight skewness towards
  \texttt{CDU}; this time, the effect of correlated features is much
  less dramatic than in the generative case.
\end{example}
\item and, finally, the \emph{avoidance of the label bias problem}, to
  which other discriminative classifiers such as MEMM
  \cite{McCallum:00} are known to be susceptible.
  \begin{example}[Label Bias Problem]
    The label bias problem arises in the cases where a locally optimal
    decision outweights globally superior solutions.  Consider, for
    example, the sentence ``Aber gerade Erwachsene haben damit
    Schwierigkeiten.'' (\textit{But especially adults have
      difficulties with it.}), for which we need to compute the most
    probable sequence of part-of-speech tags.

    \begin{center}
    \begin{tikzpicture}[node distance=5cm]
      \tikzstyle{tag}=[shape=circle split,draw=black,minimum size=2.5em,inner sep=1,fill=black!0]
      \tikzstyle{word}=[draw=none,inner sep=10pt]

      \node[word] (A) at (1, 1) {Aber};
      \node[tag] (B) at (1, 3) {\footnotesize KON \nodepart{lower} 1.};
      \node[word] (D) at (3, 1) {gerade};
      \node[tag] (E) at (3, 2) {\footnotesize ADJA \nodepart{lower} .5};
      \node[tag] (F) at (3, 4) {\footnotesize ADV \nodepart{lower} .5} ;
      \node[word] (G) at (5, 1) {Erwachsene};
      \node[tag] (I) at (5,2) {\footnotesize ADJA \nodepart{lower} .5} ;
      \node[tag] (H) at (5,4) {\footnotesize NN \nodepart{lower} .5};
      \node[word] (J) at (7,1) {haben};
      \node[tag] (K) at (7,3) {\footnotesize VA \nodepart{lower} 1.};
      \node[word] (J) at (9,1) {\ldots};

      \path [-] (B) edge node[below] {$.5$} (E);
      \path [-] (B) edge node[above] {$.5$} (F);

      \path [-] (E) edge node[below] {$.1$} (I);
      \path [-] (E) edge node[above] {$.9$} (H);
      \path [-] (F) edge node[below] {$.5$} (I);
      \path [-] (F) edge node[above] {$.5$} (H);

      \path [-] (I) edge node[below] {$.2$} (K);
      \path [-] (H) edge node[above] {$.8$} (K);
    \end{tikzpicture}\label{fig:memm-crf}
    % \caption{Feature weights for states and transitions of the
    %   part-of-speech example.}\label{fig:snt:memm-crf}
    \end{center}

    Assuming that feature weights are distributed as shown in
    Figure~\ref{fig:memm-crf}, we will first estimate the likelihood
    of different sequences using the approach of Maximum Entropy
    Markov Networks (MEMM)---a predecessor of Conditional Random
    Fields.  According to this formalism, the probability of the
    correct labeling of the initial part of the sentence
    ($KON-ADV-NN-VA$) would be equal to:
    \begin{align*}
      P(KON, ADV, NN, VA) =& P(KON)\times P(ADV|KON)\times
      P(NN|ADV)\\
      &\times P(VA)\times P(VA)
    \end{align*}
  \end{example}
\end{enumerate}
CRFs attain these useful properties thanks to a neatly formulated
objective function, in which they seek to optimize the global
log-likelihood of the gold labels $\mathbf{Y}$ conditioned on the
training data $\mathbf{X}$.  More precisely, given a set of training
instances $\mathcal{D} = \{(\mathbf{x}^{(i)},
\mathbf{y}^{(i)})\}_{i=1}^N$, CRF's training tries to find feature
coefficients $\mathbf{w}$ which maximize the log-probabilities
of dependent variables ($\mathbf{y}$) given their respective
covariates ($\mathbf{x}$) over the whole data set:
\begin{equation}\label{snt:fgsa:eq:crf-ell}
  \mathbf{w} = \argmax_{\mathbf{w}}\sum_{i=1}^N\ell
  (\mathbf{y}^{(i)}|\mathbf{x}^{(i)}),
\end{equation}
where $\ell(\mathbf{y}^{(i)}|\mathbf{x}^{(i)})$ denotes the
conditional log-likelihood $\ln P(\mathbf{y}^{(i)}|\mathbf{x}^{(i)})$
of the gold labels for the $i$-th training instance.  This likelihood
is commonly estimated using a globally normalized softmax function:
\begin{equation}
  P(\mathbf{y}^{(i)}|\mathbf{x}^{(i)}) = \frac{\prod_{m=1}^{M}
    \exp(\sum_jw_{j} \times f_j(x_{m}, y_{m-1}, y_{m}))}{Z},
\end{equation}
where $M$ means the length of the $i$-th training instance
($|\mathbf{x}^{(i)}|$), $f_j(x_{m}, y_{m-1}, y_{m})$ denotes the value
of the $j$-th feature function $f$ at position $m$, $w_j$ stands for
the corresponding weight of this feature, and $Z$ is a normalization
factor calculated over all possible label assignments:
\begin{equation}
  Z =
  \sum_{y'\in\mathcal{Y},y''\in\mathcal{Y}}\prod_{m=1}^{M} \exp(\sum_jw_{j} \times f_j(x_{m}, y'_{m-1}, y''_{m})).
\end{equation}
Taking the log-likelihood function from
Equation~\ref{snt:fgsa:eq:crf-ell} as our training objective and
computing its partial derivative w.r.t. the feature weight $w_j$, we
arrive at the solution:
\begin{equation}
  \frac{\partial}{\partial w_j}\ell = .
\end{equation}


Training

Decoding

Features

Topologies

Results

Ablation Tests

% Our next step after obtaining the training data was to train automatic
% classifiers to see how good their performance would be on this dataset
% for the two proposed sentiment interpretations.

% Based on the current state-of-the-art practices in the sentiment
% analysis research, we formulated this task as a sequence labeling
% problem.  The main objective of this problem is to assign the
% most-probable label sequence $\vec{y} =
% \langle{}y_0,\ldots,y_n\rangle$ to an input instance $\bm{x} =
% \langle{}x_0,\ldots,x_n\rangle$ given a feature representation
% $\vec{x}$ for that instance.  The tagset $\mathcal{Y}$ in our case
% consisted of only four labels $\mathcal{Y} = \{\text{TRG}, \text{SRC},
% \text{SNT}, \text{NON}\}$ which represented targets, sources,
% sentiments, and ``none of the above'' elements respectively.

% Again, relying on the state-of-the-art approaches, we chose to use the
% conditional random fields classifiers which had previously shown
% superior results due to the discriminative structured nature of their
% predictions.

% \subsubsection{Linear-chain CRFs}
% In the case of the first-order linear-chain CRF model, the conditional
% probability of $\vec{y}$ given $\vec{x}$ w.r.t. the feature parameters
% $\Theta$ is formulated as the softmax function:
% \begin{equation*}\label{eqn:prob}\small
%   \begin{split}
%     p_{\Theta}(\vec{y}|\vec{x}) =
%     \frac{\exp\Big(\sum\limits_{i=1}^n\sum\limits_{k}\theta_{k}\mathit{f}_{k}(\vec{y},
%       \vec{x}, i)\Big)}{Z_{\vec{x}}}
%   \end{split}
% \end{equation*}
% where $n$ represents the total length of the input instance,
% $\theta_k$ and $\mathit{f}_k$ are the weight and value of the $k$-th
% feature activated at the $i$-th item of that instance (in our case,
% the $i$-th token of the input sentence), and $Z_{\vec{x}}$ is the
% normalization factor which is computed over all possible label
% assignments $\mathcal{Y}^n$:
% \begin{equation*}\small
%   \begin{split}
%     Z_{\vec{x}} = \sum\limits_{\vec{y}' \in
%       \mathcal{Y}^n}\exp\Big(\sum\limits_{i=1}^n\sum\limits_{k}\theta_{k}\mathit{f}_{k}(\vec{y}',
%     \vec{x}, i)\Big)
%   \end{split}
% \end{equation*}
% In the case of the first-order linear-chain models, the feature
% function $\mathit{f}_k(\vec{y},\vec{x}, i)$ is a binary- or
% real-valued function of the token $x_i$ and its predicted tag $y_i$
% (state feature) or a pair of adjacent token labels $(y_{i-1}, y_i)$
% (transition feature).

% During the training step, feature parameters $\Theta$ are successively
% optimized in order to maximize the log conditional likelihood of the
% training data $\mathcal{D} = \{(\vec{x}_m,\vec{y}_m)\}_{m=1}^M$.  The
% partial derivatives of the log-likelihood function $\ell(\mathcal{D})$
% w.r.t. to the feature weights $\Theta$, which are needed for
% performing this optimization, are then computed as the difference
% between the empirical and the model expectation of the respective
% feature values taken over the whole training set:
% \begin{equation*}\label{eqn:pderiv}\small
%   \begin{split}
%     \frac{\partial}{\partial{\theta_k}}\ell(\mathcal{D}) = &
%     \sum\limits_{m=1}^M\sum\limits_{i=1}^{n}\big(\mathit{f}_k(\vec{y}_m,
%     \vec{x}_m, i) -
%     \bm{E}_{\bm{\Theta}}[\mathit{f}_k(\mathcal{Y}^n, \vec{x}_m, i)]
%     \big)
%   \end{split}
% \end{equation*}
% The latter expectations can be estimated dynamically at each training
% step using the belief-propagation algorithm
% \cite{Pearl:82}.\footnote{A detailed description of the
%   belief-propagation algorithm in application to different CRF
%   variants is provided in Appendix A of this paper.}

% \subsubsection{Semi-Markov CRFs}
% Instead of making their predictions over single tokens, semi-Markov
% CRFs try to partition the input sequence $\bm{x}$ into contiguous
% segments $\vec{s} = \langle{}s_1,\ldots,s_l\rangle{}$ where each
% segment $s_i$ consists of one or more tokens which all share the same
% common tag $y_i$.

% Features in semi-Markov models are defined at the level of segments
% and, like in the linear-chain case, are binary- or real-valued
% functions associated either with the segment $s_i$ and its label $y_i$
% or a pair of adjacent segment labels $(y_{i-1}, y_i)$.

% During the training and the later prediction step, the model considers
% all potential segmentations for all possible label assignments and all
% possible segment lengths ranging from 1 to $L$ where $L$ is the length
% of the longest sequence of tokens with identical tags that was
% observed in the training data.  The decoding step then jointly finds
% the most probable segmentation $\hat{s}$ and the label sequence
% $\vec{y}$ for that segmentation using either the greedy search or the
% dynamic Viterbi algorithm.

% \subsubsection{Higher-order CRFs}
% Unlike the first-order models whose transition features can only
% access a pair of adjacent element labels $(y_{i-1}, y_i)$, transition
% features of higher-order linear-chain and semi-Markov CRFs encode
% information about complete label sequences up to length $d$ that might
% precede the predicted label $y_i$ where $d$ is the maximum order of
% the model.

% But since the number of such label sequences grows exponentially in
% the order of the encoded dependencies (up to
% $\sum_{i=1}^d\left\vert\mathcal{Y}\right\vert^i$), the running time of
% the training and inference algorithms rapidly becomes prohibitively
% expensive and amounts to
% $\mathcal{O}(n(\sum_{i=1}^d\left\vert\mathcal{Y}\right\vert^{i})^2)$
% for the linear-chain case and
% $\mathcal{O}(nL(\sum_{i=1}^d\left\vert\mathcal{Y}\right\vert^{i})^2)$
% for the semi-Markov models.

% Because this problem is intractable in general \cite{Istrail:00},
% researchers are usually forced to resort to some sort of heuristics
% like an early state pruning \cite{Mueller:13} or state-transition
% reduction.

% For the purpose of our experiments, we applied the affix-based
% algorithm of \citet{Nguyen:14}.  This algorithm relies on the
% advantage that the actual number of possible sequences of $d$
% consecutive segment labels that are observed in the training set is
% typically much smaller than
% $\sum_{i=1}^d\left\vert\mathcal{Y}\right\vert^i$.  Furthermore, the
% number of possible transitions that need to be considered for the
% given label sequence $(y_{i-d}, y_{i-d+1},\ldots,y_i)$ can further be
% reduced by restricting them only to those patterns for which this
% label sequence forms the longest possible suffix.

% The running time of this algorithm for the semi-Markov case runs up to
% $\mathcal{O}(nL\left\vert\mathcal{P}\right\vert\left\vert\mathcal{Y}\right\vert)$
% where $\left\vert\mathcal{P}\right\vert$ is the total number of the
% label prefixes observed in the training set.  And even though this
% number can still be exponential in the order of the dependencies in
% the worst case, this algorithm usually leads to much faster results in
% practice especially when the label patterns in the addressed domain
% are sparse.

% \subsubsection{Tree-structured CRFs}
% In contrast to the linearly structured models, in which the predicted
% label $y_i$ only depends on the state attributes of the respective
% item $x_i$ and the label scores of the preceding tokens or segments
% and where the syntax information can only be expressed as feature
% attributes of single items, tree-CRFs allow to incorporate syntactic
% dependencies directly into the model's structure.

% In our experiments, we used the automatically constructed dependency
% trees of the sentences as the underlying graphs for the
% tree-structured CRF model.  Like in the linear CRF case, every
% predicted label $y_i$ in this model is dependent on the state features
% of the token $x_i$, but, instead of hard-coding dependencies on the
% label scores of the previous tokens, we let transition features
% connect the predicted label with every label score $y_c$ where $c$
% ranges over all indices of the child nodes of the token $x_i$ in the
% dependency tree.

% Since the resulting model graphs are acyclic, the inference in this
% type of CRFs can still be exact and its running time remains the same
% as for the first-order linear chain models:
% $\mathcal{O}(n\left\vert\mathcal{Y}\right\vert^{2})$.

\subsection{Fine-grained Sentiment Analysis Using Recurrent Neural Networks}
Introduction to RNN

GRU and LSTM

Training

Decoding

Topologies

Results

LIME

\subsection{Related Work}
\citet{Choi:05}
\citet{Choi:09}
\citet{Breck:07}
\citet{Johansson:10}
\subsection{Summary and Conclusions}

\subsubsection{Experiments}\label{sec:experiment}
For our experiments, we split our final corpus into three parts using
70\% of it as a training set, 15\% as cross-validation data, and the
remaining 15\% for testing purposes.  The data were tokenized with an
adjusted version of Christopher Potts'
Twitter-tokenizer\footnote{\url{http://sentiment.christopherpotts.net/code-data/happyfuntokenizing.py}}
and preprocessed using the rule-based normalization approach of
\citet{Sidarenka:13}.

%% During the normalization, Twitter-specific phenomena like @-mentions,
%% retweets, and URIs that were not syntactically integrated in any
%% sentence of the message were removed from the tweets and those
%% elements which played an integral syntactic role were replaced with
%% the special artificial tokens \%User, \%Link etc.  Emoticons like :-),
%% \smiley{}, \frownie{} etc. were also replaced with the placeholders
%% \%PosSmiley, \%NegSmiley, or simply \%Smiley depending on their prior
%% polarity.  Furthemore, out-of-vocabulary words which could be
%% converted to in-vocabulary terms with a pre-defined set of
%% transformations were also normalized.

We then labeled the preprocessed data with their part-of-speech tags
using \texttt{TreeTagger}\footnote{We used \texttt{TreeTagger} Version
  3.2 with the German parameter file UTF-8.}  \cite{Schmid:95} and
parsed obtained sentences with the \texttt{Mate} dependency
parser\footnote{We used \texttt{Mate} Version \texttt{3.61} with the
  German parameter model 3.6.}  \cite{Bohnet:13}.  These automatically
derived trees were then used as the source of syntactic features for
all our models and the underlying graph structures for tree-CRFs.

Since \texttt{MMAX2} did not provide a straightforward support for
storing character offsets of the tokens and the automatically
tokenized preprocessed data did not necessarily agree with the corpus
tokenization, we next applied the Needleman-Wunsch algorithm
\cite{Needleman:70} to align the manual annotation with the
automatically segmented data.

\begin{table*}
  \begin{center}
    \bgroup \setlength\tabcolsep{0.47\tabcolsep}\scriptsize
    \small
    \begin{tabular}{|p{0.12\columnwidth}| % first columm
        *{9}{>{\centering\arraybackslash}p{0.08\columnwidth}|}} % next nine columns
      \hline
          {\bfseries Element} & 1 lcCRF & 2 lcCRF & 3 lcCRF & 4 lcCRF & 1 smCRF & 2 smCRF & 3 smCRF & 4 smCRF & 1 trCRF\\\hline

          \multicolumn{10}{|c|}{\cellcolor{cellcolor}Training Set}\\\hline

          Sentiment & 85.91 & 83.82 & 84.33 & 85.45 & 82.92 & 88.13 & 83.15
          & 80.67 & 87.07\\
          Source & 78.14 & 76.74 & 77.84 & 80.59 & 73.15 & 79.88 & 73.95 &
          71.55 & 80.47\\
          Target & 79.31 & 70.52 & 71.57 & 73.47 & 67.95 & 76.18 & 70.04 &
          67.51 & 82.14\\\hline
          %% Sentiment & 87.77 & 85.25 & 86.18 & 81.91 & 86.11 & 90.06 & 86.91 & 81.96 & 80.53\\
          %% Source & 77.58 & 73.77 & 77.94 & 74.09 & 75.83 & 80.83 & 68.3 & 70.85 & 72.65\\
          %% Target & 79.07 & 78.57 & 79.83 & 75.19 & 70.95 & 77.04 & 72.3 & 75.56 & 72.45\\\hline

          \multicolumn{10}{|c|}{\cellcolor{cellcolor}Test Set}\\\hline

          Sentiment & 32.39 & 31.32 & 33.07 & 32.99 & 30.22 & \textbf{33.09}* & 30.69
          & 30.41 & 31.66\\
          Source & 25.31 & \textbf{29.64}** & 28.05 & 27.43 & 23.49 &
          26.38 & 24.24 & 23.85 & 25.15\\
          Target & \textbf{29.54} & 25.38 & 25.59 & 20.41 & 25.6 &
          26.26 & 26.49 & 26.16 & 25.52\\\hline
          %% Sentiment & 32.91 & 31.85 & 31.71 & 32.7 & 31.16 & 31.52 & 32.01 & \textbf{34} & 32.66\\
          %% Source & 26.67 & 27.09 & \textbf{27.4} & 23.16 & 26.02 & 26.17 & 21.15 & 23.77 & 27.16\\
          %% Target & 24.73 & 23 & 22.87 & \textbf{25.22} & 21.56 & 21.56 & 21.18 & 24.94 & 22.72\\\hline
    \end{tabular}
    \egroup
    \caption{Automatic sentiment analysis (broad sentiment
      interpretation).\\ {\small lcCRF -- linear-chain CRFs, smCRF --
        semi-Markov CRFs, trCRF -- tree-structured CRFs; 1, 2, and 3
        denote the order} \\ {\small token-level classification
        accuracy differs statistically significantly from the 1 lcCRF
        model at $p < 0.01$ (*) or $p < 0.05$ (**)}}
    \label{tbl:res-broad}
  \end{center}
\end{table*}

\begin{table*}
  \begin{center}
    \bgroup \setlength\tabcolsep{0.47\tabcolsep}\scriptsize
    \small
    \begin{tabular}{|p{0.12\columnwidth}| % first columm
        *{9}{>{\centering\arraybackslash}p{0.08\columnwidth}|}} % next nine columns
      \hline {\bfseries Element} & 1 lcCRF & 2 lcCRF & 3
      lcCRF & 4 lcCRF & 1 smCRF & 2 smCRF & 3 smCRF & 4 smCRF & 1
      trCRF\\\hline

          \multicolumn{10}{|c|}{\cellcolor{cellcolor}Training Set}\\\hline

          Sentiment & 87.36 & 83.43 & 83.63 & 81.17 & 82.85 & 81.59 &
          80.03 & 80.37 & 84.68\\
          Source & 67.12 & 69.38 & 71.34 & 66.58 & 61.46 & 65.06 &
          63 & 65.67 & 61.52\\
          Target & 72.84 & 62.72 & 64.77 & 60.52 & 60.02 & 61.47 &
          59.14 & 60.17 & 66.41\\\hline

          %% Sentiment & 87.01 & 83.42 & 83.4 & 81.53 & 82.36 & 81.5 & 81.8 & 80.38 & 84.32\\
          %% Source & 66.63 & 68.41 & 70.79 & 68.67 & 60.88 & 66.09 & 67.8 & 66.82 & 61.56\\
          %% Target & 73.65 & 64.85 & 66.67 & 64.76 & 61.61 & 63.61 & 65.16 & 62.23 & 67.18\\\hline

          \multicolumn{10}{|c|}{\cellcolor{cellcolor}Test Set}\\\hline

          Sentiment & 61.13 & 59.9 & 59.36 & 60.08 & 60.11 & 60.03 &
          59.31 & 60.05 & \textbf{62.92}\\
          Source & 20.18 & 22.53 & 24.51 & 23.26 & 18.13 & 23.55 &
          \textbf{27.04} & 26.06 & 22.05\\
          Target & 22.94 & 21.89 & 23.59 & 22.94 & 19.43 & 23.01 &
          22.8 & \textbf{23.36} & 21.87\\\hline

          %% Sentiment & 60.26 & 59.34 & 59.22 & 60.3 & 59.58 & 59.86 & 58.43 & 59.22 & \textbf{62.86}\\
          %% Source & 19.3 & 21.13 & 23.38 & 25.75 & 19.2 & 24.49 & 24.18 & \textbf{26.77} & 21.83\\
          %% Target & 20.43 & 20.5 & 20.96 & \textbf{22.76} & 17.86 & 21.01 & 21.78 & 20.83 & 19.99\\\hline
    \end{tabular}
    \egroup
    \caption{Automatic sentiment analysis (narrow sentiment
      interpretation).}
    \label{tbl:res-narrow}
  \end{center}
\end{table*}

\subsubsection{Features}
As our state features in the classification experiments, we used the
following types of attributes:
\begin{itemize}
\item\emph{Formal}, which included the initial three characters of
  each token, the last three characters, and the spelling class of the
  token (e.g. alphanumeric, digit, or punctuation etc.);

\item\emph{Morphological}, which encompassed the PoS tags of the
  tokens, case and gender of the inflectable PoS types, the degree of
  comparison for adjectives, and mood, tense, and person form of
  verbs;

\item\emph{Lexical}, which comprised the form of the
  token.\footnote{We also considered using neural word embeddings
    instead of unigrams but this unexpectedly lead to only marginal or
    no improvement.}  For the German modal verbs \emph{wollen}
  (\emph{want to}) and \emph{m\"ogen} (\emph{like}), we additionally
  differentiated whether they were governing another verb or had a
  noun phrase as their complement.  Finally, we also used the polarity
  class of the token which we obtained from the Zurich polarity
  lexicon \cite{Clematide:10};

\item \emph{Syntactic}, which included the dependency relation via
  which the given token was connected to its parent, the lemma of the
  parent node as a unigram feature, the PoS tag and the polarity class
  of the grandparent, and the overall polarity class of all immediate
  children.  This polarity class was computed by summing up the
  polarity scores of all child nodes that were found in the sentiment
  lexicon of \citet{Clematide:10} and checking whether this final sum
  was negative or positive.
\end{itemize}

In addition to that, we also used a set of complex lexical-syntactic
features which combined several semantic and syntactic traits into
single feature attributes.  These complex features included the
dependency relation of the previous token in the sentence + the
dependency relation of the current token, the dependency relation of
the next word in the sentence + the dependency relation of the current
token, the lemma of the child node + the dependency relation of this
child, the pos-tag of the child node + its dependency relation + the
pos-tag of the current token.

%% the lemma of the child node + the dependency relation of the
%% child to the current token + the lemma of the current token,

%% dependency relation of the preceding word to its parent + the
%% dependency relation of the current token, dependency relation of the
%% next word to its parent + the dependency relation of the current
%% token, lemma of the child node + the dependency relation of the child
%% to the current token, lemma of the child node + the dependency
%% relation of the child to the current token + lemma of the current
%% token, the pos-tag of the child node + its dependency relation + the
%% pos-tag of the current token.

We checked the utility of these features by conducting an ablation
test using the first-order linear-chain and the tree-structured models
with the broad interpretation scheme.
%%  All of these attributes proved to increase the classification
%% accuracy with the lexical and syntactic features being the most useful
%% ones.
The relative changes of the classification results on the test set
that were caused by the removal of each of these feature groups are
shown in Table \ref{tbl:ablation}.\footnote{Since we did not observe
  the test data during the feature development stage, some of the
  features that had lead to classification improvements on the
  training corpus showed a negative influence on the test data.}

\begin{table}[hb]
  \begin{center}
    \bgroup \setlength\tabcolsep{0.47\tabcolsep} \small
    \begin{tabular}{|p{0.2\columnwidth}| % first columm
        *{5}{>{\centering\arraybackslash}p{0.12\columnwidth}|}} % next five columns
      \hline
          \multirow{2}{0.2\columnwidth}{\bfseries Element} &
          \multicolumn{5}{c|}{\bfseries Removed feature type}\\\cline{2-6}
          & Form & Morph & Lex & Synt & Cmplx\\\hline

          \multicolumn{6}{|c|}{\cellcolor{cellcolor}1-st order linear-chain CRFs}\\\hline

          Sentiment & -2.12 & -1.41 & -2.7 & -1.32 & -5.25\\
          Target & -2.62 & -0.7 & -5.24 & -0.47 & -5.49\\
          Source & -1.94 & -1.91 & -2.49 & -2.92 & -7.83\\\hline

          \multicolumn{6}{|c|}{\cellcolor{cellcolor}Tree-structured CRFs}\\\hline

          Sentiment & -0.05 & -0.09 & +0.02 & -0.65 & -5.59\\
          Target & -1.4 & +1.11 & +2.71 & +3.87 & -4.28\\
          Source & -0.63 & +0.12 & +0.15 & +0.65 & -5.72\\\hline
    \end{tabular}
    \egroup
    \caption{Feature ablation tests.}
    \label{tbl:ablation}
  \end{center}
\end{table}

\subsubsection{Experiment I (broad sentiment)}
In our first set of experiments, we applied the \emph{broad}
interpretation of sentiments.  With this interpretation, we assigned
the \textsc{SNT} labels to all tokens that belonged to the sentiment
spans in our corpus annotation except for the tokens that also were
labeled as targets or sources of the opinions.  In the latter case, we
used the \textsc{TRG} and \textsc{SRC} tags respectively.  A
translated example of such annotation mapping is provided below:
\begin{example}\label{exmp:3}
  \noindent\sentiment{\target{Francis} makes a \intensifier{very}
    \emoexpression{good} impression on \source{me}!
    \emoexpression{:)}}

  $\rightarrow$

  Francis/TRG makes/SNT a/SNT very/SNT good/SNT impression/SNT on/SNT
  me/SRC !/SNT :)/SNT
\end{example}
After mapping and aligning the annotation from the corpus with the
automatically pre-processed data, we trained various modifications of
CRFs on the training set using the l-BFGS algorithm \cite{Liu:89} and
adjusted the $L1$ and $L2$ regularization parameters using the
cross-validation data.  Our final classification results for the test
set are shown in Table \ref{tbl:res-broad}.  For all reported
evaluations, we used the proportional overlap metric of
\citet{Johansson:10}.

\subsubsection{Experiment II (narrow sentiment)}
In the next attempt, we had a different take of the annotation scheme
and only assigned the label \textsc{SNT} to the tokens that were
labeled as \textsc{Emo-Expression}s in our corpus.  A translated
example of this annotation mapping is provided below:
\begin{example}\label{exmp:4}
  \noindent\sentiment{\target{Francis} makes a \intensifier{very}
    \emoexpression{good} impression on \source{me}!
    \emoexpression{:)}}

  $\rightarrow$

  Francis/TRG makes/NON a/NON very/NON good/SNT impression/NON on/NON
  me/SRC !/NON :)/SNT
\end{example}
\noindent{} The results obtained for this type of interpretation are
presented in Table \ref{tbl:res-narrow}.

%% \section{Discussion}\label{sec:discussion}
\subsubsection{Results}
As can be seen from the tables, classification performance for the
narrow sentiment spans is almost twice as high as the results obtained
with the broad interpretation (62.92\% versus 33.09\%).  The broad
approach, however, leads to better prediction scores for targets and
sources of the opinions.

For the narrow scheme, classification results for these opinion
arguments can be improved by increasing the dependency order of the
linearly structured CRF-models.  This improvement, however, comes at
the cost of a decreased accuracy for the sentiment spans themselves.

We can also see that, depending on the utilized interpretation scheme,
sentiment classification notably depends on the graph structure of the
underlying CRF model: broad sentiments, for example, are best
processed with the second-order semi-Markov CRFs, while narrow
opinions are more amenable to the tree-structured model.

\subsubsection{Discussion}\label{sec:discussion}
We also note that all our obtained results are significantly lower
than the corresponding figures achieved for other domains and other
languages.  A closer look into the errors showed us that the main
reason for such severe accuracy drop was an insufficiently good
recognition of opinionated words and phrases.  This can be partially
explained by the creativity of the users in expressing their thoughts
but also by the lack of appropriate sentiment dictionaries which would
not only contain standard language expressions but also slang words
and cusses.

An additional challenge was posed by emoticons which could not only
convey an evaluative meaning but also serve as politeness expressions
and were therefore ambiguous sentiment markers.  Furthermore, we
should also mention that the German language itself is more difficult
for automatic processing due to its relatively free word order,
ambiguous inflections, and a rich morphology.  And these problems
become even more aggravated when dealing with Twitter.

We also saw that the outcomes of our experiments crucially depended on
the definition of the sentiments that we applied and that different
CRF variants had different influence on the classification accuracy
for either interpretation.  Unfortunately, none of these models could
clearly outperform its rivals, but we could detect some general trends
for each of the proposed interpretations.  If these trends would also
hold whe using other feature sets and other corpora is one of the
questions that we want to address in future research.

\subsection{Fine-grained Sentiment Analysis Using Deep Neural Networks}

\subsection{Related Work}

\cite{Kim:06}
\subsubsection{Sentiment Analysis with CRFs}
%% Admittedly, applications of probabilistic graphical models (such as
%% hidden Markov models or CRFs) have a relatively rich history of
%% applications in the sentiment analysis research.

Starting with the work of \citet{Choi:05} who first used a
linear-chain CRF classifier in conjunction with a weakly supervised
pattern learner \texttt{AutoSlog-TS} \cite{Riloff:96} to automatically
detect sources of the opinions, increasingly more researchers turned
towards using structured prediction methods for the detection of
sentiment elements.

\citet{Breck:07}, for example, applied CRFs to automatically predict
opinion expressions in newspaper articles.  The authors trained and
tested their system on the MPQA corpus of news documents
\cite{Wiebe:05} in which they attempted to identify \emph{expressive
  subjective elements} (ESE) and \emph{direct speech events} (DSE) as
defined by the MPQA annotation scheme.  The $F$-score reported for
this system run up to 63.43\% for ESEs and 70.6\% for DSEs measured
with the binary overlap metric.

%% They considerably outperformed the baseline systems of
%% \citet{Wiebe:05b} and \citet{Wilson:05} reaching an $F$-score of
%% 70.6\% for DSEs and 63.43\% for ESEs measured with the binary overlap
%% metric.

%% Next, \citet{Jin:09} applied a first-order lexicalized HMM classifier
%% in order to predict product features and customer opinions in Amazon
%% reviews.  Since HMM is a generative model, however, its performace is
%% usually worse than the results of discriminative classifiers (such as
%% CRF) due to the inability to cope with highly correlated features
%% (i.e. features which often co-occur together).

%% \citet{Jakob:10} used CRFs in order to extract sentiment targets from
%% IMDB movie reviews, technical blog posts, and user comments posted
%% on \url{epinions.com} in both single- and cross-domain settings.  In
%% their work, the authors took opinion expressions from the manual
%% annotation as given features and significantly outperformed the
%% baseline results of \citet{Zhuang:06}.  The performance of their
%% system, however, dropped dramatically to 19--30\% $F$-score when the
%% information about subjective expressions was removed.

\citet{Li:10} compared the results obtained by the linear-chain and
tree-structured CRFs with the classification accuracy of the
respective skip-modifications of these models on a corpus of movie and
product reviews.  The best of their systems (skip-tree CRF) achieved
an average $F$-score of $\approx$ 78\% for recognizing targets,
positive and negative opinions.  A notable fact, however, was that
even a maximum entropy classifier could score impressive 67.8\% on
this dataset.
%% Tree-CRFs based on the dependency graphs were also used by
%% \citet{Nakagawa:10} to predict the overall sentiment polarity of
%% complete sentences.

\citet{Yang:12} were among the first who applied the semi-Markov CRF
model to the sentiment analysis task.  The authors used this
modification for the identification of opinion expressions. Their
system substantially improved on the results obtained by
\citet{Breck:07} reaching an $F$-score between 65.7 and 72.3\% on the
MPQA data.

Later on, \citet{Yang:13} also proposed a joint inference technique
for simultaneous prediction of opinion expressions and their arguments
(targets and sources).  The authors first obtained a set of possible
opinion expression candidates from the $n$-best sequences returned by
the linear-chain CRF classifier and then applied a log-linear model to
infer the potential arguments of these expressions.  The outputs of
these two predictors were linearly combined using an ILP technique.
The resulting $F$-score of this system amounted to 74.35\% for the
recognition of opinion expressions and to $\approx$ 65\% for the
recognition of targets and sources.  The experiments were again
conducted on the MPQA corpus.

%% Further works on CRFs for sentiment analysis include \citet{Choi:10}
%% who combined predictions over opinion expressions, their intensity,
%% and polarity into a single CRF lattice;
To the best of our knowledge, no attempts have been made so far to
increase the dependency order of the linear-chain and semi-Markov CRF
models for the sentiment analysis task.

%% Therefore, we want to address these questions in our work and,
%% similarly to \citet{Li:10}, also juxtapose these ``horizontal''
%% extensions to the ``vertical'' modification of CRFs whose structure
%% is based on the dependency trees of the sentences.  In contrast to
%% \citet{Li:10}, however, our system relies on an exact inference
%% approach (belief propagation) instead of using an approximate
%% algorithm (tree re-parameterization).

%% Furthermore, in contrast to \citet{Li:10} who used an approximate
%% inference technique for their Tree-CRF model, we confine ourselves to
%% the exact inference algorithms which we later describe in Section
%% \ref{sec:models}.

\subsubsection{Sentiment Analysis on Twitter}

With the wide spread of social media services in recent years,
sentiment researchers also have gradually started changing the focus
of their work from the analysis of official documents and newspaper
articles to the classification of user-generated content abounding on
the Web.

One of the first attempts to analyze sentiments on Twitter was made by
\citet{Go:09}.  For their experiments, the authors collected a set of
1,600,000 tweets containing smileys.  Based on these emoticons, they
automatically derived polarity classes for these messages (positive or
negative) and used them to train a Na\"{\i}ve Bayes, a MaxEnt, and an
SVM classifier.  The best $F$-score for this two-class classification
problem could be achieved by the last system and run up to 82.2\%.

Similar work was done later by \citet{Pak:10} who used a Na\"{\i}ve
Bayes approach to differentiate between neutral, positive, and
negative microblogs. \citet{Barbosa:10} also gathered a collection of
200,000 tweets from three publicly available sentiment web-services
and then trained an SVM classifier to predict the subjectivity and the
polarity class of new unseen messages.

Works attempting a more fine-grained sentiment analysis on Twitter
usually try to derive a common polarity class for each message with
respect to a particular target that is mentioned in that microblog.

\citet{Jiang:11}, for instance, tried to classify the polarity of
microblogs pertaining to a predefined set of specific topics, like
\emph{Obama}, \emph{Google}, \emph{iPad} etc.  To this end, the
authors manually labeled a corpus of 1,939 messages and trained a
binary SVM model in order to predict the subjectivity and the polarity
of the tweets with respect to the given subjects.

%% This classifier could achieve an accuracy of 68.2\% for the
%% subjectivity classification and 85.6\% for the polarity prediction.
%% The $F$-score of this system for the latter task could further be
%% improved from 66\% to 68.3\% by incorporating the information about
%% the predicted polarity class of the re-tweets, replies, and other
%% microblogs posted by the same author.

\citet{Mitchell:13} broadened the set of possible targets by allowing
any named entities found in microblogs to be associated with a
specific polarity.  For that purpose, the authors combined a CRF-based
NER system with a sentiment predicting CRF by considering three
different possibilities of such combination: a pipeline approach, a
joint multi-layer model, and a single classifier with a combined
tagset.  The best scores on their corpora of 7,105 Spanish and 2,350
English tweets could be achieved with the joint and pipeline
approaches.  The accuracy of recognizing the opinionated named
entities amounted to 31\% for Spanish and 30.4\% for English.

%% Other notable works in this direction include \citet{Chunping:14} who
%% first applied a Na\"{\i}ve Bayes classifier to predict the
%% subjectivity class of microblogs and then sequentially used two CRF
%% models to predict the particular type of subjectivity (such as anger,
%% fear, happiness etc.) for message sentences.

%% Other notable works in this direction include \citet{Dong:14} who used
%% a recurrent neural network to predict the polarity class associated
%% with the opinion targets.  They, however, assumed the targets of
%% sentiments to be apriori known and only were interested whether a
%% positive or a negative judgement was made about them.

%% Except for the work of \citet{Mitchell:13} and to some extent
%% \citet{Jiang:11}, neither of the approaches attempted to classify
%% sentiments below the sentence level.  And even in these two works, the
%% set of possible targets was either restricted to the named entities or
%% to some pre-defined list of words.  Therefore, the task that we are
%% addressing here (simultaneous classification of sentiments, source,
%% and targets) is by far more challenging, more difficult, and,
%% unfortunately, probably less successful than the work previously done.

Again, to the best of our knowledge, no work on simultaneous
prediction of sentiment targets, and sources, as well as opinion
segments has been reported for Twitter so far.

%% and we sincerely hope that our data can at least ease the efforts of
%% other researches on this field.
\subsection{Conclusions}
\newpage
