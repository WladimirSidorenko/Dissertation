\documentclass[11pt]{article}

%%%%%%%%%%%%%%%%%%%%%%%%%%%%%%%%%%%%%%%%%%%%%%%%%%%%%%%%%%%%%%%%%%
%% Packages
\usepackage{amsmath}
\usepackage{breakcites}
\usepackage{enumitem}
\usepackage[a4paper,includeheadfoot,margin=2.54cm]{geometry}
\usepackage[authoryear]{natbib}
\usepackage{titling}

%%%%%%%%%%%%%%%%%%%%%%%%%%%%%%%%%%%%%%%%%%%%%%%%%%%%%%%%%%%%%%%%%%
%% Commands
\newcommand{\ienocomma}{\textit{i.e.}}
\newcommand{\ie}{\ienocomma,}
\newcommand{\eg}{\textit{e.g.},}
\newcommand{\F}[0]{$F_1$}
\newcommand{\markable}[1]{\texttt{#1}}
\newcommand{\attribute}[1]{\emph{\texttt{#1}}}
\newcommand{\argmin}{\operatornamewithlimits{argmin}}
\renewcommand{\cite}{\citep}

%%%%%%%%%%%%%%%%%%%%%%%%%%%%%%%%%%%%%%%%%%%%%%%%%%%%%%%%%%%%%%%%%%
%% Lengths
\setlength{\droptitle}{-11em}

%%%%%%%%%%%%%%%%%%%%%%%%%%%%%%%%%%%%%%%%%%%%%%%%%%%%%%%%%%%%%%%%%%
%% Variables
\author{\normalsize{}by Uladzimir Sidarenka}
\title{\normalsize{}Main Claims of the Ph.D.~Dissertation\\[0.4em]
  {\Large{}``Sentiment Analysis of German Twitter''}\\[0.4em]
  to be defended on July 12, 2019\vspace{-3ex}}
\date{\vspace{-3ex}}

%%%%%%%%%%%%%%%%%%%%%%%%%%%%%%%%%%%%%%%%%%%%%%%%%%%%%%%%%%%%%%%%%%
%% Document
\begin{document}
\maketitle

In my dissertation, I address the problem of automatic analysis of
sentiments (\ie{} subjective evaluative opinions) in German tweets.
In particular, I investigate the ways and means by how people express
their opinions in online microblogs, examine current approaches to
automatic mining of these feelings, and propose new methods that
outperform many existing techniques.  I perform these tasks on three
different linguistic levels: \emph{subsentential} (the level of words
and phrases), \emph{sentential} (the level of single statements), and
\emph{suprasentential} (the level of discourse); making thereby the
following claims:

\begin{enumerate}
  \item I argue that \textbf{analysis of subjective evaluative
    opinions is an inherently difficult task even for human experts}.

    This argument is empirically confirmed by an extensive
    inter-annotator agreement (IAA) study that I performed during the
    creation of the Potsdam Twitter Sentiment Corpus~(PotTS), which I
    present in Chapter~2;

  \item\textbf{The difficulty of this analysis crucially depends on
    the topic and form of the tweets}, \eg{} political messages
    containing emoticons are more challenging to interpret than casual
    everyday microblogs without smileys.

    This finding is again bolstered by the IAA results that I obtained
    for tweets pertaining to different topics and formal categories;

  \item\textbf{Most annotation ambiguities can be resolved by making
    the experts aware of alternative interpretations by other
    linguists}.

    Since the IAA scores at the initial stage of working on the PotTS
    corpus were relatively low (31.21\% $\kappa$), I applied a simple
    adjudication step in which I highlighted the differences between
    the annotations of two linguists and let them resolve these
    disagreements independently of each other.  This procedure has not
    only led to a double increase of the $\kappa$-score but also
    allowed the annotators to maintain a high level of mutual
    agreement when analyzing the rest of the data.
\end{enumerate}
Turning to the evaluation of particular sentiment-analysis (SA)
resources and methods, I claim that
\begin{enumerate}[resume*]
\item\textbf{Existing German translations of English sentiment
  lexicons have a better quality than lexicons that are automatically
  inferred from German data};

\item\textbf{Among automatic lexicon-generation methods,
  dictionary-based approaches yield higher scores than corpus-- and
  word-embedding--based systems.  Nevertheless, the last group
  represents a good alternative to dictionary-based solutions in the
  cases when a manually annotated lexical ontology is absent};

\item In Chapter~4, I turn my attention to \emph{aspect-based
  sentiment analysis}, whose goal is to predict text spans of
  sentiments, their targets, and holders.  Based on the conducted
  experiments, I conclude that \textbf{aspect-based sentiment analysis
    is better addressed with probabilistic graphical models, such as
    conditional random fields (CRFs), rather than recurrent neural
    networks}.  Moreover, \textbf{the results of the standard
    first-order linear-chain CRFs can be further improved by
    increasing the order and redefining the structure of the CRF
    graph};

\item Afterwards, in Chapter~5, I compare different approaches to
  \emph{message-level sentiment analysis}, where the task is to
  classify the overall polarity of a tweet as positive, negative, or
  neutral.  For this purpose, I compare three major groups of methods:
  lexicon-, machine-learning--, and deep-learning--based ones, coming
  to the conclusion that \textbf{message-level sentiment analysis is
    more amenable to machine- and deep-learning--based paradigms} and
  that \textbf{DL-based methods can significantly benefit from the
    incorporation of sentiment lexicons};

\item Finally, in the last chapter, I prove that \textbf{one can
  improve message-level SA results by taking into account the
  discourse structure of the tweet}, \ie{} not analyzing the whole
  microblog at once, but predicting the sentiment scores for each of
  its single sentences and then inferring the overall polarity of the
  message from these scores by propagating them up the discourse tree
  instead.
\end{enumerate}

Based on these evaluations, I conclude that:
\begin{enumerate}[resume*]
  \item To a large extent, \textbf{we can apply opinion mining methods
    devised for standard English to German Twitter}, provided that the
    training objective, the evaluation metrics, as well as the size
    and reliability of the training data used in the original
    solutions are close to the respective parameters in the German SA
    task at hand;
  \item Moreover, \textbf{one can further increase the scores of
    existing approaches by overcoming the difficulties of the
    social-media genre with the help of text normalization}, which, as
    I show in Chapters~4 and~5, not only increases the scores of
    aspect-based sentiment analysis but also significantly raises the
    accuracy of message-level classification.
\end{enumerate}
\end{document}
