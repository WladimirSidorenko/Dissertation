% FILE: sentiment.tex  Version 0.01
% AUTHOR: Uladzimir Sidarenka

% This is a modified version of the file main.tex developed by the
% University Duisburg-Essen, Duisburg, AG Prof. Dr. G�nter T�rner
% Verena Gondek, Andy Braune, Henning Kerstan Fachbereich Mathematik
% Lotharstr. 65., 47057 Duisburg entstanden im Rahmen des
% DFG-Projektes DissOnlineTutor in Zusammenarbeit mit der
% Humboldt-Universitaet zu Berlin AG Elektronisches Publizieren Joanna
% Rycko und der DNB - Deutsche Nationalbibliothek

\chapter{Sentiment Analysis}

% FILE: sentiment.tex  Version 0.01
% AUTHOR: Uladzimir Sidarenka

% This is a modified version of the file main.tex developed by the
% University Duisburg-Essen, Duisburg, AG Prof. Dr. Günter Törner
% Verena Gondek, Andy Braune, Henning Kerstan Fachbereich Mathematik
% Lotharstr. 65., 47057 Duisburg entstanden im Rahmen des
% DFG-Projektes DissOnlineTutor in Zusammenarbeit mit der
% Humboldt-Universitaet zu Berlin AG Elektronisches Publizieren Joanna
% Rycko und der DNB - Deutsche Nationalbibliothek

Interpersonal communication is not only a way to share objective
information with other people but also a vibrant channel to convey
one's subjective thoughts, attitudes, and feelings.  It is, in fact,
this latter use which provides a personal touch to our discourses,
making them more grasping, more entertaining, and more living.  And it
is often this use which significantly influences our decisions,
preferences, and choices in everyday life.  Therefore, a high-quality
automatic analysis of the subjective part of information is typically
not less important than the extraction and analysis of objective
facts.

\section{Introduction to Sentiment Analysis}

The field of knowledge which deals with the analysis of people's
opinions, sentiments, evaluations, appraisals, attitudes, and emotions
towards particular entities mentioned in discourse is called
\emph{sentiment analysis} (SA) \citep{Liu:12}.  The definition of this
discipline, however, much like the definition of the term
\emph{sentiment} itself, is neither complete nor universally accepted.
The main reasons for this are
\begin{inparaenum}[\itshape i)\upshape]
  \item a frequently unclear delimitation of the subjective and
    objective components of information and
  \item the heteroginity of the language system to which SA methods
    are applied.
\end{inparaenum}

The former factor, for instance, makes it difficult to delimitate what
kinds of events and expressions should actually belong to the domain
of sentiment analysis, and which ones shall rather be excluded from
it.  A prominent example of such problematic borderline cases are the
so-called subjective facts, such as \emph{terrorist attacks} or
\emph{cancer drugs}, which some people perceive as emotionally laden
terms while others regard them as purely objective statements.

The latter factor complicates a precise definition of SA because
different levels of the language have their own concepts of
subjectivity, which in turn require different approaches and methods
to use.  Depending on the language level being analyzed, researchers
typically distinguish three different subtypes of sentiment analysis:
\begin{itemize}
  \item\emph{subsentential} or \emph{fine-grained} SA whose task is to
    determine and analyze specific subjective opinions and/or
    evaluations within single clauses,
  \item\emph{sentential} analysis which tries to ascribe a single
    polarity class to each sentence in a text, and
  \item\emph{suprasentential} or \emph{document-level} sentiment
    classification which seeks to determine the polarity and
    subjectivity classes of complete discourses.
\end{itemize}

Each of these subtypes has its own characteristic strengths and
weaknesses.  The fine-grained SA, for example, is typically considered
to be the ultimate goal of any opinion mining\footnote{Following
  \citet{Liu:12}, we do not make a distinction between the terms
  \emph{opinion mining} and \emph{sentiment analysis} and use both
  expressions interchangeably in this thesis.}  system as it aims at
the highest possible recall of all subjective expressions occurring in
a text.  At the same time, this task is unfortunately very challenging
even for human beings, as we will show in Section \ref{sec:corpus},
let alone computer programs.  The consequently low results often
intimidate researchers and prevent subsentential SA systems from being
used in industrial applications.

The last two disciplines (sentential and document-level analyses) can
be considered as approximations of the fine-grained SA.  They coarsen
the targeted recall down to the level of sentences or texts
respectively, trying to determine only one (the most prominent)
expression of subjectivity per analyzed unit.  In contrast to the
subsentential SA, these approaches typically yield better results at
the cost of sacrificing important details.

Due to these crucial disparities between different variants of
sentiment analysis, speaking of the difficulty or easiness of this
task for a specific domain in general is in the same way wrong as
making overall judgments about the amenability of this domain to the
whole natural language processing field: one always has to specify the
precise level of sentiment analysis whose difficulty is being analyzed
just as she has to set apart a particular NLP task in order to assess
its complicatedness.

In this chapter, we will primarily concentrate on the most challenging
SA objective -- that of the fine-grained SA.  After a brief summary of
related work done on the opinion mining in general and sentiment
analysis of social media in particular, we will introduce a
comprehensive corpus of German tweets that has been created for the
purpose of this work.  A detailed inter-annotator agreement study of
this dataset will reveal which linguistic and extra-linguistic factors
significantly influence the distribution of sentiments in Twitter, and
which of them cause utter confusion among human experts.  After
obtaining an upper bound on human performance, we will successively
compare with it the results attained by automatic sentiment systems,
starting with the most fundamental task of recognizing subjective
expressions and concluding with the ultimate goal of recognizing
textual spans of sentiments, their objects of evaluation
(\emph{targets}), and authors (\emph{sources}).

\subsection{Prehistory of the Field}

However, before we delve into the odds and ends of contemporary
sentiment analysis, we first would like to make a short digression
into the history of this field in order to understand its modern
trends and theories better.  Like many other subdisciplines of
computational linguistics, opinion mining has emerged from several
other areas of research, including philosophy, psychology, cognitive
sciences, and last but least narratology and linguistics.

In \emph{philosophy}, the questions about the nature of emotions,
their interaction with the human consciousness, and their influence on
people's deeds have occupied the minds of many great scholars,
starting from Plato and Aristotle.  Plato, for instance, argued in his
treatise ``The Republic'' \citep[Book~IV]{Plato:91} that the human
soul presumably consists of three fundamental parts: the rational, the
appetitive, and the passionate.  The last part -- the one by which we
become angry or get into a temper -- determines our notion of justice
by favoring either the rational or the appetitive aspect.  These ideas
were further particularized and partly revised by Plato's most
prominent student -- Aristotle.  In Book~II of ``The Rhetoric''
\citep{Aristotle:54}, Aristotle not only provides a precise taxonomy
and definitions of possible emotions that might constitute the
passionate part of the mind, but also contends that these are both the
reasons for the change of human judgements and the consequences of
those changes \cite[cf.][p. 157]{Leighton:82}.

As noted by \citet{Sousa:14}, the sheer variety of phenomena covered
by the term ``emotion'' and its closest neighbors tended to discourage
tidy philosophical theories and daunted researchers for a long period
of time.  A real renaissance of emotional studies happened, however,
in the late 19-th century in \emph{psychology} with the introduction
of the James-Lange Theory \citep{James:1884}.  In his groundbreaking
work, \citeauthor{James:1884} proclaimed the naturalistic idea that
bodily changes followed directly the perception of the exciting facts,
and that our feelings of the same changes were, in fact, the emotions.
In other words, our biological processes were the primary and the only
reason for us to perceive sentiments:
\begin{quote}
``If we fancy some strong emotion, and then try to abstract from our
  consciousness of it all the feelings of its characteristic bodily
  symptoms, we find we have nothing left behind, no `mind-stuff' out
  of which the emotion can be constituted, and that a cold and neutral
  state of intellectual perception is all that
  remains.''\citep[p. 193]{James:1884}
\end{quote}
Independently of James' work, similar views were proposed by the
Danish physician Carl Lange \citep{Lange:1885} who hypothesized that
emotional phenomena were the consequences of cardiovascular events
\citep[cf.][]{Lang:94}.

In the 1920s, James-Lange's theory was severely criticized by Philip
Bard and Walter B. Cannon \citep{Bard:28,Cannon:31}, who proposed an
alternative interpretation of emotions, refuting their connection to
the viscera and arguing that both bodily reactions and subjective
feelings were generated simultaneously in the talamic region of the
brain.

In psychology:

TODO: Schachter-Singer Theory

That emotions typically have formal objects highlights another
important feature of emotional experience which feeling theories
neglect, and which other psychological theories attempt to
accommodate: emotions involve evaluations. If someone insults me and I
become angry, his impertinence will be the aspect of his behavior that
fits the formal object of anger: I only become angry once I construe
the person's remark as a slight; the specific nature of my emotion's
formal object is a function of my appraisal of the situation. Magna
Arnold introduced the notion of appraisal into psychology,
characterizing it as the process through which the significance of a
situation for an individual is determined. Appraisal gives rise to
attraction or aversion, and emotion is equated with this ``felt
tendency toward anything intuitively appraised as good (beneficial),
or away from anything intuitively appraised as bad (harmful).''
(Arnold 1960, 171). Subsequent appraisal theories accept the broad
features of Arnold's account, and differ mainly in emphasis. Richard
Lazarus (1991) makes the strong claim that appraisals are both
necessary and sufficient for emotion, and sees the identity of
particular emotions as being completely determined by the patterns of
appraisal giving rise to them. Nico Frijda (1986) takes the patterns
of action readiness following appraisals to be what characterize
different emotions, but departs from Arnold in not characterizing
these patterns solely in terms of attraction and aversion. Klaus
Scherer and his Geneva school have elaborated appraisal theories into
sophisticated models that anatomize different emotions in terms of
some eighteen or more dimensions of appraisal. Emotions turn out to be
reliably correlated, if not identified, with patterns of such complex
appraisals. (Scherer et al., 2001). Appraisal theories can be
described as taking a functional approach to emotion, insofar as
appraisals lead to reactions whose function is to deal with specific
situation types having some significance for an individual (Scherer
2006). This approach suggests that the space of emotions can be
conceptualized as multidimensional. In practice, however, so-called
dimensional theories simplify the problem of representation by
reducing these to just two or three (Russell 2003).

In literary studies and linguistics,

TODO: Theory of Narrative Sentences

Some philosophers suggest that the directive power which emotions
exert over perception is partly a function of their essentially
dramatic or narrative structure (Rorty 1988). A particularly subtle
examination of the role of narrative in constituting our emotions over
the long term is to be found in (Goldie 2012).  de Sousa (1987) has
suggested that the stories characteristic of different emotions are
learned by association with ``paradigm scenarios.'' These are drawn
first from our daily life as small children and later reinforced by
the stories, art, and culture to which we are exposed. Later still,
they are supplemented and refined by literature and other art forms
capable of expanding the range of one's imagination of ways to live.

Among the first who applied this theory to the needs and wants of
\emph{computational linguistics} was \citet{Wiebe:90a}.  In her work,
the author proposed an algorithm for identifying characters of
narratives whose point of view was represented by subjective
sentences.  Later on, \citet{Wiebe:94} further enhanced this system
with the possibility to predict subjective sentences in stories
automatically.  This work was allegedly one of the first experiments
on the automatic prediction of sentiments on the level of single
sentences.

The real breakthrough in the sentiment analysis field, however,
happened with the introduction of the first manually annotated
corpora.  Notable contributions in this regard were done by
\cite{Wiebe:03} \cite{Wilson:03}.

Computational Linguistics: Wilensky (1983), Dyer (1983),


\cite{Hatzivassi:97} \cite{Nasukawa:03}, \cite{Yi:03},
\cite{Kanayama:04}

\subsection{Sentiment Analysis of Social Media}

One of the main problems that people working on opinion mining are
typically confronted with in the first place is that of the discourse
domain to be analyzed.  Since sentiment analysis is a highly
domain-dependent task \citep[cf.][]{Aue:05,Blitzer:07,Li:08} -- i.e.,
systems trained for specific topics and on specific text genres do not
neatly generalize to other subjects and other linguistic styles -- a
natural question that arises in this context is which of the domains
should be addressed first in that case.

While earlier sentiment works were primarily concerned with narratives
\citep{Wiebe:90a,Wiebe:94} or newspaper articles
\citep{Wiebe:03,Wiebe:05,Bautin:08}, it soon became clear that social
media provides a much more fertile ground for mining people's
opinions.  The reason for this is the virtual absence of any
moderation on many modern CMC services.  This lack of censorship
allows users to be upfront about their feelings.  Combined with the
great popularity of social networks, such freedom of expressing one's
thoughts makes social media the arguably most prolific channel for
sharing (and mining) personal emotions.

It is therefore not surprising that the rapid emergence of numerous SM
websites was accompanied by an increasing interest in their content
from many NLP practitioners.  One of the first who attempted to
extract users' evaluations automatically from CMC services were
\citet{Das:01}.  The authors investigated the correlation between
users' attitudes on economic chat boards and stock prices for about
eight stocks.  For this purpose, they trained a collection of five
classifiers on a training set of 500 messages that were manually
labeled with three classes: \emph{buy} (positive evaluations),
\emph{sell} (negative evaluations), and \emph{null} (neutral
statements).  With these classifiers, they subsequently annotated the
rest of the downloaded 25,000 posts, taking the majority votes of the
five systems as the final labels, and scrutinized the
interrelationship between these predictions and the observed stock
indices.
%% This study revealed that user chats strongly correlated with stock
%% developments in a reactive way: changes in stock prices had
%% typically lead to fluctuations of public opinions but not vica
%% versa.

\citet{Glance:05} also used sentiment analysis for business
intelligence purposes by mining users' opinions on Usenet newsgroups
with a set of hand-crafted rules. TODO: add more works

A slightly different domain and objective were addressed by
\citet{Turney:02} who performed a two-class classification of 410
Epinions comments, dividing customers' reviews of automobiles, banks,
movies, and travel destinations into \emph{recommended} (thumbs up)
and \emph{not recommended} (thumbs down) ones.  In order to make these
decisions, the proposed system first computed the sum of the pointwise
mutual information (PMI) scores between the adjectival and adverbial
phrases occurring in a review text and the word ``excellent'', and
then subtracted from this the PMIs of the found phrases in conjunction
with the word ``poor''.  Reviews with negative results were
subsequently considered as \emph{thumbs down} and comments with
non-negative total scores were correspondingly classified as
\emph{thumbs up}.

According to \citet{Turney:02}, the most challenging type of reviews
to analyze in these experiments were the movie critiques (the accuracy
on this subset only reached 66\%, whereas, for banks and automobiles,
it attained 80 and 84\% respectively).  This finding was also
confirmed by \citet{Pang:02} who tried out several machine learning
classifiers on a manually annotated corpus of $\approx2,000$ write-ups
from the Internet Movie Database (IMDb).  The best results (82.9\%
accuracy) for the two-class classification task (\emph{positive}
versus \emph{negative}) were attained by an SVM system which used the
presence of unigrams as classification features.

Due to its high commercial impact, opinion mining of customer reviews
soon became one of the most popular topics in the sentiment analysis
research. \citet{Dave:03}, for example, attempted to classify C|Net
and Amazon reviews of movies and electrical appliances as positive or
negative using the Na\"{\i}ve Bayes and SVM approaches.
%% The authors found that replacing certain entities, such as product
%% names, abbreviations, or numerals, with special meta-tokens and
%% using token n-grams of different lengths had a significant positive
%% effect on the accuracy of both classifiers.
TODO: add more works

With the increasing spread of blogging services and social networks,
sentiment researchers have also gradually shifted the focus of their
work to these nascent text genres. \citet{Mishne:05} and
\citet{Mishne:07}, for instance, tried to classify Livejournal blogs
according to the moods of their authors (e.g.  \emph{amused},
\emph{tired}, \emph{happy} etc.).  The gold labels of their
$\approx$~815,000 blog corpus were obtained automatically from blog
labels on the Livejournal website.  On this corpus, the authors
trained a supervised SVM classifier, reaching an average of 8\%
improvement over the 50\% baseline.

\citet{Chesley:06} also applied an SVM system in order to classify
user blogs into objective, positive and negative subjective ones.  The
authors derived their features from the Wiktionary and the InfoXtract
lexicon \citep{Srihari:03} and attained an accuracy increase of up to
15~\% over the majority class baseline.  This research direction was
further continued by \citet{Godbole:07}, who applied a lexicon-based
method to detect users' opinions in newspapers and blogs, and
\citet{Gill:08}, who conducted an inter-annotator agreement study of
mood detection in blogs, finding that raters' consensus on the
expressed emotions strongly increased with the length of blog entries.

Speaking of blog length, it should certainly be said that the
inception of the micro-blogging service Twitter in 2006 was a real
game changer to the opinion mining field: the sudden availability of
huge amounts of data, the presence of all possible social, national,
and age groups as well as a high idiosyncracy of the language used on
this service gave rise to numerous scientific publications, which we
think are worth to be described in a separate subsection.

\subsection{Sentiment Analysis of Twitter}

From its very onset in 2006, Twitter has constantly attracted the
attention of many NLP-practitioners and scientists, with sentiment
researchers being arguably one of the most active of these groups.
This trend seems ever exapanding in recent years: For instance, in
2015, a whole section of the international ACL Workshop on
Computational Approaches to Subjectivity, Sentiment and Social Media
Analysis (WASSA 2015) was solely dedicated to the peculiarities of the
opinion mining on Twitter.  Sentiment detection on Twitter also
remains an active track of the SegEval Shared Task TODO: cite Nakov.
But despite its great popularity, most of this work on this topic
still concentrates on English data only.

One of the first notable works on sentiment analysis of microblogs was
done by \citet{Go:09}.  In their experiments, the authors collected a
set of 1,600,000 Twitter messages containing smileys.  Based on these
emoticons, they automatically derived polarity classes of these tweets
(positive or negative) and then used these bootstrapped labels to
train the Na\"{\i}ve Bayes, MaxEnt, and SVM classifiers.  The best
$F$-score for this two-class classification problem could be achieved
by the last system and run up to 82.2\%.

Similar work was done later by \citet{Pak:10} who used the Na\"{\i}ve
Bayes approach to differentiate between neutral, positive, and
negative tweets. \citet{Barbosa:10} also gathered a collection of
200,000 tweets from three publicly available sentiment web-services.
Using the majority prediction of these three services as their gold
labels, the authors trained an SVM classifier on this corpus and then
let their system predict the subjectivity and polarity classes of new
unseen messages.

Works attempting a more fine-grained sentiment analysis on Twitter
usually try to derive a common polarity class for each message with
respect to a particular target that is mentioned in that microblog.

\citet{Jiang:11}, for instance, tried to classify the polarity of
microblogs pertaining to a predefined set of specific topics, like
\emph{Obama}, \emph{Google}, \emph{iPad} etc.  To this end, the
authors manually labeled a corpus of 1,939 messages and trained a
binary SVM model in order to predict the subjectivity and the polarity
of the tweets with respect to the given subjects.

This classifier could achieve an accuracy of 68.2\% for the
subjectivity classification and 85.6\% for the polarity prediction.
The $F$-score of this system for the latter task could further be
improved from 66\% to 68.3\% by incorporating the information about
the predicted polarity class of the re-tweets, replies, and other
microblogs posted by the same author.

\citet{Mitchell:13} broadened the set of possible targets by allowing
any named entities found in microblogs to be associated with a
specific polarity.  For that purpose, the authors combined a CRF-based
NER system with a sentiment predicting CRF by considering three
different possibilities of such combination: a pipeline approach, a
joint multi-layer model, and a single classifier with a combined
tagset.  The best scores on their corpora of 7,105 Spanish and 2,350
English tweets could be achieved with the joint and pipeline
approaches.  The accuracy of recognizing the opinionated named
entities amounted to 31\% for Spanish and 30.4\% for English.

Other notable works in this direction include \citet{Chunping:14} who
first applied a Na\"{\i}ve Bayes classifier to predict the
subjectivity class of microblogs and then sequentially used two CRF
models to predict the particular type of subjectivity (such as anger,
fear, happiness etc.) for message sentences.

Other notable works in this direction include \citet{Dong:14} who used
a recurrent neural network to predict the polarity class associated
with the opinion targets.  They, however, assumed the targets of
sentiments to be apriori known and only were interested whether a
positive or a negative judgement was made about them.

\citet{Derks:08}

The \texttt{SentiStrength} system proposed by \cite{Thelwall:12} used
an extensive list of 763 polar terms in order to predict positive and
negative scores for MySpace comments.  The manually assigned scores of
these terms were automatically fine-tuned during training using a
perceptron-like technique.  In addition to the core lexicon, the final
implementation of this system also utilized a set of heuristic methods
and auxiliary modules such as spelling correction algorithm,
dictionaries of booster words and negations as well as special rules
for emoticons, repeated letters, and exclamation marks.  It correctly
predicted positive emotions in 60.6~\% of the cases and attained
73.5~\% accuracy at predicting negative sentiment scores.  All
predictions were made at the level of complete messages.



% FILE: sentiment.tex  Version 0.01
% AUTHOR: Uladzimir Sidarenka

% This is a modified version of the file main.tex developed by the
% University Duisburg-Essen, Duisburg, AG Prof. Dr. Günter Törner
% Verena Gondek, Andy Braune, Henning Kerstan Fachbereich Mathematik
% Lotharstr. 65., 47057 Duisburg entstanden im Rahmen des
% DFG-Projektes DissOnlineTutor in Zusammenarbeit mit der
% Humboldt-Universitaet zu Berlin AG Elektronisches Publizieren Joanna
% Rycko und der DNB - Deutsche Nationalbibliothek

\chapter{Sentiment Corpus}\label{chap:corpus}

A crucial prerequisite for proving any hypotheses in computational
linguistics is the existence of sufficiently big manually annotated
datasets, on which these conjectures could be tested.  Since there
were no human-labeled sentiment data for German Twitter that we were
aware of at the time of writing this chapter, we decided to create our
own corpus, which we will introduce in this part of the thesis.

We begin our introduction by describing the selection criteria and
tracking procedure that we used to collect the initial corpus data.
After presenting the annotation scheme, we perform an extensive
analysis of the inter-annotator agreement.  For this purpose, we
introduce two new versions of the popular $\kappa$
metric~\cite{Cohen:60}---binary and proportional kappa---which have
been specifically adjusted to the peculiarities of our annotation
task.  Using these measures, we check the inter-coder reliability of
annotated sentiments, their targets and sources, polar terms and their
modifying elements (intensifiers, diminishers, and negations).  In the
final step, we estimate the correlation between the initial selection
rules and the number of labeled elements as well as the difficulty of
their annotation.

\section{Data Collection}

A common question that typically arises first when one starts creating
a new dataset is which selection criteria should be used in order to
collect the initial data.  Whereas for low-level NLP applications,
such as part-of-speech tagging or syntactic parsing, it typically
suffices to define the language domain to sample from (since the
phenomena of interest are usually frequent and uniformly spread), for
semantically demanding tasks with many diverse ways of expression one
also needs to consider various in-domain factors, which might
significantly affect the final distribution, making the resulting
corpus either utterly sparse or excessively biased.

In order to minimize both of these risks (sparseness and bias), we
decided to use a compromise approach by gathering one part of the new
dataset from microblogs that were a priori more likely to have
sentiments (thus increasing the recall) and sampling the rest of the
corpus uniformly at random (thus reducing the bias).

As criteria that could help us get more opinions, we considered the
topic and form of the tweets, assuming that some subjects, especially
social or political issues, would be more amenable to subjective
statements.  Because we started creating the corpus in spring 2013,
obvious choices of opinion-rich topics to us were \emph{the papal
  conclave}, which took place in March of that year, and \emph{the
  German federal elections}, which were held in autumn.  Since both of
these events implied some form of voting, we decided to counterbalance
the election specifics by including \emph{general political
  discussions} as the third subject in our dataset.  Finally, to obey
the second principle, \ie{} to keep the corpus bias low, we sampled
the rest of the data from \emph{casual everyday conversations} without
any prefiltering.

We collected messages for the first three groups by tracking German
microblogs between March and September 2013 through the public Twitter
API\footnote{\url{https://pypi.python.org/pypi/tweetstream}} with the
help of extensive keyword lists that described these topics.  For the
fourth category (casual everyday conversations), we used the complete
German Twitter snapshot~\cite{Scheffler:14}, which includes
$\approx97\%$ of all German microblogs posted in April 2013.  This
way, we obtained a total of 27.4~M messages, with the snapshot corpus
being by far the most prolific source of the data.

%% For our work, the in-domain factors to consider were the topics and
%% the form of tweets.  Since we wanted our corpus to be as
%% representative as possible, we had to make sure that the topics we
%% choose for sampling lend themselves as fruitful opinion sources.  At
%% the same time, we did not want automatically generated ad and news
%% tweets to spoil our data and also introduced additional formal
%% criteria (described below) that the tweets had to satisfy in order to
%% be chosen.  But then again, applying these restriction might make the
%% dataset excessively biased, so we did allow for a certain proportion
%% of tweets being selected even if they did not conform to our
%% constraints.

%% Politics: 59,531 messages, keywords: Altmaier, Wowereit, Minister,
%% Bundeskanzleramt, Schwarz-Gelb

%% Papst: 51,579 messages, keywords: papst, pabst, konklave, Vatikan
%% General Tweets: 51,579 messages, keywords: papst, pabst, konklave, Vatikan

%% Federal Elections: 3,131,315 messages, keywords:

%% General: 24,179,871 messages, keywords:

In the next step, we divided all tweets of the same topic into three
groups based on the following formal criteria:
\begin{itemize}
\item We put all messages that contained at least one polar term from
  the sentiment lexicon of \citet{Remus:10} into the first group;
\item Microblogs that did not satisfy the first condition, but had at
  least one exclamation mark or emoticon were allocated to the second
  group;
\item All remaining microblogs were assigned to the third category.
\end{itemize}
A detailed breakdown of the resulting distribution across topics and
formal groups is given in Table~\ref{snt:tbl:corp:topic-bins}.
\begin{table}[hbt!]\small
  \begin{center}
    \bgroup\setlength\tabcolsep{0.13\tabcolsep}\scriptsize
    \begin{tabular}{l*{5}{>{\centering\arraybackslash}p{0.155\textwidth}}}
      \toprule
      & \multicolumn{4}{c}{\bfseries Formal Criterion} & \\\cmidrule{2-5}

      \multirow{-2}{0.2\columnwidth}{\centering\bfseries
      Topic} & Polar Terms & Emoticons & Remaining Tweets & Total &\multirow{-2}{0.12\textwidth}{\centering\bfseries Sample\\ Keywords}\\\midrule

      Federal Elections & 537,083 (22.38\%) & 50,567 (2.1\%) & 1,811,742 (75.5\%) & 2,399,392 & \tiny\emph{Abgeordnete} (\emph{representative}), \emph{Regierung} (\emph{government})\\

      Papal Conclave & 7,859 (15.11\%) & 1,260 (2.42\%) & 42,879 (82.46\%) & 51,998 & \tiny\emph{Papst} (\emph{pope}), \emph{Pabst} (\emph{pobe})\\

      Political Discussions & 10,552 (25.8\%) & 777 (1.9\%) & 29,555 (72.29\%) & 40,884 & \tiny\emph{Politik} (\emph{politics}), \emph{Minister} (\emph{minister})\\

      General Conversations & 3,201,847 (18.7\%) & 813,478 (4.7\%) & 13,088,008 (76.5\%) & 17,103,333 & \tiny\emph{den} (\emph{the}), \emph{sie} (\emph{she})\\

      \bottomrule
    \end{tabular}
    \egroup{}
    \caption[Distribution of downloaded messages across topics and
      formal groups]{Distribution of downloaded messages across topics
      and formal groups\newline (percentages are given with respect to
      the total number of tweets pertaining to the given
      topic)\label{snt:tbl:corp:topic-bins}}
  \end{center}
\end{table}
%% Furthermore, as one also can observe, the relative proportions of the
%% formal tweets for the federal elections and political discussions are
%% approximately the same, while general conversations are apparently
%% more biased towards containing polar terms, whereas tweets about the
%% papal conclave, on the contrary, are rather supposed to be objectve.
%% We will check later in Subsection \ref{sec:snt:iaa} whether the
%% distribution of the annotated sentiments correlates with this
%% proportional split of groups and topics.

To create the final corpus, we randomly sampled 666 tweets from each
of the three formal classes for each of the four topics, getting a
total of 7,992 messages ($666\text{ microblogs} \times 3\text{ formal
  criteria} \times 4\text{ topics}$).

\section{Annotation Scheme}\label{subsec:snt:ascheme}

In the next step, we devised an annotation scheme for our data.  To
maximally cover all relevant sentiment aspects, we came up with an
extensive list of elements that had to be annotated by our experts.
This list included:

\begin{itemize}
\item
  \textbf{\markable{sentiment}}s, which we defined as \emph{polar
    subjective evaluative opinions about people, entities, or events}.
  According to our definition, a \markable{sentiment} always had to
  evaluate an entity that was explicitly mentioned in text---the
  target; and the annotators had to label both the target and its
  respective evaluative expression with this tag. Apart from tagging
  the text span, they also had to specify the following attributes of
  opinions:
  \begin{itemize}
  \item\attribute{polarity}, which reflected the attitude of opinion's
    holder to the evaluated entity.  Following
    \citet{Jindal:06a,Jindal:06b}, we distinguished between
    \emph{positive}, \emph{negative}, and \emph{comparative}
    sentiments;
  \item\attribute{intensity}, which showed the emotional strength of
    an opinion.  Possible values for this attribute were: \emph{weak},
    \emph{medium}, and \emph{strong};
  \item finally, drawing on the works of~\citet{Bosco:13} and
    \citet{Rosenthal:14}, we introduced a special boolean attribute
    \attribute{sarcasm} in order to distinguish sarcastically meant
    statements;
  \end{itemize}

  %% According to this definition, three important constraints that a
  %% potential subjective statement had to satisfy in order to be
  %% labeled as a sentiment in our dataset
  %% were: \begin{inparaenum}[(i)] \item \textit{polarity}, \ie{} the
  %% statement in question had to reflect an either positive or
  %% negative attitude; \item \textit{subjectivity}, \ie{} the
  %% expressed opinion had to be a personal belief not verifiable by
  %% any objective means; and, finally, \item an \textit{evaluative}
  %% nature, \ie{} the opinionated proposition had to refer to some
  %% clearly discernable target from its surrounding
  %% context.  \end{inparaenum}

\item
  we specified \textbf{\markable{target}}s as \emph{(real,
    hypothetical, or collective) entities, properties, or propositions
    (states or events) evaluated by opinions}.  For this item, we
  introduced the following three attributes:
  \begin{itemize}
    \item
      a boolean property \attribute{preferred}, which distinguished
      entities that were favored in comparisons;
    \item
      a link attribute \attribute{anaphref}, which had to point to the
      antecedent of a pronominal target;
    \item and, finally, another edge feature,
      \attribute{sentiment-ref}, which had to link \markable{target}s
      to their respective \markable{sentiment}s in the cases when the
      \markable{target} span was located at the intersection of two
      opinions;
  \end{itemize}

\item
  another important component of \markable{sentiment}s were
  \textbf{\markable{source}}s, which denoted \emph{the immediate
    author(s) or holder(s) of opinions}.  The only property associated
  with this element was \attribute{sentiment-ref}, which was defined
  the same way as for \markable{target}s.
\end{itemize}
To help our annotators identify exact boundaries of these elements, we
explicitly asked them to annotate \emph{smallest complete syntactic or
  discourse-level units}, \ie{} noun phrases or sentences with all
their grammatical dependents.

A sample tweet analyzed according to this rule is shown in
Example~\ref{snt:exmp:sent-anno1}.
\begin{example}\label{snt:exmp:sent-anno1}
  \upshape\sentiment{\target{Diese Milliardeneinnahmen} sind selbst
    \source{Sch\"auble} peinlich}\\[0.8em]
  \noindent\sentiment{\target{\itshape{}These billions of
      revenue\upshape{}}\itshape{} are embarrassing even for\\
    \upshape{}\source{\itshape{}Sch\"auble\upshape{}}}
\end{example}
In this message, we assigned the \markable{sentiment} tag to the
complete sentence because this grammatical unit is the smallest
syntactic constituent that simultaneously includes both the target of
the opinion (``Milliardeneinnahmen'' [\emph{billions of revenue}]) and
its evaluation (``peinlich'' [\emph{embarrassing}]).  Furthermore, we
also labeled the whole noun phrase ``diese Milliardeneinnahmen''
(\emph{these billions of revenue}), including the demonstrative
pronoun ``diese'' (\emph{these}), as \markable{target}, since this
pronoun syntactically depends on the main target word
``Milliardeneinnahmen'' (\emph{billions of revenue}).

Apart from \markable{sentiment}s, \markable{target}s,
\markable{source}s, we also asked the annotators to label elements
that could significantly affect the intensity and polarity of an
opinion.  These elements were:

\begin{itemize}
\item
  \textbf{\markable{polar term}}s, which we defined as \emph{words or
    idioms that had a distinguishable evaluative lexical meaning}.
  Typical examples of such terms were lexemes or set phrases such as
  ``ekelhaft'' (\emph{disgusting}), ``lieben'' (\emph{to love}),
  ``Held'' (\emph{hero}), ``wie die Pest meiden'' (\emph{to avoid like
    the pest}).  In contrast to \markable{target}s and
  \markable{source}s, which only could occur in the presence of a
  \markable{sentiment}, \markable{polar term}s were independent of
  other tags and always had to be labeled in the corpus.

  The main attributes of this element (\attribute{polarity},
  \attribute{intensity}, and \attribute{sarcasm}) largely coincided
  with the corresponding properties of \markable{sentiment}s, with the
  only difference that, in the case of \markable{polar term}s, these
  features had to reflect the lexical meaning of a word without taking
  into account its context (\ie{} \emph{prior} polarity and
  intensity), whereas for \markable{sentiment}s, they had to show the
  compositional meaning of the whole opinion (\ie{} its
  \emph{contextual} polarity and intensity).

  Besides these common properties, \markable{polar term}s also had
  their specific attributes: two boolean features
  (\attribute{subjective-fact} and \attribute{uncertain}) and a link
  attribute (\attribute{sen\-ti\-ment\--ref}).  The first feature
  showed whether a polar term denoted a factual entity with a clear
  emotional connotation, \eg{} ``Atombombe'' (\emph{A-bomb}) or
  ``Naturschutz'' (\emph{nature protection}); the second property
  signified cases in which the annotators were unsure about their
  decisions; finally, the last attribute was defined in the same way
  as it was previously specified for \markable{target}s and
  \markable{source}s;

\item
  \emph{elements that increased the expressivity and subjective sense
    of polar terms} had to be labeled as
  \textbf{\markable{intensifier}}s.  Typical examples of such
  expressions were adverbial modifiers such as ``sehr'' (\emph{very}),
  ``super'' (\emph{super}), ``stark'' (\emph{strongly});

\item
  \textbf{\markable{diminisher}}s, on the contrary, were \emph{words
    or phrases that reduced the strength of a polar term}.  Like
  \markable{intensifier}s, these elements were usually expressed by
  adverbs, \eg{} ``weniger'' (\emph{less}), ``kaum'' (\emph{hardly}),
  ``fast'' (\emph{almost}).

  Both of these tags (\markable{intensifier}s and
  \markable{diminisher}s) only had two attributes: a binary feature
  \attribute{degree} with two possible values: \emph{medium} and
  \emph{strong}; and a link attribute \attribute{polar-term-ref},
  which connected the modifier to its \markable{polar-term};

\item the final element, \textbf{\markable{negation}}s, was defined as
  \emph{grammatical or lexical means that reversed the semantic
    orientation of a polar term}.  These were typically represented by
  the negative particle ``nicht'' (\emph{not}) or indefinite pronoun
  ``keine'' (\emph{no}).  The only attribute associated with this tag
  was a mandatory link \attribute{polar-term-ref}.
\end{itemize}

In contrast to sentiment-level tags, which had to be assigned to
syntactic or discourse-level units, \markable{polar term}s and their
modifiers were defined as lexemes and, correspondingly, had to mark
only single words or set phrases without their grammatical dependents.

A complete tweet annotated with sentiment- and term-level elements is
shown in Example~\ref{snt:exmp:sent-anno2}. In this case, we again
labeled the whole sentence as \markable{sentiment} because only the
main verb-phrase simultaneously covers both the evaluated target
(``Die Nazi-Vergangenheit'' [\emph{The Nazi history}]) and its
respective polar expression (``nicht sehr r\"uhmlich'' [\emph{not very
    laudable}]).  The boundaries of \markable{sentiment} and
\markable{target} are determined on the syntactic level, spanning the
whole clause in the former case and including the complete noun phrase
in the latter.  The polarity of the opinion is set to \emph{negative}.
The polar term ``r\"uhmlich'' (\emph{laudable}), its intensifier
``sehr'' (\emph{very}), and negation ``nicht'' (\emph{not}), on the
other hand, only mark single words.  The polarity of the term, \ie{}
its primary semantic orientation without the context, is
\emph{positive}.
\begin{example}\label{snt:exmp:sent-anno2}
  %% \small
  \tikzstyle{every picture}+=[remember picture]
  \tikzstyle{na} = [shape=rectangle,inner sep=0pt]
  \upshape\sentiment{\target{Die Nazi-Vergangenheit} ist
    \negation{\tikz\node[na](word0){nicht};}
    \intensifier{\tikz\node[na](word1){sehr};}
    \emoexpression{\tikz\node[na](word2){r\"uhmlich};}}\\[2.2em]
  \noindent\sentiment{\target{\itshape{}The Nazi
      history\upshape{}}\itshape{} is
    \negation{\tikz\node[na](word3){not};}
    \upshape{}\intensifier{\tikz\node[na](word4){very};}
    \upshape{}
    \emoexpression{\itshape{}\tikz\node[na](word5){laudable};\upshape{}}}

  \begin{tikzpicture}[overlay]
    \path[->,deeppink4,thick](word0) edge [in=145, out=35] node
         [above] {\tiny polar-term-ref} (word2);
    \path[->,cyan,thick](word1) edge [in=145, out=30] node
         [above] {\tiny polar-term-ref} (word2);

    \path[->,deeppink4,thick](word3) edge [in=145, out=35] node
         [above] {\tiny polar-term-ref} (word5);
    \path[->,cyan,thick](word4) edge [in=145, out=30] node
         [above] {\tiny polar-term-ref} (word5);
  \end{tikzpicture}
\end{example}
A more detailed description of all annotation elements and their
possible attributes is given in the original annotation guidelines in
Appendix~\ref{chap:apdx:corp-guidelines} of this thesis.

\section{Annotation Tool and Format}\label{subsec:snt:tformat}

For annotating the collected data, we used \texttt{MMAX2}, a freely
available text-markup
tool.\footnote{\url{http://mmax2.sourceforge.net/}} Because this
program uses a token-oriented stand-off format, where all annotated
spans are stored in a separate file and only refer to the ids of words
in the original text, we first had to split all corpus messages into
tokens.  To this end, we applied a minimally modified version of
Christopher Potts' social media
tokenizer,\footnote{\url{http://sentiment.christopherpotts.net/code-data/happyfuntokenizing.py}}
which had been slightly adjusted to the peculiarities of the German
spelling (we allowed for the capitalized form of common nouns, \eg{}
``Freude'' [\emph{joy}], and the period at the end of ordinal numbers,
\eg{} ``7.''  [\emph{7th}]).

%% \begin{wrapfigure}{R}{0.5\textwidth}
%%   \begin{minipage}[t][21.5em]{0.5\textwidth}%
%%     \hspace{3em}
%%     \scalebox{0.65}{%
%%       \begin{minipage}[t][18em]{0.3\textwidth}%
%%         \small\vspace{-3em}%
%%         \dirtree{%
%%           .0 .
%%           .1 corpus/.
%%           .2 annotator-1/.
%%           .2 annotator-2/.
%%           .4 1.general.mmax.
%%           .4 ....
%%           .4 markables/.
%%           .5 1.general\_diminisher\_level.xml.
%%           .5 1.general\_polar-term\_level.xml.
%%           .5 1.general\_intensifier\_level.xml.
%%           .5 1.general\_negation\_level.xml.
%%           .5 1.general\_sentiment\_level.xml.
%%           .5 1.general\_source\_level.xml.
%%           .5 1.general\_target\_level.xml.
%%           .5 ....
%%           .2 basedata/.
%%           .3 1.general.words.xml.
%%           .3 ....
%%           .2 custom/.
%%           .3 polar-term\_customization.xml.
%%           .3 ....
%%           .2 scheme/.
%%           .3 polar-term\_scheme.xml.
%%           .3 ....
%%           .2 source/.
%%           .3 1.general.xml.
%%           .3 ....
%%           .2 style/.
%%           .1 docs/.
%%           .2 annotation\_guidelines.pdf.
%%           .2 ....
%%           .1 scripts/.
%%           .2 ....
%%         }%
%%       \end{minipage}
%%     }%

%%     \vspace{13em}
%%     \caption{Directory structure of the sentiment
%%       corpus\label{fig:snt:corpus}}%
%%   \end{minipage}
%% \end{wrapfigure}

To ease the annotation process and minimize possible data loss, we
split the corpus into 80 smaller project files with \numrange{99}{109}
tweets each.  In each such file, we put microblogs pertaining to the
same topic, ensuring an equal proportion of formal groups.

%% In the last preparation step, we finally created the corresponding
%% scheme and customization settings for the project, which specified
%% what kinds of elements with which attributes were to be annotated by
%% the human coders, and how these elements had to look like.

%% The resulting folder hierarchy of our dataset is shown in Figure
%% \ref{fig:snt:corpus}.

%% As can be seen from this listing, the top-level level structure of our
%% project consists of three main directories:
%% \begin{itemize}
%% \item\texttt{corpus/}, which includes the actual annotation data;

%% \item\texttt{docs/}, in which we placed the annotation guidelines and
%%   various supplementary documents, such as annotation tests for new
%%   coders;

%% \item and \texttt{scripts/}, which comprises auxiliary scripts for
%%   estimating the inter-annotator agreement and aligning corpus
%%   annotations with automatically parsed sentences.
%% \end{itemize}

%% The \texttt{corpus/} folder is further subdivided into the
%% subdirectories:

%% \begin{itemize}
%% \item\texttt{annotator-X/}, where X stands for the annotator's id.
%%   This directory includes the main project files, which specify the
%%   paths to the annotation directory, tokenization data, appearance
%%   settings etc.; and the subfolder \texttt{markables/}, which
%%   comprises the actual annotations;

%% \item\texttt{basedata/}, which contains files with tokenized messages;

%% \item\texttt{custom/}, which provides customization settings for the
%%   annotation elements (\eg{} their back- and foreground colors, font
%%   types and size etc.);

%% \item\texttt{scheme/}, which includes the definitions of the
%%   annotation markables,\footnote{In the \texttt{MMAX} terminology, an
%%     annotation markable is a synonym for an annotation element.}
%%   their attributes, and possible attributes' values;

%% \item\texttt{source/}, where we put the original untokenized
%%   microblogs;

%% \item and, finally, \texttt{style/}, which is the standard
%%   \texttt{MMAX} directory for storing default settings.
%% \end{itemize}

%% Examples of an actual annotation file and the underlying tokenization
%% data are given in Figures \ref{fig:snt:annofile} and
%% \ref{fig:snt:basefile}.

%% \begin{minipage}[t]{\textwidth}
%%   \begin{minipage}[t]{0.45\textwidth}
%%     \lstset{language=XML}
%%     \begin{lstlisting}
%% <?xml version="1.0" encoding="UTF-8"?>
%% <!DOCTYPE markables SYSTEM "markables.dtd">
%% <markables xmlns="www.eml.org/NameSpaces/sentiment">
%% <markable id="markable_70"
%% span="word_592..word_596" sarcasm="false"
%% mmax_level="sentiment"  polarity="positive"
%% intensity="strong" />
%% <markable id="markable_132"
%% span="word_1126..word_1139" sarcasm="false"
%% mmax_level="sentiment"  polarity="positive"
%% intensity="medium" />
%% <markable id="markable_256"
%% span="word_1056..word_1071" sarcasm="false"
%% mmax_level="sentiment"  polarity="negative"
%% intensity="medium" />
%% <markable id="markable_259"
%% span="word_1074..word_1087" sarcasm="false"
%% mmax_level="sentiment"  polarity="positive"
%% intensity="medium" />
%% ...
%% </markables>
%%     \end{lstlisting}%
%%     \captionof{figure}{Example of an annotation file\label{fig:snt:annofile}}%
%%   \end{minipage}\hfill%
%%   %
%%   \begin{minipage}[t]{0.45\textwidth}%
%%     \lstset{language=XML}
%%     \begin{lstlisting}[basicstyle=\tiny]
%% <?xml version="1.0" encoding="US-ASCII"?>
%% <!DOCTYPE words SYSTEM "words.dtd">
%% <words>
%% <word id="word_1">Gleich</word>
%% <word id="word_2">in</word>
%% <word id="word_3">Braunschweig</word>
%% <word id="word_4">mit</word>
%% <word id="word_5">Kamaraden</word>
%% <word id="word_6">Treffen</word>
%% <word id="word_7">:)</word>
%% <word id="word_8">EOL</word>
%% <word id="word_9">@graulich12</word>
%% <word id="word_10">Das</word>
%% <word id="word_11">geht</word>
%% <word id="word_12">ja</word>
%% <word id="word_13">gar</word>
%% <word id="word_14">nicht</word>
%% <word id="word_15">!</word>
%% ...
%% </words>
%%     \end{lstlisting}%
%%     \captionof{figure}{Example of tokenized data\label{fig:snt:basefile}}%
%%   \end{minipage}
%% \end{minipage}

\section{Inter-Annotator Agreement Metrics}\label{sec:snt:iaa}

For estimating the inter-annotator agreement (IAA), we adopted the
popular $\kappa$ metric \cite{Cohen:60}.  Following the standard
practice, we computed this term as:
\begin{equation*}
  \kappa = \frac{p_o - p_c}{1 - p_c},
\end{equation*}
where $p_o$ denotes the observed agreement, and $p_c$ stands for the
agreement by chance.  We estimated the observed reliability in the
normal way as the ratio of tokens with matching annotations to the
total number of tokens:
\begin{equation*}
  p_o = \frac{T - A_1 + M_1 - A_2 + M_2}{T},
\end{equation*}
where $T$ represents the total token count, $A_1$ and $A_2$ are the
numbers of tokens annotated with the given class by the first and
second annotators respectively, and the $M$ terms mean the numbers of
tokens with matching annotations.  As usual, we computed the chance
agreement $p_c$ as:
\begin{equation*}\textstyle
  p_c = c_1 \times c_2 + (1.0 - c_1) \times (1.0 - c_2).
\end{equation*}
where $c_1$ and $c_2$ are the proportions of tokens annotated with the
given class in the first and second annotations respectively, \ie{}
$c_1 = \frac{A_1}{T}$ and $c_2 = \frac{A_2}{T}$.

Two questions that arose during this computation though were
\begin{inparaenum}[(i)]
  \item whether tokens belonging to multiple overlapping annotation
    spans of the same class had to be counted several times in one
    annotation when computing the $A$ scores (for instance, whether we
    had to count the words ``dieses'' [\textit{this}], ``sch\"one''
    [\textit{nice}], and ``Buch'' [\textit{book}] in Example
    \ref{example:snt:iaa} twice as sentiments when computing $A_1$ and
    $A_2$), and
  \item whether we had to assume that two annotated spans from
    different experts agreed on all of their tokens if these spans had
    at least one word in common (\eg{} whether we had to consider the
    annotation of the token ``Mein'' [\textit{My}] in the example as
    matching, regarding that the rest of the corresponding
    \markable{sentiment}s agreed).
\end{inparaenum}

\begin{example}\label{example:snt:iaa}
\textcolor{red3}{\textbf{Annotation 1:}}\\
\upshape\sentiment{Mein Vater hasst \sentiment{dieses sch\"one Buch}.}\\
\sentiment{\itshape My father hates \upshape\sentiment{\itshape this
    nice book\upshape}.}

\noindent\textcolor{darkslateblue}{\textbf{\itshape Annotation 2:}}\\
Mein \sentiment{Vater hasst \sentiment{dieses sch\"one Buch}.}\\
\itshape My \upshape\sentiment{\itshape{}father hates \upshape\sentiment{\itshape this
    nice book\upshape}.}
\end{example}

To address these issues, we introduced two different agreement
metrics---\emph{binary} and \emph{proportional} kappa.  With the
former variant, we counted tokens belonging to overlapping annotation
spans of the same class multiple times (\ie{} $A_1$ and $A_2$ would
amount to $10$ and $9$ respectively in the above tweet) and considered
all tokens belonging to the given annotated element as matching if
this span agreed with the annotation from the other expert on at least
one token (\ie{} $M_1$ and $M_2$ would have the same values as $A_1$
and $A_2$ in this case).  With the latter metric, every labeled token
was counted only once (\ie{} the numbers of labeled words in the first
and second annotations would be $7$ and $6$ respectively), and we only
calculated the actual number of tokens with matching labels when
computing the $M$ scores (\ie{} both $M_1$ and $M_2$ would be equal to
$6$).  The final value of the binary kappa in
Example~\ref{example:snt:iaa} would consequently run up to~1.0,
because this metric would consider both annotations as perfectly
matching, since every labeled \markable{sentiment} agreed with the
other annotation on at least one token.  The proportional kappa,
however, would be equal to~0.0, since this metric would emphasize the
fact that the observed reliability $p_o$ is the same as the agreement
by chance $p_c$, and would therefore deem both labelings as
fortuitous.

\section{Annotation Procedure}\label{sec:astages}

After defining the agreement metrics, we finally let our experts
annotate the data.  The annotation procedure was performed in three
steps:
\begin{itemize}
  \item At the beginning, both annotators labeled one half of the
    corpus after only minimal training.  Unfortunately, their mutual
    agreement at this stage was relatively low, reaching only 31.21\%
    proportional-$\kappa$ for \markable{sentiment}s;
  \item In the second step, in order to improve the inter-rater
    reliability, we automatically determined all differences between
    the two annotations and highlighted non-matching tokens with a
    separate class of tags.  Then we let the experts resolve these
    discrepancies by either correcting their own decisions or
    rejecting the variants of the other coder.  As in the previous
    stage, we allowed the annotators to consult their supervisor (the
    author of this thesis), also updating the FAQ section of the
    guidelines based on their questions, but did not let them
    communicate with each other directly.  This adjudication step
    significantly improved all annotations: The agreement on
    \markable{sentiment}s increased by 30.73\%, reaching 61.94\%.
    Similar effects were observed for \markable{target}s,
    \markable{source}s, \markable{polar term}s, and their modifiers;
  \item After resolving all differences, our assistants proceeded with
    the annotation of the remaining files.  Working completely
    independently, one of the experts annotated 78.8\% of the corpus,
    whereas the second annotator labeled the complete dataset.
\end{itemize}

\section{Evaluation}\label{sec:eval}

\subsection{Initial Annotation Stage}\label{subsec:eval-initial-stage}

The agreement results of the initial annotation stage are shown in
Table~\ref{tbl:snt:agrmnt-init}.
\begin{table*}[thb!]
  \begin{center}
    \bgroup \setlength\tabcolsep{0.7\tabcolsep} \scriptsize
    \begin{tabular}{p{0.154\textwidth} % first columm
        *{10}{>{\centering\arraybackslash}p{0.065\textwidth}}} % next ten columns
      \toprule
          \multirow{2}{0.2\textwidth}{\bfseries Element} &
          \multicolumn{5}{c}{\bfseries Binary $\kappa$} & %
          \multicolumn{5}{c}{\bfseries Proportional $\kappa$}\\
          \cmidrule(r){2-6}\cmidrule(l){7-11}
          & $M_1$ & $A_1$ & $M_2$ & $A_2$ & $\mathbf{\kappa}$ %
          & $M_1$ & $A_1$ & $M_2$ & $A_2$ & $\mathbf{\kappa}$\\\midrule

          Sentiment & 4,215 & 7,070 & 3,484 & 9,827 & \textbf{38.05} &
          3,269 & 6,812 & 3,269 & 9,796 & \textbf{31.21}\\
          Target & 1,103 & 1,943 & 1,217 & 4,162 & \textbf{35.48} &
          898 & 1,905 & 898 & 4,148 & \textbf{26.85}\\
          Source & 159 & 445 & 156 & 456 & \textbf{34.53} &
          153 & 439 & 153 & 456 & \textbf{33.75}\\
          Polar Term & 1,951 & 2,854 & 2,029 & 3,188 & \textbf{64.29} &
          1,902 & 2,851 & 1,902 & 3,180 & \textbf{61.36}\\
          Intensifier & 57 & 101 & 59 & 123 & \textbf{51.71} &
          57 & 101 & 57 & 123 & \textbf{50.81}\\
          Diminisher & 3 & 10 & 3 & 8 & \textbf{33.32} &
          3 & 10 & 3 & 8 & \textbf{33.32}\\
          Negation & 21 & 63 & 21 & 83 & \textbf{28.69} &
          21 & 63 & 21 & 83 & \textbf{28.69}\\\bottomrule
    \end{tabular}
    \egroup
  \end{center}
  \captionof{table}[Inter-annotator agreement after the initial
    annotation stage]{Inter-annotator agreement after the initial
    annotation stage\\ {\small ($M1$ -- number of tokens with matching
      labels in the first annotation, $A1$ -- total number of tokens
      labeled with that class in the first annotation, $M2$ -- number
      of tokens with matching labels in the second annotation, $A2$ --
      total number of tokens labeled with that class in the second
      annotation)}}
  \label{tbl:snt:agrmnt-init}
\end{table*}

As we can see from the table, the inter-rater reliability of
\markable{sentiment}s strongly correlates with the inter-annotator
agreement on \markable{target}s and \markable{source}s, setting an
upper bound for these elements in the binary-$\kappa$ case.  With the
proportional metric, however, both \markable{sentiment}s and
\markable{target}s show worse results than \markable{source}s:
$31.21\%$ and $26.85\%$ versus $33.75\%$.  We explain this difference
by the fact, that \markable{sentiment}s and \markable{target}s are
typically represented by syntactic or discourse-level constituents
(noun phrases or clauses) and, even though the experts agreed on the
presence of these elements more often (as suggested by the
binary-$\kappa$ metric), reaching a consensus about the exact
boundaries of these elements was still a challenging task for them
despite an explicit clarification of this problem in the annotation
guidelines; \markable{source}s, on the other hand, are usually
expressed by pronouns, which rarely accept syntactic attributes, so
that their boundaries were easier to determine.  Nevertheless, even
with the binary metric, the agreement of all sentiment-level elements
is significantly below the $40\%$ threshold, which means only a slight
reliability according to the \citeauthor{Landis:77} scale
\cite{Landis:77}.

A different situation is observed for \markable{polar terms} and
\markable{intensifiers}.  The inter-annotator agreement of these
elements is above 50\%, for both $\kappa$-measures.  Obviously,
defining these entities as lexical units has significantly eased the
detection of their boundaries.  This effect becomes even more evident
when we look at \markable{diminisher}s and \markable{negation}s, where
the $A$ and $M$ scores are absolutely identical for both metrics.  It
means that both annotators always agreed on the boundaries of these
elements if they agreed on their presence.  Unfortunately, due to a
rather small number of these tags in the corpus (with only 3 cases of
\markable{diminisher}s and 21 cases of \markable{negation}s), the
overall agreement of these labels is relatively small too, amounting
to $33.32\%$ and $28.69\%$ respectively.

\subsection{Adjudication Step}\label{subsec:eval-adjudication-step}

Since these scores were unacceptable for running further experiments,
we decided to revise diverging annotations by letting our experts
recheck each other's decisions.
%% To this end, we automatically determined conflicting labelings and
%% highlighted them in the annotated \texttt{MMAX2} files.
%% Afterwards, the coders had to decide whether to ignore the
%% highlighted discrepancies or to change ther own decisions.
\begin{table*}[htb!]
  \begin{center}
    \bgroup \setlength\tabcolsep{0.7\tabcolsep} \scriptsize
    \begin{tabular}{p{0.155\textwidth} % first columm
        *{10}{>{\centering\arraybackslash}p{0.065\textwidth}}} % next ten columns
      \toprule
          \multirow{2}{0.2\textwidth}{\bfseries Element} &
          \multicolumn{5}{c}{\bfseries Binary $\kappa$} & %
          \multicolumn{5}{c}{\bfseries Proportional $\kappa$}\\
          \cmidrule(r){2-6}\cmidrule(l){7-11}
          & $M_1$ & $A_1$ & $M_2$ & $A_2$ & $\mathbf{\kappa}$ %
          & $M_1$ & $A_1$ & $M_2$ & $A_2$ & $\mathbf{\kappa}$\\
          \midrule

          Sentiment & 8,198 & 8,530 & 8,260 & 14,034 & \textbf{67.92} &
          7,435 & 8,243 & 7,435 & 13,714 & \textbf{61.94}\\

          Target & 3,088 & 3,407 & 2,814 & 5,303 & \textbf{65.66} &
          2,554 & 3,326 & 2,554 & 5,212 & \textbf{57.27}\\

          Source & 573 & 690 & 545 & 837 & \textbf{72.91} &
          539 & 676 & 539 & 833 & \textbf{71.12}\\

          Polar Term & 3,164 & 3,298 & 3,261 & 4,134 & \textbf{85.68} &
          3,097 & 3,290 & 3,097 & 4,121 & \textbf{82.64}\\

          Intensifier & 111 & 219 & 113 & 180 & \textbf{56.01} &
          111 & 219 & 111 & 180 & \textbf{55.51}\\

          Diminisher & 9 & 16 & 10 & 16 & \textbf{59.37} &
          9 & 16 & 9 & 15 & \textbf{58.05}\\

          Negation & 68 & 84 & 67 & 140 & \textbf{60.21} &
          67 & 83 & 67 & 140 & \textbf{60.03}\\\bottomrule
    \end{tabular}
    \egroup
  \end{center}
  \captionof{table}[Inter-annotator agreement after the adjudication
  step]{Inter-annotator agreement after the adjudication step\\
    {\small ($M1$ -- number of tokens with matching labels in the
      first annotation, $A1$ -- total number of labeled tokens in the
      first annotation, $M2$ -- number of tokens with matching labels
      in the second annotation, $A2$ -- total number of labeled tokens
      in the second annotation)}}
  \label{tbl:snt:agrmnt-adjud}
\end{table*}
As we can see from the results in Table~\ref{tbl:snt:agrmnt-adjud},
this procedure significantly improved the inter-rater reliability of
all annotated elements: the binary scores of \markable{sentiment}s and
\markable{target}s increased by $29.87\%$ and $30.18\%$ respectively.
An even greater improvement is observed for \markable{source}s, whose
binary kappa improved by remarkable $38.38\%$.  A similar tendency
applies to the proportional metric, where the agreement of
\markable{sentiment}s gained $30.73\%$, reaching $61.94\%$.  Likewise,
the reliability of opinion targets and holders improved by $30.42\%$
and $37.37\%$, running up to $57.27\%$ and $71.12\%$.

%% In general, however, we can see that the second annotator labeled
%% almost twice as many \markable{sentiment}s and \markable{target}s as
%% the first expert: the proportional $A_1$ scores of these two entity
%% types amount to 8,243 and 3,326 tokens, whereas the corresponding
%% $A_2$ counts run up to 13,714 and 5,212 words.

%% A better consistency in this regard is achieved by sources, where
%% the number of the labeled items in both annotations differs only by
%% a factor of 1.2.

As in the previous step, the highest agreement scores are attained by
\markable{polar term}s, whose reliability notably surpasses the 80\%
benchmark, which means an almost perfect agreement level.
Interestingly enough, only 193 out of 3,290 terms annotated by the
first expert did not match the labelings of the second annotator.
Another interesting observation is that the difference between the
binary and proportional scores of \markable{polar terms} only amounts
to 3.04\%, which implies that the assistants could unproblematically
determine the boundaries of these elements in most of the cases.

Somewhat surprisingly, the agreement of \markable{intensifier}s
improved notably less.  A closer look at the annotated cases revealed
that the majority of their disagreements stemmed from different takes
of exclamation marks: the first expert ignored these punctuation
marks, whereas the second annotator considered them as valid
intensifying elements.  Nevertheless, even despite these diverging
interpretations, the reliability of \markable{intensifier}s is above
$55\%$, which means a moderate level.

\subsection{Final Annotation Stage}\label{subsec:eval-final-annotation}

After ensuring that our annotators could reach an acceptable quality
of annotation, we eventually let them label the remaining part of the
data.  The agreement results of the final stage computed on the files
annotated by both experts are given in
Table~\ref{tbl:snt:agrmnt-final}.
\begin{table*}[thb!]
  \begin{center}
    \bgroup \setlength\tabcolsep{0.7\tabcolsep} \scriptsize
    \begin{tabular}{p{0.155\textwidth} % first columm
        *{10}{>{\centering\arraybackslash}p{0.065\textwidth}}} % next ten columns
      \toprule
          \multirow{2}{0.2\textwidth}{\bfseries Element} &
          \multicolumn{5}{c}{\bfseries Binary $\kappa$} & %
          \multicolumn{5}{c}{\bfseries Proportional $\kappa$}\\
          \cmidrule(r){2-6}\cmidrule(l){7-11}
          & $M_1$ & $A_1$ & $M_2$ & $A_2$ & $\mathbf{\kappa}$ %
          & $M_1$ & $A_1$ & $M_2$ & $A_2$ & $\mathbf{\kappa}$\\
          \midrule

          Sentiment & 14,748 & 15,929 & 14,969 & 26,047 & \textbf{65.03} &
          13,316 & 15,375 & 13,316 & 25,352 & \textbf{58.82}\\

          Target & 5,765 & 6,629 & 5,292 & 9,852 & \textbf{64.76} &
          4,789 & 6,462 & 4,789 & 9,659 & \textbf{56.61}\\

          Source & 966 & 1,207 & 910 & 1,619 & \textbf{65.99} &
          898 & 1,180 & 898 & 1,604 & \textbf{64.1}\\

          Polar Term & 5,574 & 5,989 & 5,659 & 7,419 & \textbf{82.83} &
          5,441 & 5,977 & 5,441 & 7,395 & \textbf{80.29}\\

          Intensifier & 192 & 432 & 194 & 338 & \textbf{49.97} & 192 &
          432 & 192 & 338 & \textbf{49.71}\\

          Diminisher & 16 & 30 & 17 & 34 & \textbf{51.55} & 16 & 30 &
          16 & 33 & \textbf{50.78}\\

          Negation & 111 & 132 & 110 & 243 & \textbf{58.87} & 110 &
          131 & 110 & 242 & \textbf{58.92}\\\bottomrule
    \end{tabular}
    \egroup
  \end{center}
  \captionof{table}[Inter-annotator agreement of the final
  corpus]{Inter-annotator agreement of the final corpus\\ {\small
      ($M1$ -- number of tokens with matching labels in the first
      annotation, $A1$ -- total number of labeled tokens in the first
      annotation, $M2$ -- number of tokens with matching labels in the
      second annotation, $A2$ -- total number of labeled tokens in the
      second annotation)}}
  \label{tbl:snt:agrmnt-final}
\end{table*}

This time, we can observe a slight decrease of the results: the
proportional score for \markable{sentiment}s dropped by $3.12\%$,
whereas the agreement of \markable{target}s was more persistent and
lost only $0.66\%$, going down to $56.61\%$.  The most dramatic
changes occurred for \markable{source}s, whose proportional value
deteriorated by notable $7.02\%$, sinking to $64.1\%$.  Nonetheless,
the average proportional agreement of all these elements is around
$60.5\%$, which is almost twice as high as the mean reliability
achieved in the first stage.

As before, the scores of \markable{polar term}s are in the ballpark of
almost perfect results.  Their modifying elements, however, show a
decrease: the agreement of \markable{intensifier}s deteriorated by
5.8\%, sinking to 49.71\% proportional kappa.  A similar situation is
observed for \markable{diminisher}s, whose kappa worsened from
$58.05\%$ to $50.78\%$.  The best persistence in this regard is shown
by \markable{negation}s, where the quality dropped by only $1.11\%$,
which can be considered as a very good result, regarding the small
number of these elements in the corpus.

In general, we can see that the reliability of all elements in the
final dataset is at least moderate, with \markable{polar term}s being
the most reliably annotated elements ($\kappa_{\textrm{p}}=80.29\%$)
and \markable{intensifier}s setting a lower bound on the agreement
($\kappa_{\textrm{p}}=49.71\%$).

\subsection{Qualitative Analysis}\label{subsec:eval-qualitative-analysis}

In order to understand the reasons for remaining conflicts, we decided
to have a closer look at the diverging cases.  A sample sentence with
different analyses of \markable{sentiment}s is shown in
Example~\ref{snt:exmp:sent-disagr}:
\begin{example}\label{snt:exmp:sent-disagr}
  \textcolor{red3}{\textbf{Annotation 1:}}\\ \upshape{}@TinaPannes
  immerhin ist die \#afd nicht dabei \smiley{}\\[0.8em]\itshape
  \noindent\textcolor{darkslateblue}{\textbf{\itshape Annotation
      2:}}\\ \upshape{}@TinaPannes
  \sentiment{\textcolor{red}{\target{immerhin ist die \#afd nicht
      dabei} \smiley{}}}\\[0.8em]
  \noindent\itshape{}@TinaPannes
  \upshape\sentiment{\textcolor{red}{\itshape{}\target{anyway the
        \#afd is not there} \smiley{}}\upshape{}}
\end{example}
In this tweet, the first annotator obviously overlooked the emoticon
\smiley{} at the end of the message, whereas the second expert
correctly recognized it as an evaluation of the previous sentence.
Because the first assistant did not label any \markable{sentiment} at
all, she also automatically disagreed on the assignment of the
\markable{target} tag.

%% At this point, we should note that it also was legitimate to consider
%% the noun phrase ``die \#afd'' (\emph{the \#afd}) as the object of the
%% evaluation in this message.  We, however, advised the assistants to be
%% as specific as possible when determining the target
%% elements. Therefore, when a sentiment was related to a particular
%% action performed by an agent (which, in this case, was the fact of
%% afd's being somewhere) rather than the agent herself, they better had
%% to label the complete verb phrase and not only its acting subject.
%% With this rule, we hoped to distinguish targets in sentences like
%% ``die Partei hat das Gesetz verabschiedet \smiley{}'' (\emph{the party
%%   has adopted this law \smiley{}}) from the objects of evaluations in
%% microblogs like ``die Partei hat das Gesetz abgelehnt \smiley{}''
%% (\emph{the party has rejected this law \smiley{}}), which were clearly
%% describing two completely different events so that labeling similar
%% targets, \eg{} ``die Partei'' (\emph{the party}), in both of these
%% messages would be unequivocally wrong in that case.

A much rarer case of diverging \markable{target} annotations was when
both experts actually marked a \markable{sentiment} span.  An example
of such situation is shown in the following message:
\begin{example}\label{snt:exmp:targt-disagr}
  \textcolor{red3}{\textbf{Annotation
      1:}}\\
  \upshape{}\sentiment{Koalition wirft der SPD
    \target{\textcolor{red}{Blockadehaltung}} vor}\\[0.5em]
  \noindent\itshape{}\sentiment{Coalition accuses the SPD of
    \target{\textcolor{red}{blocking politics}}}\\[0.6em]\itshape

  \noindent\textcolor{darkslateblue}{\textbf{\itshape Annotation
      2:}}\\
  \upshape{}\sentiment{Koalition wirft \target{\textcolor{red}{der SPD}}
    Blockadehaltung vor}\\[0.5em]
  \noindent\itshape{}\sentiment{Coalition accuses
    \target{\textcolor{red}{the SPD}} of blocking politics}
\end{example}
In this sentence, the first expert considered \emph{blocking politics}
as the main object of criticism, whereas the second annotator regarded
the political party accused of such behavior as sentiment's target.
In our opinion, both of these interpretations are correct and,
ideally, two \markable{sentiment}s had to be labeled in this message:
one with the target ``Blockadehaltung'' (\emph{blocking politics}) and
another one with the target ``die SPD'' (\emph{the SPD}).

Although our annotators were much more consistent about the analysis
of \markable{polar term}s, we still decided to have a look at
disagreeing labels of these elements.  A sample case of differently
annotated \markable{polar term}s is given in
Example~\ref{snt:exmp:emo-disagr}
\begin{example}\label{snt:exmp:emo-disagr}
  \textcolor{red3}{\textbf{Annotation 1:}}\\ \upshape{}Syrien vor dem
  Angriff---bringen diese Bomben den Frieden?\\[0.3em]\itshape
  \noindent\itshape{}Syria facing an attack---will these bombs bring
  peace?\\

  \noindent\textcolor{darkslateblue}{\textbf{\itshape Annotation
      2:}}\\ \upshape{}Syrien vor dem
  \emoexpression{\textcolor{red}{Angriff}}---bringen diese
  \emoexpression{\textcolor{red}{Bomben}} den
  \emoexpression{\textcolor{red}{Frieden}}?\\[0.3em]
  \noindent\itshape{}Syria facing an
  \upshape\emoexpression{\textcolor{red}{\itshape{}attack}\upshape}\itshape{}---will
  these
  \upshape\emoexpression{\textcolor{red}{\itshape{}bombs}\upshape}\itshape{}
  bring
  \upshape\emoexpression{\textcolor{red}{\itshape{}peace}\upshape}\itshape{}?
\end{example}
The obvious reason for the misclassifications in this message is the
notorious subjective facts: As you can see, the first assistant
ignored the words ``Angriff'' (\emph{attack}), ``Bombe''
(\emph{bomb}), and ``Frieden'' (\emph{peace}), while the second
annotator considered them as polar items.
%% Even though our original guidelines were underspecified in this
%% case (we only imposed the subjectivity constraint on the
%% sentiments), in our later experiments
%% (see~Section~\ref{sec:snt:lex}), labeling these terms turned out to
%% be a better solution when comparing the corpus annotation with the
%% existing sentiment lexica.
We should, however, admit that this difference is partially due to the
adjudication procedure that we used in step two, because at the
initial stage, our experts had had opposite preferences regarding
these entities (the first annotator had labeled these terms, whereas
the second assistant had usually skipped them).  During the revision,
however, both assistants changed their minds after looking at the
decisions of the other linguist.  Therefore, one needs to keep in mind
the risk of mutual concession when applying the adjudication method in
future.

%% Although the adjudication step significantly improved the reliability
%% of other \markable{polar term}s and tags, the risk of mutual
%% concession needs to be kept in mind when applying this solution in
%% future.

\subsection{Attributes Agreement}\label{subsec:eval-qualitative-analysis}

In order to see whether our annotators also agreed on the attributes
of the assigned tags, we estimated the Cohen's kappa for the polarity
and the Krippendorff's alpha \cite{Krippendorff:07} for the intensity
of matching \markable{sentiment}s and \markable{polar term}s.  The
reason for choosing two different metrics in this case is that
\attribute{polarity} is a categorical feature, whose value takes on
one of the predefined classes (\emph{positive}, \emph{negative}, or
\emph{comparison}); whereas \attribute{intensity} is an ordinal
attribute, whose value can range on a scale from zero (\emph{weak}) to
two (\emph{strong}).  Since disagreements that are further apart on
the scale need to be penalized more strongly than small divergences,
we decided to use the $\alpha$-measure for this attribute, as it
explicitly addresses this problem.
\begin{table}[thb!]
  \begin{center}
    \bgroup \setlength\tabcolsep{0.47\tabcolsep} \scriptsize
    \begin{tabular}{p{0.23\columnwidth}%
          *{2}{>{\centering\arraybackslash}p{0.2\columnwidth}}} % next five columns
      \toprule
          {\bfseries Element} & {\bfseries Polarity $\kappa$} & %
          {\bfseries Intensity $\alpha$}\\\midrule
          Sentiment & 58.8 & 73.54\\
          Polar Term & 87.12 & 78.79\\
          \bottomrule
    \end{tabular}
    \egroup
    \caption{Inter-annotator agreement on polarity and intensity of
      sentiments and polar terms}
    \label{tbl:attr-agrmnt}
  \end{center}
\end{table}

As we can see from the results in Table~\ref{tbl:attr-agrmnt},
reaching a consensus about the polarities of \markable{polar term}s is
a much easier task than agreeing on the semantic orientation of
\markable{sentiment}s.  As in Example~\ref{snt:exmp:sent-disagr}, one
of the main reasons for these disagreements is opinions containing
emoticons, especially in the cases when the polarity of the smiley
contradicts the polarity of the preceding text, \eg{} ``Ich hasse die
Piratenpartei \smiley{}'' (\emph{I hate the Pirate Party {\upshape
    \smiley{}}}).

Interestingly enough, the inter-rater agreement on the intensity of
\markable{sentiment}s ($\alpha = 73.54$) is notably higher than the
corresponding score for their polarity ($\kappa = 58.8$), although the
opposite situation is observed for \markable{polar term}s, whose
$\alpha$-value ($78.79$) is almost ten percent lower than $\kappa$
(87.12).  This means that the annotators could easily determine the
semantic orientation of a single word, but had difficulties with
agreeing on the strength of its meaning.  Vice versa, when dealing
with targeted opinions, they usually assigned the \emph{medium}
intensity to most \markable{sentiment}s, but could disagree on the
polarity of these statements.

For the sake of completeness, we compared these results with the
scores obtained on the MPQA dataset \cite[see][pp. 38, 80]{Wilson:07}.
The average $\alpha$-agreement on the intensity of direct subjective
and objective speech events (a rough counterpart of our
\markable{sentiment}s) in this corpus was around 79\%; the
corresponding results for the intensity of expressive subjective
elements (\markable{polar term}s in our case), however, were much
worse, amounting to only 46\%, even though the $\kappa$-value for
their polarity run up to 72\%.  Hence, the reliability of the
annotated attributes in our corpus still outperforms the respective
agreement in MPQA on almost all aspects except for sentiment
intensity.

%% We should, however, note that the annotation scheme used for the
%% latter dataset did not provide a specific attribute for the
%% polarity of sentiments, specifying the polarity of opinionated
%% terms as their contextual valence instead, \ie{} the polarity of
%% the respective terms with their possible modifications from the
%% surrounding context.

\subsection{Effect of the Selection Criteria}\label{subsec:eval-selection-criteria}

Finally, in order to check how the selection criteria that we applied
initially when sampling the corpus data affected the resulting
distribution of \markable{sentiment}s and \markable{polar term}s in
the final dataset, we plotted the frequencies and agreement scores of
these elements across topics and formal groups, and present these
statistics in Figures~\ref{snt:fig:crp-sent-emo-distr}
and~\ref{snt:fig:crp-sent-emo-agr}.

\begin{figure*}[htbp!]
{
\centering
\begin{subfigure}{.5\textwidth}
  \centering
  \includegraphics[width=\linewidth]{img/sentiment_stat.png}
  \caption{\texttt{Sentiments}}\label{snt:fig:crp-sent-emo-distr-a}
\end{subfigure}%
\begin{subfigure}{.5\textwidth}
  \centering
  \includegraphics[width=\linewidth]{img/emo-expression_stat.png}
  \caption{\texttt{Polar Terms}}\label{snt:fig:crp-sent-emo-distr-b}
\end{subfigure}
}
\caption{Distribution of sentiments and polar terms across topics and
  formal groups}\label{snt:fig:crp-sent-emo-distr}
\end{figure*}

\begin{figure*}[htbp!]
{
\centering
\begin{subfigure}{.5\textwidth}
  \centering
  \includegraphics[width=\linewidth]{img/sentiment_agreement.png}
  \caption{\texttt{Sentiments}}
\end{subfigure}%
\begin{subfigure}{.5\textwidth}
  \centering
  \includegraphics[width=\linewidth]{img/emo-expression_agreement.png}
  \caption{\texttt{Polar Terms}}
\end{subfigure}
}
\caption{Inter-annotator agreement on sentiments and polar terms
  across topics and formal groups}\label{snt:fig:crp-sent-emo-agr}
\end{figure*}

As we can see from the plots, topics and form clearly affect both the
number of opinions and the difficulty of their interpretation.
According to Figure~\ref{snt:fig:crp-sent-emo-distr}, the greatest
number of \markable{sentiment}s occur in tweets pertaining to the
federal elections and in messages representing casual everyday
conversations.  A similar tendency is observed for \markable{polar
  term}s, but in this case, the form of the microblogs seems to have
more impact on the elements' distribution than their topics.

%% Moreover, a higher number of opinionated terms might not necessarily
%% lead to a higher number of targeted sentiments.  We can recognize that
%% from the fact that, even though the biggest number of subjective
%% opinions show up in the first row of
%% Plot~\ref{snt:fig:crp-sent-emo-distr-a} (\ie{} in tweets with terms
%% from the SentiWS lexicon), most of the polar terms appear in row two
%% of Plot~\ref{snt:fig:crp-sent-emo-distr-b} (\ie{} in microblogs
%% containing smileys).

Regarding the inter-annotator agreement, we can see that
\markable{sentiment}s and \markable{polar term}s are most reliably
annotated in messages from the German Twitter Snapshot.  Moreover, the
former elements are apparently easiest to annotate in tweets that were
preselected using a sentiment lexicon, whereas \markable{polar term}s
are easiest to analyze in microblogs that contain emoticons.
%% Somewhat surprisingly, the emoticon group has the lowest IAA scores
%% for \markable{sentiment}s, which means that although our annotators
%% could easily recognize facial expressions as polar items, they had
%% difficulties with determining whether these expressions evaluated
%% something particular in the microblog or reflected the general mood of
%% the author.

To confirm the correlation between the topics and formal groups on the
one hand and the number and reliability of \markable{sentiment}s and
\markable{polar term}s on the other hand, we computed the correlation
coefficients ($\rho$) of these factors, considering each particular
topic and formal group as a binary variable and measuring the
association of this variable with the number and agreement of
annotated elements.
\begin{table}[thb!]
  \begin{center}
    \bgroup\setlength\tabcolsep{0.47\tabcolsep}\scriptsize
    \begin{tabular}{p{0.2\columnwidth}%
          *{4}{>{\centering\arraybackslash}p{0.185\columnwidth}}} % next five columns
      \toprule

      \multirow{3}{0.2\columnwidth}{\centering\bfseries Selection Criteria} & %
      \multicolumn{4}{c}{\bfseries Correlation Coefficients}\\\cmidrule(lr){2-3}\cmidrule(lr){4-5}

      & \multicolumn{2}{c}{\bfseries Sentiment}& %
      \multicolumn{2}{c}{\bfseries Polar Term}\\\cmidrule(lr){2-2}\cmidrule(lr){3-3}%
      \cmidrule(lr){4-4}\cmidrule(lr){5-5}

      & \# of elements & agreement & \# of elements & agreement\\\midrule

      \multicolumn{5}{c}{\cellcolor{cellcolor}Topical Groups}\\
      Federal Elections & \textbf{0.312} & 0.169 & 0.356 & 0.289\\
      Papal Conclave & 0.149 & 0.124 & 0.182 & 0.264\\
      Political Discussions & 0.195 & 0.148 & 0.218 & 0.244\\
      General Conversations & 0.183 & \textbf{0.19} & \textbf{0.372} & \textbf{0.452}\\
      \multicolumn{5}{c}{\cellcolor{cellcolor}Formal Categories}\\
      Polar Terms & \textbf{0.445} & \textbf{0.352} & 0.38 & 0.301\\
      Emoticons & 0.127 & 0.096 & \textbf{0.47} & \textbf{0.615}\\
      Random & 0.216 & 0.134 & 0.143 & 0.138\\
      \bottomrule
    \end{tabular}
    \egroup
    \caption[Correlation coefficients of topics and selection criteria
      with the number and agreement of sentiments and polar
      terms]{Correlation coefficients of topics and formal selection
      criteria with the number and agreement scores of sentiments and
      polar terms}
    \label{sent:tbl:corr-coeff}
  \end{center}
\end{table}

As we can see from the results in Table~\ref{sent:tbl:corr-coeff},
both criteria (topics and form) have a positive correlation with the
number of annotated elements and their reliability.  The highest
$\rho$-score for \markable{sentiment}s is achieved by the tweets
describing federal elections and messages containing polar terms,
where it amounts to 0.312 and 0.445 respectively.  A slightly
different situation is observed for \markable{polar term}s: the
highest scores for this element both in terms of the number of
annotated items and their reliability are achieved by casual everyday
conversations and tweets that contain emoticons.

%% In general, we can say that, to a greater or lesser extent, virtually
%% every annotation aspect of the resulting corpus except for the
%% sentiment agreement is correlated with the topic or form of the tweets
%% being labeled.  Regarding the uncorrelatednes of the sentiment
%% reliability scores, we hypothesize that the mere task of recognizing
%% subjective opinions in microblogs was so challenging to the human
%% experts (which was also confirmed by our initial annotation
%% experiments) that the difficulties with interpreting the guidelines
%% and applying their instructions to concrete cases have played a much
%% more crucial role and posed much bigger problems to the coders than
%% the particular specifics of topical and formal groups.

%% Based on these statistics, we can already say that the agreement
%% scores achieved in the last stage and the rest of the statistics
%% presented in this section suggest that our assistants could eventually
%% agree not only on the boundaries of sentiment elements but also on the
%% text spans of their respective sources and targets as well as
%% pertaining opinionated terms.  Furthermore, the results shown in
%% Table~\ref{tbl:attr-agrmnt} prove that the polarity and intensity
%% attributes of these elements could be reliably analyzed either.
%% Finally, the statistics plots on the distribution of subjective
%% opinions and polar terms show precisely how different sampling
%% criteria have affected the final composition and agreement scores of
%% the final corpus.

%% \section{Related Work}

%% As we already mentioned in the previous section, most of today's
%% resources for sentiment analysis on Twitter have been either created
%% fully automatically or obtained in a semi-supervised way.  In this
%% category also fall the popular datasets of \citet{Go:09},
%% \citet{Barbosa:10} as well as \citet{Pak:10}, whose message-polarity
%% labels were obtained either based on the emoticons contained in the
%% downloaded messages or taking the majority vote of third-party opinion
%% mining solutions.  Even though these corpora represent valuable
%% sources for developing new classifiers, the possibility of using these
%% data for training is only due to the large size of the collections
%% rather than the quality of their annotations.  Moreover, the
%% coarse-grained nature of the annotation schemes and the inherent
%% noisiness of the labels prevents the researchers from properly
%% analyzing linguistic phenomena in these collections.

%% A notable exception to these works constitutes the SemEval dataset of
%% \citet{Nakov:13}---a set of 15,000 microblogs which were manually
%% analyzed by human experts on the Amazon Mechanical Turk platform.
%% Even though this corpus is one of the biggest manually labeled tweet
%% collections to date, its data still have some potential drawbacks:
%% First of all, all messages included in this tweebank were prefiltered
%% through the SentiWordNet lexicon~\cite{Esuli:06b}, leaving out tweets
%% which did not contain any terms from this resource.  As we showed in
%% the evaluation part of this section, introducing such bias in the
%% sampled data might significantly affect the resulting distribution,
%% omitting many relevant phenomena of interest.  Furthermore, the
%% annotations of this dataset only provide information about the
%% polarity of single lexical items or complete microblogs, losing
%% valuable details about the targets, holders of opinions, and
%% compositionality effects of modifying elements.

%% This scarceness of suitable resources becomes even more aggravated
%% when speaking about Twitter corpora for non-English languages.  A few
%% attempts at creating opinion datasets for microblogs that we are aware
%% of are the sentiment subset of the TWITA corpus \cite{Basile:13} and
%% the Senti-TUT corpus of \citet{Bosco:13} created for Italian, as well
%% as the TASS shared task data \cite{Villena-Roman:13} developed for
%% Spanish.

%% The TWITA collection was used to create a smaller subset of 2,000
%% tweets, which were subsequently manually labeled with their
%% message-level polarities by three human coders.  The Senti-TUT
%% tweebank comprises 3,288 microblogs pertaining to the election of
%% Mario Monti and 1,159 messages obtained from the Twitter section of
%% the popular Italian web portal \url{http://www.spinoza.it}, which were
%% labeled by five annotators with the message-level polarity classes:
%% positive, negative, ironic, mixed, or none.  A different set of
%% sentiment valences (viz., strong negative, negative, neutral,
%% positive, and strong positive) was used for 70,000 messages of the
%% Spanish TASS corpus.

%% To the best of our knowledge, the only sentiment corpus for German
%% Twitter that existed at the time of creating our dataset was the
%% collection of 100,000~messages pertaining the German federal elections
%% 2009 gathered by \citet{Tumasjan:10}.  These tweets were automatically
%% translated into English and annotated with 12 psychological categories
%% (future orientation, past orientation, positive emotions, negative
%% emotions, sadness, anxiety, anger, etc.) using the LIWC program
%% \cite[Linguistic Inquiry and Word Count;][]{Pannebaker:07}.

%% In this regard, our presented corpus makes a valuable contribution to
%% the ecosystem of sentiment analysis data, not only filling the gap of
%% missing manual Twitter corpora with rich annotation schemes but also
%% providing a unique resource for developing new opinion mining systems
%% for German.  %% A detailed inter-annotator agreement study proesented
%% %% in this section proves that our experts could reliably annotate the
%% %% sentiment phenomena in question.  Furthermore, a detailed breakdown
%% %% of different topical and formal tweet categories unequivocally
%% %% shows which groups are more likely to contain evaluative subjective
%% %% judgements and which of these judgements might be more difficult to
%% %% agree on than the others.

\section{Summary and Conclusions}

Now that we have reached the end of the second chapter, we would like
to remind the reader that in this part of the thesis we have presented
the Potsdam Twitter Sentiment Corpus (PotTS), a collection of 7,992
German microblogs that had been manually annotated by two human
experts with sentiments, targets, sources, polar terms, and their
modifying elements.

We obtained the initial data for this corpus by tracking tweets about
German federal elections, papal conclave, discussions of general
political topics, and casual everyday conversations between spring and
autumn 2013.  Afterwards, we grouped these messages into three classes
(tweets containing polar terms, microblogs containing exclamation
marks or emoticons, and the rest of the messages) and randomly sampled
666 posts from each of these classes for each topic.

%% A rich annotation scheme, which we devised for our dataset, included
%% \markable{sentiment}s, \markable{target}s, \markable{opinion}s,
%% \markable{polar term}s, their \markable{intensifier}s,
%% \markable{diminisher}s, and \markable{negation}s.  With all these
%% elements, we associated extensive sets of attributes, such as
%% \attribute{polarity}, \attribute{intensity}, \attribute{sarcasm},
%% which also had to be specified by the experts during the annotation.

The annotation process was performed in three steps: first, the
annotators labeled one half of the data after minimal training; then,
we automatically highlighted their divergent analyses and asked them
to resolve these differences; finally, our assistants continued with
the analysis of the remaining files.

To estimate the inter-rater reliability, we introduced two modified
versions of the established $\kappa$-metric---binary and proportional
kappa---which differ in the way how they treat overlapping annotations
and partial matches.  Using these measures, we estimated the
inter-annotator agreement of our experts at different stages of their
work.  This study showed that, initially, our assistants could hardly
agree on the mere notion of targeted opinions, but their disagreements
could be resolved with the help of the adjudication procedure that we
applied in step two.  Despite a small drop of the IAA scores in the
final stage, all $\kappa$-values still remained at the level of at
least moderate reliability.

Finally, we demonstrated that our initial selection criteria had a
strong impact on the number and agreement of annotated sentiments and
polar terms, with tweets about federal elections and messages without
prefiltered topics being the most prolific sources of these elements.

That way, we not only contributed to the inventory of available
sentiment and social-media resources for German but also provided new
insights into different sampling methods that could be used to create
an opinion dataset and described the consequences of applying these
methods in practice.  A detailed inter-annotator agreement study
showed precisely which topics yield most subjective opinions
(elections and casual conversations) and which groups of messages are
especially difficult to annotate (tweets containing emoticons and
microblogs without polar terms or emoticons).  In the next step, we
are going to check whether our dataset can also serve as a basis for
building and evaluating automatic opinion mining applications.


% FILE: sentiment_lexica.tex  Version 0.01
% AUTHOR: Uladzimir Sidarenka

% This is a modified version of the file main.tex developed by the
% University Duisburg-Essen, Duisburg, AG Prof. Dr. Günter Törner
% Verena Gondek, Andy Braune, Henning Kerstan Fachbereich Mathematik
% Lotharstr. 65., 47057 Duisburg entstanden im Rahmen des
% DFG-Projektes DissOnlineTutor in Zusammenarbeit mit der
% Humboldt-Universitaet zu Berlin AG Elektronisches Publizieren Joanna
% Rycko und der DNB - Deutsche Nationalbibliothek

\section{Sentiment Lexica}\label{sec:snt:lex}

The first avenue that we are going to explore with the help of the
obtained corpus data is an automatic prediction of polar terms.
% To this end, we will first present an updated version of our dataset
% in Subsection~\ref{subsec:snt-lex:data} in which our experts revised
% the annotations of words and idioms that were present in the existing
% German sentiment lexica (GSL), but were not marked as emo-expressions
% in our data and, vice versa, were annotated as polar terms in the
% corpus, but absent in the analyzed polarity lists.
To this end, we will first evaluate the existing German sentiment
lexica (GSL) on our corpus data in order to obtain a baseline for the
subsequent experiments.  Since literally all of the current GSL were
created using an automatic translation of English opinion lists with a
manual post-editing of the translated entries, we will then look
whether the original methods that were used initially for creating the
English resources would yield better results when applied to German
data directly.  Finally, in the concluding step, we will analyze if
one of most popular areas of research in contemporary CL---distributed
vector representations of words \cite{Mikolov:13}---could be a more
perspective way for deriving new domain-specific polarity lists
without labeled data.  We will summarize and conclude in the last part
of this section, also discussing which of the presented methods
performed best on our corpus and explaining the reasons for this
success.

% \subsection{Data}\label{subsec:snt-lex:data}

% In the following experiments, we will use the updated version 0.1.0 of
% the sentiment dataset introduced in the previous section.  The main
% changes included in this version are:
% \begin{inparaenum}[\itshape a\upshape)]
%   \item the revision of the annotated emotional expressions, and
%   \item the addition of the boolean attributes \emph{subjective-fact}
%     and \emph{uncertain} to the annotation scheme of these elements.
% \end{inparaenum}

% To accomplish the fortmer objective, we compared the labelings of the
% opinionated terms in the first release of our corpus (version 0.0.1
% presented previously) with the entries from the existing German
% sentiment lexica: SentiWS \cite{Remus:10}, German Polarity Clues
% \cite{Waltinger:10}, and the Zurich Polarity List \cite{Clematide:10}.
% Similarly to the adjudication procedure used earlier, we automatically
% highlighted the differences between the corpus labels and the entries
% of these resources, letting our experts resolve the emerged conflicts.

% As it turned out, many of the contradicting cases stemmed from a
% different treatment of polar facts, such as ``Tod'' (\emph{death}),
% ``Anstieg'' (\emph{surge}), ``duften'' (\emph{to scent}) etc.: these
% entities were not labeled as emo-expressions in the full annotation of
% our dataset, but were still present as opinionated terms in the
% analyzed polarity lists.  Since these words represented a special
% class of polar clues due to their unequivocal objective nature, we
% decided to introduce a special feature called \emph{subjective-fact}
% into the attribute set of emotional expressions, explicitly asking our
% coders to annotate such terms with the emo-expression tag, but setting
% the value of the newly added attribute to \texttt{true}.

% The remaining differences were mainly due to polysemous words whose
% meaning in the corpus was not always subjective; errors in the lexica
% (GPC, for example, featured such auxiliary terms as ``sein'' (\emph{to
%   be}), ``wer'' (\emph{who}), or ``aus'' (\emph{from}) as opinionated
% entries); and, finally, numerous borderline cases which were difficult
% to resolve even in group discussions.  Examples of such challenging
% cases were words and idioms such as ``verbieten'' (\emph{to
%   prohibit}), ``verwechseln'' (\emph{to confuse}), ``Hauen und
% Stechen'' (\emph{hewing and stabbing}).  To address this issue, we
% again introduced an additional attribute, called \emph{uncertain},
% into the annotation scheme of emotional expressions and asked our
% experts to mark such borderline instances with this attribute.

% The statistics on the total number of the labeled elements and their
% agreement in the updated version are shown in
% Table~\ref{tbl:snt-lex:ucrp-agrmnt}:

% \begin{table*}[thb!]
%   \begin{center}
%     \bgroup \setlength\tabcolsep{0.7\tabcolsep} \scriptsize
%     \begin{tabular}{p{0.15\textwidth} % first columm
%         *{10}{>{\centering\arraybackslash}p{0.05\textwidth}}} % next ten columns
%       \toprule
%           \multirow{2}{0.2\textwidth}{\bfseries Element} &
%           \multicolumn{5}{c}{\bfseries Binary $\kappa$} & %
%           \multicolumn{5}{c}{\bfseries Proportional $\kappa$}\\
%           \cmidrule(r){2-6}\cmidrule(l){7-11}
%           & $M_1$ & $A_1$ & $M_2$ & $A_2$ & $\mathbf{\kappa}$ %
%           & $M_1$ & $A_1$ & $M_2$ & $A_2$ & $\mathbf{\kappa}$\\\midrule

%           EExpression &  &  &  & & \textbf{} & &  &  &  & \textbf{}\\
%           \bottomrule
%     \end{tabular}
%     \egroup
%   \end{center}
%   \captionof{table}{Inter-annotator agreement on the emotional
%     expression in the updated corpus.\\ {\small ($M1$ -- number of
%       tokens with matching labels in the first annotation, $A1$ --
%       total number of tokens labeled with that class in the first
%       annotation, $M2$ -- number of tokens with matching labels in the
%       second annotation, $A2$ -- total number of tokens labeled with
%       that class in the second annotation)}}
%   \label{tbl:snt-lex:ucrp-agrmnt}
% \end{table*}

\subsection{Evaluation of the Existing German Lexica}

In order to obtain a raw estimate of the expected scores for the
prediction of subjective expressions, we evaluated existing sentiment
lexica for German on the annotated data of our corpus.  The most
prominent of these resources are:
\begin{itemize}
\item the \textbf{German Polarity Clues} (GPC) list of
  \citet{Waltinger:10}, which comprises 10,141 subjective entries
  automatically translated from the English sentiment lexica
  \emph{Subjectivity Clues} \cite{Wilson:05} and \emph{SentiSpin}
  \cite{Takamura:05} with a subsequent manual rechecking of these
  translations and several synonyms and negated terms added by the
  authors;

\item the \textbf{SentiWS} (SWS) lexicon introduced by
  \citet{Remus:10}, which includes 1,818 positively and 1,650
  negatively connotated terms, also providing their part-of-speech
  tags and inflections (resulting in a total of 32,734 word forms).
  Similarly to the GPC, the authors used an English sentiment
  resource---the \emph{General Inquirer} list of \citet{Stone:66}---
  to bootstrap the entries for their lexicon, manually revising these
  automatic translations afterwards.  In addition to that,
  \citet{Remus:10} also expanded their polarity set with words and
  phrases frequently co-occurring with positive and negative seed
  lexemes using collocation information obtained from a corpus of
  10,200 customer reviews or extracted from the German Collocation
  Dictionary \cite{Quasthoff:10};

\item and, finally, the \textbf{Zurich Polarity List} developed by
  \citet{Clematide:10}, which comprises 8,000 subjective entries taken
  from GermaNet synsets \cite{Hamp:97}.  These synsets were manually
  annotated with their prior polarities by human experts.  Since the
  authors, however, found the number of polar adjectives obtained that
  way insufficient for running further classification experiments,
  they automatically enriched their lexicon with more attributive
  terms by analyzing conjoined collocations from a corpus using the
  method of \citet{Hatzivassi:97}.
\end{itemize}

In order to evaluate these resources on our dataset, we represented
each of the above lexicons as a trie \cite[pp. 492--512]{Knuth:98}
with the lexicon entries corresponding to paths in the resulting graph
and the polarity class(es) of these entries stored at the respective
terminal leaf nodes.  We then applied the standard match operation by
simultaneously checking the trie against contiguous runs of both
original and lemmatized corpus tokens.  All lemmatizations were done
using the \texttt{TreeTagger} \cite{Schmid:95}, and the match
operation was made case-insensitive.  Furthermore, since all analyzed
lexica only included plain textual terms, we also ignored smileys
during this comparison, solely concentrating on common lexical terms.

In this way, we estimated the precision, recall, and \F{}-score of the
positive, negative, and neutral polarity classes\footnote{We
  considered a term as neutral if it was not featured in the lexicon
  as either positive or negative entry.} of each lexicon by separately
computing these figures on each corpus file (99--109 tweets) and then
obtaining the mean and standard deviation of these scores over all
files in the dataset.  In addition to that, we also calculated the
macro- and micro-averaged results over all polarity classes.
Following the standard practice for computing these terms, we
estimated the macro-average as the mean of the \F{}-scores over all
three types (positive, negative, and neutral) and obtained the
micro-average by taking the harmonic mean of precision and recall
calculated over all true positives, false positives, and false
negatives found in the corpus.  The results of these computations are
shown in Table~\ref{snt-lex:tbl:gsl-res}.\footnote{For the sake of
  these experiments, we excluded the auxiliary words ``aus''
  (\emph{from}), ``der'' (\emph{the}), ``keine'' (\emph{no}),
  ``nicht'' (\emph{not}), ``sein'' (\emph{to be}), ``was''
  (\emph{what}), and ``wer'' (\emph{who}) with their inflection forms
  from the German Polarity Clues lexicon, since these entries
  significantly worsened the evaluation results.}

\begin{table}[h]
  \begin{center}
    \bgroup \setlength\tabcolsep{0.1\tabcolsep}\scriptsize
    \begin{tabular}{p{0.162\columnwidth} % first columm
        *{9}{>{\centering\arraybackslash}p{0.074\columnwidth}} % next nine columns
        *{2}{>{\centering\arraybackslash}p{0.068\columnwidth}}} % last two columns
      \toprule
          \multirow{2}*{\bfseries Lexicon} & %
          \multicolumn{3}{c}{\bfseries Positive Expressions} & %
          \multicolumn{3}{c}{\bfseries Negative Expressions} & %
          \multicolumn{3}{c}{\bfseries Neutral Terms} & %
          \multirow{2}{0.068\columnwidth}{\bfseries\centering Macro\newline \F{}} & %
          \multirow{2}{0.068\columnwidth}{\bfseries\centering Micro\newline \F{}}\\
          \cmidrule(lr){2-4}\cmidrule(lr){5-7}\cmidrule(lr){8-10}

          & Precision & Recall & \F{} & %
          Precision & Recall & \F{} & %
          Precision & Recall & \F{} & & \\\midrule
      %% \multicolumn{9}{|c|}{\cellcolor{cellcolor}Existing Lexica}\\\hline

      GPC & 24\stddev{7.9} & 46.4\stddev{11.2} & 31.1\stddev{8.2} & %
      22\stddev{8.1} & 41.6\stddev{10.6} & 28.1\stddev{8.3} & %
      98\stddev{0.5} & 94.3\stddev{1} & 96.1\stddev{0.5} & %
      51.8\stddev{4.7} & 92.3\stddev{0.9}\\

      SWS & 35.9\stddev{13.2} & 38.7\stddev{11.6} & 36\stddev{9.9} & %
      49\stddev{17.1} & 31.3\stddev{10.9} & \textbf{37.2}\stddev{11.6} & %
      97.5\stddev{0.5} & 97.8\stddev{1} & 97.7\stddev{0.5} & %
      \textbf{56.9}\stddev{6.2} & 95.4\stddev{0.9}\\

      ZPL & 36.4\stddev{12.3} & 21.3\stddev{7.2} & 26.3\stddev{8} & %
      41.1\stddev{15.6} & 23.3\stddev{8.7} & 29\stddev{10} & %
      97\stddev{0.6} & 98.7\stddev{0.3} & 97.8\stddev{0.3} & %
      51.1\stddev{4.4} & 95.7\stddev{0.6}\\

      GPC $\cap$ SWS $\cap$ ZPL & \textbf{54}\stddev{12.8} & %
      30.6\stddev{10.5} & \textbf{38.1}\stddev{10.1} & %

      \textbf{61.7}\stddev{17.8} & 21.6\stddev{9.1} & 31.2\stddev{11.3} & %
      97.1\stddev{0.6} & \textbf{99.3}\stddev{0.3} & \textbf{98.2}\stddev{0.3} & %
      55.8\stddev{5.4} & \textbf{96.4}\stddev{0.6}\\

      GPC $\cup$ SWS $\cup$ ZPL & 23.3\stddev{7.6} & \textbf{48.5}\stddev{11.2} & %
      30.9\stddev{8.2} & %

      22\stddev{8} & \textbf{46.1}\stddev{10.8} & 29.1\stddev{8.5} & %
      \textbf{98.1}\stddev{0.5} & 93.9\stddev{1} & 96\stddev{0.5} & %
      52\stddev{4.8} & 92\stddev{0.9}\\\bottomrule
    \end{tabular}
    \egroup
    \caption{Evaluation of the existing German sentiment lexica.\\
      {\small (GPC -- German Polarity Clues \cite{Waltinger:10}, SWS
        -- SentiWS \cite{Remus:10}, ZPL -- Zurich Polarity Lexicon
        \cite{Clematide:10})}}
    \label{snt-lex:tbl:gsl-res}
  \end{center}
\end{table}

As can be seen from the table, the intersection of all three lexica
attains the best results on both positive and neutral classes, also
yielding the best scores in terms of the micro-averaged $F$-measure.
One of the main reasons for this success is a relatively high
precision of this list for all but the neutral polarity class, where
it is outperformed by the union of the three resources.  Not
surpisingly, the union also shows the highest recall of positive and
negative terms among all compared polarity lists.

Regarding the figures attained by the individual lexica, the best
results here are achieved by the SentiWS resource \cite{Remus:10},
which not only shows the highest \F{}-score for the negative terms but
also achieves the best macro-averaged \F{}-result on all classes.

At the same time, we also can observe that the deviation of the scores
on different files is relatively high.  In order to see whether this
skewness of the distribution could significantly affect the net
statistics, we additionally recomputed all results on the whole corpus
at once.  The updated figures are shown in
Table~\ref{snt-lex:tbl:gsl-res-full}.

\begin{table}[h]
  \begin{center}
    \bgroup \setlength\tabcolsep{0.1\tabcolsep}\scriptsize
    \begin{tabular}{p{0.162\columnwidth} % first columm
        *{9}{>{\centering\arraybackslash}p{0.074\columnwidth}} % next nine columns
        *{2}{>{\centering\arraybackslash}p{0.068\columnwidth}}} % last two columns
      \toprule
          \multirow{2}*{\bfseries Lexicon} & %
          \multicolumn{3}{c}{\bfseries Positive Expressions} & %
          \multicolumn{3}{c}{\bfseries Negative Expressions} & %
          \multicolumn{3}{c}{\bfseries Neutral Terms} & %
          \multirow{2}{0.068\columnwidth}{\bfseries\centering Macro\newline \F{}} & %
          \multirow{2}{0.068\columnwidth}{\bfseries\centering Micro\newline \F{}}\\
          \cmidrule(lr){2-4}\cmidrule(lr){5-7}\cmidrule(lr){8-10}

          & Precision & Recall & \F{} & %
          Precision & Recall & \F{} & %
          Precision & Recall & \F{} & & \\\midrule
      %% \multicolumn{9}{|c|}{\cellcolor{cellcolor}Existing Lexica}\\\hline

      GPC & 23.83 & 46.8 & 31.58 & %
       22.37 & 41.58 & 29 & %
       98.01 & 94.27 & 96.1 & %
       52.26 & 91.32\\

      SWS & 34.36 & 39.23 & 36.63 & %
       49.56 & 31.45 & \textbf{38.48} & %
       97.5 & 97.84 & 97.67 & %
       \textbf{57.59} & 95.38\\

      ZPL & 36.4 & 21.22 & 26.81 & %
       41.57 & 23.14 & 29.73 & %
       96.96 & 98.73 & 97.84 & %
       51.46 & 95.67\\

      GPC $\cap$ SWS $\cap$ ZPL & \textbf{54.52} & 30.74 & \textbf{39.31} & %
       \textbf{62.92} & 21.68 & 32.25 & %
       97.13 & \textbf{99.25} & \textbf{98.18} & %
       56.58 & \textbf{96.39}\\

      GPC $\cup$ SWS $\cup$ ZPL & 23.11 & \textbf{48.86} & 31.38 & %
       22.42 & \textbf{46.05} & 30.16 & %
       \textbf{98.14} & 93.87 & 95.96 & %
       52.5 & 91.99\\\bottomrule
    \end{tabular}
    \egroup
    \caption{Evaluation of the existing German sentiment lexica on the
      complete corpus.\\ {\small (GPC -- German Polarity Clues
        \cite{Waltinger:10}, SWS -- SentiWS \cite{Remus:10}, ZPL --
        Zurich Polarity Lexicon \cite{Clematide:10})}}
    \label{snt-lex:tbl:gsl-res-full}
  \end{center}
\end{table}

As we can see from the scores, the relative placement of the
best-performing systems is the same as in the previous evaluation.
Moreover, the absolute results are only minimally higher (by typically
at most one percent) than the mean values computed in the previous
step.  Thus, even despite their high variance, the average figures
computed over single corpus files are still a reliable indicator of
the quality of the recognized opinionated terms.  For the sake of
brevity, we will therefore only present the former metric (the mean
and standard deviation of the scores estimated over individual files),
refraining from computing the statistics on the whole corpus in total.

\subsection{Evaluation of Dictionary-Based Approaches}

Since all of the presented works rely on a manual correction or
partial annotation of the lexicon entries, they clearly fall into the
category of semi-automatic approaches.  Unlike fully automated
methods, such systems typically yield more precise results, which,
however, come at the cost of tedious human efforts.  In order to see
to which extent these efforts really pay off in practice for the
lexicon generation (LG) task at hand, we additionally decided to
evaluate the most popular fully automatic LG approaches.

According to \citet[p. 79]{Liu:12}, most of such automated methods can
be divided into dictionary- and corpus-based ones.  The former
approaches try to derive polarity lists using monolingual thesauri or
lexical databases such as the Macquarie Dictionary \cite{Bernard:86}
or \textsc{WordNet} \cite{Miller:95}.  A clear advantage of these
systems is their relatively good precision as they operate on
carefully verified data enriched with hand-crafted meta information.
At the same time, this verification can become a drawback for the
recall in the domains where the language changes occur very rapidly,
and new terms are being coined in a flash.

The presumably first such approach was proposed by \citet{Hu:04}.  In
their work on sentiment classification and summarization of cutomer
reviews, the authors determined semantic orientation of adjectives
(which were supposed to be the most relevant part of speech for mining
people's opinions) by taking a set of seed terms with known polarities
and propagating these values to the synonyms of these words that were
found in \textsc{WordNet} \cite{Miller:95}.  A similar procedure was
applied to antonymous relations with the polarity orientation being
reversed during the propagation.  This expansion continued until no
more adjective could be reached via the synonymy-antonymy links.
Unfortunately, no intrinsic evaluation of the resulting lexicon was
performed in this work---the authors only report their results on
recognizing subjective sentences and classifying their polarity, where
they attain average \F-scores of 0.667 and 0.842 respectively.

Later on, this approach was refined by \citet{Blair-Goldensohn:08},
who obtained polarity labels for new words by multiplying a score
vector $v$ containing polarity scores of the known seed terms (-1 for
the negative expressions and 1 for the positive terms) with the
adjacency matrix $A$ constructed from the \textsc{WordNet} synsets.
In these experiments, the value of the adjacency cell $a_{ij}$ in the
matrix $A$ was set to $\lambda=0.2$ if there was a synonymy link
between the synsets $i$ and $j$ and to $-\lambda$ if these synsets
were antonymous to each other.  By performing this multiplication
multiple times and setting the $v$ vector to the result of the
previous iteration, the authors ensured that the polarity scores were
being propagated transitively through the network, decaying by a
constant factor ($\lambda$) with the increasing path length from the
original seeds.  This method again was evaluated only
extrinsically---the authors tested their complete sentiment
summarization system, which used the sentiment scores for individual
words as features for a maximum-entropy classifier.

With various modifications, the core idea of propagating the polarity
classes through the semantic graph was adopted in almost all of the
following dictionary-based works: \citet{Kim:04,Kim:06}, for instance,
used a similar method to determine the polarity of adjectives and
verbs given a small seed set of terms with known orientations.  In
particular, the likelihood of a new word $w$ belonging to the class $c
\in \{\textrm{postive}, \textrm{negative}, \textrm{neutral}\}$ was
computed as:
\begin{equation*}
  P(c|w) = \argmax_{c}P(c)P(w|c) = \argmax_{c}P(c)\frac{\sum\limits_{i=1}^{n}count(syn_i, c)}{count(c)},
\end{equation*}
where $P(c)$ is the prior probability of the polar class estimated as
the number of words with the given orientation $c$ divided by the
total number of terms considered, $count(syn_i, c)$ stands for the
number of times a seed term from class $c$ appears in a synset of $w$,
and $count(c)$ means the total number of synsets containing a seed
item.

Starting from a seed set of 34 adjectives and 44 verbs, the authors
successively expanded their lexicon to a list of 18,192 terms and
evaluated it on a manually labeled collection of 462 adjectives and
502 verbs taken from the TOEFL test and analyzed by two human experts.
The reported average accuracy for this method run up to 68.48\% for
adjectives and 74.28\% for verbs with their recall being equal to
93.07\% and 83.27\% respectively.  It should, however, be noted that
\citet{Kim:04} used a lenient metric for their computation by
considering neutral and positive terms as the same class which could
significantly boost the results.

% An alternative way of bootstrapping polarities for adjectives was
% proposed by \citet{Kamps:04}.  The authors estimated the orientation
% of the given term by computing the difference between the shortest
% path lengths of this word to the prototypic positive and negative
% lexemes---``good'' and ``bad''.  For example, the polarity score of
% the adjective ``honest'' was calculated as
% \begin{equation*}
%   POL(honest) = \frac{d(\textrm{honest}, \textrm{bad}) - d(\textrm{honest}, \textrm{good})}%
%   {d(\textrm{bad}, \textrm{good})} = \frac{6 - 2}{4} = 1,
% \end{equation*}
% where $d(w_1, w_2)$ means the geodesic (shortest-path) distance
% between the words $w_1$ and $w_2$ in the \textsc{WordNet} graph.  The
% respective orientation of this term was then correspondingly set to
% \texttt{positive} according to the sign of the obtained
% $POL$-value. \citet{Kamps:04} evaluated the accuracy of their method
% on the General Inquirer lexicon \cite{Stone:66} by comparing the terms
% with non-zero scores to the entries from this resource, getting
% 68.19\% of correct predictions on a set of 349 adjectives.

One of the most popular dictionary-based approaches to date, however,
was proposed by \citet{Esuli:06c}.  Starting with the positive and
negative seed sets used by \citet{Turney:03} and considering the rest
of the terms as objective if these words neither appeared in the
aforemention seed lists nor had a subjective tag in the General
Inquirer lexicon \cite{Stone:66}, the authors successively expanded
these polar sets for $k \in \{0, 2, 4, 6\}$ iterations by following
the synonymity-antonymity links similarly to the method proposed by
\citet{Hu:04}.\footnote{More precisely, following the work of
  \citet{Strapparava:04}, the authors propagated the same polarity
  values via the relations \emph{similarity}, \emph{derived-from},
  \emph{pertains-to}, and \emph{also-see}, and changed these values to
  the opposite when spreading them over the \emph{direct-antonymy}
  links.}  In each of these steps, they trained two types of ternary
classifiers---Rocchio and SVM---using tfidf-vectors of the training
glosses (whose amount was different in each iteration) as input
features.  \citet{Esuli:06c} experimented with two different ways of
obtaining such ternary predictors: using a committee of binary
classifiers and by directly training a ternary classification system,
finding that the latter option produced slightly better scores.  This
time, the evaluation was run on both the intersection with the
GI~lexicon~\cite{Stone:66} and a manually annotated subset of
\textsc{WordNet} synsets, yielding 66\% accuracy for the former
metric.\footnote{Note that different publications on
  \textsc{SentiWordNet} report different configuration settings,
  cf. \citet{Esuli:05}, \citet{Esuli:06a}, \citet{Esuli:06b}, and
  \citet{Esuli:06c}.  In our experiments, we will rely on the setup
  described in last paper as the most recent description of this
  approach.}

Finally, \citet{Rao:09} proposed another graph-based approach in which
they tried three different methods of assigning polarity scores to the
\textsc{WordNet} synsets:
\begin{inparaenum}[\itshape a\upshape)]
\item deterministic min-cut \cite{Blum:01},
\item randomized min-cut \cite{Blum:04}, and
\item the label propagation algorithm of \citet{Zhu:02}.
\end{inparaenum}
An evaluation of these systems on the GI lexicon \cite{Stone:66}
showed significant improvement over the previous baselines
\cite{Kamps:04,Kim:06}, with the deterministic min-cut yielding an
average \F{}-score of 0.833 over the three main parts of speech
(nouns, adjectives, and verbs) and the label propagation algorithm
reaching remarkable 0.916 average \F{} on nouns and verbs.\footnote{In
  order to obtain such high results, the authors had to use the
  hypernymy relation in addition to the normal synonymy links.  Since
  this relation was missing for adjectives, the results for this part
  of speech were significantly lower, only amounting to 0.73 \F{}.}

Other notable works on dictionary-based lexicon generation include
those of \citet{Mohammad:09}, who generated their seed set using
antonymous morphological patterns (e.g.,
\emph{logical}---\emph{illogical}, \emph{honest}---\emph{dishonest},
\emph{happy}---\emph{unhappy}) and subsequently expanded these seed
sets with the help of the Macquarie Thesaurus \cite{Bernard:86};
\citet{Awadallah:10}, who adopted a random walk approach, estimating
the polarity of an unknown word by taking the difference between an
average number of steps a random walker had to make in order to reach
a term from positive or negative set; and \citet{Dragut:10}, who
deduced the polarities of new words using manually specified inference
rules.

Since almost all of the presented approaches used \textsc{WordNet}---a
large lexical database with more than 117,000 synsets---and evaluated
their results in vitro (using the General Inquirer lexicon
\cite{Stone:66}), it remains unclear how these methods would work for
languages with smaller lexical resources and whether they would
perform equally well in vivo (when tested on a real-life corpus).
Moreover, because General Inquirer is a generic standard-language
dictionary, it is also not obvious whether the systems that perform
best on this list would be also applicable to more colloquial domains.

To answer these questions, we decided to reimplement the approaches of
\citet{Hu:04}, \citet{Blair-Goldensohn:08}, \citet{Kim:04,Kim:06},
\citet{Esuli:06c}, \citet{Rao:09}, and \citet{Awadallah:10}, applying
these methods to \textsc{GermaNet}---the German equivalent of the
English \textsc{WordNet} \cite{Hamp:97}\footnote{Throughout our
  experiments, we will use \textsc{WordNet} Version 3.0 and
  \textsc{GermaNet} Version 9.}--- and subsequently evaluating their
results on our presented Twitter corpus.

In order to make this comparison more fair, we used the same set of
the initial seed terms for all tested methods.  For this purpose, we
translated the original list of 14 subjectively connotated English
adjectives suggested by \citet{Turney:03}---\emph{good}$^+$,
\emph{nice}$^+$, \emph{excellent}$^+$, \emph{positive}$^+$,
\emph{fortunate}$^+$, \emph{correct}$^+$, \emph{superior}$^+$,
\emph{bad}$^-$, \emph{nasty}$^-$, \emph{poor}$^-$,
\emph{negative}$^-$, \emph{unfortunate}$^-$, \emph{wrong}$^-$, and
\emph{inferior}$^-$---into German, getting a total of 20 seeds (10
positive and 10 negative adjectives) due to multiple possible
translations of the same words.\footnote{The reimplemented methods and
  translated seed sets used in these experiments are available online
  at \url{https://github.com/WladimirSidorenko/SentiLex}.}
Furthermore, to settle the differences between the binary and ternary
approaches (i.e., those methods that only differentiate between the
positive and negative classes and those ones which also discern the
neutral terms as a separate group), we additionally enriched the
translated seed set with 10 purely objective adjectives---``neutral''
(\emph{neutral}), ``sachlich'' (\emph{objective}), ``technisch''
(\emph{technical}) ``chemisch'' (\emph{chemical}), ``physisch''
(\emph{physical}), ``materiell'' (\emph{material}), ``k\"orperlich''
(\emph{bodily}), ``finanziell'' (\emph{financial}), ``theoretisch''
(\emph{theoretical}), and ``praktisch'' (\emph{practical})---letting
all evaluated classifiers work in the ternary mode.  Finally, since
different methods relied on various notions of synonymous relations
(e.g., \citet{Hu:04} only considered two words as synonyms if they
appeared together in the same synset whereas \citet{Esuli:06c},
\citet{Rao:09}, and \citet{Awadallah:10} also considered
hyper-hyponymous connections as valid edges for propagating the
polarity of the seed terms), we decided to unify this aspect too,
letting all systems work with an extended set of links.  Into this
set, we not only included the traditional synonymity relations for the
terms co-occurring in the same synsets but also established edges
between words if their synsets were conected via the inter-synset
links \texttt{has\_participle}, \texttt{has\_pertainym},
\texttt{has\_hyponym}, \texttt{entails}, or
\texttt{is\_entailed\_by}.\footnote{For the method of
  \citet{Esuli:06c}, we only used the inter-synset links, dispensing
  with the intra-synset connections, as those were the only relations
  utilized in their original work.} We intentionally excluded the
relations \texttt{has\_hypernym} and \texttt{is\_related\_to} from
this set, since hypernyms were not guaranteed to preserve the polarity
of their children---e.g., ``bewertungsspezifisch''
(\emph{appraisal-specific}) is a neutral term in contrast to its
immediate hyponyms ``gut'' (\emph{good}) and ``schlecht''
(\emph{bad})---and the relatedness links (\texttt{is\_related\_to})
could connect both synonyms and antonyms of the same term---e.g., this
type of relation held between the words ``Form'' (\emph{shape}) and
``unf\"ormig'' (\emph{misshapen}).

We fine-tuned the settings of the LG systems that involved
hyper-parameters by using grid search and optimizing the
macro-averaged \F{}-score on a small subset of the data (four files)
from the other annotator.  In particular, instead of waiting for the
full convergence of the eigenvector in the approach of
\citet{Blair-Goldensohn:08}, we set the maximum number of times the
polarity vector was multiplied with the adjacency matrix to five.  Our
experiments showed that this limitation had a crucial impact on the
quality of the resulting polarity lists (e.g., after five
multiplications, the average precision of the recognized positive
terms amounted to 49.9\%, reaching an average \F{}-score of 26\% for
that class; after ten more iterations, however, this precision
decreased dramatically to 4.3\%, pulling the class-specific \F{}-score
down to 7.8\%).  Similarly to that, we limited the maximum number of
iterations in the label-propagation method of \citet{Rao:09} to 300,
although the effect of this setting was much weaker than in the
previous case (by comparison, the scores obtained after 30 runs
differed only by a couple percent).  Finally, in the method suggested
by \citet{Awadallah:10}, we allowed for seven simultaneous walkers
with a maximum of 17 steps each, considering a term as polar if more
than a half of these walkers agreed on the polarity of the analyzed
term.

The results of our re-implementations are shown in
Table~\ref{snt-lex:tbl:lex-res}.

\begin{table}[h]
  \begin{center}
    \bgroup \setlength\tabcolsep{0.1\tabcolsep}\scriptsize
    \begin{tabular}{p{0.1\columnwidth} % first columm
        *{9}{>{\centering\arraybackslash}p{0.078\columnwidth}} % next nine columns
        *{2}{>{\centering\arraybackslash}p{0.078\columnwidth}}} % last two columns
      \toprule
          \multirow{2}*{\bfseries Lexicon} & %
          \multicolumn{3}{c}{\bfseries Positive Expressions} & %
          \multicolumn{3}{c}{\bfseries Negative Expressions} & %
          \multicolumn{3}{c}{\bfseries Neutral Terms} & %
          \multirow{2}{0.068\columnwidth}{\bfseries\centering Macro\newline \F{}} & %
          \multirow{2}{0.068\columnwidth}{\bfseries\centering Micro\newline \F{}}\\
          \cmidrule(lr){2-4}\cmidrule(lr){5-7}\cmidrule(lr){8-10}

          & Precision & Recall & \F{} & %
          Precision & Recall & \F{} & %
          Precision & Recall & \F{} & & \\\midrule

          \textsc{Seeds Set} & \textbf{75.8}\stddev{23.8} & 9.1\stddev{5.7} & 15.8\stddev{8.8} & %
          24.4\stddev{41.8} & 1\stddev{1.8} & 1.9\stddev{3.4} & %
          96.3\stddev{0.8} & \textbf{99.9}\stddev{0.1} & \textbf{98.1}\stddev{0.4} & %
          38.6\stddev{3.2} & \textbf{96.2}\stddev{0.8}\\

          HL & 13.8\stddev{5.6} & 19.3\stddev{7} & 15.7\stddev{5.8} & %
          16.7\stddev{9.6} & 9.4\stddev{5.6} & \textbf{11.7}\stddev{6.5} & %
          96.7\stddev{0.7} & 96.5\stddev{0.6} & 96.6\stddev{0.5} & %
          41.3\stddev{3.3} & 93.3\stddev{0.9}\\

          BG & 49.9\stddev{14.5} & 18\stddev{6.6} & \textbf{26}\stddev{8.3} & %
          22.8\stddev{20.6} & 5.8\stddev{4.8} & 9\stddev{7.4} & %
          96.9\stddev{0.7} & 99.3\stddev{0.3} & 97.9\stddev{0.4} & %
          \textbf{44.3}\stddev{4} & 95.9\stddev{0.8}\\

          KH & 70.8\stddev{19.7} & 12.9\stddev{6.3} & 21.3\stddev{9.1} & %
          21.6\stddev{29.1} & 2.5\stddev{2.8} & 4.3\stddev{4.9} & %
          96.4\stddev{0.8} & 99.7\stddev{0.2} & 98\stddev{0.4} & %
          41.2\stddev{3.5} & 96.1\stddev{0.8}\\

          ES & 3.6\stddev{1.3} & \textbf{43.9}\stddev{10.8} & 6.6\stddev{2.3} & %
          2.6\stddev{1.5} & 18\stddev{6.9} & 4.5\stddev{2.3} & %
          \textbf{97.44}\stddev{0.6} & 71.5\stddev{2.8} & 82.5\stddev{1.8} & %
          31.2\stddev{1.6} & 66.8\stddev{2.8}\\

          RR$_{\textrm{mincut}}$ & 74.6\stddev{23.1} & 10.1\stddev{5.7} & 17.3\stddev{8.8} & %
          2.6\stddev{1.2} & \textbf{41.5}\stddev{10.2} & 4.8\stddev{2.1} & %
          97.4\stddev{0.6} & 72.5\stddev{2} & 83.1\stddev{1.3} & %
          35.1\stddev{3.2} & 70.6\stddev{1.8}\\

          RR$_{\textrm{lbl-prop}}$ & 55.6\stddev{18.2} & 14.6\stddev{6.2} & 22.7\stddev{8.4} & %
          \textbf{42.6}\stddev{41.5} & 2.7\stddev{3} & 5.1\stddev{5.5} & %
          96.5\stddev{0.7} & 99.7\stddev{0.2} & \textbf{98.1}\stddev{0.4} & %
          41.9\stddev{3.5} & \textbf{96.2}\stddev{0.8}\\

          AR & 75.5\stddev{25.3} & 9\stddev{5.6} & 15.6\stddev{8.7} & %
          25.3\stddev{42.1} & 1.1\stddev{2.1} & 2\stddev{3.9} & %
          96.3\stddev{0.8} & \textbf{99.9}\stddev{0.1} & \textbf{98.1}\stddev{0.4} & %
          38.6\stddev{3.2} & \textbf{96.2}\stddev{0.8}\\\bottomrule
    \end{tabular}
    \egroup
    \caption{Evaluation of dictionary-based approaches.\\ {\small (HL
        -- \citet{Hu:04}, BG -- \citet{Blair-Goldensohn:08}, KH --
        \citet{Kim:04,Kim:06}, ES -- \citet{Esuli:06c}, RR --
        \citet{Rao:09}, AR -- \citet{Awadallah:10})}}
    \label{snt-lex:tbl:lex-res}
  \end{center}
\end{table}

As can be seen from the table, the results of the purely automatic
approaches are significantly lower than the scores obtained by the
semi-automatic systems discussed previously.  The best macro-averaged
\F{}-score for the three classes (0.443) is attained by the method of
\citet{Blair-Goldensohn:08} which is still more than 0.12 points worse
than the peak result reached by the SentiWS lexicon (0.569).  Somewhat
surprisingly, no one of the compared systems could outperform the
micro-averaged baseline established by the initial seed set, even
though the label-propagation approach of \citet{Rao:09} and the
random-walk method of \citet{Awadallah:10} were on a par with this
result.

Regarding the figures for the positive and negative classes, we can
see the top-performing systems (\citet{Blair-Goldensohn:08} for the
positive terms and \citet{Hu:04} for the negative entries) achieve
these results mainly due to a relatively good balance of their
precision (P) and recall (R).  At the same time, approaches showing
the best scores for either of these two aspects (P or R) usually have
significant difficulties with the other metric: for example, the
manually selected set of the positive seed terms attains the highest
observed precision (75\%) for this polarity class, but suffers from a
rather low recall (9.1\%).  Vice versa, the system with the highest
recall---the SentiWordNet approach of \citet{Esuli:06c}---has a
dramatically low precision (3.6\%) on recognizing positive
expressions.  A similar situation can be observed for the negative
polarity class too where the best performing method---the system of
\citet{Hu:04}---has a well-maintained equilibrium of P and R in
contrast to the approaches with the best achieved precision or highest
recall---the label propagation and min-cut systems of \citet{Rao:09}
respectively---which only reach 2.6 to 2.7\% on the other evaluation
metric.

In general, however, the results shown by all systems tested in this
section are notably lower than the respective figures reported in
their original papers.  We can explain this divergence by the
following reasons:
\begin{inparaenum}[\itshape a\upshape)]
\item the evaluation metrics that we applied in our experiments are
  considerably different from the testing methods used in the previous
  works (we estimated the results on a real-life corpus, counting
  every false positive and false negative case, whereas the English
  approaches evaluated their results on the intersection with the
  General Inquirer Lexicon \cite{Stone:66}, omitting false positive
  matches and therefore artifically boosting their scores);
\item both the domain and the language that we addressed in this
  section are apparently more challenging than the standard English
  invariant for which these methods have been developed; and, finally,
\item the notion of the synonymous relations and the set of the
  initial seed terms could differ from the original settings of the
  evaluated approaches.
\end{inparaenum}

In order to test how the last factor (diverging settings) affected the
obtained results, we decided to re-generate the polarity lists with
the above approaches using a constrained set of synonymous links (this
time, we only considered two terms as synonyms if they both appeared
in at least one common synset).  The results of this experiment are
presented in Table~\ref{snt-lex:tbl:lex-res-constr-syn}.

\begin{table}[h]
  \begin{center}
    \bgroup \setlength\tabcolsep{0.1\tabcolsep}\scriptsize
    \begin{tabular}{p{0.1\columnwidth} % first columm
        *{9}{>{\centering\arraybackslash}p{0.078\columnwidth}} % next nine columns
        *{2}{>{\centering\arraybackslash}p{0.078\columnwidth}}} % last two columns
      \toprule
          \multirow{2}*{\bfseries Lexicon} & %
          \multicolumn{3}{c}{\bfseries Positive Expressions} & %
          \multicolumn{3}{c}{\bfseries Negative Expressions} & %
          \multicolumn{3}{c}{\bfseries Neutral Terms} & %
          \multirow{2}{0.068\columnwidth}{\bfseries\centering Macro\newline \F{}} & %
          \multirow{2}{0.068\columnwidth}{\bfseries\centering Micro\newline \F{}}\\
          \cmidrule(lr){2-4}\cmidrule(lr){5-7}\cmidrule(lr){8-10}

          & Precision & Recall & \F{} & %
          Precision & Recall & \F{} & %
          Precision & Recall & \F{} & & \\\midrule

         \textsc{Seeds Set} & \textbf{75.8}\stddev{23.8} & 9.1\stddev{5.7} & 15.8\stddev{8.8} & %
          24.4\stddev{41.8} & 1\stddev{1.8} & 1.9\stddev{3.4} & %
          96.3\stddev{0.8} & \textbf{99.9}\stddev{0.1} & \textbf{98.1}\stddev{0.4} & %
          38.6\stddev{3.2} & \textbf{96.2}\stddev{0.8}\\

          HL & 73.66\stddev{23.9} & 9.2\stddev{5.6} & 15.6\stddev{8.7} & %
          \textbf{33.1}\stddev{44.5} & 1.7\stddev{2.5} & 3.2\stddev{4.7} & %
          96.3\stddev{0.8} & \textbf{99.9}\stddev{0.1} & \textbf{98.1}\stddev{0.4} & %
          39.1\stddev{3.5} & \textbf{96.2}\stddev{0.8}\\

          BG & 75.4\stddev{21.9} & 9.7\stddev{5.6} & \textbf{16.7}\stddev{8.6} & %
          32.2\stddev{41.1} & \textbf{2}\stddev{2.6} & \textbf{3.7}\stddev{4.8} & %
          96.3\stddev{0.8} & \textbf{99.9}\stddev{0.1} & \textbf{98.1}\stddev{0.4} & %
          \textbf{39.5}\stddev{3.4} & \textbf{96.2}\stddev{0.8}\\

          KH & \textbf{75.8}\stddev{23.8} & 9.1\stddev{5.7} & 15.8\stddev{8.8} & %
          25\stddev{41.8} & 1\stddev{1.9} & 2\stddev{3.5} & %
          96.3\stddev{0.8} & \textbf{99.9}\stddev{0.1} & \textbf{98.1}\stddev{0.4} & %
          38.6\stddev{3.2} & \textbf{96.2}\stddev{0.8}\\

          ES & 3.9\stddev{1.3} & \textbf{51.9}\stddev{10.7} & 7.3\stddev{2.3} & %
          13\stddev{20.5} & 1.8\stddev{2.6} & 3\stddev{4.4} & %
          \textbf{97.6}\stddev{0.5} & 71.7\stddev{2.7} & 82.6\stddev{1.8} & %
          31\stddev{1.9} & 70\stddev{2.5}\\

          RR$_{\textrm{mincut}}$ & \textbf{75.8}\stddev{23.8} & 9.1\stddev{5.7} & 15.8\stddev{8.8} & %
          32.3\stddev{44.4} & 1.4\stddev{2.1} & 2.7\stddev{3.9} & %
          96.3\stddev{0.8} & \textbf{99.9}\stddev{0.1} & \textbf{98.1}\stddev{0.4} & %
          38.9\stddev{3.4} & \textbf{96.2}\stddev{0.8}\\

          RR$_{\textrm{lbl-prop}}$ & \textbf{75.8}\stddev{23.8} & 9.1\stddev{5.7} & 15.8\stddev{8.8} & %
          32.3\stddev{44.4} & 1.4\stddev{2.1} & 2.7\stddev{3.9} & %
          96.3\stddev{0.8} & \textbf{99.9}\stddev{0.1} & \textbf{98.1}\stddev{0.4} & %
          38.9\stddev{3.4} & \textbf{96.2}\stddev{0.8}\\

          AR & 72.9\stddev{27} & 7.3\stddev{4.7} & 12.9\stddev{7.6} & %
          23.8\stddev{41.1} & 1\stddev{1.9} & 1.9\stddev{3.5} & %
          96.2\stddev{0.8} & \textbf{99.9}\stddev{0.1} & \textbf{98.1}\stddev{0.4} & %
          37.6\stddev{2.8} & \textbf{96.2}\stddev{0.8}\\
          \bottomrule
    \end{tabular}
    \egroup
    \caption{Evaluation of dictionary-based approaches with a
      constrained set of synonymous links.\\ {\small (HL --
        \citet{Hu:04}, BG -- \citet{Blair-Goldensohn:08}, KH --
        \citet{Kim:04,Kim:06}, ES -- \citet{Esuli:06c}, RR --
        \citet{Rao:09}, AR -- \citet{Awadallah:10})}}
    \label{snt-lex:tbl:lex-res-constr-syn}
  \end{center}
\end{table}

This time, the results are spread much more uniformly: in particular,
seven of the eight tested models including the seed set attain
identical \F{}-scores on the neutral terms and get almost the same (up
to a few hundredths) micro-averaged results.  The only exception from
this tendency is formed by the method of \citet{Esuli:06c} whose
original implementation only used inter-synset links and, as we can
see from the results, did not generalize well to the purely
intra-synset connections.  Nevertheless, this system reached the
highest observed precision on recognizing the neutral terms and also
attained the best recall of the positively connotated expressions.
The best observed precision for the latter class is, again, attained
by the manually compiled seed set, with the method of
\citet{Kim:04,Kim:06} and the two proposed systems of
\citet{Rao:09}---the min-cut and label propagation
algorithms---showing identical results.  Similarly to the previous
experiments, the highest overall \F{}-score for the positive category
(0.167 \F{}) is shown by the approach of \citet{Blair-Goldensohn:08},
which, this time, also performs best on predicting the negative
expressions, getting the best recall and \F{}-result for this polarity
group.

In general, however, the results obtained after constraining the set
of the possible synonymity links are significantly lower than the ones
achieved with the extended relation set.  This effect can partially be
explained by an extreme sparsity of the resulting \textsc{GermaNet}
graph which prevents graphical and matrix-based approaches from
propagating the polarity scores to other nodes.  To check this
hypothesis, we computed the mean in-degrees of the graph nodes with
the extended and constrained sets of relations, getting an average
number of 5.91 incident links per node for the former case and 1.86
edges per node for the latter setting.

Another important factor that could significantly affect the quality
of the resulting polarity lists was the set of the seed terms which we
used to initialize the polarity scores in the evaluated methods.  In
order to estimate the impact of this setting on the final scores, we
re-run our experiments with the extended set of synonymity links using
the seed lists proposed by \citet{Hu:04}, \citet{Kim:04},
\citet{Esuli:06c}, and \citet{Remus:10}.  Since \citet{Hu:04},
however, only partially specified their polarity set in the original
paper, and \citet{Kim:04} did not provide any examples at all, we
filled the missing terms in these lists with common polar German terms
we came up with to match the size of these seed sets.  Furthermore,
since the list of the neutral terms from the polarity set of
\citet{Esuli:06c} comprised 4,122 words---the authors considered as
neutral all terms from the General Inquirer lexicon \cite{Stone:66}
which were not marked as either positive or negative and did not
appear in the seed list of \citet{Turney:03}---and was therefore
cumbersome to translate manually, we automatically obtained
translations for these terms using the publicly available online
dictionary \texttt{dict.cc}\footnote{\url{http://www.dict.cc}}, taking
the first suggested German translation for each of the neutral
entries.\footnote{We also tried to use all possible translations of
  the original terms, but it lead to a considerable boost in the
  number of neutral items (45,252 words) and resulted in a substantial
  decrease of the final system scores.} A short statistics on the
cardinality and composition of the resulting seed sets is presented in
Table~\ref{snt-lex:tbl:alt-seed-sets}.

\begin{table}[h]
  \begin{center}
    \bgroup \setlength\tabcolsep{0.1\tabcolsep}\scriptsize
    \begin{tabular}{ %
        >{\centering\arraybackslash}p{0.2\columnwidth} % first columm
        *{4}{>{\centering\arraybackslash}p{0.2\columnwidth}}} % next four columns
      \toprule
      {\bfseries Seed Set} & %
      {\bfseries Cardinality} & %
      {\bfseries Parts of Speech} & %
      {\bfseries Examples} & %
      {\bfseries Comments}\\
      \midrule
      \citet{Hu:04} & 14 positive, 15 negative, and 10 neutral terms & adjectives %
      & {{\itshape fantastisch, lieb, sympathisch, %
          b\"ose, dumm, schwierig}} & polar terms translated from the original paper %
      \cite{Hu:04}; neutral terms added by us;\\
      \citet{Kim:04} & 60 positive, 60 negative, and 60 neutral terms & any & %
      {\itshape fabelhaft, Hoffnung, lieben, h\"asslich, Missbrauch, t\"oten} %
      & devised by us so as to match the cardinality of the original set with %
      neutral terms added extra;\\
      \citet{Esuli:06c} & 16 positive, 35 negative, and 4,122 neutral terms & %
      any & {\itshape angenehm, ausgezeichnet, freundlich, %
        arm, bedauernswert, d\"urftig} & polar terms translated from the seed %
      set of \citet{Turney:03}; neutral terms automatically translated from %
      objective entries in the General Inquirer lexicon \cite{Stone:66};\\
      \citet{Remus:10} & 12 positive, 12 negative, and 10 neutral terms & %
      adjectives & {\itshape gut, sch\"on, richtig, %
        schlecht, unsch\"on, falsch} & %
      polar terms translated from the seed set of \citet{Turney:03}; %
      neutral terms added by us.\\
      \\\bottomrule
    \end{tabular}
    \egroup
    \caption{Cardinality and composition of the alternative polarity lists.\\
      (all cardinalities are given with respect to the resulting
      German translations)}
    \label{snt-lex:tbl:alt-seed-sets}
  \end{center}
\end{table}

The results of the lexicons generated with the help of these seed sets
are shown in Figure~\ref{snt:fig:sent-lex-alt-seeds}.  As we can see
from the statistics, the choice of the initial seed terms might
significantly affect the overall scores of the resulting polarity
lists, with the macro-averaged \F{}-results deviating by up to 10\%
depending on the seed set being applied at the beginning---this is for
example the case for the approach of \citet{Esuli:06c} whose
macro-averaged \F{} runs up to $31.45^{\pm 0.19}$\% when generated
with the initial polar terms of \citet{Turney:03}, \citet{Hu:04},
\citet{Kim:04}, or \citet{Remus:10}, but achieves remarkable $41.7$\%
when used with the original seeds of the authors.

In general, however, the best scores for all compared classifieres are
achieved with either of two seed sets which by themselves (without any
expansion) attain the best macro-averaged \F{}-results.  These are the
seed list of \citet{Kim:04}, which yields the best scores for the
approaches of \citet{Blair-Goldensohn:08}, \citet{Kim:04,Kim:06},
\citet{Rao:09}, and \citet{Awadallah:10}, and the initial seed set
proposed by \citet{Esuli:06c}, which leads to the best results when
used in combination with the method of \citet{Hu:04} and the original
system of \citet{Esuli:06c}.

\begin{figure}[hbtp!]
{
  \centering
  \includegraphics[width=\linewidth]{img/sentilex-alt-seed-sets.png}
}
\caption{Macro-averaged \F{}-scores of dictionary-based methods with
  different seed sets.\\ {\small (HL -- \citet{Hu:04}, BG --
    \citet{Blair-Goldensohn:08}, KH -- \citet{Kim:04,Kim:06}, ES --
    \citet{Esuli:06c}, RR -- \citet{Rao:09}, AR --
    \citet{Awadallah:10})}}\label{snt:fig:sent-lex-alt-seeds}
\end{figure}

For convenience, we will summarize the results of three top-performing
configurations---the methods of \citet{Blair-Goldensohn:08},
\citet{Kim:04,Kim:06}, and the label-propagation approach of
\citet{Rao:09} used in combination with the initial seed sed of
\citet{Kim:04}---in Table~\ref{snt-lex:tbl:lex-kh-seedset} and use
these results as baselines in our later experiments.

\begin{table}[h]
  \begin{center}
    \bgroup \setlength\tabcolsep{0.1\tabcolsep}\scriptsize
    \begin{tabular}{p{0.1\columnwidth} % first columm
        *{9}{>{\centering\arraybackslash}p{0.078\columnwidth}} % next nine columns
        *{2}{>{\centering\arraybackslash}p{0.078\columnwidth}}} % last two columns
      \toprule
          \multirow{2}*{\bfseries Lexicon} & %
          \multicolumn{3}{c}{\bfseries Positive Expressions} & %
          \multicolumn{3}{c}{\bfseries Negative Expressions} & %
          \multicolumn{3}{c}{\bfseries Neutral Terms} & %
          \multirow{2}{0.068\columnwidth}{\bfseries\centering Macro\newline \F{}} & %
          \multirow{2}{0.068\columnwidth}{\bfseries\centering Micro\newline \F{}}\\
          \cmidrule(lr){2-4}\cmidrule(lr){5-7}\cmidrule(lr){8-10}

          & Precision & Recall & \F{} & %
          Precision & Recall & \F{} & %
          Precision & Recall & \F{} & & \\\midrule

          BG & 22.7\stddev{6.5} & \textbf{29.9}\stddev{8.4} & 25.4\stddev{6.7} & %
          17.1\stddev{7.4} & \textbf{16.9}\stddev{7.2} & \textbf{16.6}\stddev{6.7} & %
          \textbf{97}\stddev{0.6} & 96.4\stddev{0.6} & 96.7\stddev{0.5} & %
          46.3\stddev{3.5} & 93.5\stddev{0.9}\\

          KH & 51.6\stddev{12.4} & 21.7\stddev{7.2} & \textbf{29.9}\stddev{8.1} & %
          38.9\stddev{24.7} & 7.5\stddev{4.9} & 12.2\stddev{7.8} & %
          96.7\stddev{0.7} & \textbf{99.4}\stddev{0.2} & \textbf{98}\stddev{0.4} & %
          \textbf{46.7}\stddev{3.9} & \textbf{96}\stddev{0.7}\\

          RR$_{\textrm{lbl-prop}}$ & \textbf{52.9}\stddev{15.2} & 18.4\stddev{6.5} & 26.7\stddev{8.2} & %
          \textbf{42.8}\stddev{29.7} & 8.9\stddev{5.8} & 14.1\stddev{8.8} & %
          96.6\stddev{0.7} & \textbf{99.4}\stddev{0.3} & \textbf{98}\stddev{0.4} & %
          46.3\stddev{4.3} & \textbf{96}\stddev{0.8}\\
          \bottomrule
    \end{tabular}
    \egroup
    \caption{Results of the top-scoring dictionary-based approaches
      with the best observed seed set configuration. {\small (BG --
        \citet{Blair-Goldensohn:08}, KH -- \citet{Kim:04,Kim:06}, RR
        -- \citet{Rao:09})}}
    \label{snt-lex:tbl:lex-kh-seedset}
  \end{center}
\end{table}

% Seed Sets:

% Hu-Liu were using 30 adjectives, but they only provided some
% examples: great, fantastic, nice, cool, bad, and dull

% Blair-Goldensohn: do not report (In our experiments, the original
% seed set contained 20 negative and 47 positive words that were
% selected by hand to maximize domain coverage, as well as 293 neutral
% words that largely consist of stop words.)

% Kim-Hovy (2004): To start the seed lists we selected verbs (23
% positive and 21 negative) and adjectives (15 positive and 19
% negative), adding nouns later.  But they, again, do not report
% specific examples.

% Kim-Hovy (2006): We described a word classification system to de-
% tect opinion-bearing words in Section 2.1. To ex- amine its
% effectiveness, we annotated 2011 verbs and 1860 adjectives, which
% served as a gold stan- dard 7 . These words were randomly selected
% from a collection of 8011 English verbs and 19748 English
% adjectives. We use training data as seed words for the WordNet
% expansion part of our algorithm.

% Esuli/Sebastiani: Lp and Ln are two small sets, which we have
% defined by manually selecting the intended synsets4 for 14
% "paradigmatic" Positive and Negative terms (e.g., the Positive term
% nice, the Negative term nasty) which were used as seed terms in
% (Turney and Littman, 2003).  The Lo set is treated differently from
% Lp and Ln, because of the inherently "complementary" nature of the
% Objective category (an Objective term can be defined as a term that
% does not have either Positive or Negative characteristics). We have
% heuristically defined Lo as the set of synsets that (a) do not
% belong to either T rK p or T rK n , and (b) contain terms not marked
% as either Positive or Negative in the General Inquirer lexicon
% (Stone et al., 1966); this lexicon was chosen since it is, to our
% knowledge, the largest manually annotated lexicon in which terms are
% tagged according to the Positive or Negative categories.

% Rao-Ravichandran: All experiments reported in Sections 4.1 to 4.5
% use the data described above with a 50-50 split so that the first half
% is used as seeds and the sec- ond half is used for test.

% Awdallah: After (Turney, 2002), we use our method to predict
% semantic orientation of words in the General Inquirer lexicon (Stone
% et al., 1966) using only 14 seed words.

% seed sets: (Turney and Littman, 2003); SentiWS (Remus, 2010)

\subsection{Evaluation of Corpus-Based Approaches}

An alternative way to generate polarity lists represent corpus-based
methods.  In contrast to dictionary-based approaches, these systems
typically do not require any large-scale hand-crafted linguistic
thesauri and draw their information directly from (untagged) corpus
data, harnessing the co-occurrence statistics of the initial seed
terms.  A clear advantage of such methods are their virtual
independence of any (expensive) language resources and the ability to
keep pace with frequent changes in the targeted discourse domain.
These benefits, however, come at the cost of a reduced quality as
these systems operate almost completely unsupervised.

A pioneering work on these methods was done by \citet{Hatzivassi:97}.
Based on the assumption that coordinatively connected adjectives
typically share the same semantic orientation, the authors trained a
supervised logisitic regression classifier, which predicted the degree
of dissimilarity between two conjoined adjectival terms.  In the next
step, they constructed a word graph, drawing a link between any two
adjectives if they appeared in the same coordination pair and taking
the predicted dissimilarity scores as edge weights in the constructed
graph.  In the final step, \citet{Hatzivassi:97} partitioned the final
graph into two parts and ascribed positive polarity to the cluster
which had more nodes.  This method achieved an overall accuracy of
82.05\% on predicting the polarity of a subset of manually annotated
adjectives when trained on the rest of these hand-labeled data.

Later on, this approach was further refined by \citet{Takamura:05},
who tried to unite dictionary- and corpus-based methods into a unified
probabilistic framework.  To this end, the authors adopted the Ising
spin model from the statistical mechanics, considering terms from
\textsc{WordNet} \cite{Miller:95}, the Wall Street Journal and Brown
corpora as electrons in a single ferromagnetic lattice.  A link was
established between any two electrons, if their corresponding terms
appeared in synonymically connected synsets or a coordinatively
conjoined pair in any of the two corpora.  Taking into account the a
priori known polarities of the seed terms, the Ising model then tried
to find an approximation of the most likely polarity combination of
all terms in the graph over all possible polarity assignments,
reaching 91.5\% accuracy at predicting polarity of the manually
labeled subjective terms from the General Inquirer lexicon
\cite{Stone:66}.

An alternative way of inducing sentiment lexicons was proposed by
\citet{Turney:03}, who derived a set of polarity terms by computing
the ratio of PMI-associations between the potential items and
predefined sets of positive and negative seeds.  In particular, the
semantic orientation score of the given word $w$ was defined as:
\begin{equation*}
  \textrm{SO-A}(w) = \sum_{w_p\in\mathcal{P}}PMI(w, w_p) - \sum_{w_n\in\mathcal{N}}PMI(w, w_n),
\end{equation*}
where $\mathcal{P}$ was the set of the positive seed terms (presented
earlier in this sections), $\mathcal{N}$ denoted the collection of the
negative words, and the $PMI$ score was normally computed as the
log-ratio $PMI(w, w_x) = \log_2\frac{p(w, w_x)}{p(w)p(w_x)}$.  The
joint probability $p(w, w_x)$ was calculated with the help of the
AltaVista\footnote{\url{www.altavista.com}} NEAR operator as the
number of hits returned by the $w NEAR w_x$ query divided by the total
number of indexed documents.

With great success, this method was later adapted to the Twitter
domain by \citet{Kiritchenko:14}.  Using the distantly supervised
corpus of \citet{Go:09} and a separate collection of 775,000 tweets
gathered solely for the purpose of their experiments, the authors
created two sentiment lexicons---the Sentiment140 Base Lexicon and
Hashtag Sentiment Base Lexicon respectively---using the semantic
orientation formula described above.  However, instead of querying
AltaVista for computing the joint probability scores, the authors
estimated these values as the number of times a potential candidate
appeared in a positive (negative) tweet divided by the total number of
tweets in the collection.  The polarity classes of the tweets were
distantly assigned based on the occurrence of predefined emoticons in
the \citet{Go:09} corpus or emotional hashtags in the compiled tweet
collection.


In order to see which of the above paradigms bears more potential for
generating high-quality polarity lists for Twitter---the
dictionary-based methods with a supposedly higher precision or the
corpus-based approaches with an allegedly better recall---we
re-implemented the most popular systems from either groups and
evaluated these systems on our corpus.  The results of these
computations are presented in the following sections.

Among the first who addressed the problem of the automatic generation
of sentiment lexica were \citet{Hatzivassi:97}.  In their work, the
authors relied on the hypothesis that coordinatively conjoined terms
often share the same semantic orientation while adversatively linked
words rather express opposite polarities.  To test this conjecture,
they automatically extracted all pairs of conjoined adjectives from
the Wall Street Journal (WSJ) corpus and represented those adjectives
as nodes in a graph.  The arcs weights of this graph were to show the
strength and direction by which two coordinatively conjoined terms
influenced each others' polarity.  To derive these weights,
\citeauthor{Hatzivassi:97} trained a log-linear regression model on
those pairs of terms in which both nodes belonged to a manually
labeled seed set of 1,336 adjectives and then let this model predict
the weights for the rest of the arcs.  In the final step, the
resulting graph was partitioned into two clusters -- that of positive
and that of negative terms -- which were subsequently used to enrich
the initial seed set.
%% This method, enhanced by the possibility of recognizing gradable
%% adjectives, was later used in the classification experiments of
%% \citet{Hatzivassi:00} to predict subjective and objective sentences in
%% the WSJ.

A different way of bootstrapping polarity terms from large text
corpora was proposed by \citet{Turney:03}.  Following
\citeauthor{Turney:02}'s original approach for classifying reviews
\citep{Turney:02}, the authors generated a polarity lexicon by first
taking a seed set of 14 a priory known polar adjectives (seven
negative and seven positive ones) and then expading this set with the
words that had the strongest pointwise mutual information associations
with the chosen seeds.  The PMI scores were computed as the log ratio
between the number of times a new word $w$ appeared with any of the
seed terms, divided by the total number of search hits for the
complete seed set.  As search hits, the authors considered the number
of relevant documents returned by the \texttt{AltaVista} search engine
for the given queries.  This system attained an accuracy of 82.84\% on
the General Inquirer Lexicon \citep{Stone:66} and correctly predicted
polar terms from the lexicon of \cite{Hatzivassi:97} in 87.13\% of the
cases. %% This method could also be further improved by using cosine
%% similarities between word vectors from an LSA matrix instead of
%% web-based PMIs.

Other notable works on corpus-based lexicon induction include
\citet{Kanayama:06}, who enhanced the method of
\citeauthor{Hatzivassi:97} by incorporating iter-sentential coherence
relations.  \citet{Kaji:07} also followed a corpus-based approach as
they mined opinionated sentences from HTML pages using structural and
linguistic clues and then extracted new polar terms from these
sentences, considering words having the highest PMI association scores
with the rest of the sentences as subjective.

%% More precisely, the social orientation PMI (SO-PMI) of a new word
%% $w$ was defined as: \begin{equation*} \text{SO-PMI}(w) =
%% \log_2\bigg(\frac{\prod_{p\in\mathcal{P}}\text{hits}(w\text{NEAR}p)%
%% \prod_{n\in\mathcal{N}}\text{hits}(w\text{NEAR}n)}%
%% {\prod_{p\in\mathcal{P}}\text{hits}(p)\prod_{n\in\mathcal{N}}\text{hits}(n)}\bigg) \end{equation*}
%% where $\mathcal{P}$ and $\mathcal{N}$ are the seed sets of positive
%% and negative adjectives respectively and
%% $\text{hits}(\cdot\text{NEAR}\cdot)$ is the number

%% The authors evaluated the accuracy of their model on the General
%% Inquirer lexicon \cite{Stone:66}.  Its final results (81.9\%) were
%% comparable to the figures obtained by \citet{Turney:03} in their
%% method (82.84\%) and significantly outperformed the precision of
%% the approach proposed by \citet{Kamps:04} (73.4 versus 70.8).

\citet{Takamura:05} attempted to unite corpus- and dictionary-based
approaches into a single framework.  To this end, the authors adopted
the Ising spin model from the statistical mechanics and represented
all words found in \textsc{WordNet}, in the Wall Street Journal, and
the Brown corpus as nodes in a graph.  The edges of this graph
represented associativity links and were established between any two
words, if one of these word was a synonym, an antonym, or a hyponym of
the other one or appeared in its gloss or context.  Taking into
account the a priori known polarities of some of the terms, the Ising
model then tried to find an approximation of the most likely polarity
combination of all terms in the graph over all possible polarity
assignments.

Another popular alternative to the Ontology-based methods is lexicon
induction on the basis of actual corpus data.  Considering that
Twitter vocabulary is typically very different from the entries that
are usually included into standard-language dictionaries, applying
this strategy directly to tweets might potentially significantly
outperform the results of both translated resources and polar term
lists generated from GermaNet.

To check this hypothesis, we have re-implemented the Ising Spin system
of \citet{Takamura:05} -- one of the arguably most competitive methods
for unsupervised lexicon induction -- and applied it to the German
Twitter snapshot of \cite{Scheffler:14}.

The results of this method are shown in Table~\ref{snt-lex:tbl:ispn-res}.

\begin{table}[h]
  \begin{center}
    \bgroup \setlength\tabcolsep{0.1\tabcolsep}\scriptsize \small
    \begin{tabular}{|p{0.21\columnwidth}| % first columm
        *{8}{>{\centering\arraybackslash}m{0.1\columnwidth}|}} % next nine columns
      \hline
          \multirow{2}*{\bfseries Element} & \multicolumn{3}{c|}{Positive
        Expressions} & %
      \multicolumn{3}{c|}{Negative Expressions} & %
      \multirow{2}*{Macro-\F{}} & %
      \multirow{2}*{Micro-\F{}}\\\cline{2-7}

      & Precision & Recall & \F{} & Precision & Recall & \F{} & & \\\hline
      \multicolumn{9}{|c|}{\cellcolor{cellcolor}Existing Lexica}\\\hline

      Ising Spin Model & \stddev{} & \stddev{} & \stddev{} & \stddev{}
      & \stddev{} & \stddev{} & \stddev{} & \stddev{}\\\hline

      \multicolumn{9}{|c|}{\cellcolor{cellcolor}Our Method}\\\hline
    \end{tabular}
    \egroup
    \caption{Classification results.\\ {\small (GPC -- German Polarity
        Clues \cite{Waltinger:10}, SWS -- SentiWS \cite{Remus:10}, ZPL
        -- Zurich Polarity Lexicon \cite{Clematide:10})}}
    \label{snt-lex:tbl:ispn-res}
  \end{center}
\end{table}

\subsection{Lexicon Generation Using Neural Word Embeddings}

A new family of lexicon induction methods builds on learned vector
representations of words -- the neural word embeddings
\cite{Mikolov:13}.

The results of this method are shown in Table~\ref{snt-lex:tbl:w2v}.

\begin{table}[h]
  \begin{center}
    \bgroup \setlength\tabcolsep{0.1\tabcolsep}\scriptsize \small
    \begin{tabular}{|p{0.21\columnwidth}| % first columm
        *{8}{>{\centering\arraybackslash}m{0.1\columnwidth}|}} % next nine columns
      \hline
          \multirow{2}*{\bfseries Element} & \multicolumn{3}{c|}{Positive
        Expressions} & %
      \multicolumn{3}{c|}{Negative Expressions} & %
      \multirow{2}*{Macro-\F{}} & %
      \multirow{2}*{Micro-\F{}}\\\cline{2-7}

      & Precision & Recall & \F{} & Precision & Recall & \F{} & & \\\hline
      \multicolumn{9}{|c|}{\cellcolor{cellcolor}Existing Lexica}\\\hline

      \textsc{SentiWordNet}$^{\mathrm{ternary}}_{\mathrm{Rocchio}}$ & 67.09\stddev{22.16} &
      14.74\stddev{7.79} & 11.73\stddev{5.59} & 4.57\stddev{4.57} &
      5.88\stddev{5.4} & 2.48\stddev{2.37} & 21.08\stddev{2.15} &
      96.04\stddev{0.72}\\

      Ising Spin Model & \stddev{} & \stddev{} & \stddev{} & \stddev{}
      & \stddev{} & \stddev{} & \stddev{} & \stddev{}\\\hline

      \multicolumn{9}{|c|}{\cellcolor{cellcolor}Our Method}\\\hline
    \end{tabular}
    \egroup
    \caption{Classification results.\\ {\small (GPC -- German Polarity
        Clues \cite{Waltinger:10}, SWS -- SentiWS \cite{Remus:10}, ZPL
        -- Zurich Polarity Lexicon \cite{Clematide:10})}}
    \label{snt-lex:tbl:w2v}
  \end{center}
\end{table}

\subsection{Discussion}

Since \textsc{GermaNet}, however, is significantly different from its
English counterpart, both quantitatively and qualitatively, we should
first present some key statistics (shown in
Table~\ref{snt-lex:tbl:germanet-wordnet}) and visualize the synset
graphs (demonstrated in Figures~\ref{snt-lex:fig:germanet}
and~\ref{snt-lex:fig:wordnet}) of these lexical databases.

\begin{table}[h]
  \begin{center}
    \bgroup \setlength\tabcolsep{0.1\tabcolsep}\scriptsize \small
    \begin{tabular}{p{0.15\textwidth} % first columm
        *{8}{>{\centering\arraybackslash}m{0.085\textwidth}}} % next nine columns
      \toprule
      & \multicolumn{2}{c}{\bfseries Noun} & %
      \multicolumn{2}{c}{\bfseries Verb} & %
      \multicolumn{2}{c}{\bfseries Adjective} & & \\
      \multirow{-2}{0.12\columnwidth}{\centering\bfseries Resource} & %
      Words & Synsets & Words & Synsets & Words & Synsets & %
      \multirow{-2}{0.085\columnwidth}{\centering\scriptsize\bfseries{}Hy\-pon.\newline{}Rels} & %
      \multirow{-2}{0.085\columnwidth}{\centering\scriptsize\bfseries{}Anto\-n.\newline{}Rels}\\
      \midrule

      \textsc{GermaNet} & 85,662 & 71,575 & 9,340 & 11,026 & 12,890 & 10,645 & %
      97,712 & 1,741\\
      \textsc{WordNet}  & 117,798 & 82,115 & 11,529 & 13,767 & 21,479 & 18,156 & %
      95,322 & 7,394\\
      \bottomrule
    \end{tabular}
    \egroup
    \caption{Key statistics on \textsc{GermaNet} and
      \textsc{WordNet}.}
    \label{snt-lex:tbl:germanet-wordnet}
  \end{center}
\end{table}

\begin{figure*}[htbp!]
{
\centering
\begin{subfigure}{.5\textwidth}
  \centering
  \includegraphics[width=\linewidth]{img/germanet.png}
  \caption{\textsc{GermaNet}}\label{snt-lex:fig:germanet}
\end{subfigure}%
\begin{subfigure}{.5\textwidth}
  \centering
  \includegraphics[width=\linewidth]{img/wordnet.png}
  \caption{\textsc{WordNet}}\label{snt-lex:fig:wordnet}
\end{subfigure}
}
\caption{Graphical visualization of \textsc{GermaNet} and
      \textsc{WordNet}.}\label{snt:fig:crp-sent-emo-distr}
\end{figure*}

As can be seen from the table, \textsc{GermaNet} has significantly
fewer words and synsets for all common parts of speech with the
largest gaps observed for nouns and adjectives.  Moreover, as shown in
Figures~\ref{snt-lex:fig:germanet} and~\ref{snt-lex:fig:wordnet}, such
PoS-classes as adverbs and adjective satellites are completely missing
in the German resource.  The reason for this is that the form (and
meaning) of most German adverbs typically coincides with that of the
adjectives; therefore, both categories are treated in the same way,
being represented through the adjectival synsets.

A slightly different situation can be observed for the semantic links
(relations) between the synsets: here, \textsc{GermaNet} features
almost 2,500 more hypernym-hyponym pairs than the English resource,
whereas the number of antonyms is more than four times less than in
\textsc{WordNet}.

An especially interesting pattern, however, appears with the relations
connecting different parts of speech: As can be seen from the figures,
the strongest inter-PoS connections in \textsc{GermaNet} are the
pertainym links between the adjectives and nouns and the participle
edges between the adjectives and verbs.  The interlinks between the
nouns and verbs, however, are both much fewer in number and more
diverse in their nature.  This situation is different in
\textsc{WordNet} where the prevailing majority of the
inter-part-of-speech connections are represented through the
\texttt{related\_to} (especially between the nouns, adjective
satellites, verbs, and verbs and adjectives) and
\texttt{derived\_from} links (especially between adverbs and
adjectives with their satellites).  The relations between the
adjectives and nouns are mixed though, featuring both
\texttt{related\_to} and \texttt{derived\_from} connections.  As we
should see later, these links are crucial for breaking part-of-speech
dependencies of seed sets in the cases when all seeds belong to the
same PoS class.

Another lexical sentiment resource (\textsc{WordNet-Affect}) was
proposed by \citet{Strapparava:04} who manually compiled a list of
1,903 subjective terms and projected these polarities to the
respective synononyms set in \textsc{WordNet}.  The resulting database
included 2,874 synsets with a total of 4,787 words.

% \subsubsection{Domain-Specific Sentiment Lexica}

% \citet{Chetviorkin:14} obtained a set of possible subjective terms
% from English and Russian microblogs by using an ensemble of supervised
% machine learning classifiers that had previously been trained on a
% manually annotated corpus of movie reviews.  In order to determine the
% prior polarity of the extracted terms, the authors first calculated
% approximate polarity scores of the processed messages using general
% polarity lexicons and then took these rough estimates as prior
% polarity expectations of the candidate expressions.  The posterior
% scores of these expressions were computed using the Ising spin model
% in a similar way to the approach proposed by \citet{Takamura:05}.  The
% resulting lexicon comprised 2,772 words for Russian and 2,786 lexical
% items for English.

\subsection{Summary and Conclusions}

In this section, we presented the first attempt of a practical
evaluation of our corpus.  In doing so, we addressed the task of
automatic prediction of polar terms (emotional expressions) with the
help of sentiment dictionaries.  To obtain a rough baseline estimate,
we first evaluated the quality of the existing sentiment lists for
German: German Polarity Clues~\cite{Waltinger:10},
SentiWS~\cite{Remus:10}, and Zurich Polarity List~\cite{Clematide:10}.
We showed that \ldots achieved the best quality, reaching an average
\F{}-score of \ldots on recognizing positive expressions and \ldots on
predicting negative polar terms.

In the next step, we analyzed whether the methods that were used for
creating the original English resources whose translations formed the
basis of the German lexica could yield better results than the
manually revised tranlated lists when applied to German data directly.

Another popular approach to an unsupervised induction of sentiment
lexica relies on the cooccurrence information about the words taken
directly from corpus.  One of the most popular methods from this
category is the Ising spin model adopted from the statistical
mechanics which interprets words as magnetic spins in a crystal grid
and tries to derive the most probable orientation of these spins in a
magnetic field.  This model was first applied to the needs of
computational linguistics by \citet{Takamura:05}, who induced a
sentiment lexicon for English using \ldots corpus data.  We have
reimplemented this approach in our program suite and applied to the
German Twitter snapshop of \citet{Scheffler:14}.  The results of this
approach are shown in Table~\ref{snt-lex:tbl:ispn-res}.

A different way of incorporating corpus data is to encode the
cooccurrence statistics directly into word information, representing
the latter as vectors.  We explored this direction in the final part
of this section, first obtaining word2vec embeddings for tokens from
the aforementioned snapshot and then applying clustering algorithms to
these representations.

Our results show that \ldots.

\newpage


\section{Conclusions}\label{sec:snt:concl}


%% \section{Identification of Targets, Sources, and Sentiments}\label{sec:fgsa}
%% \subsection{Rule-based Approaches to Sentiment Tagging}
%% \subsection{Machine Learning Approaches to Sentiment Tagging}
%% \subsubsection{Compared Systems}
%% \subsubsection{Features}
%% \subsubsection{Ablation Tests}
%% \subsubsection{Evaluation}
%% \subsection{Related Work}
%% \subsection{Conclusions}

%% \section{Coarse-grained Sentiment Analysis}
%% \subsection{Sentiment Analysis on the Level of Discussions}
%% \subsection{Sentiment Analysis on the Level of Messages}
%% \subsection{Sentiment Identification on the Level of Sentences}
%% \subsection{Related Work}
%% \subsection{Conclusions}

%% \section{Effects of Text Normalization and Domain Adaptation}
%% \subsection{Text Normalization Impact on Fine-grained Sentiment Analysis}
%% \subsection{Text Normalization Effect on Coarse-grained Sentiment Analysis}
%% \subsection{Comparison with other Corpora and other Domains}
%% \subsection{Related Work}
%% \subsection{Conclusions}
