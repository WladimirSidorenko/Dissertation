% FILE: appendixA.tex  Version 2.1
% AUTHOR:
% Universit�t Duisburg-Essen, Standort Duisburg
% AG Prof. Dr. G�nter T�rner
% Verena Gondek, Andy Braune, Henning Kerstan
% Fachbereich Mathematik
% Lotharstr. 65., 47057 Duisburg
% entstanden im Rahmen des DFG-Projektes DissOnlineTutor
% in Zusammenarbeit mit der
% Humboldt-Universitaet zu Berlin
% AG Elektronisches Publizieren
% Joanna Rycko
% und der
% DNB - Deutsche Nationalbibliothek

\chapter{Gradient Computation of the Optimized Projection
  Line}\label{chap:apdx:lex-grad}

In order to prove the correctness of the gradient shown in
Equation~\ref{eq:prj-line-grad}, let us first compute the partial
derivative of the optimized distance function $f =
\sum_{\vec{p}_+}\sum_{\vec{p}_-}\frac{1}{2}\left(%
\frac{\vec{b}\cdot\left(\vec{p}_+ - \vec{p}_-\right)}{\vec{b}^2}%
\vec{b}\right)^{2}$ w.r.t. to a single element $\vec{b}_j$ of the
projection vector $\vec{b}$.  Assuming that the length of this vector
is normalized at each iteration step prior to calculating the
derivative, we obtain:
{\small
  \begin{align}
    \begin{split}
    \frac{\partial}{\partial\vec{b}_j}f &= %
    \frac{\partial}{\partial\vec{b}_j}\sum_{\vec{p}_+}\sum_{\vec{p}_-}\frac{1}{2}\left(%
\frac{\vec{b}\cdot\left(\vec{p}_+ - \vec{p}_-\right)}{\vec{b}^2}%
\vec{b}_j\right)^2\\
&=\sum_{\vec{p}_+}\sum_{\vec{p}_-}\gamma\vec{b}_j%
\frac{\partial}{\partial\vec{b}_j}%
\frac{\vec{b}\cdot\left(\vec{p}_+ - \vec{p}_-\right)}{\vec{b}^2}\vec{b}_j\\
&=\sum_{\vec{p}_+}\sum_{\vec{p}_-}\gamma\vec{b}_j%
\left(\frac{(\vec{p}_+ - \vec{p}_-)_{j}\vec{b}^2 - 2\gamma\vec{b}_j}{\vec{b}^4}\vec{b}_j%
+\frac{\gamma}{\vec{b}^2}\right)\\
&=\sum_{\vec{p}_+}\sum_{\vec{p}_-}\gamma\vec{b}_j%
\left(\left(\vec{p}_+ - \vec{p}_-\right)_j\vec{b}_j - 2\gamma\vec{b}^2_j+\gamma\right)\\
&=\sum_{\vec{p}_+}\sum_{\vec{p}_-}\gamma\left(%
\left(\vec{p}_+ - \vec{p}_-\right)_j\vec{b}^2_j - 2\gamma\vec{b}_j\vec{b}^2_j%
+\gamma\vec{b}_j\right),\label{eq:prj-line-partial}%
    \end{split}%
  \end{align}}%
where $\gamma$ is defined as previously:
{\small
  \begin{align*}
    \gamma = \vec{b}\cdot\left(\vec{p}_+ - \vec{p}_-\right).%
  \end{align*}}%
Since Expression~\ref{eq:prj-line-partial} will be identical for all
$j$, we can estimate the final form of the gradient as:
{\small
  \begin{align}
    \nabla f &= \sum_{\vec{p}_+}\sum_{\vec{p}_-}\gamma\left(%
\left(\vec{p}_+ - \vec{p}_-\right)\vec{b}^2 - 2\gamma\vec{b}\vec{b}^2%
+\gamma\vec{b}\right)\\
&= \sum_{\vec{p}_+}\sum_{\vec{p}_-}\gamma\left(\Delta - \gamma\vec{b}\right),%
\end{align}}%
which is exactly the solution we provide in
Equation~\ref{eq:prj-line-grad}.
