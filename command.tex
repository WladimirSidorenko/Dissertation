%% -*- coding: utf-8; -*-
% FILE: command.tex  Version 2.1
% AUTHOR:
% Universität Duisburg-Essen, Standort Duisburg
% AG Prof. Dr. Günter Törner
% Verena Gondek, Andy Braune, Henning Kerstan
% Fachbereich Mathematik
% Lotharstr. 65., 47057 Duisburg
% entstanden im Rahmen des DFG-Projektes DissOnlineTutor
% in Zusammenarbeit mit der
% Humboldt-Universitaet zu Berlin
% AG Elektronisches Publizieren
% Joanna Rycko
% und der
% DNB - Deutsche Nationalbibliothek

%%%%%%%%%%%%%%%%%%%%%%%%%%%%%%%%%%%%%%%%%%%%%%%%%%%%%%%%%%%%%%%%%%
% Packages
\usepackage{algorithm}
\usepackage{algorithmicx}
\usepackage{algpseudocode}
\usepackage{amsmath}
\usepackage{amssymb}
% \usepackage{amsthm}
\usepackage{bm}
\usepackage{booktabs}
\usepackage[font={small,it},justification=centering]{caption}
\usepackage{color}              % \definecolor
\usepackage{colortbl}           % \cellcolor (colored tables)
\usepackage{dirtree}            % \dirtree (draw directory structure)
\usepackage{etoolbox}           % \ifstrempty
\usepackage{fancyhdr}
\usepackage{footnote}
\usepackage{graphicx}           % \scalebox
\usepackage{hhline}             % \ifstrempty
\usepackage{hyperref}           % clickable table of contents
\usepackage{listings}           % for \lstlisting
\usepackage{mathtools}          % \stackrel, \mathclap
\usepackage{mdframed}           % gray background boxes
\usepackage{multicol}           % for multiple columns in a table
\usepackage{multirow}
\usepackage[authoryear]{natbib}
\usepackage{ntheorem}
\usepackage{paralist}
\usepackage{relsize}
\usepackage{rst}
\usepackage{subcaption}
\usepackage{tabularx}
\usepackage{thmtools}
\usepackage{tikz}
\usepackage[colorinlistoftodos,prependcaption,textsize=tiny]{todonotes}
\usepackage{wasysym}            % smiley symbols
\usepackage{wrapfig}
\usepackage{xcolor}             % \colorlet

\usetikzlibrary{arrows,backgrounds,calc,decorations.markings,%
        positioning,shapes,shapes.misc,snakes}

%%%%%%%%%%%%%%%%%%%%%%%%%%%%%%%%%%%%%%%%%%%%%%%%%%%%%%%%%%%%%%%%%%
%% Lengths
\newlength{\oosixthClmnWidth}
\setlength{\oosixthClmnWidth}{0.055\textwidth}

\newlength{\clmnwidth}
\setlength{\clmnwidth}{0.65\textwidth}
\captionsetup[wraptable]{margin=1cm,width=4cm}

\newlength{\exampleSep}
\setlength{\exampleSep}{0.6em}

%% RST drawing
%% \setlength{\compressionWidth}{100pt}
%% \setlength{\terminalWidth}{0pt}
\setlength{\rstmiddleskip}{1.5em}

%%%%%%%%%%%%%%%%%%%%%%%%%%%%%%%%%%%%%%%%%%%%%%%%%%%%%%%%%%%%%%%%%%
%% Colors
\definecolor{cyan}{rgb}{0.0,0.6,0.6}
\definecolor{darkblue}{rgb}{0.0,0.0,0.6}
\definecolor{darkgoldenrod4}{RGB}{139,101,8}
\definecolor{darkred}{RGB}{139,0,0}
\definecolor{darkslateblue}{RGB}{72,61,139}
\definecolor{deeppink4}{RGB}{139,10,80}
\definecolor{firebrick}{RGB}{178,34,34}
\definecolor{gray69}{RGB}{176,176,176}
\definecolor{gray76}{RGB}{194,194,194}
\definecolor{gray84}{RGB}{214,214,214}
\definecolor{gray}{rgb}{0.4,0.4,0.4}
\definecolor{highlighter green}{RGB}{27,252,6}
\definecolor{lightcyan4}{RGB}{122,139,139}
\definecolor{midnightblue}{RGB}{25,25,112}
\definecolor{red3}{RGB}{205,0,0}
\definecolor{seagreen}{RGB}{46,139,87}
\definecolor{violetred4}{RGB}{139,34,82}
\definecolor{chartreuse}{RGB}{127,255,0}
\definecolor{green}{RGB}{0,255,0}
\definecolor{green2}{RGB}{0,238,0}
\definecolor{green3}{RGB}{0,205,0}


\definecolor{green3}{RGB}{0,205,0}

\colorlet{cellcolor}{gray84}

%%%%%%%%%%%%%%%%%%%%%%%%%%%%%%%%%%%%%%%%%%%%%%%%%%%%%%%%%%%%%%%%%%
%% Commands
\newcommand{\vars}{\texttt}
\newcommand{\func}{\textrm}
\newcommand{\ttranslate}[2]{#1

\noindent\emph{#2}}

\renewcommand{\cite}{\citep}
\newcommand{\titlerule}{\rule{0.8\textwidth}{.4pt}}
\providecommand{\shortcite}[1]{\cite{#1}}
\newcommand{\texample}[1]{``\textit{#1}''}
\newcommand{\done}[2][]{\todo[color=green3,#1]{#2}}

\newcommand{\mtag}[3]{{{\upshape[}\ifstrempty{#3}{#2}{\textcolor{#3}{#2}}{\upshape]$_{\textrm{\bfseries\emph{\tiny
          #1}}}$}}}
\newcommand{\sentiment}[2][]{\mtag{sen\-ti\-ment\ifstrempty{#1}{}{:#1}}{#2}{}}
\newcommand{\source}[2][]{\mtag{source\ifstrempty{#1}{}{:#1}}{#2}{darkgoldenrod4}}
\newcommand{\target}[2][]{\mtag{tar\-get\ifstrempty{#1}{}{:#1}}{#2}{darkslateblue}}
\newcommand{\emoexpression}[2][]{\mtag{eex\-pression\ifstrempty{#1}{}{:#1}}{#2}{red}}
\newcommand{\intensifier}[2][]{\mtag{in\-ten\-si\-fier\ifstrempty{#1}{}{:#1}}{#2}{cyan}}
\newcommand{\diminisher}[2][]{\mtag{di\-mi\-ni\-sher\ifstrempty{#1}{}{:#1}}{#2}{}}
\newcommand{\negation}[2][]{\mtag{ne\-ga\-tion\ifstrempty{#1}{}{:#1}}{#2}{deeppink4}}

\newcommand{\argone}[2][]{\mtag{\ifstrempty{#1}{}{rel#1:}arg1}{#2}{darkgoldenrod4}}
\newcommand{\argtwo}[2][]{\mtag{\ifstrempty{#1}{}{rel#1:}arg2}{#2}{darkslateblue}}
\newcommand{\connective}[2][]{\mtag{\ifstrempty{#1}{}{rel#1:}con\-nec\-ti\-ve}{#2}{cyan}}

\newcommand{\argmax}{\operatornamewithlimits{argmax}}
\newcommand{\argmin}{\operatornamewithlimits{argmin}}
\newcommand{\stddev}[1]{{\tiny$^{\pm#1}$}}
\newcommand{\negdelta}[1]{\textsuperscript{\textcolor{red3}{-#1}}}
\newcommand{\posdelta}[1]{\textsuperscript{\textcolor{seagreen}{+#1}}}
\newcommand{\F}[0]{$F_1$}
\newcommand{\NA}{\textcolor{gray69}{NA}}
\newcommand{\expnumber}[2]{{#1}\mathrm{e}^{#2}}
\newcommand{\cev}[1]{\reflectbox{\ensuremath{\vec{\reflectbox{\ensuremath{#1}}}}}}
\newcommand{\norm}[1]{\left\lVert#1\right\rVert}
\newcommand\given[1]{\:#1\vert\:}
\newcommand{\heart}{\ensuremath\heartsuit}
\newcommand{\defeq}{\vcentcolon=}
\newcommand{\eqdef}{=\vcentcolon}
\newcommand{\eg}{\textit{e.g.},}
\newcommand{\ienocomma}{\textit{i.e.}}
\newcommand{\ie}{\ienocomma,}

\newcommand{\specialcell}[2][c]{%
  \begin{tabular}[#1]{@{}c@{}}#2\end{tabular}}

\newcommand{\crfEdges}[2]{
    \begin{scope}[on background layer]
      \foreach \prnt in {Neg, Neut, Pos}
        {
           \foreach \chld in {Neg, Neut, Pos}
             \path [-] (#1\prnt) edge node [factor] {} (#2\chld);
        }
    \end{scope}
}

\newcommand{\crfEdgesX}[2]{
    \begin{scope}[on background layer]
      \foreach \prnt in {SNT, TRG, SRC, NON}
        {
           \foreach \chld in {SNT, TRG, SRC, NON}
             \path [-] (\prnt#1) edge node [factor] {} (\chld#2);
        }
    \end{scope}
}

\newcommand{\crfEdgesLatent}[2]{
    \begin{scope}[on background layer]
      \foreach \prnt in {Neg, Neut, Pos}
        {
           \foreach \chld in {Neg, Neut, Pos}
           {
             \path [-,transform canvas={xshift=-1pt}] (#1\prnt) edge node [factor] {} (#2\chld);
             \path [-,transform canvas={xshift=1pt}] (#1\prnt) edge node [factor] {} (#2\chld);
           }
        }
    \end{scope}
}

%% 3 arguments:
%% Arg1: list of (i, x, w) tuples, where i is the number of feature, x is its x position, and w stands for the feature weight
%% Arg2: index of the parent node
%% Arg3: y position of feature nodes
\newcommand{\crfFeatures}[3]{
    \foreach \i/\x/\w in {#1}
    {
        \node[xnode] (#2FEAT\i) at (\x, #3) {\w};
        \foreach \prnt in {Neg, Neut, Pos}
        {
            \path [-] (#2FEAT\i) edge node [factor] {} (#2\prnt);
        }
    }
}

\newcommand{\crfFeaturesX}[3]{
    \foreach \i/\x in {#1}
    {
        \node[xnode] (FEAT\i) at (\x, #3) {};
        \begin{scope}[on background layer]
          \foreach \prnt in {SNT, TRG, SRC, NON}
          {
            \path [-] (FEAT\i) edge node [factor] {} (\prnt#2);
          }
        \end{scope}
    }
}

\newcommand{\crfFeaturesXX}[3]{
    \foreach \i/\x in {#1}
    {
        \node[xnode] (FEAT\i) at (\x, #3) {};
        \begin{scope}[on background layer]
          \foreach \prnt in {SNT, TRG, SRC, DOTS, NON}
          {
            \path [-] (FEAT\i) edge node [factor] {} (\prnt#2);
          }
        \end{scope}
    }
}

\newcommand{\crfFeaturesSemiMarkov}[3]{
    \foreach \i/\x/\d in {#1}
    {
      \node[xnode] (FEAT\i) at (\x, #3) {};
      \begin{scope}[on background layer]
        \foreach \prnt in {SNT, TRG, SRC, NON}
        {
          \path [-] (FEAT\i) edge node [factor] {} (\prnt\d);
        }
      \end{scope}
    }
    \foreach \i/\x/\d in {#2}
    {
      \node[xnode] (FEAT\i) at (\x, #3) {};
      \begin{scope}[on background layer]
        \foreach \prnt in {SNT, TRG, SRC, NON}
        {
          \path [-] (FEAT\i) edge node [factor] {} (\prnt\d);
        }
      \end{scope}
    }
}

%% 3 arguments:
%% Arg1: left node
%% Arg2: middle node
%% Arg3: right node
%% Arg4: hypernode label
\newcommand{\hyperNode}[4]{
    \begin{scope}[on background layer,fill opacity=0.4]
    %% Root
     \filldraw[fill=darkslateblue!45] ($(#1)+(-0.49,0.)$)
        to[out=90,in=180] ($(#2) + (0.,0.47)$)
        to[out=0,in=90] ($(#3) + (0.49,0.)$)
        to[out=270,in=0] ($(#2) + (0.,-0.47)$)
        to[out=180,in=270] ($(#1) + (-0.49,0.)$);
     \node[label] (Root) [above right=0em and 0.3em of #3] {%
                  \textcolor{darkslateblue}{\textsc{\bfseries #4}}};
    \end{scope}
}

\newcommand{\hyperNodeX}[5]{
    \begin{scope}[on background layer,fill opacity=0.4]
    %% Root
     \filldraw[fill=darkslateblue!45] ($(#1)+(0.,0.49)$)
        to[out=0,in=90] ($(#1) + (0.47,0.)$)
        to[out=270,in=90] ($(#2) + (0.47,0.)$)
        to[out=270,in=90] ($(#3) + (0.47,0.)$)
        to[out=270,in=90] ($(#4) + (0.47,0.)$)
        to[out=270,in=0] ($(#4) + (0.,-0.49)$)
        to[out=180,in=270] ($(#4) + (-0.47,0.)$)
        to[out=90,in=270] ($(#3) + (-0.47,0.)$)
        to[out=90,in=270] ($(#2) + (-0.47,0.)$)
        to[out=90,in=270] ($(#1) + (-0.47,0.)$)
        to[out=90,in=180] ($(#1) + (0.,0.49)$);
     \node[label] (Root) [above right=0em and 0.3em of #1] {%
                  \textcolor{darkslateblue}{\textsc{\bfseries #5}}};
    \end{scope}
}

\newcommand{\hyperNodeXX}[6]{
    \begin{scope}[on background layer,fill opacity=0.4]
    %% Root
     \filldraw[fill=darkslateblue!45] ($(#1)+(0.,0.49)$)
        to[out=0,in=90] ($(#1) + (0.87,0.)$)
        to[out=270,in=90] ($(#2) + (0.87,0.)$)
        to[out=270,in=90] ($(#3) + (0.87,0.)$)
        to[out=270,in=90] ($(#4) + (0.87,0.)$)
        to[out=270,in=90] ($(#5) + (0.87,0.)$)
        to[out=270,in=0] ($(#5) + (0,-0.49)$)
        to[out=180,in=270] ($(#5) + (-0.87,0.)$)
        to[out=90,in=270] ($(#4) + (-0.87,0.)$)
        to[out=90,in=270] ($(#3) + (-0.87,0.)$)
        to[out=90,in=270] ($(#2) + (-0.87,0.)$)
        to[out=90,in=270] ($(#1) + (-0.87,0.)$)
        to[out=90,in=180] ($(#1) + (0.,0.49)$);
     \node[label] (Root) [above right=0em and 0.3em of #1] {%
                  \textcolor{darkslateblue}{\textsc{\bfseries #6}}};
    \end{scope}
}

\newcommand{\hyperNodeTreeCRF}[5]{
    \begin{scope}[on background layer,fill opacity=0.4]
    %% Root
     \filldraw[fill=darkslateblue!45] ($(#1)+(-0.49,0.)$)
        to[out=90,in=180] ($(#2) + (0.,0.47)$)
        to[out=0,in=180] ($(#3) + (0.,0.47)$)
        to[out=0,in=90] ($(#4) + (0.49,0.)$)
        to[out=270,in=0] ($(#3) + (0.,-0.47)$)
        to[out=180,in=0] ($(#2) + (0.,-0.47)$)
        to[out=180,in=270] ($(#1) + (-0.49,0.)$);
     \node[label] (Root) [above right=0.3em and 0.em of #4] {%
                  \textcolor{darkslateblue}{\textsc{\bfseries #5}}};
    \end{scope}
}

%% Header, Footer, and Page Layout
\fancyhf{}
\fancyhead[L]{\leftmark}
\rfoot{\thepage}
\pagestyle{fancy}

%% Tikz
\makeatletter
\tikzset{nomorepostaction/.code={\let\tikz@postactions\pgfutil@empty}}
\tikzset{circle split part fill/.style  args={#1,#2}{%
alias=tmp@name, % Jake's idea !!
postaction={%
insert path={
\pgfextra{%
\pgfpointdiff{\pgfpointanchor{\pgf@node@name}{center}}%
{\pgfpointanchor{\pgf@node@name}{east}}%
\pgfmathsetmacro\insiderad{\pgf@x}
%\begin{scope}[on background layer]
%\fill[#1] (\pgf@node@name.base) ([xshift=-\pgflinewidth]\pgf@node@name.east) arc
%                    (0:180:\insiderad-0.5\pgflinewidth)--cycle;
%\fill[#2] (\pgf@node@name.base) ([xshift=\pgflinewidth]\pgf@node@name.west)  arc
%                     (180:360:\insiderad-0.5\pgflinewidth)--cycle;
\fill[#1] (\pgf@node@name.base) ([xshift=-\pgflinewidth]\pgf@node@name.east) arc
(0:180:\insiderad-\pgflinewidth)--cycle;
\fill[#2] (\pgf@node@name.base) ([xshift=\pgflinewidth]\pgf@node@name.west)  arc
(180:360:\insiderad-\pgflinewidth)--cycle;            %  \end{scope}
}}}}}
\makeatother

%% Math
\DeclareMathOperator{\E}{\mathop{\mathbb{E}}}

%%%%%%%%%%%%%%%%%%%%%%%%%%%%%%%%%%%%%%%%%%%%%%%%%%%%%%%%%%%%%%%%%%
%% Environments
% uncomment this and the `amsthm` package to get plain uncolored
% examples
% \newtheoremstyle{break}
%   {\topsep}{\topsep}%
%   {\itshape}{}%
%   {\bfseries}{}%
%   {\newline}{}%
% \theoremstyle{break}
% \newtheorem{example}{Example}[section]

\theoremstyle{break}
\theoremheaderfont{\bfseries\upshape}
\newmdtheoremenv[%
linecolor=gray,%
leftmargin=35,%
rightmargin=35,%
backgroundcolor=gray!20,%
innertopmargin=5pt]{example}{Example}[section]

% allow fotnotes in tables
\makesavenoteenv{tabular}

%%%%%%%%%%%%%%%%%%%%%%%%%%%%%%%%%%%%%%%%%%%%%%%%%%%%%%%%%%%%%%%%%%
%% Hypersetup
\hypersetup{
    colorlinks=true,
    linktoc=all,
    citecolor=black,
    filecolor=black,
    linkcolor=midnightblue,
    urlcolor=black
}

%%%%%%%%%%%%%%%%%%%%%%%%%%%%%%%%%%%%%%%%%%%%%%%%%%%%%%%%%%%%%%%%%%
%% XML Highlighting
\lstset{
  basicstyle=\tiny,
  columns=fullflexible,
  showstringspaces=false,
  commentstyle=\color{gray}\upshape
}

\lstdefinelanguage{XML}
{
  morestring=[b]",
  morestring=[s]{>}{<},
  morecomment=[s]{<?}{?>},
  stringstyle=\color{black},
  identifierstyle=\color{darkblue},
  keywordstyle=\color{cyan},
  morekeywords={xmlns,id,intensity,mmax_level,span,sarcasm,polarity,version,encoding}%
}
