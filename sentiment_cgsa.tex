\chapter{Coarse-Grained Sentiment Analysis}\label{sec:snt:cgsa}

Having familiarized ourselves with the peculiarities of a sentiment
corpus, the different possibilities to automatically induce new
polarity lists, and the difficulties of fine-grained opinion mining,
we now move on to the presumably most popular sentiment analysis
objective---the coarse-grained analysis or CGSA, in which we need to
determine the overall polarity of a message.

Traditionally, this task has been addessed with either of three
popular groups of methods:
\begin{inparaenum}[(i)]
  \item lexicon-based approaches,
  \item machine-learning-based (ML) techniques, and
  \item deep-learning-based (DL) applications.
\end{inparaenum}
In this chapter, we are going to scrutinize the most prominent
representatives of each of these paradigms and also tackle a much more
ambitious goal, namely to check whether we can achieve results
comparable with the scores of these methods when the language of the
domain we train on is completely different from the language of the
test data.

We begin our comparison by first presenting the metrics that we will
use in our subsequent evaluation.  After a brief description of the
data preparation step, we proceed to the actual estimation of popular
lexicon-, ML-, and DL-based approaches, explaining and evaluating them
in Sections~\ref{sec:cgsa:lexicon-based}, \ref{sec:cgsa:ml-based},
and~\ref{sec:cgsa:dl-based} respectively.  Then,
in~Section~\ref{sec:cgsa:domain-adaptation}, we also show which
results can be obtained by using cross-lingual transfer, where we
train a classifier on English microblogs and then adapt this system to
German messages.  Finally, we conclude with an extensive evaluation of
different hyperparameters and settings (including various types of
sentiment lexicons, different kinds of word embeddings, the utility of
the text normalization step, and the impact of additional noisily
labeled training data), summarizing our results and recapping our
findings at the end of this chapter.

\section{Evaluation Metrics}\label{sec:cgsa:eval-metrics}

To estimate the quality of the compared systems, we will use two
established evaluation metrics which are commonly applied for
measuring CGSA results: One of these metrics is the macro-averaged
\F-score over the two major polarity classes~(positive and negative):
{ \small%
  \begin{equation*}
    F_1 = \frac{F_{pos} + F_{neg}}{2}.
  \end{equation*}%
  \normalsize%
}%
This measure was first introduced by the organizers of the SemEval
competition~\cite{Nakov:13,Rosenthal:14,Rosenthal:15} and has become a
de facto standard not only for the SemEval dataset, but virtually for
all related sentiment tasks and corpora.

The second metric is the micro-averaged \F-score over all three
possible semantic orientations (positive, negative, and neutral),
which basically corresponds to the accuracy over the whole labeled
dataset~\cite[cf.][p.~577]{Manning:99}.  This measure both predates
and supersedes the SemEval evaluation as it had already been applied
in the very first works on coarse-grained opinion
mining~\cite{Wiebe:99,Das:01,Read:05,Kennedy:06,Go:09} and was again
reintroduced in the GermEval Shared Task on Sentiment
Analysis~2017~\cite{Biemann:17}.

Moreover, in addition to these two metrics, we will also give a
detailed information on precision, recall, and \F-scores of each
particular polarity class in order to get a better intuition about
precise strengths, weaknesses, and biases of each evaluated CGSA
method.

\section{Data Preparation}\label{sec:cgsa:data}

Similarly to the data preparation steps used for fine-grained
sentiment analysis, we preprocessed all tweets of the PotTS
corpus~\cite{Sidarenka:16} with the text normalization system
of~\citet{Sidarenka:13}, tokenized them using the same adjusted
version of Potts'
tokenizer,\footnote{\url{http://sentiment.christopherpotts.net/code-data/happyfuntokenizing.py}}
lemmatized and assigned part-of-speech tags to these tokens with the
\texttt{TreeTagger} of~\citet{Schmid:95}.  Furthermore, like in the
previous chapter, we automatically obtained morphological features for
each word and induced syntactic trees for each sentence with the help
of the \texttt{Mate} dependency parser~of~\citet{Bohnet:13}. Apart
from the PotTS dataset, we also applied this procedure to the
microblogs of the German Twitter snapshot~\cite{Scheffler:14}, which
will be used in our subsequent experiments on noisy supervision.

We again divided the corpus data into a training, development, and
test set, using 70\% of the tweets for learning, 10\% for tuning and
picking the optimal hyperparameters, and the remaining 20\% for
evaluating the results.  We inferred polarity labels for these
microblogs with a simple heuristic rule, assigning the positive
(negative) class to all messages which contained exclusively positive
(negative) sentiments, considering all tweets without any polar
opinions as neutral, and skipping all microblogs that contained
sentiments with opposite semantic orientations.  Finally, to derive
these classes for the snapshot posts, we followed the procedure
proposed by~\citet{Read:05} and~\citet{Go:09}, and assigned the
positive (negative) labels to the tweets which contained respective
emoticons.

\todo[inline]{Add statistics on the noisily labeled corpus.}

\section{Coarse-Grained Sentiment Analysis with Lexicon-Based
  Methods}\label{sec:cgsa:lexicon-based}

% Das and Chen, 2001; Turney, 2002;  Kim and Hovy, 2004
The usage of lexicon-based approaches for estimating the semantic
orientation of complete texts dates back to the very origins of the
sentiment analysis research in general: For example, \citet{Das:01}
already used an example of five classifiers (two of which were purely
lexicon-based and the other three )

two purely lexicon-based and three machine-learning
methods, which heavily relied on lexicon features, in an ensemble of
five classifiers to predict the polarity of stock messages (buy, sell,
or neutral), attaining an accuracy of 62\% on a corpus of several
hundreds stock board messages.  \citet{Turney:02}, while proposing his
PMI method for automatically generating sentiment lexicons, also
suggested computing the orientation of the whole review by averaging
the sentiment scores of its terms found in the lexicon.  With this
approach, the author reached an accuracy of 74\% on a corpus of 410
manually labeled Epinions comments.  Finally, \citet{Kim:04} compared
three different methods of determining the overall polarity of a
sentence:
\begin{inparaenum}[(i)]
  \item by multiplying the signs of polar terms found in sentence,
  \item taking their sum, and
  \item taking the geometric mean of polarity scores;
\end{inparaenum}
finding that the first and last options worked best for the Document
Undestanding Corpus.\footnote{\url{http://duc.nist.gov/}}

% Hatzivassiloglou and Wiebe, 2000
\citet{Hatzivassi:00} proved, for instance, via statistical
significance tests that the mere presence of a subjective or even just
gradable adjective from a lexicon was a highly reliable indicator that
the sentence it appeared in also was subjective.

% Hu and Liu, 2004
Similarly, \citet{Hu:04} determined the semantic orientation of
sentences in customer reviews by simply comparing the number of
positive and negative terms found in these passages. Since the
authors, however, were primarily interested in estimating the polarity
towards particular product features mentioned in the clauses, they
additionally applied a fallback strategy in case of a tie by checking
which of the polar lexicon terms appeared closer to the features, and
assuming the polarity of the preceding sentence if these numbers were
also equal.

% Taboada et al., 2004
Largely inspired by the Appraisal theory of~\citet{Martin:00},
\citet{Taboada:04} enhanced the original method of~\citet{Turney:02}
by increasing the weights of polar adjectives which occurred in the
middle and at the end of a document, and also augmenting these values
with the affect, judgement, and appreciation scores.  Similarly to
polarity, the appraisal scores were calculated automatically by
computing the PMI of their cooccurrence with different pronouns using
a web search engine.

% Polanyi and Zaenen, 2006
An early analysis of lexicon-based approaches to sentiment analysis
was made by~\citet{Polanyi:06}, who argued that, besides considering
the lexical valence (i.e., semantic orientation) of polar terms, it
was also necessary to incorporate syntactic, discourse-level, and
extra-linguistic factors such as negations, intensifiers, modal
operators (e.g., \emph{could} or \emph{might}), presuppositional items
(e.g., \emph{barely} or \emph{failure}), irony, reported speech,
discourse connectors and structure, genre and attitude assessment,
reported speech, multi-entity evaluation, and so on.

% Kennedy and Inkpen, 2006
This theoretical hypothesis was later proven empirically by
\citet{Kennedy:06}, who compared two lexicon-based approaches:
\begin{inparaenum}[(i)]
\item simply counting positive and negative expressions, and assigning
  the input text (a review) to the polarity class with the greater
  number of found terms (considering a review as neutral if it had an
  equal number of expressions with either polarities); and
\item the same procedure as above, but additionally enhanced with the
  information about contextual valence shifters~(intensifiers,
  downtoners, and negations).
\end{inparaenum}
In the latter method, the authors swapped the polarity of a polar
expression to the opposite if it followed a negation, and decreased or
increased the score of this term by a constant factor if it was
preceded by a diminisher or intensifier.  This enhancement was shown
to bring a statistically significant improvement, boosting the
accuracy of the two-class prediction on a corpus of product and movie
reviewa from 67.9 to 69.3\%.

% Taboada et al., 2006
Another important contribution to the development of lexicon-based
approaches was made by~\citet{Taboada:06}, who compared three popular
polarity lists---a PMI lexicon computed with the original method
of~\citet{Turney:02} using the AltaVista's NEAR operator; a similar
polarity list obtained with the help of Google's AND queries; and,
finally, the manually compiled General Inquirer lexicon
of~\citet{Stone:66}.  The authors evaluated these resources both
intrinsically (by comparing them with GI entries) and extrinsically
(by computing the polarity of 400 manually annotated Epinions
reviews).  To estimate the overall polarity of a review for the second
task, \citeauthor{Taboada:06} calculated the average SO value of all
polar terms found in the review, obtaining these scores from the
mean-normalized lexicons, and flipping the polarity sign to the
opposite in case of the negation.

% Taboada et al., 2011
Finally, a veritably seminal work on lexicon-based approaches was
introduced by~\citet{Taboada:11} who presented a manually compiled
sentiment lexicon,\footnote{The authors hand-annotated all occurrences
  of adjectives, nouns, and verbs found in a corpus of 400 Epinions
  reviews with ordinal categories ranging from -5 to 5 which reflected
  the semantic orientation of a term (positive vs. negative) and its
  polar strength (weak vs. strong).} and used this resource to compute
the overall \emph{semantic orientation} (SO) of text.  Inspired by the
ideas of~\citet{Polanyi:06}, to make this calculation more precise,
the authors also incorporated a set of additional heuristic rules by
changing the prior SO values of negated, itensified, and downtoned
terms, ignoring irrealis sentences, and adjusting weights of specific
sections of a document.  An extensive evaluation of this method showed
a superior performance of the manual lexicon in comparison with other
polarity lists including the Subjectivity Dictionary~\cite{Wilson:05},
Maryland Polarity Set~\cite{Mohammad:09}, and
\textsc{SentiWordNet}~\cite{Esuli:06c}.  Moreover, the authors also
proved the robustness of their SO computation procedure for other
topics and text genres, hypothesizing that lexion-based approaches
were more domain-independent than traditional supervised learning
techniques.

% Musto et al., 2014
A lexicon-based system created specifically for Twitter data was
presented by~\citet{Musto:14}, who examined four different strategies
for computing the overall polarity scores of tweets---\emph{basic},
\emph{normalized}, \emph{emphasized}, and
\emph{normalized-emphasized}; checking these approaches with four
distinct lexicons---\textsc{SentiWordNet}~\cite{Esuli:06c},
\textsc{WordNet-Affect}~\cite{Strapparava:04},
\textsc{MPQA}~\cite{Wiebe:05}, and
\textsc{SenticNet}~\cite{Cambria:14}.  In all of these strategies, the
authors first split an input messages into \emph{micro-phrases} based
on the occurrence of punctuation marks and conjunctions; then
calculated the polarity score of each of these segments by summing
(\emph{basic} and \emph{emphasized}) or averaging (\emph{normalized}
and \emph{normalized-emphasized}) the lexicon scores of their tokens;
and, finally, estimated the overall polarity of the whole tweet by
summing or averaging the scores of the micro-phrases.
\citeauthor{Musto:14} obtained their best results (58.99\% accuracy on
the SemEval-2013 dataset) with the \textsc{SentiWordNet} lexicon
of~\citet{Esuli:06c}, using the normalized-emphasized approach, in
which they averaged the polarity scores of segment tokens, boosting
these values by 50\% for informative parts of speech; and regarded the
sum of the segements' scores as the final overall polarity of the
microblog.

% Jurek et al., 2015
Another Twitter-tailored approach was proposed by~\citet{Jurek:15}.
Drawing on the ideas of~\citet{Taboada:11}, the authors introduced a
new formula for estimating the polarity of a message, in which they
explicitly encoded the presumably most important sentiment factors:
polar terms, intensifiers, and negations.  In particular, given a
message~$m$, \citet{Jurek:15} calculated the postive and negative
scores ($F_p$ and $F_n$ respectively) of this microblog using the
following equation:%
{ \small%
  \begin{align}
    F_P &= \min\left(\frac{A_P}{2 - \log(3.5\times W_P + I_P)}, 100\right),\\
    F_N &= \max\left(\frac{A_N}{2 - \log(3.5\times W_N + I_N)}, -100\right);\label{cgsa:eq:jurek}
  \end{align}%
  \normalsize%
}%
where $A_P$ and $A_N$ represent the average scores of positive and
negative lexicon terms found in the tweet, $W_P$ and $W_N$ stand for
the raw counts of these items, and $I_P$ and $I_N$ denote the number
of intensifiers preceding the lexicon terms in the context.  In
addition to that, in case of negation, the authors modified the
individual polarity score $S_w$ of negated word $w$ as follows: {
  \small%
  \begin{equation*}
neg(S_w) =
    \begin{cases}
        \min\left(\frac{S_w - 100}{2}, -10\right) & \text{if } S_w > 0,\\
        \max\left(\frac{S_w + 100}{2}, 10\right), & \text{if } S_w < 0.
    \end{cases}
\end{equation*}%
\normalsize%
}%
Besides estimating the polarity scores $F_p$ and $F_n$,
\citeauthor{Jurek:15} also computed the degree of subjectivity for the
message by replacing the $A_P$ and $A_N$~terms in
Equations~\ref{cgsa:eq:jurek} with averaged conditional probabilitites
of polar terms.  These probabilities were calculated automatically on
the noisily labeled data set of~\citet{Go:09} and reflected how likely
a message was subjective given that a specific lexicon term appeared
in its text.  In the final step, the authors assigned the message to
the most probable polarity class if its overall polarity score was
greater than 25 and the subjectivity value was more than 0.5,
considering the tweet as neutral otherwise.  With this approach, the
authors achieved an accuracy of~77.3\% on the manually annotated
subset of \citeauthor{Go:09}'s corpus, and reached 74.2\% on the IMDB
corpus~\cite{Maas:11}.

% Kolchyna et al., 2015
Finally, \citet{Kolchyna:15} compared lexicon- and ML-based systems,
evaluating these approaches on the SemEval-2013 data
set~\cite{Nakov:13}.  For the former group of methods, the authors
explored two different ways of estimating the overall polarity of a
microblog:
\begin{inparaenum}[(i)]
\item by simply averaging the scores of the lexicon terms found in the
  message, and
\item by taking the signed logarithm of this average:
\end{inparaenum}
\begin{equation*}
  \text{Score}_{\log} =
  \begin{cases}
    \text{sign}(\text{Score}_{\text{AVG}})\log_{10}(|\text{Score}_{\text{AVG}}|) & %
    \text{if |Score}_{\text{AVG}}| > 0.1,\\
    0, & \text{otherwise}.
  \end{cases}
\end{equation*}%
To estimate the final label of the tweet, \citeauthor{Kolchyna:15}
trained a $k$-means clustering algorithm, which utilized either of the
above polarity scores as its features.  They showed that the
logarithmic approach performed better than the simple average
solution, yielding an accuracy of 61.74\% when used with a manually
compiled lexicon enriched with colloquial and slang terms.  For
ML-based methods, the authors proposed using cost-sensitive
SVMs~\cite{Masnadi:12}, which allowed them to improve on the results
of the top-performing system from the SemEval
competition~\cite{Mohammad:13} by 4\%, reaching 0.71~macro-averaged
\F{} on the two polarity classes (positive and negative).

\todo[inline]{reimplement:}

\begin{itemize}
\item Taboada (2011),
\item Musto et al. (2014),
\item Jurek et al. (2015),
\item Kolchyna (2015)
\end{itemize}

\begin{table}[h]
  \begin{center}
    \bgroup \setlength\tabcolsep{0.1\tabcolsep}\scriptsize
    \begin{tabular}{p{0.162\columnwidth} % first columm
        *{9}{>{\centering\arraybackslash}p{0.074\columnwidth}} % next nine columns
        *{2}{>{\centering\arraybackslash}p{0.068\columnwidth}}} % last two columns
      \toprule
      \multirow{2}*{\bfseries Method} & %
      \multicolumn{3}{c}{\bfseries Positive} & %
      \multicolumn{3}{c}{\bfseries Negative} & %
      \multicolumn{3}{c}{\bfseries Neutral} & %
      \multirow{2}{0.068\columnwidth}{\bfseries\centering Macro\newline \F{}} & %
      \multirow{2}{0.068\columnwidth}{\bfseries\centering Micro\newline \F{}}\\
      \cmidrule(lr){2-4}\cmidrule(lr){5-7}\cmidrule(lr){8-10}

      & Precision & Recall & \F{} & %
      Precision & Recall & \F{} & %
      Precision & Recall & \F{} & & \\\midrule

       &  &  &  & %
       &  &  & %
       &  &  & %
       & \\\bottomrule
    \end{tabular}
    \egroup
    \caption[Evaluation of lexicon-based coarse-grained SA methods.]{
      Evaluation of lexicon-based coarse-grained SA methods.\\
      {\small }}
    \label{snt-cgsa:tbl:lex-res}
  \end{center}
\end{table}

\section{Coarse-Grained Sentiment Analysis with ML-Based
  Methods}\label{sec:cgsa:ml-based}

One of the first works on an automatic sentiment classification with
ML-based methods was presented by~\citet{Wiebe:99}, who trained a
Na{\"i}ve Bayes system to differentiate between subjective and
objective statements, relying primarily on binary features which
reflected the presence of a pronoun, an adjective, a cardinal number,
or a modal other than ``will'' in the analyzed sentence.  The authors
achieved an accuracy of~72.17\%, outperforming the majority class
baseline by more than 20~percentage points.  An even better result
(81.5\%) could be reached when the data set was restricted only to the
examples with the most confident annotations.

A further step in this direction was taken by~\citet{Pang:02}, who
compared Na{\"i}ve Bayes, Maximum Entropy, and SVM approaches on the
polarity classification task for movie reviews, getting their best
results (82.9\% accuracy) with the SVM system that used only unigram
features.

\todo[inline]{}

From Barbosa 2010: A variety of features have been exploited on the
problem of sentiment detection (Pang and Lee, 2004; Pang et al., 2002;
Wiebe et al., 1999; Wiebe and Riloff, 2005; Riloff et al., 2006).

ReviewSA: this is the approach proposed by Pang and Lee (Pang and Lee,
2004) for sentiment analysis in regular online re- views. It performs
the subjectivity detec- tion on a sentence-level relying on the prox-
imity between sentences to detect subjectiv- ity. The set of sentences
predicted as subjec- tive is then classified as negative or positive
in terms of polarity using the unigrams that compose the sentences. We
used the imple- mentation provided by LingPipe (LingPipe, 2008);

Unigrams: Pang et al. (Pang et al., 2002) showed unigrams are
effective for sentiment detection in regular reviews. Based on that,
we built unigram-based classifiers for the subjectivity and polarity
detections over the training data. Another approach that uses un-
igrams is the one used by TwitterSentiment website. For polarity
detection, they select the positive examples for the training data
from the tweets containing good emoticons and negative examples from
tweets contain- ing bad emoticons. (Go et al., 2009). We built a
polarity classifier using this approach (Unigrams-TS)



Wiebe 2002, Riloff 2003

Wiebe, Bruce, \& O'Hara 1999
Hatzivassiloglou \& Wiebe 2000
Wiebe 2000;
Wiebe et al. 2002
Yu \& Hatzivassiloglou 2003

Bruce and Wiebe (1999) annotated 1,001 sentences as sub- jective or
objective, and Wiebe et al. (1999) de- scribed a sentence-level Naive
Bayes classifier using as features the presence or absence of
particular syn- tactic classes (pronouns, adjectives, cardinal num-
bers, modal verbs, adverbs), punctuation, and sen- tence position.

More recently, Wiebe et al.  (2002) report on document-level
subjectivity classi- fication, using a k-nearest neighbor algorithm
based on the total count of subjective words and phrases within each
document.

\todo[inline]{}

A semi-supervised classification approach was proposed
by~\citet{Yu:03}, who presented a three-stage method, in which they
first distinguished between subjective and objective documents, then
differentiated between polar and neutral sentences, and, finally,
classified the polarity of opinionated clauses.  The authors used a
Na{\"i}ve Bayes classifer for the document-level task, reaching a
remarkable \F-score of~0.96 on this objective; and applied an ensemble
of NB systems to predict the subjectivity of the sentences.  In the
final step, they determined the semantic orientation of subjective
clauses by averaging the polarity scores of their tokens, getting
these scores from an automatically constructed sentiment
lexicon~\cite{Hatzivassi:97}.  With this approach, \citeauthor{Yu:03}
attained an accuracy of~91\% on a set of 38 sentences which had a
perfect inter-annotator agreement in their data.

\todo[inline]{}

To the best of our knowledge, the idea of utilizing web texts
containing emoticons as noisily labeled training data was first
proposed by~\citet{Read:05}, who collected a set of 26,000 Usenet
posts featuring smileys or frownies and used these documents to train
a Na{\"i}ve Bayes and SVM classifier.  The author demonstrated that,
despite some encouraging results obtained on the instances from the
same domain (up to 70\% accuracy), the trained systems did not
generalize well to other text genres, barely outperforming the chance
baseline and reaching a maximum accuracy of~54.4\% on news data and
56.8\% on movie reviews.

The presumably first known attempt to adopt distant supervision for
the sentiment analysis of Twitter data was made by~\citet{Go:09} who
collected a set of 800,000 positive and 800,000 negative microblogs
relying on emoticons as their noisy labels.  After stripping off these
smileys from text, the authors trained three independent
ML-classifiers (Na{\"i}ve Bayes, Maximum Entropy, and Support Vector
Machines) on this collection, achieving their best results (82.7\%
accuracy) with the NB and MaxEnt systems thaat utilized unigrams and
bigrams as features.

Another distantly supervised approach was presented
by~\citet{Barbosa:10}, who gathered a collection of automatically
labeled tweets from three popular sentiment web sites (Twendz, Twitter
Sentiment, and TweetFeel), and trained two binary SVM systems on this
corpus.  The first of these classifiers had to distinguish between
subjective and objective microblogs, attaining an error rate of~18.1\%
on a subset of 1,000 manually annotated messages.  In the next step,
the second system had to determine the semantic orientation of
opinionated posts (positive or negative), reaching an error rate
of~18.7\% on this prediction.

In a similar way, \citet{Pak:10} gathered a collection of 300,000
noisily labeled tweets, ensuring an even distribution of positive,
negative, and neutral messages.  After a brief exploration of PoS tag
statistics in these different classes, they presented a Na{\"i}ve
Bayes system which utilized highly relevant binary part-of-speech and
$n$-gram features.\footnote{\citet{Pak:10} determined the relevance of
  a feature $f$ using a special \emph{salience} metric, which they
  defined as a negative ratio between the minimum and maximum
  conditional probabilities of this feature belonging to different
  target classes:
  \begin{equation*}
    salience(f) = \frac{1}{N}\sum_{i=1}^{N-1}\sum_{j=i+1}^N 1 - \frac{\min(P(f, s_i), P(f, s_j))}{\max(P(f, s_i), P(f, s_j))}.
  \end{equation*}
  The $N$~term in this formula denotes the number of training
  examples, and the expression $s_i$ means the sentiment class of the
  $i$-th training instance.} With this approach, the authors attained
an accuracy slighlty above 0.6 on the manually labeled test set
of~\citet{Go:09}, also demonstrating a particular utility of bigrams,
negation rules, and feature pruning heuristics.

A slightly different task was addressed by~\citet{Davidov:10}, who
sought to predict hashtags and emoticons occurring in tweets using a
$k$-NN classifier trained on a large collection of messages.  The
authors achieved an \F-measure of~0.31 on the former task, and reached
an \F-score of~0.64 on predicting smileys.

\citet{Kouloumpis:11} trained an AdaBoost
classifier~\cite{Schapire:00} on two large collections of noisily
labeled tweets---the emoticon tweebank of~\citet{Go:09} and the
Edinburgh hashtag corpus.\footnote{\url{http://demeter.inf.ed.ac.uk}}
Using $n$-gram (up to length two), lexicon, part-of-speech, and
micro-blogging features (such as emoticons, abbreviations, and slang
expressions), the authors achieved a macro-averaged \F-measure of~0.68
on the three-class prediction task.

% One of the first attempts to analyze message-level sentiments on
% Twitter was made by \citet{Go:09}.  For their experiments, the authors
% collected a set of 1,600,000 tweets containing smileys.  Based on
% these emoticons, they automatically derived polarity classes for these
% messages (positive or negative) and used them to train a Na\"{\i}ve
% Bayes, MaxEnt, and SVM classifier.  The best $F$-score for this
% two-class classification problem could be achieved by the last system
% and run up to 82.2\%.

% Similar work was also done by \citet{Pak:10} who used the Na\"{\i}ve
% Bayes approach to differentiate between neutral, positive, and
% negative microblogs; and \citet{Barbosa:10} who gathered a collection
% of 200,000 tweets, subsequently analyzing them with three publicly
% available sentiment web-services and training an SVM classifier on the
% results of these predictors.  In a similar way, \citet{Agarwal:11}
% compared a simple unigram-based SVM approach with two other
% full-fledged systems, one which relied on a rich set of manually
% defined features, and another used partial tree
% kernels~\cite{Moschitti:06}.  The authors evaluated these methods on a
% commercially acquired corpus of 8,753 foreign-language tweets, which
% were automatically translated into English, finding that a combination
% of these methods worked best for both two- and three-way prediction
% tasks.

% The state-of-the-art results for message level polarity prediction on
% tweets were established by~\citet{Mohammad:13}, whose system (a
% supervised SVM classifier) used a rich set of various features
% including word and character n-grams, PoS statistics, Brown
% clusters~\cite{Brown:92}, etc., and also strongly benefitted from
% automatic corpus-based polarity lists---Sentiment~140 and NRC
% Hashtag~\cite{Mohammad:12,Kiritchenko:14}.  This approach ranked first
% at the SemEval competition~2013~\cite{Nakov:13} and anchieved the
% fourth place on the rerun of this task one year
% later~\cite{Rosenthal:14}, being outperformed by the supervised
% logistic regression approach of~\citet{Miura:14}, who used a heavy
% preprocessing of the data and a special balancing scheme for
% underrepresented classes.  Later on, these results were further
% improved by the apporaches of~\citet{Hagen:15} and \citet{Deriu:16},
% which both relied on ensembles of multiple independent classifiers.

\begin{table}[h]
  \begin{center}
    \bgroup \setlength\tabcolsep{0.1\tabcolsep}\scriptsize
    \begin{tabular}{p{0.162\columnwidth} % first columm
        *{9}{>{\centering\arraybackslash}p{0.074\columnwidth}} % next nine columns
        *{2}{>{\centering\arraybackslash}p{0.068\columnwidth}}} % last two columns
      \toprule
      \multirow{2}*{\bfseries Method} & %
      \multicolumn{3}{c}{\bfseries Positive} & %
      \multicolumn{3}{c}{\bfseries Negative} & %
      \multicolumn{3}{c}{\bfseries Neutral} & %
      \multirow{2}{0.068\columnwidth}{\bfseries\centering Macro\newline \F{}} & %
      \multirow{2}{0.068\columnwidth}{\bfseries\centering Micro\newline \F{}}\\
      \cmidrule(lr){2-4}\cmidrule(lr){5-7}\cmidrule(lr){8-10}

      & Precision & Recall & \F{} & %
      Precision & Recall & \F{} & %
      Precision & Recall & \F{} & & \\\midrule

       &  &  &  & %
       &  &  & %
       &  &  & %
       & \\\bottomrule
    \end{tabular}
    \egroup
    \caption[Evaluation of ML-based coarse-grained SA methods.]{
      Evaluation of ML-based coarse-grained SA methods.\\
      {\small }}
    \label{snt-cgsa:tbl:ml-res}
  \end{center}
\end{table}

\section{Coarse-Grained Sentiment Analysis with DL-Based
  Methods}\label{sec:cgsa:dl-based}

\citet{Yessenalina:11}

A real breakthrough in the use of deep neural networks for the
sentence-level sentiment analysis happened with the pioneering work
of~\citet{Socher:11}, who first introduced a recursive autoencoder
(RAE).  In this system, the authors obtained a fixed-width vector
representation for complex phrases $\vec{v}$ by recursively merging
the vectors of adjacent tokens (say $\vec{w}_1$ and $\vec{w}_2$),
first multiplying these vectors with a compositional matrix $W$ and
then applying a non-linear function ($softmax$) to the resulting
product:
\begin{align*}
  \vec{c} &= softmax\left(W\cdot\begin{bmatrix}
  \vec{w}_1\\
  \vec{w}_2
  \end{bmatrix}\right)
\end{align*}
Using a max-margin classifier on top of the resulting phrase
representation, \citet{Socher:11} could improve the state-of-the-art
results on predicting the sentence-level polarity of user's blog
posts~\cite{Potts:10} and also outperformed the system
of~\citet{Nasukawa:03} on the MPQA data set~\cite{Wiebe:05}.

Later on, \citet{Socher:12} further improved these scores with the
help of a recursive matrix-vectors space model (RMVSM), in which each
word was associated with a 2-tuple of a vector and matrix---e.g.,
$(\vec{w}_1, W_1)$ and $(\vec{w}_2, W_2)$---and the compositionality
function was redefined as follows:
\begin{align*}
  \vec{c} &= softmax\left(W\cdot\begin{bmatrix}
  W_2\cdot\vec{w}_1\\
  W_1\cdot\vec{w}_2
  \end{bmatrix}\right)
\end{align*}

\citet{Wang:15}

\begin{table}[h]
  \begin{center}
    \bgroup \setlength\tabcolsep{0.1\tabcolsep}\scriptsize
    \begin{tabular}{p{0.162\columnwidth} % first columm
        *{9}{>{\centering\arraybackslash}p{0.074\columnwidth}} % next nine columns
        *{2}{>{\centering\arraybackslash}p{0.068\columnwidth}}} % last two columns
      \toprule
      \multirow{2}*{\bfseries Method} & %
      \multicolumn{3}{c}{\bfseries Positive} & %
      \multicolumn{3}{c}{\bfseries Negative} & %
      \multicolumn{3}{c}{\bfseries Neutral} & %
      \multirow{2}{0.068\columnwidth}{\bfseries\centering Macro\newline \F{}} & %
      \multirow{2}{0.068\columnwidth}{\bfseries\centering Micro\newline \F{}}\\
      \cmidrule(lr){2-4}\cmidrule(lr){5-7}\cmidrule(lr){8-10}

      & Precision & Recall & \F{} & %
      Precision & Recall & \F{} & %
      Precision & Recall & \F{} & & \\\midrule

       &  &  &  & %
       &  &  & %
       &  &  & %
       & \\\bottomrule
    \end{tabular}
    \egroup
    \caption[Evaluation of DL-based coarse-grained SA methods.]{
      Evaluation of DL-based coarse-grained SA methods.\\
      {\small }}
    \label{snt-cgsa:tbl:ml-res}
  \end{center}
\end{table}

\section{Coarse-Grained Sentiment Analysis Using Language and Domain
  Adaptation}\label{sec:cgsa:domain-adaptation}

One of the first works which pointed out the importance of domain
adaptation for sentiment analysis was introduced by~\citet{Aue:05}.
In their experiments, the authors trained separate SVM classifiers on
four different document sets: movie reviews, book reviews, customer
feedback from a product support service, and a feedback survey from a
customer knowledge base; finding that each classifier performed best
when applied to the same domain as it was trained on.  In order to
find an optimal way of overcoming this domain specificity,
\citet{Aue:05} tried out four different options:
\begin{inparaenum}[(i)]
\item\label{sent-cgsa:lst:rel-wrk1} training one classifier on all but
  the target domain and applying it to the latter;
\item using the same procedure as above, but limiting the features to
  only those which also appeared in the target texts;
\item taking an ensemble of individual classifiers each of which was
  trained on a different data collection; and, finally,
\item using a minimal subset of labeled in-domain data to train a
  Na{\"i}ve Bayes system with the expectation-maximization algorithm
  \cite[EM;][]{Dempster:77}.
\end{inparaenum}
The authors found that the ensemble and EM options worked best for
their cross-domain task, achieving an accuracy of up to 82.39\% for
the two-class prediction (positive vs negative) on new unseen text
genres.

Another notable milestone in the domain adaptation research was set
by~\citet{Blitzer:07}.  Relying on their previous work on structural
correspondence learning~\cite{Blitzer:07}, in which they used a set of
\emph{pivot features} (features which frequently appeared in both
target and source domains) to find an optimal correspondence of the
remaining attributes,\footnote{In particular, the authors trained $m$
  binary predictors for each of their $m$ pivot features in order to
  find other attributes which frequently co-occurred with the pivots.
  Afterwards, they composed these $m$ resulting weight vectors into a
  single matrix $W := [\vec{w}_{1},\ldots,\vec{w}_{m}]$, took an SVD
  decomposition of this matrix, and used the top $h$ left singular
  vectors to translate source features to the new domain.} the authors
refined their method by pre-selecting the pivots using their PMI
scores and improving misaligned feature projections using a small set
of labeled target examples.  With these modifications,
\citeauthor{Blitzer:07} were able to reduce the average adaptation
loss (the accuracy drop when transferring a classifier to a different
domain) from 9.1 to 4.9~percent when testing a sentiment predictor on
the domains of book, dvd, electical appliances, and kitchen reviews.

Other important works on domain adaptation for opinion mining include
those of~\citet{Read:05}, who pointed out that sentiment
classification might not only depend on the domain but also on topic,
time, and language style in which the text was written;
\citet{Tan:07}, who proposed using the classifier trained on the
source domain to classify unlabeled instances from the target genre,
and then iteratively retrain the system on the enriched data set.
Finally, \citet{Andreevskaia:08} proposed a combination of a lexicon-
and ML-based systems, claiming that this ensemble would be more
resistible to the domain shift than each of these classifiers on their
own.

Another line of research was introduced by~\citet{Glorot:11} who
proposed stacked denoising autoencoders (SDA)---a neural network
architecture in which an input vector $\vec{x}$ was first mapped to a
smaller representation $\vec{x}'$ via some function
$h: \vec{x}\mapsto\vec{x}'$, and then restored to its approximate
original state via an inverse transformation
$g: \vec{x}'\mapsto\vec{x}''\approx\vec{x}$.  In their experiments,
the authors optimized the parameters of the functions $h$ and $g$ on
both target and source data, getting approximate representations of
instances from both data sets; and then trained a linear SVM
classifier on the restored representations of the source instances,
subsequently applying this classifier to the target domain.  This
approach was further refined by~\citet{Chen:12} who analytically
computed the reconstruction function~$g$, and used both original and
restored features to predict the polarity labels of the target
data.\footnote{Both approaches were trained tested on the Amazon
  Review Corpus of~\citet{Blitzer:07}.}


Further notable contributions to domain adaptation in general were
made by~\citet{Daume:07} who proposed to replicate each extracted
feature three times and train the first replication on both domains,
the second repetion only on source, and the third copy only on target
domain, for which he assumed a small subset of labeled examples was
available; \citet{Yang:15} who trained neural embeddings of features,
trying to predict which instance attributes frequently co-occured with
each other;

\section{Evaluation}
\subsection{Effect of Lexicons}
\subsection{Effect of Distant Supervision}
\subsection{Effect of Word Embeddings}
\subsection{Effect of Normalization}

\section{Summary and Conclusions}\label{slsa:subsec:conclusions}
