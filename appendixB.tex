% FILE: appendixA.tex  Version 2.1
% AUTHOR:
% Universit�t Duisburg-Essen, Standort Duisburg
% AG Prof. Dr. G�nter T�rner
% Verena Gondek, Andy Braune, Henning Kerstan
% Fachbereich Mathematik
% Lotharstr. 65., 47057 Duisburg
% entstanden im Rahmen des DFG-Projektes DissOnlineTutor
% in Zusammenarbeit mit der
% Humboldt-Universitaet zu Berlin
% AG Elektronisches Publizieren
% Joanna Rycko
% und der
% DNB - Deutsche Nationalbibliothek

\chapter{Annotation Guidelines of the Sentiment Corpus}\label{chap:apdx:corp-guidelines}

{
  \setlength{\parindent}{0ex}
\tocless\section{Introduction}

In this assignment, your task is to annotate sentiments in a corpus of
Twitter messages.  We define \emph{sentiments} as polar (either
positive or negative) evaluative opinions about some persons,
entities, or events.  Your goal is to annotate both: text spans
denoting the opinions (\emph{sentiments}) and text spans signifying
the evaluated entities and events (\emph{sentiment targets}).  In
addition to that, you also have to label opinions' holders
(\emph{sentiment sources}) and lexical elements that might
significantly affect the polarity or the intensity of a sentiment.
These elements are:
\begin{itemize}
  \item\emph{polar terms}, which are words or phrases that
    unequivocally possess an evaluative lexical meaning in and of
    themselves (these are typically words like \emph{hassen}
    [\emph{hate}], \emph{bewundern} [\emph{admire}], \emph{sch\"on}
    [\emph{nice}] etc.);
  \item\emph{intensifiers} and \emph{diminishers} (or
    \emph{downtoners}), which are words and expressions that increase
    or decrease the evaluative sense of a polar term.  Examples of
    intensifiers are words like \emph{sehr} (\emph{very}),
    \emph{besonders} (\emph{especially}), or \emph{insbesondere}
    (\emph{particularly}).  Typical examples of diminishers are
    \emph{ein wenig} (\emph{a little}), \emph{ein bisschen} (\emph{a
      bit}), \emph{gewisserma\ss{}en} (\emph{to a certain degree}),
    etc.;
    \item and, finally, \emph{negations}, which are words or
      expressions that completely flip the polarity of a polar term or
      sentiment to the opposite (\eg{} \emph{nicht} gut [\emph{not}
        good] or \emph{kein} Talent [\emph{not} a talent]).
\end{itemize}

\tocless\section{Annotation Tool}

For annotating this corpus, you need to install \texttt{MMAX2}, a
freely available annotation tool, which you can download at:

\small\url{http://sourceforge.net/projects/mmax2/files/mmax2/mmax2_1.13.003/MMAX2_1.13.003b.zip/download}

After you have downloaded this file, unzip the received archive,
change to the newly created directory \texttt{1.13.003/MMAX2} in your
shell and execute the following commands: \code{chmod u+x~./mmax2.sh\\
  nohup~./mmax2.sh \&}
An \texttt{MMAX2} window will then appear on your screen.
If you have never used \texttt{MMAX2} before, please read its user
manual \texttt{mmax2quickstart.pdf}, which you can find in the
subdirectory \texttt{MMAX2/Docs} of the downloaded archive.

\tocless\section{Corpus Files}

You should also have received a copy of corpus files either as a
tar-gzipped archive or via a version control system.  In the former
case, you need to unpack the downloaded \texttt{.tgz} file using the
following command: \code{tar -xzf archive-name.tgz} After that, a
directory called \corpusDir{} will appear in your current folder.

You can find your annotation files in the subdirectory
\texttt{\corpusDir{}/corpus/annotator-ANNOTATOR\_ID}, where
ANNOTATOR\_ID is the ID number that has been previously assigned to
you by the supervisor.  In order to load an annotation file into your
\texttt{MMAX2} program, click on the menu \texttt{File -> Load}.  In
the displayed pop-up window, select the path to the
\corpusDir{}\texttt{/anno\-ta\-tor-ANNOTATOR\_ID} directory, and click
on one of the \texttt{*.mmax} files in this folder.

%%%%%%%%%%%%%%%%%%%%%%%%%%%%%%%%%%%%%%%%%%%%%%%%%%%%%%%%%%%%%%%%%%

\tocless\section{Tags and Attributes}\label{sec:markables}

Below, you can find a short list of all labels and their possible
attributes that will be used in this assignment:
\begin{multicols}{2}
  \begin{enumerate}
  \item \texttt{sentiment}s with the attributes:
    \begin{enumerate}
    \item \texttt{polarity},
    \item \texttt{intensity},
    \item \texttt{sarcasm};
    \end{enumerate}
  \item \texttt{target}s with the attributes:
    \begin{enumerate}
    \item \texttt{preferred},
    \item \texttt{anaph-ref},
    \item \texttt{sentiment-ref};
    \end{enumerate}
  \item \texttt{source}s with the attributes:
    \begin{enumerate}
    \item \texttt{anaph-ref},
    \item \texttt{sentiment-ref};
    \end{enumerate}
  \item \texttt{polar-term}s with the attributes:
    \begin{enumerate}
    \item \texttt{polarity},
    \item \texttt{intensity},
    \item \texttt{sarcasm},
    \item \texttt{sentiment-ref};
    \end{enumerate}
  \item \texttt{intensifier}s with the attributes:
    \begin{enumerate}
    \item \texttt{degree},
    \item \texttt{polar-term-ref};
    \end{enumerate}
  \item \texttt{diminisher}s with the attributes:
    \begin{enumerate}
    \item \texttt{degree},
    \item \texttt{polar-term-ref};
    \end{enumerate}
  \item and, finally, \texttt{negation}s with the single attribute:
    \begin{enumerate}
    \item \texttt{polar-term-ref}.
    \end{enumerate}
  \end{enumerate}
\end{multicols}
A more detailed description of these attributes is given in the
following sections.

\tocless\subsection{sentiment}\label{sec:sentiment}
\paragraph{Definition.} \emph{Sentiments} are polar subjective
evaluative opinions about people, entities, or events.

According to this definition, a sentiment must always fulfill the
following three criteria:
\begin{itemize}
\item it has to be \textbf{polar}, \ie{} it must always reflect either
  positive or negative attitude to its respective target.  Neutral,
  non-evaluative statements such as \textit{Ich glaube, er wird heute
    fr\"uher kommen} (\textit{I think he will be earlier today}) must
  not be marked as \texttt{sentiment}s;

\item it has to be \textbf{subjective}, \ie{} you must not assign this
  tag to statements of objective facts, such as \textit{Beim Angriff
    wurden 14 Glasscheiben besch\"adigt} (\textit{14 glass plates were
    broken during the attack}), even if you have a personal polar
  attitude to such events.  Sentiments should always reflect \emph{the
    personal opinion of their holder, not yours};

\item a sentiment has to be \textbf{evaluative}, which means that it
  must always refer to an explicit target and judge about its
  properties.  You should not regard cases like \textit{Ich bin heute
    so gl\"ucklich} (\textit{I am so happy today}) as
  \texttt{sentiment}s, because such statements do not evaluate
  anything in particular, but only express general mood of the author.
\end{itemize}

\paragraph{Example.} Typical examples of sentiments are evaluative
sentences similar to the one shown below.
\begin{example}
  \sentiment{Ich mag den neuen James Bond Film nicht.}

  (\sentiment{I don't like the new James Bond movie.})\label{ex:sentiment}
\end{example}
This example expresses a personal subjective evaluation; the opinion
is strictly negative; and it also has an explicit evaluation
target---the \textit{movie}.  Therefore, we enclose this sentence in
the \texttt{sentiment} tags.

We also consider contrastive comparisons as a special type of
evaluations.  But unlike other sentiments, comparisons typically
express a relative subjective judgment, \ie{} an object is regarded as
better or worse than another, but we usually do not know whether the
author actually likes or dislikes any of them.  To distinguish such
cases, we have introduced a special \texttt{comparison} value for the
\texttt{polarity} attribute of this element, which you should use to
distinguish such cases.

You should not label as \texttt{sentiments} polar opinions whose truth
status is unknown.  These are sentences like \textit{Ich wei\ss{}
  nicht, ob ich meinen Bruder mag} (\textit{I don't know whether I
  like my brother}), where neither we nor the author actually know
whether the author likes or dislikes her brother.  Exceptions from
this rule are cases like \textit{Ich zweifle, dass er ein guter Mensch
  ist} (\textit{I doubt that he is a good man}) or \textit{Ich glaube
  nicht, dass er diesen Preis verdient hat} (\textit{I don't think
  that he has deserved this award}), which express author's
disagreement with positive evaluations and, consequently, acts as a
negative judgment.  Special care should be taken when dealing with
questions and subjunctive sentences though (see FAQ Section in the
extended
version\footnote{\url{https://github.com/WladimirSidorenko/PotTS/blob/master/docs/annotation_guidelines.pdf}}
of these guidelines).

\paragraph{Boundaries.} \texttt{sentiment} tags should
enclose both the evaluated object (target) and the evaluative
expression (typically a polar-term), \ie{} you should put these tags
around the \emph{minimal complete syntactic or discourse-level unit in
  which both (target and evaluation expression) appear together}.

In Example~\ref{exmp:book}, for instance, the evaluated object is
\textit{Buch} (\textit{book}), the evaluative expression is
\textit{langweiliges} (\textit{boring}), and the minimal syntactic
unit that simultaneously comprises both of these elements is the noun
phrase \textit{ein langweiliges Buch} (\textit{a boring book}).
Therefore, we annotate the noun phrase with the \texttt{sentiment}
tags, but do not enclose anything else inside these labels.
\begin{example}
  Auf dem Tisch lag \sentiment{ein langweiliges Buch}.

  (There was \sentiment{a boring book} on the table.)\label{exmp:book}
\end{example}
Sentiments are not restricted to just noun phrases, they can also be
expressed by complete clauses or even multiple sentences (discourse
units).  In these cases, a \texttt{sentiment} span still has to be
\emph{complete}, \ie{} it should capture the common syntactic or
discourse-level ancestor of the evaluative expression and its target,
as well as all other descendents of that common ancestor element; and
it has to be \emph{minimal}, \ie{} it should only enclose the closest
possible ancestor, without including its parent or sibling elements.

Example~\ref{exmp:petterson} demonstrates a sentiment expressed by a
clause:
\begin{example}
  Wir akzeptieren das, weil \sentiment{wir alle ein bisschen in
    Petterson verliebt sind}.

  (We accept this because \sentiment{we all are a little bit in love
    with Petterson}.)\label{exmp:petterson}
\end{example}
In this sentence, the evaluative statement is made about
\textit{Petterson}, who acts as sentiment's target; the author says
that they all \textit{in ihn verliebt sind} (\textit{are in love with
  him}), which is a subjective evaluation.  Both (target and
evaluative expression) appear in one verb phrase, whose head is the
link verb \textit{sein} (\textit{to be}).  Consequently, we enclose
the complete verb phrase including its grammatical subject
\textit{wir} (\textit{we}) in the \texttt{sentiment} tags.

%% \begin{itemize}
%% \item single noun phrases, possibly with their prepositional
%%   attributes, \eg{} \textit{Auf dem Tisch lag \sentiment{ ein
%%       langweiliges Buch} (There was \sentiment{a boring book} on the
%%     table)};
%% \item clauses, \eg{} \textit{\sentiment{Ich hasse B\"ucher ohne
%%   Inhaltsangabe}. (\sentiment{I hate books without table of
%%   contents})};
%% \item multiple sentences in cases when these sentences jointly form a
%%   sentiment relation, \eg{}\\\textit{\sentiment{Sie denken, reden,
%%       riechen, lieben, schmecken, f*cken Plastik. Sie haben\\das so
%%       gelernt in der Plastik-Werbewelt}.\\ (\sentiment{They think and
%%       speak about, smell at, love, taste, f*ck plastic.  They
%%       have\\ learned it so in the advertising world of plastic}.)}.
%% \end{itemize}

\paragraph{Attributes.} After you have annotated the \texttt{sentiment} span,
you should next set the values of its attributes, which are summarized
in Table~\ref{tbl:sentiment}.
\begin{center}
  \begin{table}[htb!]
    \begin{tabular}{m{0.25\clmnwidth}>{\centering\arraybackslash}m{0.25\clmnwidth}m{0.92\clmnwidth}}
      \toprule
      \textbf{Attribute} & \textbf{Value} & \multicolumn{1}{c}{\textbf{Meaning}}\\\midrule
      %%%%%%%%%%%%%%%%%%%

      \multirow[c]{3}{*}[-2.5cm]{polarity} & \multirow[c]{1}{*}{\textit{positive}} & sentiment expresses a positive attitude to
      its respective target, \eg{} \textit{Es war ein fantastischer
        Abend (It was a fantastic evening)};\\\cmidrule{2-3}

      & \textit{\shortstack{negative\\(default)}} & sentiment expresses a
      negative attitude to its respective target, \eg{} \textit{Seine
        Schwester ist einfach unausstehlich (His sister is simply
        obnoxious)}\\\cmidrule{2-3}

      & \textit{comparison} & sentiment
      expresses a comparison of two objects with preference given to
      one of them, \eg{} \textit{Mir gef\"allt das rote Kleid mehr als
        das blaue (I like the red dress more than the blue
        one)}\\\midrule

      %%%%%%%%%%%%%%%%%%%

      \multirow[c]{3}{*}[-1.5cm]{intensity} & \textit{weak} & sentiment expresses a weak evaluative opinion,
      \eg{} \textit{Der Auftritt war mehr oder weniger gut (The
        appearance was more or less good)}\\\cmidrule{2-3}

      & \textit{\shortstack{medium\\(default)}} & sentiment has a middle
      emotional expressivity, \eg{} \textit{Mir hat das neue Album gut
        gefallen (I enjoyed the new album)}\\\cmidrule{2-3}

      & \textit{strong} & sentiment
      expresses a very emotional polar statement, \eg{} \textit{Dieses
        Festival war einfach umwerfend!!! (This festival was simply
        terrific!!!)}\\\midrule
      %%%%%%%%%%%%%%%%%%%

       & \textit{true} & the opinion is
      derisive, \ie{} its actual polarity is the opposite of its
      literal meaning, although there are no immediate modifiers in
      the nearby context.  An example of a sarcastic sentiment is the
      following passage: \textit{Mein J\"ungerer ist in der Pr\"ufung
        durchgefallen.  Klasse! (My youngest has failed his exam.
        Well done!)}  In this case, you should set the polarity
      attribute of the sentiment to \texttt{negative} and the value of
      the \texttt{sarcasm} attribute to \texttt{true}.\\\cmidrule{2-3}

      \multirow[c]{-2}{*}[2.7cm]{sarcasm} &
      \textit{\shortstack{false\\(default)}} & no sarcasm is
      present---polar attitude has its literal meaning.\\\bottomrule
    \end{tabular}
    \caption{Attributes of \texttt{sentiment}s}\label{tbl:sentiment}
  \end{table}
\end{center}

%%%%%%%%%%%%%%%%%%%%%%%%%%%%%%%%%%%%%%%%%%%%%%%%%%%%%%%%%%%%%%%%%%%%%%%%%%%%%%%%%%%%%%%%%%
\tocless\subsection{target}
\paragraph{Definition.} \emph{Targets} are objects or events
that are evaluated by a sentiment.

Because sentiments are required to be evaluative, there always must be
at least one target for each \texttt{sentiment} element.

\paragraph{Example.} An example of a sentiment target is given in
sentence~\ref{exmp:target}:
\begin{example}
Mein Bruder ist nicht begeistert von \target{dem neuen Call of Duty}.

(My brother is not impressed by \target{the new Call of
  Duty}.)\label{exmp:target}
\end{example}
In this message, the author tells us about the opinion of her brother
regarding the new version of a computer game.  The computer game is
the object of this evaluation, so you shall label it as a
\texttt{target}.

\paragraph{Boundaries.} As for \texttt{sentiment}s, you have to put
the \texttt{target} tags around the minimal complete syntactic or
discourse-level unit that denotes the evaluated entity or event.
These are usually noun phrases (\eg{} \textit{Mir wird's schlecht,
  wenn ich \target{diese Werbung} im Fernsehen sehe} [\textit{I feel
    sick when I see this \target{ad} on TV}]) or clauses (\eg{}
\textit{Ich hasse wenn \target{Voldemort mein Shampoo benutzt}.}
       [\textit{I hate when \target{Voldemort is using my shampoo}}]).

If a sentiment has multiple targets, you shall label each one of them
separately (see Example~\ref{exmp:trg-conj}).
\begin{example}
  Meiner Mutter haben \target{Nelken} und \target{Dahlien} immer gefallen.

  (My mother has always liked \target{carnations} and
  \target{dahlias}.)\label{exmp:trg-conj}
\end{example}
Similarly, in comparisons, you have to annotate each compared object
with a separate tag.  In addition to that, you should also set the
value of the \texttt{preferred} attribute to \texttt{false} for the
object that is dispreferred in the comparison (see Example~\ref{exmp:trg-comp}).
\begin{example}
  Ich mag \target[preferred=true]{Domino-Eis} mehr als
  \target[preferred=false]{Magnum}.

  (I like \target[preferred=true]{Domino ice cream} more than
  \target[preferred=false]{Magnum}.)\label{exmp:trg-comp}
\end{example}

\paragraph{Attributes.} Further possible attributes of \texttt{target}s
are given in Table~\ref{tbl:target}.
\begin{center}
  \begin{table}[htb!]
    \begin{tabular}{m{0.25\clmnwidth}>{\centering\arraybackslash}m{0.25\clmnwidth}m{0.92\clmnwidth}}\toprule
      \textbf{Attribute} & \textbf{Value} & \multicolumn{1}{c}{\textbf{Meaning}}\\\midrule

      \multirow{-2}{*}[-2cm]{preferred} &
      \textit{\shortstack{true\\(default)}} & in comparisons,
      this value means that the respective target is considered better
      than another compared object, \eg{} \textit{\emph{Die neue
          Frisur} passt ihr garantiert besser als die alte (\emph{The
          new hairstyle} suits her definitely better than the old
        one)};\\\cmidrule{2-3}

      & \textit{false} & in comparisons,
      this value signifies the target element that is considered worse
      than its counterpart, \eg{} \textit{Die zweite Saison von
        Breaking Bad war viel spannender als \emph{die dritte} (The
        second season of Breaking Bad was much more exciting than
        \emph{the third one})};\\\midrule

      sentiment-ref & \textit{\shortstack{$\longrightarrow$\\(directed
        edge)}} & a directed edge pointing from \texttt{target} to its
      respective \texttt{sentiment}.  You need to draw this edge in
      two cases:
      \begin{itemize}
      \item when the \texttt{target} is located at intersection of two
        different \texttt{sentiment}s (in this case, you should draw
        an edge from \texttt{target} to \texttt{sentiment}, which this
        \texttt{target} actually belongs to),

      \item when the target of an opinion is expressed outside the
        \texttt{sentiment} span;
      \end{itemize}\\\midrule

      anaph-ref & \textit{\shortstack{$\longrightarrow$\\(directed edge)}} &
      a directed edge pointing from \texttt{target} expressed by a
      pronoun or pronominal adverb to its respective non-pronominal
      antecedent (in order to draw this edge, you also need to
      annotate the antecedent as \texttt{target})\\\bottomrule
    \end{tabular}
    \caption{Attributes of \texttt{target}s}\label{tbl:target}
  \end{table}
\end{center}

%%%%%%%%%%%%%%%%%%%%%%%%%%%%%%%%%%%%%%%%%%%%%%%%%%%%%%%%%%%%%%%%%%
%% Source
\tocless\subsection{source}
\paragraph{Definition.} Sentiment \emph{sources} are immediate author(s) % chktex 36
or holder(s) of evaluative opinions.  These are typically the author % chktex 36
of a message, or officials whose opinion is cited.

If sentiment's holder is not explicitly mentioned in the tweet, it is
implicitly assumed that it is the user who wrote that microblog, and
you need not annotate anything as a source in this case.

\paragraph{Example.} An example of an explicitly mentioned source
is the pronoun \textit{Sie} (\textit{she}) in the following sentence.
\begin{example}
  \source{Sie} mag die neue Farbe nicht

  (\source{She} doesn't like the new color)\label{exmp:source}
\end{example}

Note that in citations you should only label the immediate person or
the institution whose original opinion is cited, but should not
annotate the citing person as a \texttt{source} (see
Example~\ref{exmp:source-citation}).
\begin{example}
  Laut Staatsanwalt soll die \source{Angeklagte} sich missbilligend \"uber
  ihren Vorgesetzten ge\"au\ss{}ert haben.

  (According to the attorney, the \source{defendant} had made
  disapproving remarks about her boss.)\label{exmp:source-citation}
\end{example}

\paragraph{Boundaries.} For determining the boundaries of
\texttt{source}s, you should proceed in a similar way as you did for
\texttt{target}s and \texttt{sentiment}s, \ie{} only annotate complete
minimal syntactic units.  Sources are most commonly expressed by noun
phrases.  As with \texttt{target}s, if the source of a sentiment is
expressed by multiple separate noun phrases, you should label each of
them separately (see Example~\ref{exmp:source2}).

\begin{example}
  \source{Ihr} und \source{ihrer Mutter} gef\"allt die neue Farbe
  nicht.\\ (Neither \source{she} and \source{her mother} likes the new
  color)\label{exmp:source2}
\end{example}

\paragraph{Attributes.} The attributes of the \texttt{source}
tag are fully identical to the attributes of the \texttt{target}
elements and are recapped in Table~\ref{tbl:source}.
\begin{center}
  \begin{table}[htb!]
    \begin{tabular}{m{0.25\clmnwidth}>{\centering\arraybackslash}m{0.25\clmnwidth}m{0.92\clmnwidth}}\toprule
      \textbf{Attribute} & \textbf{Value} & \multicolumn{1}{c}{\textbf{Meaning}}\\\midrule

      sentiment-ref & \textit{\shortstack{$\longrightarrow$\\(directed
        edge)}} & see Table~\ref{tbl:target}\\\midrule

      anaph-ref & \textit{\shortstack{$\longrightarrow$\\(directed edge)}} &
      see Table~\ref{tbl:target}\\\bottomrule
    \end{tabular}
    \caption{Attributes of \texttt{source}s}\label{tbl:source}
  \end{table}
\end{center}

%%%%%%%%%%%%%%%%%%%%%%%%%%%%%%%%%%%%%%%%%%%%%%%%%%%%%%%%%%%%%%%%%%
%% polar-term
\tocless\subsection{polar-term}
\paragraph{Definition.} \emph{polar-terms} are words or
phrases that have an inherent evaluative meaning.

\paragraph{Example.} An example of a polar-term is the word
\textit{ekelhaft} (\textit{disgusting}) in sentence~\ref{exmp:emo-expr1}.
\begin{example}
  Beim Aufr\"aumen des Zimmers haben wir einen
  \emoexpression{ekelhaften} Teller mit verschimmeltem Essen unter dem
  Bett gefunden.

  (When we cleaned the room, we found a \emoexpression{disgusting}
  plate with moldy food under the bed.)\label{exmp:emo-expr1}
\end{example}

In contrast to \texttt{source}s and \texttt{target}s, which should
only be annotated in the presence of a \texttt{sentiment}, you always
have to label polar terms in the text irrespective of any other
tags.

Note, however, that because many words and idioms are ambiguous and
can have several different meanings, it can often be the case that
only some of these meanings are evaluative and subjective.  In such
cases, you should only label such words if their actual sense in the
given context is polar.  If these words denote an objective entity or
fact, you must not use this tag.
\begin{example}
  Dieser Wein ist ein echtes \emoexpression{Juwel} in meiner
  Kollektion.

  (This wine is a real \emoexpression{jewel} in my collection.)

  Koh-i-Noor ist das teuerste Juwel heutzutage.

  (Koh-i-Noor is the most expensive jewel nowadays.)\label{exmp:polar-term-jewel}
\end{example}
In Example~\ref{exmp:polar-term-jewel}, for instance, the meaning of
the word \textit{Juwel} (\textit{jewel}) is metaphoric and subjective
in the first sentence, but literal and objective in the second
statement.  So you should only annotate this word as
\texttt{polar-term} in the former case, but disregard it in the
latter.

\paragraph{Boundaries.} \texttt{polar-term}s are typically
expressed by:
\begin{itemize}
  \item nouns, \eg{} \textit{Held (hero)}, \textit{Ideal (ideal)},
    \textit{Betr\"uger (fraudster)};

  \item adjectives or adverbs, \eg{} \textit{sch\"on (nice)},
    \textit{zuverl\"assig (reliably)}, \textit{hinterh\"altig
      (devious)}, \textit{heimt\"uckisch (insidiously)};

  \item verbs, \eg{} \textit{lieben (to love)}, \textit{bewundern (to
    admire)}, \textit{hassen (to hate)};

  \item idioms, \eg{} \textit{auf die Nerven gehen (to get on one's
    nerves)};

  \item smileys, \eg{} :), :-(, \smiley{}, \frownie{}. % chktex 26 9 36
\end{itemize}
If a \texttt{polar-term} represents an idiomatic phrase, you shall
always annotate the complete idiom.  If a verb has an evaluative sense
only in conjunction with certain prepositions (\eg{} \textit{to go for
  sth.} in the sense of \textit{to like}), you shall annotate
both the verb and the preposition with a single pair of tags (check
the \texttt{MMAX} manual to see how to annotate discontinuous spans).

\paragraph{Attributes.}

When determining the \texttt{polarity} of a \texttt{polar-term}, you
should disregard any possible contextual modifiers such as
intensifiers or negations and set the value of this attribute to the
lexical (\ie{} \emph{prior}) polarity of that term (see
Example~\ref{exmp:polar-term-polarity}).
\begin{example}
Es war keine \emoexpression[polarity=positive]{gute} Idee.

(It was not a \emoexpression[polarity=positive]{good} idea.)\label{exmp:polar-term-polarity}
\end{example}
Apart from that, when determining the value of the \texttt{polarity}
attribute of a \texttt{polar-term}, you should analyze its polarity
from the perspective of the holder of the opinion towards the
evaluated object.  This means that in cases like \textit{Ich vermisse
  meine Freundin} (\textit{I miss my girlfriend}), the polarity of the
polar-term \textit{vermissen} (\textit{to miss}) is still positive
because the author has a positive attitude to his girlfriend, and
consequently feels sad about of her absence.

Further attributes of \texttt{polar-term}s include \texttt{intensity},
\texttt{sarcasm}, and \texttt{sentiment-ref}; their possible values
are summarized in Table~\ref{tbl:polar-term}.
\begin{center}
  \begin{table}[htb!]
    \begin{tabular}{m{0.25\clmnwidth}>{\centering\arraybackslash}m{0.25\clmnwidth}m{0.92\clmnwidth}}\toprule
      \textbf{Attribute} & \textbf{Value} & \multicolumn{1}{c}{\textbf{Meaning}}\\\midrule
      %%%%%%%%%%%%%%%%%%%

      \multirow{2}{*}[-0.7cm]{polarity} & \textit{positive} & polar term has a positive evaluative
      meaning, \eg{} \textit{gut (good), verhimmeln (to ensky),
        Prachtkerl (corker)} etc.\\\cmidrule{2-3}

      & \textit{\shortstack{negative\\(default)}}
      & polar term expresses a negative evaluation of its target,
      \eg{} \textit{versauen (to botch up), rotzig (snotty),
        Dreckskerl (scum)} etc.\\\midrule

      %%%%%%%%%%%%%%%%%%%

      \multirow{3}{*}[-1.1cm]{intensity} & \textit{weak} & polar-term has a weak evaluative sense, \eg{}
      \textit{solala (so-so), nullachtf\"unfzehn (vanilla),
        durchschnittlich (mediocre)} etc.\\\cmidrule{2-3}

      & \textit{\shortstack{medium\\(default)}} & polar-term has middle
      stylistic expressivity, \eg{} \textit{gut (good), schlecht
        (bad), robust (tough)} etc.\\\cmidrule{2-3}

      & \textit{strong} & polar-term
      expresses a very strong positive or negative evaluation, \eg{}
      \textit{allerbeste (bettermost), zum Kotzen (to make one puke),
        Kacke (shit)} etc.\\\midrule

      %%%%%%%%%%%%%%%%%%%%

      \multirow{2}{*}[-0.4cm]{sarcasm} & \textit{true} & polar-term is
      derisive, \ie{} its actual polarity is the opposite of its
      primary lexical sense even though there are no negations in the
      surrounding context\\\cmidrule{2-3}

      & \textit{\shortstack{false\\(default)}} & no sarcasm is present---the
      term has its literal polar meaning; this is the
      default\\\midrule

      %%%%%%%%%%%%%%%%%%%

      sentiment-ref & \textit{\shortstack{$\longrightarrow$\\(directed
        edge)}} & an arrow pointing to the \texttt{sentiment} that this
      \texttt{polar-term} belongs to.  You should only draw this edge
      if a \texttt{polar-term} is located at an intersection of two
      \texttt{sentiment}s or outside of the \texttt{sentiment} span
      that it belongs to\\\bottomrule
    \end{tabular}
    \caption{Attributes of \texttt{polar-term}s}\label{tbl:polar-term}
  \end{table}
\end{center}

%%%%%%%%%%%%%%%%%%%%%%%%%%%%%%%%%%%%%%%%%%%%%%%%%%%%%%%%%%%%%%%%%%
%% Intensifier
\tocless\subsection{intensifier}
\paragraph{Definition.} \emph{Intensifiers} are elements that increase
the expressivity or the evaluative sense of a polar term.

\paragraph{Example.}An example of intensifier is the word
\textit{sehr} (\textit{very}) in sentence~\ref{exmp:intensifier}.
\begin{example}
  Wir suchen eine \intensifier{sehr} zuverl\"assige Polin als
  Haushaltshilfe.

  (We are looking for a \intensifier{very} reliable Polish woman as
  domestic help.)\label{exmp:intensifier}
\end{example}
\paragraph{Boundaries.}

Intensifiers are usually expressed by adverbs or adjectives such as
\textit{sehr} (\textit{very}) or \textit{sicherlich}
(\textit{certainly}), but other ways of intensification are still
possible (see Example~\ref{exmp:intensifier-comp}).
\begin{example}
  Dieser Junge ist stark \intensifier{wie ein Pferd}.

  (This boy is strong \intensifier{as a
    horse}.)\label{exmp:intensifier-comp}
\end{example}

\paragraph{Attributes.} An \texttt{intensifier} must always
relate to some \texttt{polar-term}, and you always have to explicitly
show this relation by drawing a \texttt{polar-term-ref} edge from the
\texttt{intensifier} to its modified polar expression.

Further possible attributes of \texttt{intensifier}s are shown in
Table~\ref{tbl:intensifier}.
\begin{center}
  \begin{table}[htb]
    \begin{tabular}{m{0.25\clmnwidth}>{\centering\arraybackslash}m{0.25\clmnwidth}m{0.92\clmnwidth}}\toprule
      \textbf{Attribute} & \textbf{Value} & \multicolumn{1}{c}{\textbf{Meaning}}\\\midrule

      \multirow{-2}{*}[-1.25cm]{degree} & \textit{\shortstack{medium\\(default)}} & the intensifier moderately
      increases the polar sense of the polar term,
      \eg{} \textit{ziemlich (quite), recht (fairly)} etc.\\\cmidrule{2-3}

      & \textit{strong} & the intensifier strongly increases the polar
      sense and stylistic markedness of the polar term, \eg{}
      \textit{sehr (very), super (super), stark (strongly)}
      etc.\\\midrule

      %%%%%%

      polar-term-ref & \textit{\shortstack{$\longrightarrow$\\(directed
        edge)}} & a directed edge pointing from the intensifier to the
      \texttt{polar-term} whose meaning is being
      intensified\\\bottomrule
    \end{tabular}
    \caption{Attributes of \texttt{intensifier}s}\label{tbl:intensifier}
  \end{table}
\end{center}

\tocless\subsection{diminisher}
\paragraph{Definition.} \emph{Diminisher}s or \emph{downtoner}s are words or phrases
that decrease the polar lexical sense of a \texttt{polar-term}.

\paragraph{Example.} In Example~\ref{exmp:diminisher}, the
diminisher is expressed by the adverb \textit{weniger}
(\textit{less}).
\begin{example}
  \diminisher{Weniger} erfolgreiche Unternehmen verzichten auf externe
  Berater.\label{exmp:diminisher}

  The \diminisher{less} successful companies do not use external
  consultants.
\end{example}
\paragraph{Attributes.} Like intensifiers, diminishers must
always relate to a polar term, and you also have to explicitly show
this relation by using the \texttt{polar-term-ref} attribute; other
attributes of \texttt{diminisher}s mainly coincide with those of
intensifiers and are summarized in Table~\ref{tbl:diminisher}.
\begin{center}
  \begin{table}[hb]
    \begin{tabular}{m{0.25\clmnwidth}>{\centering\arraybackslash}m{0.25\clmnwidth}m{0.92\clmnwidth}}\toprule
      \textbf{Attribute} & \textbf{Value} & \multicolumn{1}{c}{\textbf{Meaning}}\\\midrule

      \multirow{2}{*}[-0.5cm]{degree} & \textit{\shortstack{medium\\(default)}} & diminisher moderately decreases
      the polar sense of its respective \texttt{polar-term},
      \eg{} \textit{wenig (few), bisschen (little)} etc.\\\cmidrule{2-3}

      & \textit{strong} & diminisher strongly
      decreases the polar sense of the \texttt{polar-term},
      \eg{} \textit{kaum (hardly)} etc.\\\midrule

      polar-term-ref & \textit{\shortstack{$\longrightarrow$\\(directed
        edge)}} & see Table~\ref{tbl:intensifier}\\\bottomrule
    \end{tabular}
    \caption{Attributes of \texttt{diminisher}s}\label{tbl:diminisher}
  \end{table}
\end{center}

\tocless\subsection{negation}
\paragraph{Definition.} \emph{Negation}s are elements that turn
the polarity of a \texttt{polar-term} to the opposite.

\paragraph{Example.} In Example~\ref{exmp:negation}, for
instance, the negative article \textit{kein} (\textit{not}) makes the
\emph{contextual} polarity of the word \textit{interessant}
(\textit{interesting}) negative, even though the prior semantic
orientation of this term is positive.
\begin{example}
Diese Geschichte war \"uberhaupt nicht \negation{interessant}!

This story was \negation{not} interesting at all!\label{exmp:negation}
\end{example}

The role of negations is closely related to that of diminishers.  In
order to help you better distinguish between these entities, we have
listed the most obvious differences between the two elements:
\begin{itemize}
  \item\textit{Semantic differences:} diminishers only decrease the
    lexical sense of an polar-term, a but part of its original sense
    still remains active (\ie{} \textit{a hardly understandable
      speech} is still understandable); negations, on the other hand,
    fully deny that meaning and turn it to the complete opposite
    (\textit{a not understandable speech} is absolutely
    unintelligible);

  \item\textit{Part-of-speech differences:} diminishers are usually
    expressed by adjectives or adverbs, whereas negations are
    typically represented by the negative article \textit{kein (no)},
    the negation particle \textit{nicht (not)}, or verbs or
    adjectives, \eg{} \textit{Es ist sehr zweifelhaft, dass die neue
      Version von Windows besser wird} (\textit{It is very doubtful
      that the new Windows version will be any better})

  %% \item\textit{Syntactic differences}.  While diminishers are usually
  %%   expressed by either adjectives or adverbs; negations, on the contrary,
  %%   are commonly represented by either the negative article \textit{kein
  %%   (no)} or the negation particle \textit{nicht (not)}, or even complete
  %%   clauses (\textit{Er ist nicht einverstanden, dass die neue Version von
  %%   Windows besser ist (He disagrees that the new Windows version is any
  %%   better)}).
\end{itemize}

\paragraph{Attributes.} The only attribute of negations is the
mandatory edge \texttt{polar-term-ref}.  You have to draw this edge
from the \texttt{negation} to that \texttt{polar-term} that is
negated.  Like intensifiers and diminishers, negations must always
refer to at least one polar item.
\begin{center}
  \begin{table}
    \begin{tabular}{m{0.25\clmnwidth}>{\centering\arraybackslash}m{0.25\clmnwidth}m{0.92\clmnwidth}}\toprule % noqa
      \textbf{Attribute} & \textbf{Value} &
      \multicolumn{1}{c}{\textbf{Meaning}}\\\midrule polar-term-ref &
      \textit{\shortstack{$\longrightarrow$\\(directed edge)}} & an edge from
      \texttt{negation} to the \texttt{polar-term} being
      negated\\\bottomrule
    \end{tabular}
    \caption{Attributes of \texttt{negation}s}\label{tbl:negation}
  \end{table}
\end{center}

\tocless\section{Summary}\label{sec:summary}

Summarizing all of the above, your task in this assignment is to find
subjective evaluative opinions about some entities or events.  You
need to annotate these opinions with the \texttt{sentiment} tags and
also determine the polarity and the intensity of the expressed
attitudes.  After that, you should assign the \texttt{target} tags to
objects or events that are evaluated, and label the holders of these
attitudes as \texttt{source}s.  Both, \texttt{source}s and
\texttt{target}s, can only exist in the presence of a
\texttt{sentiment}.

Another important task is to annotate words and phrases that have a
polar evaluative meaning.  We call these words \texttt{polar-term}s,
and you need to annotate them always, regardless of whether there is a
targeted sentiment or not.  If a \texttt{polar-term} is intensified,
diminished, or negated by another word or phrase, you should also
annotate the modifying element as well.

\tocless\section{Examples}\label{sec:examples}

We conclude these guidelines with a couple of real-world annotation
examples from our corpus, explaining our decisions for these
annotations.

\begin{example}
  \footnotesize WAS HABEN ALLE MIT
  \sentiment[polarity=negative,intensity=strong,sarcasm=false]{IHREN\\
    \emoexpression[polarity=negative,intensity=strong,sarcasm=false]{VERF*CKTEN}\\
    \target{GR\"UNEN AUGEN}}

  {\scriptsize(WHAT DO THEY ALL HAVE WITH
    \sentiment[polarity=negative,intensity=strong,sarcasm=false]{THEIR\\
      \emoexpression[polarity=negative,intensity=strong,sarcasm=false]{F*CKED}\\
      \target{GREEN EYES}})}\label{exmp:sarcasm}
\end{example}

\textbf{Explanation:} In this case, there is an evaluative opinion
about the green eyes of some persons.  It is, however, unclear what is
the author's attitude to the people themselves, we only can see that
she thinks that the eyes of these people are \textit{verf*ckt}
(\textit{f*cked}).  Therefore, the \texttt{target} of this sentiment is
the word \textit{Augen} (\textit{eyes}), and the \texttt{sentiment}
span should enclose the noun phrase comprising that target and its
evaluative term \textit{f*cked}. Since the polar term is an intense
abusive word, we set the polarity of this word and its enclosing
\texttt{sentiment} to \texttt{negative} and the intensity of both tags
to \texttt{strong}.\footnote{In the cases where we do not specify an
  attribute in the example, this attribute is assumed to have the
  default value.}

\begin{example}
  \footnotesize\sentiment[polarity=negative,intensity=medium,sarcasm=true]{Wo
    ist der
    \emoexpression[polarity=positive,intensity=strong,sarcasm=true]{\#Jubel}
    von \target{\#CDU} \target{\#CSU} \& \target{\#FDP} \"uber den Tod
    der Mieterin nach\\
    \#Zwangsr\"aumung?}

  {\scriptsize\sentiment[polarity=negative,intensity=medium,sarcasm=true]{Where
      is the
      \emoexpression[polarity=positive,intensity=strong,sarcasm=true]{\#exultation}
      of \target{\#CDU} \target{\#CSU} \& \target{\#FDP} about the
      death of the renter after
      forced\\ \#eviction?}}\label{exmp:sarcasm}
\end{example}

\textbf{Explanation:} In Example~\ref{exmp:sarcasm}, we have not
labeled \textit{Jubel von \#CDU~\ldots{} \"uber den Tod von \ldots{}}
(\textit{the exultation of the \#CDU~\ldots{} about the death
  of~\ldots{}}) as \texttt{sentiment}, because the truth status of
this statement is unknown.  But, on the other hand, the mere
hypothesis that a political party could experience a glee feeling
because of a renter's death is sarcastic.  We can recognize it from
the \texttt{polar-term} \textit{\#Jubel} (\textit{\#exultation}),
whose prior semantic orientation is positive, but which suggests a
negative attitude to the CSU party in this context, without any
explicit contextual modifiers.  Accordingly, we set the (prior)
\texttt{polarity} of the term to \texttt{positive}, the polarity of
its \texttt{sentiment} to \texttt{negative}, and the \texttt{sarcasm}
attribute of both labels to \texttt{true}.  Apart from having
different polarities, \texttt{sentiment} and \texttt{polar-term} also
have different intensities: since \textit{\#Jubel}
(\textit{\#exultation}) expresses a higher degree of excitement than
the word \textit{Freude} (\textit{joy}), we set its \texttt{intensity}
to \texttt{high}.  On the other hand, the overall sentiment expression
is rather subtle and does not show high exaggeration of the author.
So, we set the \texttt{intensity} of the \texttt{sentiment} to
\texttt{medium} rather than \texttt{high}.

Another non-trivial case is shown in Example~\ref{exmp:nested-2},
which we will analyze step by step:
\begin{example}
  \footnotesize RT @JochenFlasbarth~: Guter \#Spiegel-Titel~, wie
  Welzer~, Sloterdijk und andere Promi \#Nichtw\"ahler die Demokratie
  verspielen~: Tr\"age~, frustriert

  {\scriptsize(RT @JochenFlasbarth~: A good \#Spiegel title~, how
    Welzer~, Sloterdijk, and other celebrity non-voters squander the
    democracy~: Sluggish~, frustrated)}\label{exmp:nested-2}
\end{example}

\textbf{Explanation:} First of all, we have to look at words with
unambiguous lexical polarity (polar-terms), as they are our primary
cues for detecting sentiments.  This tweet features one positive
terms, \textit{guter} (\textit{good}), and there negative polar items,
\textit{verspielen} (\textit{to squander}), \textit{tr\"age}
(\textit{sluggish}), and \textit{frustriert} (\textit{frustrated}).
Since we have two sets of polar-terms with contradicting polarities,
it is most likely that there also are two sentiments---one positive
and one negative.  The positive evaluation obviously pertains to the
suggested \#Spiegel title ``wie Welzer~, Sloterdijk und andere Promi
\#Nichtw\"ahler die Demokratie verspielen: Tr\"age~, frustriert''
(``\textit{how Welzer~, Sloterdijk, and other celebrity non-voters
  squander the democracy: Sluggish~, frustrated}'').  The author finds
this title good, and the annotation of this sentiment then looks as
follows:
\begin{example}
  \sentiment[polarity=positive,intensity=medium,sarcasm=false,id=1]{RT
    \source[sentiment\_ref=1]{@JochenFlasbarth}~:\\ \emoexpression[polarity=positive,intensity=medium,sarcasm=false,
      sentiment\_ref=1]{Guter} \#Spiegel-Titel~,
    \target[sentiment\_ref=1]{wie Welzer~, Sloterdijk und andere Promi
      \#Nichtw\"ahler die Demokratie verspielen~:
      Tr\"age~,\\ frustriert}}

  (\sentiment[polarity=positive,intensity=medium,sarcasm=false,id=1]{RT
    \source[sentiment\_ref=1]{@JochenFlasbarth}~: A\\
    \emoexpression[polarity=positive,intensity=medium,sarcasm=false,
      sentiment\_ref=1]{good} \#Spiegel title~,
    \target[sentiment\_ref=1]{how Welzer~, Sloterdijk, and other
      celebrity non-voters squander the democracy~: Sluggish~,\\
      frustrated}})
\end{example}
The negative opinion, which is expressed by the terms
\textit{verspielen} (\textit{to squander}), \textit{tr\"age}
(\textit{sluggish}), and \textit{frustriert} (\textit{frustrated}),
obviously relates to the celebrity non-voters, \textit{Welzer} and
\textit{Sloterdijk}; so, we annotate this evaluation as:

\begin{example}
  \scriptsize
  \sentiment[{\tiny polarity=negative,intensity=medium,sarcasm=false,id=2}]{RT
    \source[sentiment\_ref=2]{@JochenFlasbarth}~: Guter
    \#Spiegel-Titel~, wie\\\target[sentiment\_ref=2]{Welzer}~,
    \target[sentiment\_ref=2]{Sloterdijk}\\und
    \target[sentiment\_ref=2]{andere Promi \#Nichtw\"ahler} die
    Demokratie\\
    \emoexpression[polarity=negative,intensity=medium,sarcasm=false,sentiment\_ref=2]{verspielen}~:\\
    \emoexpression[polarity=negative,intensity=medium,sarcasm=false,sentiment\_ref=2]{Tr\"age}~,\\ \emoexpression[polarity=negative,intensity=medium,sarcasm=false,sentiment\_ref=2]{frustriert}\\
  }

  {\scriptsize(\sentiment[polarity=negative,intensity=medium,sarcasm=false,id=2]{RT
      \source[sentiment\_ref=2]{@JochenFlasbarth}~: A good \#Spiegel
      title~, how\\\target[sentiment\_ref=2]{Welzer}~,
      \target[sentiment\_ref=2]{Sloterdijk},\\ and
      \target[sentiment\_ref=2]{other celebrity non-voters}\\
      \emoexpression[polarity=negative,intensity=medium,sarcasm=false,sentiment\_ref=2]{squander}
      the democracy~:\\
      \emoexpression[polarity=negative,intensity=medium,sarcasm=false,
        sentiment\_ref=2]{Sluggishly}~,\\
      \emoexpression[polarity=negative,intensity=medium,sarcasm=false,
        sentiment\_ref=2]{frustrated}\\
    })}
\end{example}

In both cases, \textit{@JochenFlasbarth} is the original author of the
cited opinion, so we should label it as a \texttt{source}.  But since
there are two sentiment relations, we assign this tag twice, drawing
an edge (in our example denoted by attribute \texttt{sentiment\_ref})
to the respective \texttt{sentiment} element in each case.}
