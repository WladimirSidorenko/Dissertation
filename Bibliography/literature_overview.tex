\documentclass[a4paper,11pt]{article}

%%%%%%%%%%%%%%%%%%%%%%%%%%%%%%%%%%%%%%%%%%%%%%%%%%%%%%%%%%%%%%%%%%
%% Packages
\usepackage{paralist}

%%%%%%%%%%%%%%%%%%%%%%%%%%%%%%%%%%%%%%%%%%%%%%%%%%%%%%%%%%%%%%%%%%
%% Variables and Methods

%%%%%%%%%%%%%%%%%%%%%%%%%%%%%%%%%%%%%%%%%%%%%%%%%%%%%%%%%%%%%%%%%%
%% Documents
\begin{document}
\section{Pre-Processing}

\section{Sentiment Analysis}
\subsubsection{Turney: Thumbs Up or Thumbs Down? Semantic Orientation Applied to
               Unsupervised Classification of Reviews\cite{Turney02}}

\subsubsection{Bing and Liu: Mining and summarizing customer reviews\cite{Bing-Liu-04}}
In their article, Bing and Liu \cite{Bing-Liu-04} propose a system for
automatically mining subjective judgements on product features in customer
reviews.  This system searches the reviews for frequently occurring mentions
of prominent features of specific products by using an association mining
algorithm.  The algorithm extracts nouns and noun phrases which frequently
occur across multiple documents.  Implicit features and features which are not
represented by nouns are not taken into consideration.  Extracted feature
candidates are then pruned off if their components do not occur in a specific
order in sentences or if some features appear to be full subsets of other
ones.

In the following processing step, text passages surrounding feature candidates
are searched for adjectives, since words of this part-of-speech are considered
to be the most obvious indicators of subjective opinions.  Found adjectives
are classified into subjective and objective ones.  The subjective class is
thereby subdivided into positive and negative subclasses depending on the type
of adjectives' polarity.  Both subjectivity and polarity are determined by
using WordNet information.  For this, two manually constructed sets of
adjectives with opposite polarities are used.  Synonyms and antonyms of
adjectives found in text are searched for occurrences of any adjectives from
one of the two seed sets.  If such an occurrence is found, the text adjective
is added to the respective seed set or its counterpart depending on whether
the occurrence was found in synonyms or antonyms.  The algorithm stops when
both seed sets stop growing or when the list of text adjectives is exhausted.
Adjectives which were not assigned to any of the seed cluster during this
procedure are assumed to have neutral polarity and are skipped.

In the next step, sentences which don not contain any frequent features but
which contain at least one opinion adjective are analyzed.  The aim of this
analysis is to find less frequent features of a product which still can be
important for the user.  It is conjectured by the authors that less frequent
features will usually be expressed by nouns or noun phrases which are nearest
to the opinion word.

Further, overall polarity of sentences containing frequent and less frequent
product features is determined.  Here three different possible cases are
distinguished:
\begin{enumerate}
  \item The user likes or dislikes most or all the features in one sentence.
  \item The user likes or dislikes most of the features in one sentence, but
    there is an equal number of positive and negative opinion words.
  \item All the other cases.
\end{enumerate}
For case 1, the dominant orientation of a sentence is easily identified as
average orientation of adjectives used in this sentence.  For case 2, the
orientation of effective opinions of features is used instead.  Effective
opinion is assumed to be the most related opinion for a feature.  For case 3,
the orientation of the opinion sentence is assumed to be the same as the
orientation of previous opinion sentence unless the former sentence is
introduced by a contrastive connector like ``but'' or ``however'' in which
case the orientation is swapped to the opposite.

In the final step, a review summary is generated for each sentence containing
a subjective opinion on a frequent or less frequent feature of a product.  The
summary generation procedure generally consists of two steps:
\begin{itemize}
  \item For each discovered feature, related opinion sentences are put into
    positive and negative categories according to the opinion sentences'
    orientations.

  \item All features are ranked according to the frequency of their
    appearances in the reviews.
\end{itemize}

\section{Markov Logic Network}

\section{Discourse Analysis}

\nocite{*}
\bibliography{bibliography}     % relative to the project's root directory
\bibliographystyle{plain}
\end{document}
