% FILE: abstract.tex  Version 2.1
% AUTHOR:
% Universit�t Duisburg-Essen, Standort Duisburg
% AG Prof. Dr. G�nter T�rner
% Verena Gondek, Andy Braune, Henning Kerstan
% Fachbereich Mathematik
% Lotharstr. 65., 47057 Duisburg
% entstanden im Rahmen des DFG-Projektes DissOnlineTutor
% in Zusammenarbeit mit der
% Humboldt-Universitaet zu Berlin
% AG Elektronisches Publizieren
% Joanna Rycko
% und der
% DNB - Deutsche Nationalbibliothek
%% \begin{abstract}
%% Here is the english abstract.
%% \end{abstract}
%-englische-Zusammenfassung---------------------------------------
\selectlanguage{english}
\section*{Abstract}
\addcontentsline{toc}{section}{Abstract}
This thesis presents a comprehensive study of politics-related discussions on
German Twitter.  In our work, we analyze political online discourses from both
liguistic and computational points of views by analyzing manually annotated
corpora from the linguistic perspective and finding the most efficient
solutions for imposed questions from the computational perspective.


investigate how much the language used on the Web differs
from standard used in newspaper articles; how strong the ungrammaticality of
users' posts affects the performace of existing NLP-applications; what ways of
expressing opionions

the peculiarities of political online
discourses from both linguistic and computational points of view.


We use linguistic knowledge for creating manually annotated text corpora,
resolving controversial annotation cases and classifying different phenomena
found in text.  We uilize our knowledge of computational algorithms, data
structures, and probability theory for finding optimal ways of dealing with
Twitter-specific phenomena in automatic text processing applications.

We first investigate how much the language used in tweets differs from
standard German used in newspaper articles.  We do that from both linguistic
and computational point of view by measuring the rate of out-of-ocabulary
tokens for two most popular open-source text processing tools
(\texttt{TreeTagger} \cite{} and \texttt{hunspell})
