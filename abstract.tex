% FILE: abstract.tex  Version 2.1
% AUTHOR:
% Universit�t Duisburg-Essen, Standort Duisburg
% AG Prof. Dr. G�nter T�rner
% Verena Gondek, Andy Braune, Henning Kerstan
% Fachbereich Mathematik
% Lotharstr. 65., 47057 Duisburg
% entstanden im Rahmen des DFG-Projektes DissOnlineTutor
% in Zusammenarbeit mit der
% Humboldt-Universitaet zu Berlin
% AG Elektronisches Publizieren
% Joanna Rycko
% und der
% DNB - Deutsche Nationalbibliothek
%% \begin{abstract}
%% Here is the english abstract.
%% \end{abstract}
%-englische-Zusammenfassung---------------------------------------
\selectlanguage{english}
\section*{Abstract}
\addcontentsline{toc}{section}{Abstract}
This thesis presents a comprehensive study of politically oriented discussions
on German Twitter.  In our work, we investigate how much users' conversations
about political topics on this popular social media platform differ from
standard language texts found in official documents and newspapers.  We
perform this analysis from both computational and linguistic perspectives on
three different NLP-subtasks with distinct levels of granularity.

On the first level of analysis -- which we also call \emph{subsentential} --
we investigate how well existing NLP-applications, such as sentence splitters,
tokenizers, PoS-taggers, and syntactic parsers, perform on tweets and what
difficulties may be encountered by these tools in short messages which usually
do not occur in other text genres.  In order to assess the unconventionality
of Twitter posts quantitatively, we estimate the rate of out-of-vocabulary
(OOV) tokens in users' messages, classify found OOV's, and suggest several
techniques of how to deal with such phenomena in order to mitigate their
negative effect on the performance of NLP-applications.

We then turn to the \emph{sentential} level of automatic text analysis on the
example of the sentiment extraction task.  Here, we first present a novel
comprehensive corpus of German tweets manually annotated with fine-grained
sentiment relations.  From the psychological and linguistic perspectives, we
first show how difficult the task of sentiment annotation is for humans, what
language phenomena are typically used to express evaluative opinions, and by
which linguistic means the polarity and intensity of evaluations can be
affected.  With these observations in mind, we then demonstrate how well the
sentiment classification task can be done automatically using machine learning
and lexicon-based techniques.  In our experiments, we consder three related
subtasks:
\begin{inparaenum}[\itshape a)\upshape]
 \item automatic classification of prior polarity of words,
 \item machine learning-based prediction of fine-grained sentiment
 constituents (i.e. opinion holders, targets, and evaluation spans), and,
 finally,
 \item automatic detection of sentiment intensity and polarity.
\end{inparaenum}

In an attempt to improve the performance of our automatic sentiment extraction
system, we focus our attention on the \emph{suprasentential}
or \emph{discourse} level of natural language analysis and show how difficult
the automatic recognition of discourse segments and relations between those
segments is for Twitter.  In this part of the work, we adapt the traditional
definition of the rhetorical structure theory (\shortcite{Mann-88}) to the
peculiarities of tweets and extend this formalism by adding dialogue relations
for definining relations between tweets.


In the concluding step, we demonstrate how different levels of analysis might
beefit from each other by showing the impact of the improvements in low-level
processing and incorporation of discourse information on the performance of
the sentiment extraction module.
