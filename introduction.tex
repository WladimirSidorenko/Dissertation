% FILE: introduction.tex  Version 0.01
% AUTHOR: Uladzimir Sidarenka

% This is a modified version of the file main.tex developed by the
% University Duisburg-Essen, Duisburg, AG Prof. Dr. G�nter T�rner
% Verena Gondek, Andy Braune, Henning Kerstan Fachbereich Mathematik
% Lotharstr. 65., 47057 Duisburg entstanden im Rahmen des
% DFG-Projektes DissOnlineTutor in Zusammenarbeit mit der
% Humboldt-Universitaet zu Berlin AG Elektronisches Publizieren Joanna
% Rycko und der DNB - Deutsche Nationalbibliothek

\chapter*{Introduction}
%% \epigraph{\itshape Romance should never begin with sentiment. It should begin
%%   with science and end with a settlement.}{Oscar Wilde: \textit{An Ideal
%%     Husband}}


The growing popularity of online conversation platforms along with the
increasing accessibility of their data to the research community made
Internet-based communication (IBC) an attractive research object for many
scientific disciplines, including economics, political science, sociology etc.
But dealing with the abundance of unstructured text data in manual way, as it
used to be the case in several human sciences for many years, is both
cumbersome and inefficient.  Therefore, if we want to understand the way the
Internet-aware society works, we urgently need new effecient automatic natural
language processing (NLP) tools which are able to deal even with such
``unconventional'' linguistic variations as the Web language.

Unfortunately, until the last few years, online conversations have not
attracted as much attention of the scientific NLP community as they probably
deserved.  This concerns both the number of theoretical works and the number
of practical applications adapted to the peculiarities of the online language.
Furthermore, this situation becomes even more aggravated for languages other
than English.  In our thesis, we aim to fill the existing theoretical and
practical gaps in the research of the German Web sociolect.  We conduct our
analysis on the basis of Twitter\footnote{\url{https://twitter.com/}}
microblogs.  The reason for choosing this platform is, on the one hand, its
rapidly growing popularity and, on the other hand, the richness of its
messages in many diverse IBC phenomena.
%% The reasons for choosing this type of social media were, on the one hand,
%% the rapidly growing popularity of the platform and, on the other hand, the
%% high density and diversity of IBC phenomena in users' posts found there.

From the linguistic perspective, we investigate which language features
actually account for the peculiarities of the online sociolect making it
different from the standard language form.  From the computational
perspective, we analyze how these features affect the performance of existing
NLP applications and how this performance could be improved.  And since
language is not a uniform object but much more a hierarchical system, we also
perform our study on three different levels of the linguistic analysis:
\begin{inparaenum}[i)]
        \item the \emph{subsentential},
        \item the \emph{sentential}, and
        \item the \emph{suprasentential} one.
\end{inparaenum}


%% When Oscar Wilde's famous masterpiece ``An Ideal Husband'' was first performed
%% on the stage in 1895, its audience hardly knew anything about the Internet,
%% Twitter or any other kind of social networks. What people at that time,
%% however, knew about was politics and what they felt were real emotions.

%% Today, the world has changed.  It is now much more likely to meet a person who
%% knows almost everything about followers, likes and re-tweets on Twitter or
%% Facebook than one who knows who Oscar Wilde is or who cares about political
%% issues in the local government.

%% The worlds did not diverge completely though.  There are still people
%% interested in political maters and many of them are also fimiliar with the
%% Internet-based communication means.  But even more important is that we still
%% feel emotions, only that the ways of expressing our feelings have probably
%% changed -- from face to face to screen to screen mode.

%% This dissertation investigates the cut surface of emotions, politics, and
%% web-based talks.

\section{Terminology}
\section{Hypotheses}
\section{Methodology}
\section{Research Overview}
