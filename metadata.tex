% FILE: metadata.tex Version 2.1
% AUTHOR:
% Universit�t Duisburg-Essen, Standort Duisburg
% AG Prof. Dr. G�nter T�rner
% Verena Gondek, Andy Braune, Henning Kerstan
% Fachbereich Mathematik
% Lotharstr. 65., 47057 Duisburg
% entstanden im Rahmen des DFG-Projektes DissOnlineTutor
% in Zusammenarbeit mit der
% Humboldt-Universitaet zu Berlin
% AG Elektronisches Publizieren
% Joanna Rycko
% und der
% DNB - Deutsche Nationalbibliothek

%-------Eingabe der Metadaten-------------------------------------------
% Bitte f�llen Sie alle Angaben aus, f�r die dies m�glich ist. Falls
% Ihre Arbeit z.B. keinen Untertitel hat, schreiben Sie in das
% entsprechende Feld bitte ~ . Nachname und Titel  werden �ber ein
% zus�tzliches Feld in der Schreibweise ohne Umlaute abgefragt, bitte
% lesen Sie die beigef�gte Anleitung um diese korrekt auszuf�llen!

%-------Daten des Autors------------------------------------------------
\Anrede{M.A.}
\Fname{Uladzimir}
\Surname{Sidarenka}
\SurnameASCII{Sidarenka}
\DOB{1984/16/06} %alle Daten bitte in 8-stelliger Schreibweise Jahr/Monat/Tag angeben
\Birthplace{Homel}

%-------Titel und Untertitel--------------------------------------------
\Type{Dissertation}
\Title{Discourse-Aware Sentiment Analysis of German Twitter}
\TitleASCII{Discourse-Aware Sentiment Analysis of German Twitter}
\Subtitle{}

%-------Eingabe der Gutachternamen--------------------------------------
\Supervisor{Prof.~Dr.~Manfred~Stede}
\GutachterA{Prof.~Dr.~Jacob~Eisenstein}
\GutachterB{Dr.~Tatjana~Scheffler}

%-------Informationen zur Universitaet----------------------------------
\Degree{Dr. phil.}
\Fach{Linguistik}
\Faculty{Humanwissenschaftlichen Fakult\"at}
\University{Universit\"at Potsdam}
\Dekan{Prof. Dr. sc. Heinz  Mustermann}
\Rektor{Prof. Dr. Dr. h.c. Wilhelm Muster}

%-------Pruefungsdaten: eingereicht und mdl. Pruefung-------------------
\SubmissionDate{2018/10/01} %alle Daten bitte in 8-stelliger Schreibweise Jahr/Monat/Tag angebe
\Pruefungsdatum{2018/10/01} %alle Daten bitte in 8-stelliger Schreibweise Jahr/Monat/Tag angebe

%-deutsche Schlagwoerter(bitte getrennt durch Kommata auflisten)---------------------------
\Schlagwoerter{Sentiment-Analyse, Social Media, Diskurs-Analyse}

%--englische Schlagwoerter(bitte getrennt durch Kommata auflisten)--------------------------
\Keywords{sentiment analysis, social media, discourse analysis}


