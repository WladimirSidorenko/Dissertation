% FILE: abstract.tex Version 2.1 AUTHOR: Universit�t Duisburg-Essen,
% Standort Duisburg AG Prof. Dr. G�nter T�rner Verena Gondek, Andy
% Braune, Henning Kerstan Fachbereich Mathematik Lotharstr. 65., 47057
% Duisburg entstanden im Rahmen des DFG-Projektes DissOnlineTutor in
% Zusammenarbeit mit der Humboldt-Universitaet zu Berlin AG
% Elektronisches Publizieren Joanna Rycko und der DNB - Deutsche
% Nationalbibliothek
%% \begin{abstract}
%% Here is the english abstract.
%% \end{abstract}
%-englische-Zusammenfassung---------------------------------------
\selectlanguage{english}
\section*{Abstract}
\addcontentsline{toc}{section}{Abstract}

The immense popularity of online communication services in the last
decade has not only massively changed our life (with news spreading
like wildfire on the Web, presidents announcing their decisions on
Twitter, and the outcome of political elections being determined on
Facebook) but also dramatically increased the amount of data exchanged
on these platforms.  Therefore, if we wish to understand the needs of
modern society better and if we want to protect it from new threats,
we urgently need more robust, higher-quality natural language
processing (NLP) applications that can recognize such necessities and
menaces automatically, by analyzing uncensored speech.  Unfortunately,
most NLP programs that exist today have been either created for
standard language, as we know it from newspapers, or, in the best
case, adapted to English social media texts.

This thesis reduces the existing arrears by entering the Neuland of
German online communication and addressing one of its most prolific
forms---users' conversations on Twitter.  In particular, it explores
ways and means how people express their evaluations on this service,
examines current approaches to automatic mining of such opinions, and
proposes novel methods that outperform state-of-the-art techniques.
For this purpose, I introduce a new corpus of German tweets that have
been manually annotated with evaluative opinions, their targets and
holders, as well as lexical polarity items and contextual modifiers of
these terms.
%% A detailed inter-annotator agreement study of our dataset shows that
%% not only are sentiments difficult to analyze for people, but this task
%% becomes even more challenging in the presence of emoticons or when
%% dealing with political topics.
Using these data, I explore four major areas of sentiment research:
\begin{inparaenum}[(i)]
\item generation of sentiment lexicons,
\item aspect-based opinion mining,
\item message-level polarity classification, and
\item discourse-aware sentiment analysis.
\end{inparaenum}
In the first task, I compare three popular groups of lexicon
generation methods: dictionary-, corpus-, and word-embedding--based
ones, finding that dictionary-based systems generally yield better
polarity lists than the last two groups.  Apart from this, I propose a
novel linear projection algorithm, whose results surpass many existing
automatically generated lexicons.  Afterwords, in the second task, I
examine two common approaches to automatic prediction of sentiment,
source, and target spans: conditional random fields and recurrent
neural networks, obtaining higher scores with the former model and
improving these results even further by redefining the structure of
CRF graphs.  When dealing with message-level polarity classification,
I juxtapose three major sentiment paradigms: lexicon-,
machine-learning--, and deep-learning--based systems, and also try to
unite the first and the last of these method groups by introducing a
bidirectional neural network with lexicon-based attention. Finally, in
order to make the new classifier aware of microblogs' discourse
structure, I let it separately analyze elementary discourse units of
each tweet and infer the overall polarity of a message from the scores
of its EDUs with the help of two novel approaches: latent-marginalized
CRFs and Recursive Dirichlet Process.
