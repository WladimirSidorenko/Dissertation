\chapter{Coarse-Grained Sentiment Analysis}\label{sec:snt:cgsa}

\section{Coarse-Grained Sentiment Analysis with Lexicon-Based Methods}

\section{Coarse-Grained Sentiment Analysis with Machine Learning
  Frameworks}

\section{Coarse-Grained Sentiment Analysis with Deep Neural Networks}

\section{Coarse-Grained Sentiment Analysis using Language and Domain
  Adaptation}

\section{Evaluation}
\subsection{Effect of Distant Supervision}
\subsection{Effect of Normalization}

\section{Related Work}

\textbf{Lexicon-Based Methods}

Vasileios Hatzivassiloglou and Janyce Wiebe. 2000. Ef- fects of
adjective orientation and gradability on sen- tence subjectivity. In
Conference on Computational Linguistics (COLING-2000).

\todo[inline]{Das and Chen, 2001}

\todo[inline]{\citet{Turney:02}}

\todo[inline]{}

More recently, Wiebe et al.  (2002) report on document-level
subjectivity classification, using a k-nearest neighbor algorithm
based on the total count of subjective words and phrases within each
document.

\todo[inline]{\citet{Hu:04}}

\todo[inline]{}

% Taboada et al., 2004
Largely inspired by the Appraisal theory of~\citet{Martin:00},
\citet{Taboada:04} enhanced the original method of~\citet{Turney:02}
by increasing the weights of polar adjectives which occurred in the
middle and at the end of a document, and also augmenting these values
with the affect, judgement, and appreciation scores.  Similarly to
polarity, the appraisal scores were calculated automatically by
computing the PMI of their cooccurrence with different pronouns using
a web search engine.

\citet{Kim:04} experimented with three different methods of
determining the overall polarity of a sentence:
\begin{inparaenum}[(i)]
  \item by multiplying the signs of polar terms found in sentence,
  \item taking their sum, and
  \item taking the geometric mean of polarity scores;
\end{inparaenum}
finding that the first and last options worked best for the Document
Undestanding Corpus.\footnote{\url{http://duc.nist.gov/}}

% Polanyi and Zaenen, 2006
An early analysis of lexicon-based approaches to sentiment analysis
was made by~\citet{Polanyi:06}, who argued that, besides considering
the lexical valence (i.e., semantic orientation) of polar terms, it
was also necessary to incorporate syntactic, discourse-level, and
extra-linguistic factors such as negations, intensifiers, modal
operators (e.g., \emph{could} or \emph{might}), presuppositional items
(e.g., \emph{barely} or \emph{failure}), irony, reported speech,
discourse connectors and structure, genre and attitude assessment,
reported speech, multi-entity evaluation, and so on.

% Kennedy and Inkpen, 2006
This theoretical hypothesis was later proven empirically by
\citet{Kennedy:06}, who compared two lexicon-based approaches:
\begin{inparaenum}[(i)]
\item simply counting positive and negative expressions, and assigning
  the input text (a review) to the polarity class with the greater
  number of found terms (considering a review as neutral if it had an
  equal number of expressions with either polarities); and
\item the same procedure as above, but additionally enhanced with the
  information about contextual valence shifters~(intensifiers,
  downtoners, and negations).
\end{inparaenum}
In the latter method, the authors swapped the polarity of a polar
expression to the opposite if it followed a negation, and decreased or
increased the score of this term by a constant factor if it was
preceded by a diminisher or intensifier.  This enhancement was shown
to bring a statistically significant improvement, boosting the
accuracy of the two-class prediction on a corpus of product and movie
reviewa from 67.9 to 69.3\%.

\todo[inline]{Taboada, Anthony, and Voll (2006)}

Finally, a veritably seminal work on lexicon-based approaches was
introduced by~\citet{Taboada:11} who presented a manually compiled
sentiment lexicon,\footnote{The authors hand-annotated all occurrences
  of adjectives, nouns, and verbs found in a corpus of 400 Epinions
  reviews with ordinal categories ranging from -5 to 5 which reflected
  the semantic orientation of a term (positive vs. negative) and its
  polar strength (weak vs. strong).} and used this resource to compute
the overall \emph{semantic orientation} (SO) of text.  Inspired by the
ideas of~\citet{Polanyi:06}, to make this calculation more precise,
the authors also incorporated a set of additional heuristic rules by
changing the prior SO values of negated, itensified, and downtoned
terms, ignoring irrealis sentences, and adjusting weights of specific
sections of a document.  An extensive evaluation of this method showed
a superior performance of the manual lexicon in comparison with other
polarity lists including the Subjectivity Dictionary~\cite{Wilson:05},
Maryland Polarity Set~\cite{Mohammad:09}, and
\textsc{SentiWordNet}~\cite{Esuli:06c}.  Moreover, the authors also
proved the robustness of their SO computation procedure for other
topics and text genres, hypothesizing that lexion-based approaches
were more domain-independent than traditional supervised learning
techniques.

\todo[inline]{\citet{Eisenstein:17}}

\textbf{Traditional Supervised Machine Learning Frameworks}

\todo[inline]{\citet{Pang:02}}

Pang et al.  (2002) adopted a more direct approach, using super- vised
machine learning with words and n-grams as features to predict
orientation at the document level with up to 83\% precision.

\todo[inline]{}

One of the first attempts to analyze message-level sentiments on
Twitter was made by \citet{Go:09}.  For their experiments, the authors
collected a set of 1,600,000 tweets containing smileys.  Based on
these emoticons, they automatically derived polarity classes for these
messages (positive or negative) and used them to train a Na\"{\i}ve
Bayes, MaxEnt, and SVM classifier.  The best $F$-score for this
two-class classification problem could be achieved by the last system
and run up to 82.2\%.

Similar work was also done by \citet{Pak:10} who used the Na\"{\i}ve
Bayes approach to differentiate between neutral, positive, and
negative microblogs; and \citet{Barbosa:10} who gathered a collection
of 200,000 tweets, subsequently analyzing them with three publicly
available sentiment web-services and training an SVM classifier on the
results of these predictors.  In a similar way, \citet{Agarwal:11}
compared a simple unigram-based SVM approach with two other
full-fledged systems, one which relied on a rich set of manually
defined features, and another used partial tree
kernels~\cite{Moschitti:06}.  The authors evaluated these methods on a
commercially acquired corpus of 8,753 foreign-language tweets, which
were automatically translated into English, finding that a combination
of these methods worked best for both two- and three-way prediction
tasks.

The state-of-the-art results for message level polarity prediction on
tweets were established by~\citet{Mohammad:13}, whose system (a
supervised SVM classifier) used a rich set of various features
including word and character n-grams, PoS statistics, Brown
clusters~\cite{Brown:92}, etc., and also strongly benefitted from
automatic corpus-based polarity lists---Sentiment~140 and NRC
Hashtag~\cite{Mohammad:12,Kiritchenko:14}.  This approach ranked first
at the SemEval competition~2013~\cite{Nakov:13} and anchieved the
fourth place on the rerun of this task one year
later~\cite{Rosenthal:14}, being outperformed by the supervised
logistic regression approach of~\citet{Miura:14}, who used a heavy
preprocessing of the data and a special balancing scheme for
underrepresented classes.  Later on, these results were further
improved by the apporaches of~\citet{Hagen:15} and \citet{Deriu:16},
which both relied on ensembles of multiple independent classifiers.

\cite{Nakagawa:10}

Wiebe 2002, Riloff 2003

Wiebe, Bruce, \& O'Hara 1999
Hatzivassiloglou \& Wiebe 2000
Wiebe 2000;
Wiebe et al. 2002
Yu \& Hatzivassiloglou 2003

Bruce and Wiebe (1999) annotated 1,001 sentences as sub- jective or
objective, and Wiebe et al. (1999) de- scribed a sentence-level Naive
Bayes classifier using as features the presence or absence of
particular syn- tactic classes (pronouns, adjectives, cardinal num-
bers, modal verbs, adverbs), punctuation, and sen- tence position.

More recently, Wiebe et al.  (2002) report on document-level
subjectivity classi- fication, using a k-nearest neighbor algorithm
based on the total count of subjective words and phrases within each
document.

A semi-supervised classification approach was proposed
by~\citet{Yu:03}, who presented a three-stage method, in which they
first distinguished between subjective and objective documents, then
differentiated between polar and neutral sentences, and, finally,
classified the polarity of opinionated clauses.  The authors used a
Na{\"i}ve Bayes classifer for the document-level task, reaching a
remarkable \F-score of~0.96 on this objective; and applied an ensemble
of NB systems to predict the subjectivity of the sentences.  In the
final step, they determined the semantic orientation of subjective
clauses by averaging the polarity scores of their tokens, getting
these scores from an automatically constructed sentiment
lexicon~\cite{Hatzivassi:97}.  With this approach, \citeauthor{Yu:03}
attained an accuracy of~91\% on a set of 38 sentences which had a
perfect inter-annotator agreement in their data.

\textbf{Deep Neural Networks}

\citet{Yessenalina:11}

A real breakthrough in the use of deep neural networks for the
sentence-level sentiment analysis happened with the pioneering work
of~\citet{Socher:11}, who first introduced a recursive autoencoder
(RAE).  In this system, the authors obtained a fixed-width vector
representation for complex phrases $\vec{v}$ by recursively merging
the vectors of adjacent tokens (say $\vec{w}_1$ and $\vec{w}_2$),
first multiplying these vectors with a compositional matrix $W$ and
then applying a non-linear function ($softmax$) to the resulting
product:
\begin{align*}
  \vec{c} &= softmax\left(W\cdot\begin{bmatrix}
  \vec{w}_1\\
  \vec{w}_2
  \end{bmatrix}\right)
\end{align*}
Using a max-margin classifier on top of the resulting phrase
representation, \citet{Socher:11} could improve the state-of-the-art
results on predicting the sentence-level polarity of user's blog
posts~\cite{Potts:10} and also outperformed the system
of~\citet{Nasukawa:03} on the MPQA data set~\cite{Wiebe:05}.

Later on, \citet{Socher:12} further improved these scores with the
help of a recursive matrix-vectors space model (RMVSM), in which each
word was associated with a 2-tuple of a vector and matrix---e.g.,
$(\vec{w}_1, W_1)$ and $(\vec{w}_2, W_2)$---and the compositionality
function was redefined as follows:
\begin{align*}
  \vec{c} &= softmax\left(W\cdot\begin{bmatrix}
  W_2\cdot\vec{w}_1\\
  W_1\cdot\vec{w}_2
  \end{bmatrix}\right)
\end{align*}

\citet{Wang:15}

\textbf{Language and Domain Adaptation}

One of the first works which pointed out the importance of domain
adaptation for sentiment analysis was introduced by~\citet{Aue:05}.
In their experiments, the authors trained separate SVM classifiers on
four different document sets: movie reviews, book reviews, customer
feedback from a product support service, and a feedback survey from a
customer knowledge base; finding that each classifier performed best
when applied to the same domain as it was trained on.  In order to
find an optimal way of overcoming this domain specificity,
\citet{Aue:05} tried out four different options:
\begin{inparaenum}[(i)]
\item\label{sent-cgsa:lst:rel-wrk1} training one classifier on all but
  the target domain and applying it to the latter;
\item using the same procedure as above, but limiting the features to
  only those which also appeared in the target texts;
\item taking an ensemble of individual classifiers each of which was
  trained on a different data collection; and, finally,
\item using a minimal subset of labeled in-domain data to train a
  Na{\"i}ve Bayes system with the expectation-maximization algorithm
  \cite[EM;][]{Dempster:77}.
\end{inparaenum}
The authors found that the ensemble and EM options worked best for
their cross-domain task, achieving an accuracy of up to 82.39\% for
the two-class prediction (positive vs negative) on new unseen text
genres.

Another notable milestone in the domain adaptation research was set
by~\citet{Blitzer:07}.  Relying on their previous work on structural
correspondence learning~\cite{Blitzer:07}, in which they used a set of
\emph{pivot features} (features which frequently appeared in both
target and source domains) to find an optimal correspondence of the
remaining attributes,\footnote{In particular, the authors trained $m$
  binary predictors for each of their $m$ pivot features in order to
  find other attributes which frequently co-occurred with the pivots.
  Afterwards, they composed these $m$ resulting weight vectors into a
  single matrix $W := [\vec{w}_{1},\ldots,\vec{w}_{m}]$, took an SVD
  decomposition of this matrix, and used the top $h$ left singular
  vectors to translate source features to the new domain.} the authors
refined their method by pre-selecting the pivots using their PMI
scores and improving misaligned feature projections using a small set
of labeled target examples.  With these modifications,
\citeauthor{Blitzer:07} were able to reduce the average adaptation
loss (the accuracy drop when transferring a classifier to a different
domain) from 9.1 to 4.9~percent when testing a sentiment predictor on
the domains of book, dvd, electical appliances, and kitchen reviews.

Other important works on domain adaptation for opinion mining include
those of~\citet{Read:05}, who pointed out that sentiment
classification might not only depend on the domain but also on topic,
time, and language style in which the text was written;
\citet{Tan:07}, who proposed using the classifier trained on the
source domain to classify unlabeled instances from the target genre,
and then iteratively retrain the system on the enriched data set.
Finally, \citet{Andreevskaia:08} proposed a combination of a lexicon-
and ML-based systems, claiming that this ensemble would be more
resistible to the domain shift than each of these classifiers on their
own.

Another line of research was introduced by~\citet{Glorot:11} who
proposed stacked denoising autoencoders (SDA)---a neural network
architecture in which an input vector $\vec{x}$ was first mapped to a
smaller representation $\vec{x}'$ via some function
$h: \vec{x}\mapsto\vec{x}'$, and then restored to its approximate
original state via an inverse transformation
$g: \vec{x}'\mapsto\vec{x}''\approx\vec{x}$.  In their experiments,
the authors optimized the parameters of the functions $h$ and $g$ on
both target and source data, getting approximate representations of
instances from both data sets; and then trained a linear SVM
classifier on the restored representations of the source instances,
subsequently applying this classifier to the target domain.  This
approach was further refined by~\citet{Chen:12} who analytically
computed the reconstruction function~$g$, and used both original and
restored features to predict the polarity labels of the target
data.\footnote{Both approaches were trained tested on the Amazon
  Review Corpus of~\citet{Blitzer:07}.}


Further notable contributions to domain adaptation in general were
made by~\citet{Daume:07} who proposed to replicate each extracted
feature three times and train the first replication on both domains,
the second repetion only on source, and the third copy only on target
domain, for which he assumed a small subset of labeled examples was
available; \citet{Yang:15} who trained neural embeddings of features,
trying to predict which instance attributes frequently co-occured with
each other;

\section{Summary and Conclusions}\label{slsa:subsec:conclusions}
