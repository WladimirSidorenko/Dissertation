% FILE: appendixA.tex  Version 2.1
% AUTHOR:
% Universit�t Duisburg-Essen, Standort Duisburg
% AG Prof. Dr. G�nter T�rner
% Verena Gondek, Andy Braune, Henning Kerstan
% Fachbereich Mathematik
% Lotharstr. 65., 47057 Duisburg
% entstanden im Rahmen des DFG-Projektes DissOnlineTutor
% in Zusammenarbeit mit der
% Humboldt-Universitaet zu Berlin
% AG Elektronisches Publizieren
% Joanna Rycko
% und der
% DNB - Deutsche Nationalbibliothek

%%%%%%%%%%%%%%%%%%%%%%%%%%%%%%%%%%%%%%%%%%%%%%%%%%%%%%%%%%%%%%%%%%
% Packages
\usepackage{amsmath}
% \usepackage{amsthm}
\usepackage{bm}
\usepackage{booktabs}
\usepackage[font={small,it},justification=centering]{caption}
\usepackage{color}              % \definecolor
\usepackage{colortbl}           % \cellcolor (colored tables)
\usepackage{dirtree}            % \dirtree (draw directory structure)
\usepackage{etoolbox}           % \ifstrempty
\usepackage{footnote}
\usepackage{graphicx}           % \scalebox
\usepackage{hhline}             % \ifstrempty
\usepackage{listings}           % for \lstlisting
\usepackage{mdframed}           % gray background boxes
\usepackage{multicol}           % for multiple columns in a table
\usepackage{multirow}
\usepackage[authoryear]{natbib}
\usepackage{ntheorem}
\usepackage{paralist}
\usepackage{subcaption}
\usepackage{tabularx}
\usepackage{tikz}
\usepackage{wasysym}            % smiley symbols
\usepackage{wrapfig}
\usepackage{xcolor}             % \colorlet

\usetikzlibrary{arrows,backgrounds,calc,positioning,shapes,snakes}

%%%%%%%%%%%%%%%%%%%%%%%%%%%%%%%%%%%%%%%%%%%%%%%%%%%%%%%%%%%%%%%%%%
%% Lengths
\newlength{\oosixthClmnWidth}
\setlength{\oosixthClmnWidth}{0.055\textwidth}

\newlength{\clmnwidth}
\setlength{\clmnwidth}{0.65\textwidth}

%%%%%%%%%%%%%%%%%%%%%%%%%%%%%%%%%%%%%%%%%%%%%%%%%%%%%%%%%%%%%%%%%%
%% Commands
\newcommand{\ttranslate}[2]{#1

\noindent\emph{#2}}

\renewcommand{\cite}{\citep}
\newcommand{\titlerule}{\rule{0.8\textwidth}{.4pt}}
\providecommand{\shortcite}[1]{\cite{#1}}
\newcommand{\texample}[1]{``\textit{#1}''}

\newcommand{\mtag}[3]{{{\upshape[}\ifstrempty{#3}{#2}{\textcolor{#3}{#2}}{\upshape]$_{\textrm{\bfseries\emph{\tiny
          #1}}}$}}}
\newcommand{\sentiment}[2][]{\mtag{sen\-ti\-ment\ifstrempty{#1}{}{:#1}}{#2}{}}
\newcommand{\source}[2][]{\mtag{source\ifstrempty{#1}{}{:#1}}{#2}{darkgoldenrod4}}
\newcommand{\target}[2][]{\mtag{tar\-get\ifstrempty{#1}{}{:#1}}{#2}{darkslateblue}}
\newcommand{\emoexpression}[2][]{\mtag{eex\-pression\ifstrempty{#1}{}{:#1}}{#2}{red}}
\newcommand{\intensifier}[2][]{\mtag{in\-ten\-si\-fier\ifstrempty{#1}{}{:#1}}{#2}{cyan}}
\newcommand{\diminisher}[2][]{\mtag{di\-mi\-ni\-sher\ifstrempty{#1}{}{:#1}}{#2}{}}
\newcommand{\negation}[2][]{\mtag{ne\-ga\-tion\ifstrempty{#1}{}{:#1}}{#2}{deeppink4}}
\newcommand{\argmax}{\operatornamewithlimits{argmax}}
\newcommand{\stddev}[1]{{\tiny$^{\pm#1}$}}
\newcommand{\F}[0]{$F_1$}

%% Tikz
\makeatletter
\tikzset{circle split part fill/.style  args={#1,#2}{%
alias=tmp@name, % Jake's idea !!
postaction={%
insert path={
\pgfextra{%
\pgfpointdiff{\pgfpointanchor{\pgf@node@name}{center}}%
{\pgfpointanchor{\pgf@node@name}{east}}%
\pgfmathsetmacro\insiderad{\pgf@x}
%\begin{scope}[on background layer]
%\fill[#1] (\pgf@node@name.base) ([xshift=-\pgflinewidth]\pgf@node@name.east) arc
%                    (0:180:\insiderad-0.5\pgflinewidth)--cycle;
%\fill[#2] (\pgf@node@name.base) ([xshift=\pgflinewidth]\pgf@node@name.west)  arc
%                     (180:360:\insiderad-0.5\pgflinewidth)--cycle;
\fill[#1] (\pgf@node@name.base) ([xshift=-\pgflinewidth]\pgf@node@name.east) arc
(0:180:\insiderad-\pgflinewidth)--cycle;
\fill[#2] (\pgf@node@name.base) ([xshift=\pgflinewidth]\pgf@node@name.west)  arc
(180:360:\insiderad-\pgflinewidth)--cycle;            %  \end{scope}
}}}}}
\makeatother

%%%%%%%%%%%%%%%%%%%%%%%%%%%%%%%%%%%%%%%%%%%%%%%%%%%%%%%%%%%%%%%%%%
%% Environments
% uncomment this and the `amsthm` package to get plain uncolored
% examples
% \newtheoremstyle{break}
%   {\topsep}{\topsep}%
%   {\itshape}{}%
%   {\bfseries}{}%
%   {\newline}{}%
% \theoremstyle{break}
% \newtheorem{example}{Example}[section]

\theoremstyle{break}
\theoremheaderfont{\bfseries\upshape}
\newmdtheoremenv[%
linecolor=gray,%
leftmargin=40,%
rightmargin=40,%
backgroundcolor=gray!20,%
innertopmargin=5pt]{example}{Example}[section]

% allow fotnotes in tables
\makesavenoteenv{tabular}

%%%%%%%%%%%%%%%%%%%%%%%%%%%%%%%%%%%%%%%%%%%%%%%%%%%%%%%%%%%%%%%%%%
%% Colors
\definecolor{cyan}{rgb}{0.0,0.6,0.6}
\definecolor{darkblue}{rgb}{0.0,0.0,0.6}
\definecolor{darkgoldenrod4}{RGB}{139,101,8}
\definecolor{darkred}{RGB}{139,0,0}
\definecolor{darkslateblue}{RGB}{72,61,139}
\definecolor{deeppink4}{RGB}{139,10,80}
\definecolor{firebrick}{RGB}{178,34,34}
\definecolor{gray69}{RGB}{176,176,176}
\definecolor{gray76}{RGB}{194,194,194}
\definecolor{gray84}{RGB}{214,214,214}
\definecolor{gray}{rgb}{0.4,0.4,0.4}
\definecolor{lightcyan4}{RGB}{122,139,139}
\definecolor{midnightblue}{RGB}{25,25,112}
\definecolor{red3}{RGB}{205,0,0}
\definecolor{violetred4}{RGB}{139,34,82}
\colorlet{cellcolor}{gray84}

%%%%%%%%%%%%%%%%%%%%%%%%%%%%%%%%%%%%%%%%%%%%%%%%%%%%%%%%%%%%%%%%%%
%% XML Highlighting
\lstset{
  basicstyle=\tiny,
  columns=fullflexible,
  showstringspaces=false,
  commentstyle=\color{gray}\upshape
}

\lstdefinelanguage{XML}
{
  morestring=[b]",
  morestring=[s]{>}{<},
  morecomment=[s]{<?}{?>},
  stringstyle=\color{black},
  identifierstyle=\color{darkblue},
  keywordstyle=\color{cyan},
  morekeywords={xmlns,id,intensity,mmax_level,span,sarcasm,polarity,version,encoding}%
}
