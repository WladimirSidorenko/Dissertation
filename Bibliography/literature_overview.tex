\documentclass[a4paper,11pt]{article}

%%%%%%%%%%%%%%%%%%%%%%%%%%%%%%%%%%%%%%%%%%%%%%%%%%%%%%%%%%%%%%%%%%
%% Packages
\usepackage{paralist}

%%%%%%%%%%%%%%%%%%%%%%%%%%%%%%%%%%%%%%%%%%%%%%%%%%%%%%%%%%%%%%%%%%
%% Variables and Methods

%%%%%%%%%%%%%%%%%%%%%%%%%%%%%%%%%%%%%%%%%%%%%%%%%%%%%%%%%%%%%%%%%%
%% Documents
\begin{document}
\section{Pre-Processing}

\section{Sentiment Analysis}
\subsubsection{Turney: Thumbs Up or Thumbs Down? Semantic Orientation Applied to
               Unsupervised Classification of Reviews\cite{Turney02}}

\subsubsection{Bing and Liu: Mining and summarizing customer reviews\cite{Bing-Liu-04}}
In their article, Bing and Liu \cite{Bing-Liu-04} propose a system for
automatically mining subjective judgements on product features described in
customer reviews.  This system first searches the reviews for frequently
occurring mentions of prominent features of specific products by using an
association mining algorithm.  This algorithm extracts nouns and noun phrases
which occur frequently across muliple documents.  Implicit features and
features which are not represented by nouns are not taken into consideration
by this algorithm.  The extracted feature candidates are then pruned if the
components of detected noun phrases do not occur in a specific order in
sentences or if some features are full subsets of other ones.

As the next processing step, text passages surrounding detected feature
candidates in text are searched for adjectives, since words belonging to this
part-of-speech are assumed to be the most prominent indicators of subjective
opinions.  Found adjectives are then classified into subjective and objective
ones.  Adjectives of the former class are subsequently subclassified into
positive and negative ones depending on their polarity.  Both subjectivity and
polarity are determined on the basis of WordNet information.

\section{Markov Logic Network}

\section{Discourse Analysis}

\nocite{*}
\bibliography{bibliography}     % relative to the project's root directory
\bibliographystyle{plain}
\end{document}
