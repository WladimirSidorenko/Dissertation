% FILE: sentiment.tex  Version 0.01
% AUTHOR: Uladzimir Sidarenka

% This is a modified version of the file main.tex developed by the
% University Duisburg-Essen, Duisburg, AG Prof. Dr. Günter Törner
% Verena Gondek, Andy Braune, Henning Kerstan Fachbereich Mathematik
% Lotharstr. 65., 47057 Duisburg entstanden im Rahmen des
% DFG-Projektes DissOnlineTutor in Zusammenarbeit mit der
% Humboldt-Universitaet zu Berlin AG Elektronisches Publizieren Joanna
% Rycko und der DNB - Deutsche Nationalbibliothek

\section{Sentiment Corpus}\label{sec:snt:corpus}

A vital prerequisite for proving or disproving any hypotheses in
computational liguistics is the existence of sufficiently big manually
labeled data for the targeted domain, on which the made conjectures
could be tested.  Since there was no corresponding sentiment corpus
for German Twitter that we were aware of at the time of writing this
thesis, we had to create our own dataset, which we will introduce in
this section.

We begin our introduction by describing the selection criteria and
tracking procedure we used to initially collect the raw data for our
``tweebank''.  After presenting the utilized annotation tool and
detailing the annotation scheme, we turn to an extensive analysis of
the inter-annotator agreement, which will serve as an upper bound in
our later experiments.  In Subsection \ref{subsec:snt:iaa}, we propose
two new versions of the popular $\kappa$ metric, which were
specifically adjusted to the peculiarities of our annotation task:
binary and proportional kappa.  With these two measures, we estimate
the inter-coder reliability on annotating spans of subjective
opinions, their targets, and sources, also checking whether the
annotators agreed on the polarity and intensity of sentiments, once
they agreed on their boundaries.  Finally, with statistical means, we
investigate the correlations between the selection criteria we applied
initially and the number of annotated elements and disagreements of
our experts in order to understand which linguistic and
extra-linguistic factors might have significantly influenced the
distribution of sentiments in the corpus and which ones exacerbated
their analysis.

\subsection{Data Collection}

An important problem that one has to solve prior to creating a new
dataset is that of the selection criteria to use in order to collect
the data.  While, for low-level NLP tasks, such as part-of-speech
tagging or syntactic parsing, it typically suffices to define the
language domain to sample from (since the phenomena of interest are
usually frequent and uniformly spread), for semantically demanding
tasks with many diverse ways of expressions, one also has to pay
attention to in-domain factors as they might significantly affect the
desired distribution, making the resulting corpus either utterly
sparse or excessively biased.

To minimize both of these risks -- scarceness and bias, we decided to
use a compromise approach by gathering one part of the dataset from
tweets which were a priori more likely to contain sentiments, and
sampling the rest of the corpus from a big collection of messages
uniformly at random.  That way, we hoped to increase the recall by
using the former strategy and mitigate the introduced bias with the
latter method.

As criteria which could help us find more opinion-laden microblogs, we
considered the topic and form of tweets.  Our motivation was that some
political or social issues by themselves might incite people to be
more subjective in their judgements.  Therefore, including such topics
in our corpus would automatically increase the coverage of sentiment
expressions.  On the other hand, opinions also need to be expressed by
some linguistic means, therefore the mere form of a message can also
suggest whether it is likely to be emotional or not.

As we started creating the corpus in the spring 2013, obvious choices
for subjectively rich topics to us were the \emph{papal conclave},
which took place in March of that year, and the \emph{federal
  elections in Germany}, which were to be held in autumn.  Because
both of these events implied to some sort of voting, we decided to
include \emph{general discussions of political issues} as a third
topic in our dataset in order to counterbalance the election
specifics.  Finally, to obey the second objective, i.e., to keep the
corpus bias low, we considered \emph{casual everyday conversations}
without any pre-filtering as the last source of the new data.

To collect tweets for the first three topics, we created extensive
keyword lists (with several dozens entries each) for each of these
subjects.  Based on these lists, we subsequently were tracking
microblogs between March and September 2013 using the public Twitter
API.\footnote{\url{https://pypi.python.org/pypi/tweetstream}}
Afterwards, we additionally enriched the federal elections data with
the messages we obtained from the University of M\"unster, who were
concurrently working on related problems within the joint FMER project
``Discourse Analysis of Social Media''.  For sampling the fourth
topic, we used the German Twitter snapshot of \citet{Scheffler:14},
which comprised all German microblogs posted in April 2013.

%% For our work, the in-domain factors to consider were the topics and
%% the form of tweets.  Since we wanted our corpus to be as
%% representative as possible, we had to make sure that the topics we
%% choose for sampling lend themselves as fruitful opinion sources.  At
%% the same time, we did not want automatically generated ad and news
%% tweets to spoil our data and also introduced additional formal
%% criteria (described below) that the tweets had to satisfy in order to
%% be chosen.  But then again, applying these restriction might make the
%% dataset excessively biased, so we did allow for a certain proportion
%% of tweets being selected even if they did not conform to our
%% constraints.

%% Politics: 59,531 messages, keywords: Altmaier, Wowereit, Minister,
%% Bundeskanzleramt, Schwarz-Gelb

%% Papst: 51,579 messages, keywords: papst, pabst, konklave, Vatikan
%% General Tweets: 51,579 messages, keywords: papst, pabst, konklave, Vatikan

%% Federal Elections: 3,131,315 messages, keywords:

%% General: 24,179,871 messages, keywords:

This tracking procedure gave us a total of 27.4 M messages with the
M\"unster corpus and Scheffler's snapshot being by far the most
prolific sources of the data.  In the next step, we separated all
tweets pertaining to the same topic into three groups based on the
following formal criteria:
\begin{itemize}
\item We put messages that contained at least one polar term from the
  sentiment lexicon SentiWS \cite{Remus:10} into the first group;
\item Microblogs which did not satisfy the first criterion but
  featured at least one emoticon or exclamation mark were put into the
  second set;
\item Eventually, all remaining tweets were allocated to the third
  category of their respective topic.
\end{itemize}
The reason for choosing this strategy was that messages containing
known polar terms were by default more probable to express some
subjective opinions.  To find such terms, we considered three major
German sentiment lexica: SentiWS \cite{Remus:10}, German Polarity
Clues \cite{Waltinger:10}, and Zurich Sentiment Lexicon
\cite{Clematide:10}, eventually choosing the first one due to its
moderate size, acceptably high precision, and the availability of
inflection forms of its etries.

Since no polar lexicon, however, is guaranteed to provide for the full
coverage of opinionated phrases and, moreover, because Twitter users
are renowned for their creativity in instantly inventing new language
forms, we also applied a bail-out approach by separately collecting
messages which did not have any lexicon terms but did contain a smiley
or exclamation point.  Our assumption was that either these elements
alone would suffice to express subjective opinions or that they would
reinforce the meaning of accidentally missed polar words.

Finally, as we did not make any hypotheses about the distribution of
sentiments in the rest of the tweets, we allocated all remaining
microblogs to the same group, hoping that a uniform sampling would
provide us with further positive and negative opinion examples.  A
detailed breakdown of the resulting distribution of messages across
the four topics and their formal groups is given in
Table~\ref{snt:tbl:corp:topic-bins}:
\begin{table}[hbt!]\small
  \begin{tabular}{|l|*{5}{>{\centering\arraybackslash}p{0.12\textwidth}|}}
    \hline

    \cellcolor{cellcolor}& \multicolumn{4}{c|}{{\cellcolor{cellcolor}}
      Formal Criterion} &
    \cellcolor{cellcolor}\\\hhline{|>{\arrayrulecolor{gray76}}-*{4}{>{\arrayrulecolor{black}}|-}|>{\arrayrulecolor{gray76}}-|}\arrayrulecolor{black}

    \multirow{-2}{0.2\columnwidth}{\centering\bfseries\cellcolor{cellcolor}
      Topic} & {\cellcolor{cellcolor}} Polar Terms &{\cellcolor{cellcolor}}
    Emoticons and Exclamations &{\cellcolor{cellcolor}} Rest &
    {\cellcolor{cellcolor}}Total
    &\multirow{-2}{0.12\textwidth}{\centering\cellcolor{cellcolor}
      Sample\\ Tracking\\ Keywords}\\\hline

    Federal Elections & 537,083 (22.38\%) & 50,567 (2.1\%) & 1,811,742
    (75.5\%) & 2,399,392 & \tiny\emph{Abgeordnete}
    (\emph{representative}), \emph{Bundesregierung}
    (\emph{federal government})\\\hline

    Papal Conclave & 7,859 (15.11\%) & 1,260 (2.42\%) & 42,879
    (82.46\%) & 51,998 & \tiny\emph{Papst} (\emph{pope}), \emph{Pabst} (\emph{pobe})\\\hline

    Political Discussions & 10,552 (25.8\%) & 777\newline (1.9\%) & 29,555
    (72.29\%) & 40,884 &\tiny\emph{Politik} (\emph{politics}),
    \emph{Minister} (\emph{minister})\\\hline

    General Conversations &  &  &  & & \tiny\emph{den} (\emph{the}),
    \emph{sie} (\emph{she})\\
    \hline
  \end{tabular}
  \caption{Distribution of the downloaded messages across the topics
    and formal groups.\newline (percentage numbers are given with
    respect to the topic)\label{snt:tbl:corp:topic-bins}}
\end{table}

As can be seen from the table, the vast majority of tweets in each
topic fall into the third category, i.e., the one where neither polar
terms no common emoticons exist.  The average expected proportion of
such tweets $\mu$ is , and the standard deviation $\sigma$ from this
proportion runs up to .

The second biggest set is that of microblogs containing emotional
terms ($\mu=, \sigma=$).

Finally, for generating the final corpus, we randomly sampled 666
messages from each of the three bins of each of the four topics. This
gave us our final dataset of 7,992 microblogs: $666\text{ tweets}
\times 4\text{ topics} \times 3\text{ formal criteria}$.

\subsection{Annotation Tool and Procedure}\label{subsec:snt:tformat}

All annotations were done using \texttt{MMAX2} -- a freely available
text markup tool.\footnote{\url{http://mmax2.sourceforge.net/}} Since
this program relies on a token-oriented stand-off XML format where all
annotations are stored separately from the text and only refer to the
ids of the tokens they are spanning, we first had to split the
downloaded tweets into tokens in order to create an annotation
project\footnote{In \texttt{MMAX2}, an annotation project refers to a
  collection of all XML files pertaining to one corpus, including text
  data, annotation files, scheme definition etc.}  for our corpus.
For this purpose, we applied a minimally modified version of
Christopher Potts' social media
tokenizer\footnote{\url{http://sentiment.christopherpotts.net/code-data/happyfuntokenizing.py}}
which we slightly adjusted to the peculiarities of German spelling (we
accounted for the capitalized form of German nouns and the dot at the
end of ordinal numbers).

To ease the annotation process and to minimize possible data loss
during the labeling, we also split our complete dataset into 80
smaller project files with 99 -- 109 tweets each.

%% For each such project, we took only tweets pertaining to one
%% particular topic (elections, conclave, politics, or casual talks)
%% but made sure that the microblogs from the three bins of that topic
%% were represented equally in the project.


\subsection{Annotation Scheme}\label{subsec:snt:ascheme}
%% In the next step, we defined an annotation scheme for our
%% corpus.\footnote{Our complete annotation guidelines and corpus are
%%   provided as supplementary material for this paper and will be made
%%   available online after review.}  Since our goal was to get a
%% maximally full coverage of all sentiment-relevant aspects, we devised
%% an extensive list of elements that had to be annotated.  This list
%% included:
%% \begin{itemize}
%%   \item\textbf{emotional expressions} which we defined as words or
%%     phrases that unequivocally possessed some evaluative lexical
%%     meaning (e.g. \emph{gut} ``good'', \emph{schlecht} ``bad'', or
%%     \emph{lieben} ``to love'');
%%   \item\textbf{intensifiers} which were specified as elements that
%%     increased the expressivity or the polar sense of emotional
%%     expressions (for instance, \emph{sehr} ``very'' or
%%     \emph{au\ss{}ergew\"ohnlich} ``extraordinarily'');
%%   \item\textbf{diminishers} which we described as lexical items that
%%     decreased the polar sense of an emotional expression (for example,
%%     \emph{wenig} ``little'' or \emph{kaum} ``hardly'');
%%   \item and \textbf{negations} which were linguistic means that turned
%%     the polarity of an emotional expression to the complete opposite.

%%     %% (these were typically represented by the negative particle
%%     %% \emph{nicht} ``not'' or the attributive indefinite pronoun
%%     %% \emph{kein} ``no'').\footnote{The assignment to parts of speech is
%%     %%   done according to the rules of German grammar.}
%% \end{itemize}
%% It should, however, be noted that these elements did not reflect the
%% interrelationship between the polar terms (emotional expressions, in
%% our case) and the entities they were evaluating.  To compensate for
%% this disadvantage, we additionally introduced the following elements
%% to capture these links:
%% \begin{itemize}
%%   \item\textbf{targets} which we described as objects or events that
%%     were being evaluated by sentiment expressions;
%%   \item\textbf{sources} which were immediate author(s) or holder(s) of
%%     evaluative opinions;
%%   \item and actual \textbf{sentiments} which we specified as minimal
%%     complete syntactic or discourse-level units in which both target
%%     and evaluative expression appeared together.
%% \end{itemize}

%% \noindent{}A sample tweet annotated according to this scheme is
%% provided below:
%% \begin{example}
%% \upshape\sentiment{\target{Diese Milliardeneinnahmen} sind selbst
%%   \source{Sch\"auble} \emoexpression{peinlich}}\\[0.8em]
%% \noindent\sentiment{\target{\itshape{}These billions of
%%     revenues\upshape{}}\itshape{} are\upshape{}
%%   \emoexpression{\itshape{}embarrassing\upshape{}}\itshape{} even for
%%   \upshape{}\source{\itshape{}Sch\"auble\upshape{}}}
%% \end{example}
%% In addition to annotating spans of sentiment-relevant items, our
%% experts also set the values of the attributes associated with these
%% elements. For emotional expressions and sentiments, they specified the
%% polarity and intensity of the respective evaluations.  For sources and
%% targets that were expressed by pronouns, our annotators labeled the
%% respective antecedents of these pro-forms.  Furthermore, we also
%% endowed diminishers and intensifiers with a special attribute
%% \texttt{degree} which showed by how much these elements were changing
%% the intensity of their respective emotional expressions.


%% Finally, we let our experts annotate the data.  The annotation
%% procedure was carried out in three steps:
%% \begin{itemize}
%%   \item first, both annotators labeled one half of the corpus after
%%     only minimal training;
%%   \item then, we automatically determined the differences between the
%%     two annotations and let our experts resolve these discrepancies
%%     without communicating with each other;
%%   \item eventually, our assistants annotated the rest of the
%%     corpus.\footnote{At this step, only one annotator labeled the full
%%       dataset, another one labeled 78.8\% of it.}
%% \end{itemize}

\subsection{Corpus Statistics}\label{subsec:snt:stat}

%% \begin{table}[h]
%%   \centering\small
%%   \caption[Sentiment corpus statistics]{Statistics on the annotated
%%     sentiment corpus.\\ POL = corpus part with discussions about
%%     general politic topics; FE = corpus part describing the federal
%%     election 2013; PE = corpus part with discussions about the Pope
%%     election 2013; GEN = part of the corpus containing tweets with no
%%     particular topic}
%%   \begin{tabular}{|>{\centering}p{0.15\textwidth}|*{4}{>{\centering}p{\oosixthClmnWidth}|}
%%       >{\centering\bfseries}p{\oosixthClmnWidth}|*{4}{>{\centering}p{\oosixthClmnWidth}|}
%%       >{\centering\bfseries}p{\oosixthClmnWidth}|}
%%     \hline

%%     \multirow{2}{*}{\parbox{0.13\textwidth}{\centering Markable Type}}
%%     & \multicolumn{5}{>{\centering}p{7\oosixthClmnWidth}|}{Annotator
%%       1} &
%%     \multicolumn{5}{>{\centering}p{7\oosixthClmnWidth}|}{Annotator
%%       2}\tabularnewline\cline{2-11}

%%     & POL & FE & PE & GEN & Total & POL & FE & PE & GEN &
%%     Total\tabularnewline\hline

%%     Sentiment & 212 & 222 & 163 & 131 & 728 & 317 & 335 & 314 & 305 & 1271
%%     \tabularnewline\hline

%%     Source & 101 & 119 & 68 & 73 & 361 & 114 & 109 & 94 & 85 & 402
%%     \tabularnewline\hline

%%     Target & 229 & 279 & 184 & 151 & 843 & 342 & 369 & 328 & 324 & 1363
%%     \tabularnewline\hline

%%     Emotional Expression & 727 & 689 & 581 & 811 & 2808 & 662 & 669 & 671 & 768 & 2770
%%     \tabularnewline\hline

%%     Intensifier & 16 & 32 & 14 & 44 & 106 & 31 & 35 & 31 & 58 & 155
%%     \tabularnewline\hline

%%     Diminisher & 2 & 4 & 3 & 2 & 11 & 2 & 9 & 4 & 2 & 17
%%     \tabularnewline\hline

%%     Negation & 18 & 15 & 23 & 14 & 70 & 33 & 33 & 31 & 23 & 120
%%     \tabularnewline\hline
%%   \end{tabular}
%%   \label{table:sentiment-agreement-topics}
%% \end{table}

\subsection{Inter-annotator Agreement}\label{subsec:snt:iaa}
%% For estimating the inter-annotator agreement of the final dataset, we
%% adopted the Cohen's $\kappa$ metric \cite{Cohen:60}.  Following the
%% standard practice for computing this metric, we first calculated the
%% observed agreement $p_o$ as the ratio of the matching token
%% annotations to the total number of tokens:
%% \begin{equation*}\textstyle
%%   p_o = \frac{T - A_1 + M_1 - A_2 + M_2}{T}
%% \end{equation*}
%% where $T$ denotes the total number of tokens, $A_1$ and $A_2$ are the
%% numbers of tokens annotated with the given class by the first and
%% second annotator respectively, and $M_1$ and $M_2$ represent the
%% number of tokens with matching annotations for that class.  And we
%% then estimated the chance agreement $p_c$ in the usual way as:
%% \begin{equation*}\textstyle
%%   p_c = c_1 \times c_2 + (1.0 - c_1) \times (1.0 - c_2)
%% \end{equation*}
%% where $c_1$ and $c_2$ are the proportions of tokens annotated with the
%% given class in the first and second annotation respectively, i.e. $c_1
%% = \frac{A_1}{T}$ and $c_2 = \frac{A_2}{T}$.

%% Two questions that arose during this computation were:
%% \begin{inparaenum}[\itshape i)\upshape]
%% \item if tokens belonging to several overlapping annotation spans of
%%   the same class in one annotation had to be counted multiple times
%%   when computing the $A$ scores (for instance, if we had to count
%%   the words \textit{this}, \textit{nice}, and \textit{book} in
%%   Example \ref{exmp:1} twice as sentiments when computing $A_1$ and
%%   $A_2$), and
%% \item if we had to assume that two annotated spans from different
%%   experts agreed on all of their tokens when these spans had at
%%   least one token in common or if we rather had to count the actual
%%   number of tokens whose annotation was matching (i.e., if we had to
%%   consider the annotation of the word \textit{My} in Example
%%   \ref{exmp:1} as matching or not considering that the rest of the
%%   sentiment spans agreed).
%% \end{inparaenum}

%% \begin{example}\label{exmp:1}
%%   \textrm{Annotation 1:}\\ \sentiment{My father hates \sentiment{this
%%       nice book}.}\\\\ \textrm{Annotation 2:}\\ My \sentiment{father
%%     hates \sentiment{this nice book}.}
%% \end{example}

%% To answer these questions, we introduced two agreement metrics:
%% \emph{binary} and \emph{proportional} kappa.  With the former variant,
%% we counted tokens belonging to multiple eponymous annotation spans
%% multiple times and considered all tokens belonging to the given
%% annotation span as matching if this span agreed on at least one token
%% with the annotation from the other expert.  With the latter metric,
%% every token labeled as sentiment (target or source) was counted only
%% once, and we only calculated the actual number of tokens with matching
%% annotations when computing the $M$ scores.  The final agreement
%% results measured with these two metrics are shown in Table
%% \ref{tbl:agrmnt}.

%% \begin{table*}[bth!]
%%   \begin{center}
%%     \bgroup \setlength\tabcolsep{0.4\tabcolsep}\scriptsize
%%     \begin{tabular}{|p{0.1\columnwidth}| % first columm
%%         *{15}{>{\centering\arraybackslash}p{0.05\columnwidth}|}} % next five columns
%%       \hline
%%           \multirow{2}{0.2\columnwidth}{\bfseries Element} &
%%           \multicolumn{5}{c|}{Initial annotation} & %
%%           \multicolumn{5}{c|}{Adjudication step} & %
%%           \multicolumn{5}{c|}{Final corpus}\\\cline{2-16}
%%           & $M_1$ & $A_1$ & $M_2$ & $A_2$ & $\mathbf{\kappa}$ %
%%           & $M_1$ & $A_1$ & $M_2$ & $A_2$ & $\mathbf{\kappa}$ %
%%           & $M_1$ & $A_1$ & $M_2$ & $A_2$ & $\mathbf{\kappa}$\\\hline

%%           \multicolumn{16}{|c|}{\cellcolor{cellcolor}Binary Kappa}\\\hline

%%           Sentiment & 4,215 & 7,070 & 3,484 & 9,827 & \textbf{38.05} &
%%           %
%%           8,198 & 8,530 & 8,260 & 14,034 & \textbf{67.92} &%
%%           14,748 & 15,929 & 14,969 & 26,047 & \textbf{65.03}\\
%%           Target & 1,103 & 1,943 & 1,217 & 4,162 & \textbf{35.48} & %
%%           3,088 & 3,407 & 2,814 & 5,303 & \textbf{65.66} &%
%%           5,765 & 6,629 & 5,292 & 9,852 & \textbf{64.76}\\
%%           Source & 159 & 445 & 156 & 456 & \textbf{34.53} & %
%%           573 & 690 & 545 & 837 & \textbf{72.91} & 966 & 1,207 & 910 & 1,619 & \textbf{65.99}\\
%%           EExpression & 1,951 & 2,854 & 2,029 & 3,188 & \textbf{64.29} &%
%%           3,164 & 3,298 & 3,261 & 4,134 & \textbf{85.68} & 5,574 & 5,989 & 5,659 & 7,419 & \textbf{82.83}\\
%%           Intensifier & 57 & 101 & 59 & 123 & \textbf{51.71} &%
%%           111 & 219 & 113 & 180 & \textbf{56.01} & 192 & 432 & 194 & 338 & \textbf{49.97}\\
%%           Diminisher & 3 & 10 & 3 & 8 & \textbf{33.32} &%
%%           9 & 16 & 10 & 16 & \textbf{59.37} & 16 & 30 & 17 & 34 & \textbf{51.55}\\
%%           Negation & 21 & 63 & 21 & 83 & \textbf{28.69} &%
%%           68 & 84 & 67 & 140 & \textbf{60.21} & 111 & 132 & 110 & 243 & \textbf{58.87}\\\hline

%%           \multicolumn{16}{|c|}{\cellcolor{cellcolor}Proportional Kappa}\\\hline
%%           Sentiment & 3,269 & 6,812 & 3,269 & 9,796 & \textbf{31.21} & %
%%           7,435 & 8,243 & 7,435 & 13,714 & \textbf{61.94} & 13,316 & %
%%           15,375 & 13,316 & 25,352 & \textbf{58.82}\\
%%           Target & 898 & 1,905 & 898 & 4,148 & \textbf{26.85} & 2,554 %
%%           & 3,326 & 2,554 & 5,212 & \textbf{57,27} & 4,789 & 6,462 & %
%%           4,789 & 9,659 & \textbf{56.61}\\
%%           Source & 153 & 439 & 153 & 456 & \textbf{33.75} & 539 & 676 & %
%%           539 & 833 & \textbf{71.12} & 898 & 1,180 & 898 & 1,604 & \textbf{64.1}\\
%%           EExpression & 1,902 & 2,851 & 1,902 & 3,180 & \textbf{61.36}
%%           & %
%%           3,097 & 3,290 & 3,097 & 4,121 & \textbf{82.64} & 5,441 & 5,977 & %
%%           5,441 & 7,395 & \textbf{80.29}\\
%%           Intensifier & 57 & 101 & 57 & 123 & \textbf{50.81} & %
%%           111 & 219 & 111 & 180 & \textbf{55.51} & 192 & 432 & 192 & 338 & \textbf{49.71}\\
%%           Diminisher & 3 & 10 & 3 & 8 & \textbf{33.32} & %
%%           9 & 16 & 9 & 15 & \textbf{58.05} & 16 & 30 & 16 & 33 & \textbf{50.78}\\
%%           Negation & 21 & 63 & 21 & 83 & \textbf{28.69} & %
%%           67 & 83 & 67 & 140 & \textbf{60.03} & 110 & 131 & 110 & 242 & \textbf{58.92}\\\hline
%%     \end{tabular}
%%     \egroup
%%     \caption{Inter-coder agreement at different annotation
%%       stages.\\ {\small ($M1$ -- number of tokens with matching labels
%%         in the first annotation, $A1$ -- total number of labeled
%%         tokens in the first annotation, $M2$ -- number of tokens with
%%         matching labels in the second annotation, $A2$ -- total number
%%         of labeled tokens in the second annotation)}}
%%     \label{tbl:agrmnt}
%%   \end{center}
%% \end{table*}

%% %% \begin{table}
%% %%   \begin{center}
%% %%     \bgroup \setlength\tabcolsep{0.47\tabcolsep}
%% %%     \small
%% %%     \begin{tabular}{|p{0.2\columnwidth}| % first columm
%% %%         *{5}{>{\centering\arraybackslash}p{0.12\columnwidth}|}} % next five columns
%% %%       \hline
%% %%           {\bfseries Element} & $M_1$ & $A_1$ & $M_2$ & $A_2$ & $\kappa$\\\hline

%% %%           \multicolumn{6}{|c|}{\cellcolor{cellcolor}Binary Kappa}\\\hline

%% %%           Sentiment & 14,748 & 15,929 & 14,969 & 26,047 & 65.03\\
%% %%           Target & 5,765 & 6,629 & 5,292 & 9,852 & 64.76\\
%% %%           Source & 966 & 1,207 & 910 & 1,619 & 65.99\\
%% %%           EExpression & 5,574 & 5,989 & 5,659 & 7,419 & 82.83\\\hline

%% %%           \multicolumn{6}{|c|}{\cellcolor{cellcolor}Proportional Kappa}\\\hline

%% %%           Sentiment & 13,316 & 15,375 & 13,316 & 25,352 & 58.82\\
%% %%           Target & 4,789 & 6,462 & 4,789 & 9,659 & 56.61\\
%% %%           Source & 898 & 1,180 & 898 & 1,604 & 64.1\\
%% %%           EExpression & 5,441 & 5,977 & 5,441 & 7,395 & 80.29\\\hline
%% %%     \end{tabular}
%% %%     \egroup
%% %%     \caption{Sentiment corpus agreement.\\
%% %%       {\small ($M1$ -- number of tokens with matching labels in the
%% %%         first annotation, $A1$ -- total number of labeled tokens in
%% %%         the first annotation, $M2$ -- number of tokens with matching
%% %%         labels in the second annotation, $A2$ -- total number of
%% %%         labeled tokens in the second annotation)}}
%% %%     \label{tbl:agrmnt}
%% %%   \end{center}
%% %% \end{table}

%% \begin{table}[htb]
%%   \centering\small
%%   \caption[Inter-annotator agreement for the sentiment corpus across
%%     topics]{Inter-annotator agreement for the sentiment corpus across
%%     topics.\\ POL = corpus part with discussions about general politic
%%     topics; FE = corpus part describing the federal election 2013; PE
%%     = corpus part with discussions about the Pope election 2013; GEN =
%%     part of the corpus containing tweets with no particular topic}
%%   \begin{tabular}{|>{\centering}p{0.15\textwidth}|*{4}{>{\centering}p{\oosixthClmnWidth}|}
%%       >{\centering\bfseries}p{\oosixthClmnWidth}|*{4}{>{\centering}p{\oosixthClmnWidth}|}
%%       >{\centering\bfseries}p{\oosixthClmnWidth}|}
%%     \hline

%%     \multirow{2}{*}{\parbox{0.13\textwidth}{\centering Markable Type}}
%%     &
%%     \multicolumn{5}{>{\centering}p{7\oosixthClmnWidth}|}{$\kappa$-Agreement
%%       for Binary Overlap} &
%%     \multicolumn{5}{>{\centering}p{7\oosixthClmnWidth}|}{$\kappa$-Agreement
%%       for Proportional Overlap}\tabularnewline\cline{2-11}

%%     & POL & FE & PE & GEN & Total & POL & FE & PE & GEN &
%%     Total\tabularnewline\hline

%%     Sentiment & 0.35 & 0.35 & 0.45 & 0.41 & 0.39 & 0.27 & 0.29 & 0.36 & 0.34 & 0.32
%%     \tabularnewline\hline

%%     Source & 0.39 & 0.27 & 0.41 & 0.41 & 0.37 & 0.38 & 0.28 & 0.4 & 0.4 & 0.36
%%     \tabularnewline\hline

%%     Target & 0.32 & 0.38 & 0.4 & 0.39 & 0.38 & 0.26 & 0.28 & 0.31 & 0.32 & 0.3
%%     \tabularnewline\hline

%%     Emotional Expression & 0.64 & 0.57 & 0.68 & 0.66 & 0.64 & 0.6 & 0.54 & 0.65 & 0.63 & 0.61
%%     \tabularnewline\hline

%%     Intensifier & 0.46 & 0.48 & 0.21 & 0.62 & 0.52 & 0.46 & 0.48 & 0.21 & 0.6 & 0.51
%%     \tabularnewline\hline

%%     Diminisher & 0.67 & 0.44 & 0.0 & 0.4 & 0.37 & 0.67 & 0.44 & 0.0 & 0.4 & 0.37
%%     \tabularnewline\hline

%%     Negation & 0.44 & 0.1 & 0.36 & 0.21 & 0.28 & 0.44 & 0.1 & 0.36 & 0.21 & 0.28
%%     \tabularnewline\hline
%%   \end{tabular}
%%   \label{table:sentiment-agreement-topics}
%% \end{table}

\subsection{Analysis of Disagreements}

\subsection{Related Work}
%% The presumably first reported attempt at creating a labeled sentiment
%% dataset for Twitter was made by \citet{Go:09}.  Drawing on the work of
%% \citet{Read:05}, the authors gathered a large collection of 1.6~M
%% microblogs by downloading messages which contained positive or
%% negative smileys and then automatically assigned polarity labels
%% (positive or negative) to the collected tweets based on these
%% emoticons.  In the same vein, \citet{Pak:10} compiled a corpus of
%% 300~K short messages by retrieving microblogs with emotionally charged
%% smileys and then expanded this collection with objective tweets
%% obtained from Twitter accounts of popular newswires.

%% A slightly different way of automatically bootstrapping corpus labels
%% was proposed by \citet{Barbosa:10} who fetched 200~K microblogs along
%% with their subjectivity classes and ${\approx}\,150$~K Twitter posts
%% along with their polarity scores from three publicly available
%% sentiment web-services and used the majority votes of these services
%% as gold tags for their data.

%% As already mentioned previously, all of these datasets were created in
%% an unsupervised manner and therefore cannot serve as a reliable source
%% of information.  An attempt to alleviate this problem was made by
%% \citet{Nakov:13} as part of the SemEval competition.  The SemEval
%% corpus -- the supposedly largest hand labeled sentiment collection of
%% microblogs to date -- consists of more than 15~K tweets which were
%% manually annotated with their overall polarities and subjective
%% expressions by five experts using Amazon Mechanical Turk
%% (cf. \citet{Diakopoulos:10} for a similar crowdsourcing
%% experiment). These annotations, however, did not capture the relation
%% between the polarity expressed by tweet and the particular entity that
%% this tweet was evaluating.

%% Target-oriented sentiment annotation was attempted by \citet{Jiang:11}
%% and \citet{Saif:13}.  The former authors collected a corpus of 1,939
%% messages pertaining to one of a predefined set of five entities
%% (\emph{Obama}, \emph{Google}, \emph{iPad}, \emph{Lakers}, and
%% \emph{Lady Gaga}).\footnote{Similar work was done by Sanders group
%%   (cf. \url{http://www.sananalytics.com/lab/twitter-sentiment/}).} The
%% latter group obtained a short list of 28 popular targets that were
%% frequenty metioned on Twitter and then randomly sampled 3,000
%% microblogs related to one of these topics.  Both annotations were
%% again done manually.

%% The dataset presented in this paper significantly outperforms current
%% state of the art in many noticeable regards: First of all, except for
%% SemEval, the size of our corpus substantially surpasses any existing
%% manually labeled sentiment collections created for Twitter.  Secondly,
%% our annotation scheme is considerably richer than the one applied by
%% Nakov et al. (It not only captures polarities of subjective terms but
%% also reflects their intensities, possible negations, and modifiers as
%% well as polarities and intensities of complete opinions, also
%% capturing their respective sources and targets.) Last but not least,
%% our corpus was created for German -- a language segment that only
%% marginally was touched on in Twitter literature.

%% %% \citet{Kouloumpis:11}
%% %% \citet{Villena-Roman:13}1

%% \subsection{Conclusions}
%% As can be seen from the table, recognizing subjective evaluations of
%% particular entities (sentiments) poses a considerable problem to human
%% experts.  Even with detailed guidelines but practically no training at
%% the beginning, their mutual agreement only reached 38.05~\% when
%% measured with the less conservative binary $\kappa$ metric.  This
%% problem, however, could be alleviated by letting annotators unify
%% their judgments.  Once the differences were resolved, the experts
%% could proceed with the annotation without loosing their high level of
%% reliability (compare columns $\kappa$ for the adjudication step and
%% final corpus).

%% The low inter-coder agreement for sources and targets at the initial
%% stage, however, can be considered as a consequential error of
%% diverging sentiment concepts, i.e., when the annotators marked
%% different sentiments in the corpus, they automatically associated
%% different sources and targets with these evaluations.  But this,
%% again, could be corrected during the adjudication (after the
%% unification, binary $\kappa$ increased from $\approx\,35\%$ to more
%% than $65\%$ and remained high in the final corpus).

%% In contrast to sentiments, which always pertain to some objects and
%% are apparently difficult to agree on, independent emotional
%% expressions are seemingly easier to recognize.  As we can see from the
%% statistics, even in the initial step, our experts reached notable
%% 64.29~\%, which, according to \citet{Landis:77}, means a substantial
%% agreement.

%% The agreement figures for diminishers and negations at the first stage
%% were admittedly low though.  But this was mostly due to the low number
%% of these elements rather than contradicting interpretations of our
%% experts and also could be corrected in the later steps.

%% In general, our annotators reached a high agreement level of more than
%% 50~\% on almost all annotated elements (except for intensifiers) in
%% the final version of our corpus.  Regarding the moderately big size of
%% our data and the diversity of annotated items, we believe that our
%% dataset could thus be a valuable contribution to the broad scientific
%% community, fostering further research and providing more insight into
%% the ways how people express their opinions on the Web.
\newpage
