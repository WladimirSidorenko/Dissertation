% FILE: introduction.tex  Version 0.01
% AUTHOR: Uladzimir Sidarenka

% This is a modified version of the file main.tex developed by the
% University Duisburg-Essen, Duisburg, AG Prof. Dr. Günter Törner
% Verena Gondek, Andy Braune, Henning Kerstan Fachbereich Mathematik
% Lotharstr. 65., 47057 Duisburg entstanden im Rahmen des
% DFG-Projektes DissOnlineTutor in Zusammenarbeit mit der
% Humboldt-Universitaet zu Berlin AG Elektronisches Publizieren Joanna
% Rycko und der DNB - Deutsche Nationalbibliothek

\chapter{Introduction}

\section{Overview}
As social media become more and more ubiquitous nowadays, the need for
automatic analysis of their data rises.  But this analysis is greatly
exacerbated by the fact that the language used in these media is fundamentally
different from the one used in newspapers or scientific articles.  As we shall
see later in this work, some researchers consider the discrepancies between
the two language variants so overwhelming that they even suggest using a
machine translation approach in order to transform one form into another
\citep[cf.][]{Aw:06,Pennell:11}.

Indeed, messages like the tweets in Example \ref{intro:exmp:tweets:en} are
arguably hard to understand even for humans, let alone computer programs.
\begin{example}\label{intro:exmp:tweets:en}
Boom! Ya ur website suxx bro {\upshape (Sarah Silverman)}

\dots dats why pluto is pluto it can neva b a star {\upshape (Shaquille o'Neil)}
\end{example}
But even though almost everyone today agrees that the language of the Web is
different, not to say \emph{bad}, no formal definition exists until today of
what this difference actually is, how it is expressed, and no ultimate answer
is given to the question of what one should do about this difference.


As noted by \cite{Eisenstein:13}, ``the rise of social media has brought
computational linguistics to ever-closer contact with \emph{bad language}:
text that defies our expectations about vocabulary, spelling, and syntax.''
To prove his statement, the author gives the following examples of tweets
posted by american celebrities:

\begin{example}
Boom! Ya ur website suxx bro {\upshape (Sarah Silverman)}

... dats why pluto is pluto it can neva b a star {\upshape (Shaquille o'Neil)}
\end{example}
These messages are arguably hard to understand even for humans, let alone
computer programs.  But while almost everyone today agrees that the language
of the Web is essentially \emph{different} from the standard norm, not to say
\emph{bad}, no one can precisely define what this \emph{difference} or
\emph{badness} actually is, what it is caused by, and how it, in fact, affects
the analysis.


Another question which is closely related with the previous ones is what one
should do about the \emph{badness} of casual online texts.  Should it be
curated or should it be accepted as is, following the famous ``that's not a
bug, that's an undocumented feature!'' approach.  As we should see later, this
question remains unsolved untill today, though both approaches have their
distinct pros and contras.
