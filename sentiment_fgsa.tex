% FILE: sentiment_fgsa.tex  Version 0.0.1
% AUTHOR: Uladzimir Sidarenka

% This is a modified version of the file main.tex developed by the
% University Duisburg-Essen, Duisburg, AG Prof. Dr. G�nter T�rner
% Verena Gondek, Andy Braune, Henning Kerstan Fachbereich Mathematik
% Lotharstr. 65., 47057 Duisburg entstanden im Rahmen des
% DFG-Projektes DissOnlineTutor in Zusammenarbeit mit der
% Humboldt-Universitaet zu Berlin AG Elektronisches Publizieren Joanna
% Rycko und der DNB - Deutsche Nationalbibliothek

\chapter{Aspect-Based Sentiment Analysis}\label{chap:fgsa}

Even though polar lexicons play a crucial role in opinion-mining
research, they still serve only as a building block for achieving more
challenging and more sophisticated objectives.  One of the most
prominent such objectives is the fine-grained sentiment analysis
(FGSA), which deals with the identification of subjective evaluative
opinions (\emph{sentiments}), the holders of these opinions
(\emph{sources}), and their respectively evaluated objects
(\emph{targets}) in text.  Since an accurate automatic prediction of
these elements would enable people to track public attitude to
literally any object (\eg{} a product, service, or political
decision), FGSA is commonly considered to be one of the most
attractive, necessary, but, unfortunately, also challenging goals in
computational linguistics.

Researchers usually regard this analysis as a sequence labeling (SL)
task, and address it with either of the two popular SL techniques:
conditional random fields (CRFs) or recurrent neural networks (RNNs).
The former approach represents a discriminative probabilistic
graphical framework, which relies on an extensive set of hand-crafted
features; the latter methods utilize a recursive computational loop,
which can learn feature representations completely automatically.  In
this section, we are going to evaluate each of these systems in detail
in order to find out which of these algorithms is better suited for
the domain of German Twitter.  However, before we proceed with our
evaluation, we should first make a short linguistic digression and
briefly discuss the definition of textual spans, to which these
approaches should assign their labels, and the evaluation metrics,
with which we will estimate the quality of this assignment.

\section{Definition of the Sentiment, Target, and Source Spans}
Despite some notable advances and an ongoing active research on
fine-grained opinion extraction, the crucial task of defining the
exact boundaries of sentiment spans and the spans of their respective
targets and sources has not been addressed in the literature with the
due attention yet.  Researchers typically overlook this problem,
leaving its solution to the discretion of their annotators
\cite[cf.][]{Wiebe:05,Klinger:13}.

In contrast to these works, instead of relying on rather intuitive
decisions of our coders, we explicitly provided a rule for determining
opinions' boundaries by telling the experts to assign the
\textsc{sentiment} label to ``\emph{minimal complete syntactic or
  discourse-level units that included both the target of an opinion
  and its actual evaluation}.''

% According to this instruction, during the annotation, linguists first
% had to identify evaluated objects (targets) in text, then find the
% respective evaluative expressions of these objects (usually but not
% necessarily polar terms), and, finally, determine the smallest
% syntactic components (typically noun or verb phrases) or discourse
% units (clauses or sentences) where both of these entities appeared
% together.

A sample annotation analyzed in compliance with this rule is shown in
Example~\ref{snt:fgsa:exmp:sent-anno1}:
\begin{example}[Annotation of a Sentiment Span]\label{snt:fgsa:exmp:sent-anno1}
  \upshape\sentiment{Der neue Papst gilt als
    bescheidener, zur\"uckgenommener Typ.}\\[0.8em]
  \noindent\sentiment{The new Pope is believed to be a sober, modest
    man.}
\end{example}
\noindent In this sentence, an expert had to label the complete
sentence as a sentiment, since this unit was the minimal syntactic
constituent which included both the object of the evaluation---``der
neue Papst'' (\textit{the new pope})---and the evaluation
itself---``bescheidener, zur\"uckgenommener Typ'' (\textit{a sober,
  modest man}).

We applied the same principles of minimality and completeness to the
annotation of targets and sources, requiring the main components of
these elements (typically nouns or verbs) to be labeled along with all
their syntactic dependents.  Accordingly, the correct annotation of
the target in the previous example had to look as follows:
\begin{example}[Annotation of a Target Span]\label{snt:fgsa:exmp:sent-anno2}
  \upshape\sentiment{\target{Der neue Papst} gilt als
    bescheidener, zur\"uckgenommener Typ.}\\[0.8em]
  \noindent\sentiment{\target{The new Pope} is believed to be a sober,
    modest man.}
\end{example}
\noindent with the \textsc{target} span assigned to the whole noun
phrase---``der neue Papst'' (\textit{the new pope})---and not only its
main word.

Similarly, source elements had to cover complete syntactic structures
as shown in Example~\ref{snt:fgsa:exmp:src-anno1}:
\begin{example}[Annotation of a Source Span]\label{snt:fgsa:exmp:src-anno1}
  \upshape\sentiment{Die Homosexuellenehe war f\"ur \source{den Kardinal, der jetzt Papst ist,} eine Zerst\"orung von Gottes Plan}\\[0.8em]
  \noindent\sentiment{For \source{the cardinal, who is the Pope now,}
    the same-sex marriage was a destruction of God's plan.}
\end{example}
\noindent This time, again, the whole noun phrase including the
dependent attributive clause---``den Kardinal, der jetzt Papst ist,''
(\textit{the cardinal, who is the Pope now,})---had to be labeled with
the \textsc{source} tag because this constituent was the only
\emph{minimal complete} syntactic node which encompassed both the
immediate holder of the opinion---``Kardinal'' \textit{cardinal}---and
its grammatical dependents, without including any of its parental
elements.

\section{Evaluation Metrics}
The next question which naturally arose after defining the span
boundaries for human coders was that of the best way to compare these
spans with the automatically assigned labels.  One possibility to
estimate the quality of such automatic assignment was to compute the
precision, recall, and \F{}-scores using either the binary overlap or
exact match metrics \cite[cf.][]{Choi:06,Breck:07} .  The former
method considers an automatically labeled span as correct if it has at
least one token in common with a labeled entity from the gold
annotation.  The latter metric only regards as true positives those
automatic spans which have absolutely identical boundaries with the
expert's assignment.  Unfortunately, both of these approaches are
problematic to some extent: While the binary overlap might be overly
optimistic, always assigning perfect scores to automatic spans which
cover the whole sentence; the exact match metric might, vice versa, be
too drastic, considering the whole assignment as false if only one
(possibly irrelevant) token of it is missing.

Instead of relying on these measures, we opted for the ``golden mean''
solution to this problem that was proposed by \citet{Johansson:10a}.
In their work, the authors introduced another way of estimating the
quality of an automatic label assignment, in which they penalized the
predicted spans proportionally to the number of tokens whose labels
were different from the gold annotation.  More precisely, given a set
of manually annotated entities $\mathcal{S}$ and automatically tagged
spans $\widehat{\mathcal{S}}$, they estimated the precision of the
automatic assignment as:
\begin{equation}\label{eq:fgsa:jmmetric}
  P(\mathcal{S}, \widehat{\mathcal{S}}) = \frac{C(\mathcal{S},
    \widehat{\mathcal{S}})}{|\widehat{\mathcal{S}}|},
\end{equation}
where $C(\mathcal{S},\widehat{\mathcal{S}})$ stands for the \emph{span
  coverage} metric, which is computed as the proportion of overlapping
tokens across all pairs of manually ($s_i$) and automatically ($s_j$)
annotated entities:
\begin{equation*}
  C(\mathcal{S}, \widehat{\mathcal{S}}) = \sum_{s_i \in
    \mathcal{S}}\sum_{s_j \in \widehat{\mathcal{S}}}c(s_i, s_j),
\end{equation*}
and the $|\widehat{\mathcal{S}}|$ term denotes the total number of
spans automatically labeled with the given tag.  Similarly, the recall
of this assignment was estimated as:
\begin{equation*}
  R(\mathcal{S}, \widehat{\mathcal{S}}) = \frac{C(\mathcal{S},
    \widehat{\mathcal{S}})}{|\mathcal{S}|},
\end{equation*}
and the \F{}-measure was normally computed as the harmonic mean of the
precision and recall scores:
\begin{equation*}
  F_1 = 2\times\frac{P \times R}{P + R}.
\end{equation*}

Since this proportional estimation could adequately accommodate both
extrema of an automatic annotation---too long and too short
spans---and also penalized for erroneous labels, we decided to use
this measure throughout our subsequent experiments.

\section{Data Preparation}\label{snt:fgsa:subsec:data}

In order to evaluate the CRF and RNN approaches on our data set, we
split the whole corpus into three parts, using 70\% of it as training
data, 10\% as a development set, and the remaining 20\% as a test
corpus.  We tokenized all tweets with an adjusted version of the
Christopher Potts'
tokenizer,\footnote{\url{http://sentiment.christopherpotts.net/code-data/happyfuntokenizing.py}}
and preprocessed them using the rule-based normalization technique of
\citet{Sidarenka:13}.
%% During the normalization, Twitter-specific phenomena like @-mentions,
%% retweets, and URIs that were not syntactically integrated in any
%% sentence of the message were removed from the tweets and those
%% elements which played an integral syntactic role were replaced with
%% the special artificial tokens \%User, \%Link etc.  Emoticons like :-),
%% \smiley{}, \frownie{} etc. were also replaced with the placeholders
%% \%PosSmiley, \%NegSmiley, or simply \%Smiley depending on their prior
%% polarity.  Furthemore, out-of-vocabulary words which could be
%% converted to in-vocabulary terms with a pre-defined set of
%% transformations were also normalized.
Afterwards, we labeled the normalized data with their part-of-speech
tags using \texttt{TreeTagger}\footnote{We used \texttt{TreeTagger}
  Version 3.2 with the German parameter file UTF-8.} \cite{Schmid:95},
and parsed the resulting sentences with the \texttt{Mate} dependency
parser\footnote{We used \texttt{Mate} Version \texttt{3.61} with the
  German parameter model 3.6.}  \cite{Bohnet:13}.  Finally, since
\texttt{MMAX2} did not provide a straightforward support for the
character offsets of the annotated tokens, and the automatically
tokenized data could disagree with the original corpus tokenization,
we aligned manual annotation with the automatically split words using
the Needleman-Wunsch alignment algorithm~\cite{Needleman:70}.

\section{Conditional Random Fields}

The first method which we evaluated on the obtained data was that of
the conditional random fields.  First introduced by
\citet{Lafferty:01}, CRFs had rapidly grown in popularity, turning
into one of the most commonly used probabilistic frameworks, which had
been dominating the NLP field for more than a decade.

The main reasons for the huge success of this model are:
\begin{enumerate}[1)]
\item the \emph{structural nature} of CRFs, which, in contrast to
  single-entity classifiers, such as logistic regression or SVM, make
  their predictions over a sequence of covariates, trying to find the
  most likely label assignment to the whole input chain and not only
  its individual elements;
\item the \emph{discriminative power} of this framework, which, in
  contrast to generative probabilistic models such as HMMs
  \cite{Rabiner:86}, optimizes the conditional probability
  $P(\boldsymbol{Y}|\boldsymbol{X})$ instead of maximizing the joint
  distribution $P(\boldsymbol{X},\boldsymbol{Y})$ and consequently can
  efficiently deal with overlapping and correlated features;
\begin{example}[Overlapping and Correlated Features]
  In order to demonstrate the different effects of correlated and
  overlapping features on generative and discriminative models, let us
  go through an example where we need to predict whether a tweet
  mentioning ``Merkel'' and ``Steinmeier'' is about the Christian
  Democratic Union (\texttt{CDU}) or Social Democratic Party of
  Germany (\texttt{SPD}).

  As features for this task, we will use lexical unigrams appearing in
  the training data.  Assuming that our training set consists of three
  messages mentioning ``Merkel'' and one microblog mentioning
  ``Steinmeier'' which are labeled as \texttt{CDU}, plus one tweet
  mentioning ``Merkel'' and three posts mentioning ``Steinmeier''
  which are annotated as \texttt{SPD}, the generative Na\"{i}ve Bayes
  model would estimate the probability of the two competing classes
  as:
  \begin{align*}
    P(\mathbf{x}, CDU) =& P(\textrm{Merkel},\textrm{Steinmeier}|CDU)\times P(CDU)\\
    =& P(\textrm{Merkel}|CDU)\times P(\textrm{Steinmeier}|CDU) \times P(CDU)\\
    =&\frac{3}{4}\times\frac{1}{4}\times\frac{4}{8}\approx 0.0938\\
    P(\mathbf{x}, SPD) =& P(\textrm{Merkel},\textrm{Steinmeier}|SPD)\times P(SPD)\\
    =& P(\textrm{Merkel}|SPD)\times P(\textrm{Steinmeier}|SPD) \times P(SPD)\\
    =&\frac{1}{4}\times\frac{3}{4}\times\frac{4}{8}\approx 0.0938.\\
  \end{align*}
  After normalizing these probabilities, we would get equal 50\%
  chances for each of the parties, which is fair regarding the token
  distribution in our corpus.  However, if we replace ``Merkel'' with
  ``von der Leyen'' both in the training data and test example, and
  rerun this experiment once again, the probability would get
  significantly skewed towards the CDU class:
  \begin{align*}
    P(\mathbf{x}, CDU) =& P(\textrm{von},\textrm{der},\textrm{Leyen},\textrm{Steinmeier}|CDU)\times P(CDU)\\
    =& P(\textrm{von}|CDU)\times P(\textrm{der}|CDU)\times P(\textrm{Leyen}|CDU)\\
    &\times P(\textrm{Steinmeier}|CDU) \times P(CDU)\\
    =&\frac{3}{4}\times\frac{3}{4}\times\frac{3}{4}\times\frac{1}{4}\times\frac{4}{8}\approx 0.0527\\
    P(\mathbf{x}, SPD) =& P(\textrm{von},\textrm{der},\textrm{Leyen},\textrm{Steinmeier}|SPD)\times P(SPD)\\
    =& P(\textrm{von}|SPD)\times P(\textrm{der}|SPD)\times P(\textrm{Leyen}|SPD)\\
    &\times P(\textrm{Steinmeier}|SPD) \times P(SPD)\\
    =&\frac{1}{4}\times\frac{1}{4}\times\frac{1}{4}\times\frac{3}{4}\times\frac{4}{8}\approx 0.0059,\\
  \end{align*}
  which, after normalization, would result in 90\% chances for
  \texttt{CDU}, and a 10\% score for \texttt{SPD}, even though we only
  changed the name of the politician.

  A different situation can be observed with discriminative models
  such as maximum entropy classifier: Instead of optimizing the joint
  distribution $P(\mathbf{x}, y)$ as it is done in the generative
  frameworks, discriminative systems seek to optimize the conditional
  likelihood $P(y|\mathbf{x})$ by maximizing the total probability of
  the training set $\sum_{i=1}^N\log P(y_i|\mathbf{x}_i, \mathbf{w})$.
  This probability is usually estimated using the sigmoid function
  $\frac{1}{1 + e^{-(\mathbf{x}_i, \mathbf{w})}}$, where
  $\mathbf{x}_i$ denotes the input features of the $i$-th training
  instance, and the vector $\mathbf{w}$ stands for the respective
  weights of these features.  By optimizing this function using
  gradient descent, we will arrive at the optimal solution
  $w_1 \approx 0.5$ for the feature ``Merkel'' and $w_2 \approx -0.5$
  for the feature ``Steinmeier'' for the first example, which would
  again result in equal 50\% chances for both classes.  In the second
  example, however, all three features ``von,'' ``der,'' and ``Leyen''
  would get an equal weight of $\approx 0.3$, and the ``Steinmeier''
  feature would receive a coefficient of $\approx -0.4$, which would
  result in 60\% probability for the test message being about the CDU,
  and 40\% that the tweet is about the SPD.  Even though this still
  means a slight skewness towards \texttt{CDU}; this time, the effect
  of correlated features is much less dramatic than in the generative
  case.
\end{example}
\item and, finally, the \emph{avoidance of the label bias problem},
  which other discriminative classifiers, such as maximum entropy
  Markov networks \cite{McCallum:00}, are known to be susceptible to.
  \begin{example}[Label Bias Problem]
    The label bias problem arises in the cases where a locally optimal
    decision outweighs globally superior solutions.  Consider, for
    example, the sentence ``Aber gerade Erwachsene haben damit
    Schwierigkeiten.'' (\textit{But especially adults have
      difficulties with it.}), for which we need to compute the most
    probable sequence of part-of-speech tags.

    \begin{center}
      \begin{tikzpicture}[node distance=5cm]
        \tikzstyle{tag}=[circle split,draw=gray!50,%
          minimum size=2.5em,inner ysep=2,inner xsep=0,%
          circle split part fill={yellow!20,blue!30}]
      \tikzstyle{word}=[draw=none,inner sep=10pt]

      \node[word] (A) at (1, 1) {Aber};
      \node[tag] (B) at (1, 3) {\footnotesize KON \nodepart{lower} 1.};
      \node[word] (D) at (3, 1) {gerade};
      \node[tag] (E) at (3, 2) {\footnotesize ADJA \nodepart{lower} .5};
      \node[tag] (F) at (3, 4) {\footnotesize ADV \nodepart{lower} .5} ;
      \node[word] (G) at (7, 1) {Erwachsene};
      \node[tag] (I) at (7,2) {\footnotesize ADJA \nodepart{lower} .5} ;
      \node[tag] (H) at (7,4) {\footnotesize NN \nodepart{lower} .5};
      \node[word] (J) at (9,1) {haben};
      \node[tag] (K) at (9,3) {\footnotesize VA \nodepart{lower}\small 1.};
      \node[word] (J) at (11,1) {\ldots};

      \path [-] (B) edge node[below] {$.5$} (E);
      \path [-] (B) edge node[above] {$.5$} (F);

      \path [-] (E) edge node[below] {$.3$} (I);
      \path [-] (E) edge node[below left=0.4] {$.7$} (H);
      \path [-] (F) edge node[above left=0.4] {$.8$} (I);
      \path [-] (F) edge node[above] {$.2$} (H);

      \path [-] (I) edge node[below] {$.1$} (K);
      \path [-] (H) edge node[above] {$.9$} (K);
    \end{tikzpicture}
    \captionof{figure}{Example of a CRF graph}\label{fig:snt:memm-crf}
    \end{center}
    Assuming that feature weights are distributed as shown in
    Figure~\ref{fig:snt:memm-crf}, we will first estimate the
    probability of the correct label sequence for the initial part of
    this sentence using the Maximum Entropy Markov Model (MEMM)---the
    predecessor of the Conditional Random Fields.  According to the
    MEMM's definition, the probability of the correct labeling
    ($KON-ADV-NN-VA$) is equal to:
    \begin{align*}
      P(KON, ADV, NN, VA) &= P(KON)\times P(ADV|KON)\\
      &\times P(NN|ADV)\times P(VA|NN)\\
      &=\frac{\exp(1)}{\exp(1)}\times\frac{\exp(0.5 + 0.5)}{\exp(0.5 + 0.5) + \exp(0.5 + 0.5)}\\%
      &\times\frac{\exp(0.2 + 0.5)}{\exp(0.2 + 0.5) + \exp(0.8 + 0.5)}\\
      &\times\frac{\exp(0.9 + 1.)}{\exp(0.9 + 1.)} \approx 0.177
    \end{align*}
    At the same time, the probability of the wrong variant
    ($KON-ADV-ADJA-VA$) amounts to $\approx$ 0.323 and will therefore
    be preferred by the automatic tagger.

    A different situation is observed with the CRFs, where the
    normalizing factor in the denominator is computed over the whole
    input sequence without factorizing into individual terms for each
    transition as it is done in MEMM.  This way, the probability of
    the correct labels will run up to:
    \begin{align*}
      P(KON, ADV, NN, VA) =& P(KON)\times
      P(ADV|KON)\times P(NN|ADV)\\
      &\times P(VA|NN)\\ =&\frac{\exp(1 + 0.5
        \times 3 + 0.2 + 0.9 + 1)}{Z} \approx 0.252,
    \end{align*}
    where $Z = \exp(1 + 0.5 \times 3 + 0.2 + 0.9 + 1) + \exp(1 + 0.5
    \times 3 + 0.8 + 0.1 + 1) + \exp(1 + 0.5 \times 3 + 0.7 + 0.9 + 1)
    + \exp(1 + 0.5 \times 3 + 0.3 + 0.1 + 1)$ is the total score of
    all possible label assignments; the incorrect alternative
    ($KON-ADV-ADJA-VA$), however, will get a probability score of
    $\approx$ 0.207, which is less than the score of the correct
    labeling.
  \end{example}
\end{enumerate}
\textbf{Training.}\quad{}CRFs get these useful properties thanks to a
neatly formulated objective function in which they seek to optimize
the global log-likelihood of the gold labels $\mathbf{Y}$ conditioned
on the training data $\mathbf{X}$.  In particular, given a set of
training instances $\mathcal{D} = \{(\mathbf{x}^{(n)},
\mathbf{y}^{(n)})\}_{n=1}^N$, where $\mathbf{x}^{(n)}$ stands for the
input variables of the $n$-th instance, and $\mathbf{y}^{(n)}$ denotes
its respective gold labels, CRF's training adds up to finding feature
coefficients $\mathbf{w}$ which maximize the log-probabilities $\ell$
of $\mathbf{y}^{(i)}$ given their covariates $\mathbf{x}^{(i)}$ over
the whole corpus:
\begin{equation}\label{snt:fgsa:eq:crf-w}
  \mathbf{w} = \argmax_{\mathbf{w}}\sum_{n=1}^N\ell
  \left(\mathbf{y}^{(n)}|\mathbf{x}^{(n)}\right).
\end{equation}
The likelihood term $\ell(\mathbf{y}^{(n)}|\mathbf{x}^{(n)})$ in this
equation is commonly estimated using a globally normalized softmax
function:
\begin{equation}\label{snt:fgsa:eq:crf-ell}
  \ell\left(\mathbf{y}^{(n)}|\mathbf{x}^{(n)}\right) =
  \ln\left(P(\mathbf{y}^{(n)}|\mathbf{x}^{(n)})\right) =
  \ln\left(\frac{ \exp\left(\sum_{m=1}^{M}\sum_jw_{j} \cdot f_j(x_{m},
    y_{m-1}, y_{m})\right)}{Z}\right),
\end{equation}
where $M$ stands for the length of the $n$-th training instance,
$f_j(x_{m}, y_{m-1}, y_{m})$ denotes the value of the $j$-th feature
function $f$ at the sequence position $m$, $w_j$ represents the
corresponding weight of this feature, and $Z$ is a normalization
factor calculated over all possible label assignments:
\begin{equation}
  Z =
  \sum_{y'\in\mathcal{Y},y''\in\mathcal{Y}}\exp\left(\sum_{m=1}^{M}\sum_jw_{j}
  \times f_j(x_{m}, y'_{m-1}, y''_{m})\right).
\end{equation}
Since this normalizing term appears in the denominator, and couples
together all feature weights that need to be optimized, it becomes
prohibitively expensive to find the best solution to
Equation~\ref{snt:fgsa:eq:crf-w} analytically with a single shot.  A
possible remedy to this problem is to resort to other optimization
techniques such as gradient descent, in which the weights of the
features are successively changed towards the direction of the
gradient until the global minimum of the loss function is reached.

From Equation~\ref{snt:fgsa:eq:crf-ell}, we can see that the partial
derivative of the log-likelihood function $\ell$ w.r.t. a single
feature weight $w_j$ amounts to the following solution:
\begin{equation}
  \frac{\partial}{\partial w_j}\ell =%
  \sum_{n=1}^N\sum_{m=1}^Mf_j(x_{m}, y_{m-1}, y_{m}) -%
  \sum_{n=1}^N\sum_{m=1}^{M}\sum_{y'\in\mathcal{Y},y''\in\mathcal{Y}}f_j(x_{m},%
  y'_{m-1}, y''_{m})P(y',y''|\mathbf{x}^{(n)}),
\end{equation}
which, after dividing both parts of the equation with the constant
term $N$---the size of the corpus---can in turn be transformed into:
\begin{equation}
  \frac{1}{N}\frac{\partial}{\partial w_j}\ell = \E[f_j(\mathbf{x},
  \mathbf{y})] - \E_{\mathbf{w}}[f_j(\mathbf{x}, \mathbf{y})],
\end{equation}
where the first term ($\E[f_j(\mathbf{x}, \mathbf{y})]$) is the
expectation of the feature $f_j$ under the empirical distribution, and
the second term ($\E_{\mathbf{w}}[f_j(\mathbf{x}, \mathbf{y})]$) is
the same expectation under the model's parameters $\mathbf{w}$.  In
other words, the optimal solution to the log-likelihood objective in
Equation~\ref{snt:fgsa:eq:crf-ell} is the one, where the model's
expectation of the features matches their (true) empirical expectation
in the corpus.

The marginal probabilities of the feature attributes, which are
required for computing these expectations, can be estimated
dynamically using the forward-backward (FB) algorithm
\cite{Rabiner:90}, which is a particular case of the more general
belief propagation method \cite[cf.][p.~81]{Barber:12}.

\noindent\textbf{Inference.}\quad{}Once the optimal feature weights are
learned, one can unproblematically compute the most likely label
assignment for a new instance by using the Viterbi algorithm
\cite{Viterbi:67}, which effectively corresponds to the forward pass
of the FB method with the summation over the alternative preceding
states replaced by the maximum operator (hence the other name for this
algorithm---``max-product'').

\noindent\textbf{Features.}\quad{}A crucial component which accounts
for a huge part of the success (or failure) of a CRF system are
feature attributes which are defined by the developer.

Traditionally, feature functions in CRFs are divided into transition-
and state-based ones.  The former attributes represent real- or
binary-valued functions $f(\mathbf{x}, y'', y')\rightarrow\mathbb{R}$
associated with some data predicate
$\phi(\mathbf{x})\rightarrow\mathbb{R}$ and the labels $y''$ and $y'$.
The value of such attribute at position $m$ in sequence $\mathbf{x}$
is usually defined as:
\begin{equation}
  f(\mathbf{x}_m, y'', y') = \begin{cases} \phi(\mathbf{x}_m), &
    \mbox{if } \mathbf{y}_{m-1} = y''\mbox{ and }\mathbf{y}_{m} =
    y'\\ 0, & \mbox{otherwise;}
  \end{cases}
\end{equation}
where the predicate~$\phi$ typically represents a simple unit
function: $\phi(\mathbf{x}_m)\mapsto 1$, $\forall\mathbf{x}_m$.

In contrast to the ternary transition features, state attributes are
associated with binary predicates, whose output depends on the input
data at the given position and the label $y'$ at the respective state:
\begin{equation}
  f(\mathbf{x}_m, y') = \begin{cases} \phi(\mathbf{x}_m), & \mbox{if }
    \mathbf{y}_{m} = y'\\ 0, & \mbox{otherwise.}
  \end{cases}
\end{equation}
This time, the predicate~$\phi$ is usually much more sophisticated, as
it reflects various properties of the input sequence at the respective
position, such as whether the current token is capitalized or whether
it begins with a specific prefix or ends with a specific suffix.  This
type of features commonly accounts for the overwhelming majority of
all attributes used in a CRF system.

As state attributes in our experiments, we used the following types of
predicates (which, for simplicity, are listed here in groups):
\begin{itemize}
\item\emph{formal}, which included the initial three characters of
  each token (\eg{} $\phi_{abc}(\mathbf{x}_m) = 1\mbox{ if
  }\mathbf{x}_m\sim\mbox{ /\textasciicircum abc/ else } 0$), its last
  three characters, and the general spelling class of the token (\eg{}
  alphanumeric, digit, or punctuation);

\item\emph{morphological}, which encompassed the part-of-speech tags
  of the analyzed tokens as well as case and gender values for
  inflectable PoS types, the degree of comparison for adjectives, and
  mood, tense, and person forms for verbs;

\item\emph{lexical}, which comprised the actual lemma and form of the
  tokens (using one-hot encoding), their polarity classes (positive,
  negative, or neutral), which we obtained from the Zurich Polarity
  Lexicon~\cite{Clematide:10};

\item and, finally, \emph{syntactic}, which included the dependency
  relation via which token $x_m$ was connected to its parent; two
  binary features reflecting whether the previous token in the sentence
  was the parent or child of the current word; as well as two other
  features, one of which encoded the dependency relation of the
  previous token in the sentence to its parent + dependency relation
  of the current token to its ancestor, and the other reflected the
  dependency link of the next token + dependency relation of the
  current token to its parent.
\end{itemize}

In addition to the above attributes, we also used a set of complex
lexico-syntactic features, which simultaneously combined several
semantic and syntactic traits.  These included:
\begin{itemize}
\item the lemma of the syntactic parent;
\item the part-of-speech tag and polarity class of the grandparent;
\item the lemma of the child node + dependency relation between the
  current token and this child;
\item the PoS tag of the child node + its dependency relation + PoS
  tag of the current token;
\item the lemma of the child node + its dependency relation + lemma of
  the current token;
\item the overall polarity of the children, which was computed by
  summing up the polarity scores of all immediate syntactic
  descendants, and checking whether the resulting value was greater,
  less than or equal to zero.\footnote{We again used the Zurich
    Polarity Lexicon of~\citet{Clematide:10} for computing these
    scores.}
\end{itemize}

\textbf{Results.}\quad{}The results of our experiments are shown in
Table~\ref{snt-fgsa:tbl:crf-res}.  As we can see from the table, with
the given set of features, the model can perfectly well fit the
training data, achieving a macro-averaged \F-score of~0.904.  The
learned parameters, however, only partially generalize to unseen
tweets, leading to notably lower \F-results on the test corpus
(0.287).  This disbalance indicates a strong ``overfitting'' of the
model to the training data (\ie{} the assignment of unreasonably high
weights to rather sporadic, noisy features, which only accidentally
co-occurred with the target classes in the corpus).

Another notable tendency, which can be observed both on the training
and test splits, is that the recall of the CRF system is generally
lower than its precision.  This again can be attributed to the
overfitting effect, due to which, less indicative features became more
important than attributes which actually gave rise to subjective
evaluations.  Since the former features were unlikely to appear in the
test data or, even if they did, rarely correlated with the sentiment
entities, the model often failed to recognize sentiments in new
contexts which did have important but underweight traits.

\begin{table*}
  \begin{center}
    \bgroup \setlength\tabcolsep{0.1\tabcolsep}\scriptsize
    \begin{tabular}{p{0.162\columnwidth} % first columm
        *{9}{>{\centering\arraybackslash}p{0.074\columnwidth}} % next nine columns
        *{1}{>{\centering\arraybackslash}p{0.136\columnwidth}}} % last two columns
      \toprule
      \multirow{2}*{\bfseries Data Set} & \multicolumn{3}{c}{\bfseries Sentiment} & %
      \multicolumn{3}{c}{\bfseries Source} & %
      \multicolumn{3}{c}{\bfseries Target} & %
      \multirow{2}{0.136\columnwidth}{\bfseries\centering Macro\newline \F{}}\\\cmidrule(lr){2-4}\cmidrule(lr){5-7}\cmidrule(lr){8-10}

      & Precision & Recall & \F{} & %
      Precision & Recall & \F{} & %
      Precision & Recall & \F{} &\\\midrule

      Training Set & 0.949 & 0.908 & 0.928 & 0.903 & 0.87 & 0.886 & %
      0.933 & 0.865 & 0.898 & 0.904\\
      Test Set & 0.37 & 0.28 & 0.319 & 0.305 & 0.244 & 0.271 & 0.304 & %
      0.244 & 0.271 & 0.287\\\bottomrule
    \end{tabular}
    \egroup
    \caption{Results of fine-grained sentiment analysis with the
      first-order linear-chain CRFs}
    \label{snt-fgsa:tbl:crf-res}
  \end{center}
\end{table*}

\subsection{Effect of Features}

Since redefining the graph structures of the presented models did not
bring the expected improvements, we decided to investigate the effect
of the input provided to these systems on their net results, and also
analyze more thoroughly what these models have actually learned from
this input.  For this purpose, we first performed an ablation test of
the CRF's state features, removing one feature group at a time and
rechecking the performance of the model on the development set.

\begin{table}[hbt]
  \begin{center}
    \bgroup \setlength\tabcolsep{0.47\tabcolsep}\scriptsize
    \begin{tabular}{p{0.14\columnwidth} % first columm
        *{6}{>{\centering\arraybackslash}p{0.13\columnwidth}}} % next five columns
      \toprule
          \multirow{2}{0.2\columnwidth}{\bfseries Element} &
          \multirow{2}{0.1\columnwidth}{\bfseries Original\newline \F-Score} &
          \multicolumn{5}{c}{\bfseries \F-Score after Feature Removal}\\\cline{3-7}
          & & Formal & Morphological & Lexical & Syntactic & Complex\\\midrule

          Sentiment & 0.346 & 0.343\negdelta{0.003} & 0.344\negdelta{0.002} & 0.326\negdelta{0.02} & 0.345\negdelta{0.001} & 0.324\negdelta{0.022}\\
          Source & 0.309 & 0.321\posdelta{0.012} & 0.313\posdelta{0.004} & 0.265\negdelta{0.044} & 0.359\posdelta{0.05} & 0.271\negdelta{0.038}\\
          Target & 0.26 & 0.282\posdelta{0.022} & 0.252\negdelta{0.008} & 0.263\posdelta{0.003} & 0.233\negdelta{0.027} & 0.263\posdelta{0.003}\\\bottomrule
    \end{tabular}
    \egroup
    \caption[Results of the feature ablation tests for the CRF
    model]{Results of the feature ablation tests for the CRF
      model\\{\small\itshape (negative changes w.r.t. the original
        scores on the development set are shown in
        \textsuperscript{\textcolor{red3}{red}}; positive changes are
        depicted in \textsuperscript{\textcolor{seagreen}{green}}
        superscripts)\footnotemark}}
    \label{tbl:ablation}
  \end{center}
\end{table}

The results of this test are shown in
Table~\ref{tbl:ablation}.\footnotetext{Negative changes indicate good
  features in this context, since their removal leads to a degradation
  of results.}  As we can see from the table, all feature types turn
out to be useful for predicting sentiments as their removal
unequivocally leads to a degradation of the scores.  This quality
drop, however, is usually quite small, suggesting that other attribute
types can easily make up for the removed ones.  A different situation
is observed for sources and targets though.  In the first case,
removing formal, morphological, and syntactic features shows a
strongly positive effect, improving the \F-results for sources by up
to five percent.  However, removing lexical and lexico-syntactic
features, on the contrary, worsens these results, tearing the
\F-scores down by up to 4.4\%.  Except for the formal group, all these
attributes behave completely differently when applied to targets,
which seem to benefit from morphological and syntactic features, while
suffering a slight degradation from lexical and complex attributes.

\begin{table}[hbt]
  \begin{center}
    \bgroup \setlength\tabcolsep{0.47\tabcolsep}\scriptsize
    \begin{tabular}{%
        >{\centering\arraybackslash}p{0.045\columnwidth} % first columm
        >{\centering\arraybackslash}p{0.3\columnwidth} % second columm
        >{\centering\arraybackslash}p{0.1\columnwidth} % third columm
        >{\centering\arraybackslash}p{0.23\columnwidth} % fourth columm
        >{\centering\arraybackslash}p{0.1\columnwidth}} % next four columns
      \toprule
          \multirow{2}{0.2\columnwidth}{Rank} &
          \multicolumn{2}{c}{\bfseries State Features} &
          \multicolumn{2}{c}{\bfseries Transition Features}\\\cmidrule(lr){2-3}\cmidrule(lr){4-5}
          & Feature & Score & Feature & Score\\\midrule

          1 & prntLemma=meiste $\rightarrow$ TRG & 18.68 & NON $\rightarrow$ TRG & -7.01\\
          2 & prntLemma=rettungsschirme $\rightarrow$ TRG & 18.3 & NON $\rightarrow$ SRC & -6.85\\
          3 & initChar=sty $\rightarrow$ NON & -16.04  & NON $\rightarrow$ SNT & -5.39\\
          4 & form=meisten $\rightarrow$ NON & 15.99 & TRG $\rightarrow$ SRC & -2.99\\
          5 & prntLemma=urlauberin $\rightarrow$ SNT & 14.74 & NON $\rightarrow$ NON & 2.69\\
          6 & lemma=anfechten  $\rightarrow$ SNT & 14.07 & SRC $\rightarrow$ NON & -2.59\\
          7 & form=thomasoppermann  $\rightarrow$ TRG & 13.44 & SNT $\rightarrow$ SNT & 2.54\\
          8 & form=bezeichnete $\rightarrow$ SNT & 13.25 & TRG $\rightarrow$ TRG & 2.31\\
          9 & deprel[0]|deprel[1]=NK|AMS $\rightarrow$ NON & 12.92 & SRC $\rightarrow$ SRC & 2.19\\
          10 & trailChar=te. $\rightarrow$ NON & 12.77 & SRC $\rightarrow$ TRG & -2.07\\\bottomrule
    \end{tabular}
    \egroup
    \caption[Top-10 state and transition features learned by the CRF
    model]{Top-10 state and transition features learned by the CRF
      model\\{\small (sorted by the absolute values of their
        weights)}}
    \label{fgsa:tbl:ablation}
  \end{center}
\end{table}

In order to get a better overview of the learned model's parameters,
we also extracted top ten state and transition features ranked by the
absolute values of their learned weights (see
Table~\ref{fgsa:tbl:ablation}).  As can be seen from the statistics,
three of the five highest ranked state attributes are complex features
reflecting the lemma of the parent token: ``meiste'' (\emph{most}) and
``rettungsschirme'' (\emph{bailout}), which typically indicate a
target, and ``urlauberin'' (\emph{holiday}), which frequently
correlates with sentiments.  Another common group of features are the
lemma and form of the current token: here, we again encounter
``meisten'' (\emph{most}), which, however, indicates the absence of
any sentiment entities at the current position; two other
attributes---``anfechten'' (\emph{doubt}) and ``bezeichnete''
(\emph{called})---represent the so-called \emph{direct speech events}
and expectedly correlate with the sentiment tags; the remaining
feature---``thomasopperman''---is a person name, which frequently
appeared as a target of an opinion in the corpus.

An interesting pattern can be observed for the transition attributes:
As we can see from the results, the top-three transition features
indicate a strong belief that an objective token is highly unlikely to
be followed by a sentiment entity (hence, the high negative weights of
the transitions emanating from \textsc{NON}).  It is, however, quite
common that a \textsc{NON} tag will precede another \textsc{NON}
instance (cf. line 5 of the table).  Other transition attributes also
mainly reflect plausible regularities: It is, for instance, uncommon
that a target of an opinion will appear immediately before a source
(\textsc{TRG}$\rightarrow$\textsc{SRC} $= -2.99$); in the same vein,
it is rather uncommon that a source tag will precede a target
(\textsc{SRC}$\rightarrow$\textsc{TRG} $= -2.07$); nonetheless, is is
perfectly acceptable that the same tag will continue over multiple
words (\eg{} \textsc{SNT}$\rightarrow$\textsc{SNT} $= 2.54$,
\textsc{TRG}$\rightarrow$\textsc{TRG} $= 2.31$).

\section{Recurrent Neural Networks}

A competitive alternative to the conditional random fields are deep
recurrent neural networks (RNNs).  Introduced in the
mid-nineties~\cite{Hochreiter:97}, RNNs have become one of the most
popular trends in the recent surge of deep learning research, showing
superior performance on many important NLP tasks including
part-of-speech tagging~\cite{Wang:15:pos}, dependency
parsing~\cite{Kiperwasser:16a}, machine
translation~\cite{Kalchbrenner:13,Bahdanau:14,Sutskever:14} etc.  The
key factors which account for this success are
\begin{enumerate}[1)]
\item \emph{the ability of these systems to learn optimal feature
    representations automatically}, which favorably sets them apart
  from traditional supervised machine learning frameworks such as SVMs
  or CRFs where all features need to be defined by user; and
\item \emph{the ability to deal with arbitrary sequence lengths},
  which advantageously distinguishes these approaches from other NN
  architectures such as convolutional or plain feed-forward networks
  where the size of all layers has to be constant.
\end{enumerate}

The main component which underlies any modern RNN approach is a
fixed-size hidden vector $\vec{h}$, which gets recurrently updated
over the input sequence $\mathbf{x}$, and is supposed to encode the
meaning of the sequence seen so far.  The general form of this vector
at input position $t$ is usually defined as:
\begin{align}
  \vec{h}^{(t)} = f(\vec{h}^{(t-1)}, \mathbf{x}^{(t)});
\end{align}
where $f$ represents some pointwise non-linear transformation
function, $\vec{h}^{(t-1)}$ denotes the state of the hidden vector at
the previous time step, and $\mathbf{x}^{(t)}$ is the input vector at
position $t$.

\textbf{LSTM.}\quad{}A fundamental problem which arises from the above
definition is that the gradients of the trained parameters (the ones
involved in computing the $\vec{h}$ vector), rapidly vanish to zero or
explode to infinity (depending on whether the absolute values of
$\vec{h}$ are less or greater than one) as the length of the input
sequence increases.  In order to solve this issue,
\citet{Hochreiter:97} proposed the long short-term memory mechanism
(LSTM), in which they explicitly incorporated the goal of dropping
parts of the input which appeared to be irrelevant for the final
outcome.  In particular, given an input sequence $\mathbf{x}$, they
introduced a special \emph{activation unit} $\vec{i}^{(t)}$:
\begin{align*}
  \vec{i}^{(t)} &= \sigma\left(W_i\cdot \mathbf{x}^{(t)} + U_i \cdot \vec{h}^{(t-1)} + \vec{b}_i\right);
\end{align*}
where $\sigma$ denotes the sigmoid function; $W_i$, $U_i$, and
$\vec{b_i}$ represent the optimized model's matrices and vector; and
$\mathbf{x}^{(t)}$ and $\vec{h}^{(t-1)}$ stand for the input and
previous hidden states respectively.  In addition to that, the authors
also estimated a dedicated \emph{forget gate}~$\vec{f}^{(t)}$:
\begin{align*}
  \vec{f}^{(t)} &= \sigma\left(W_f\cdot \mathbf{x}^{(t)} + U_f
  \cdot \vec{h}^{(t-1)} + \vec{b}_f\right),
\end{align*}
which was then used to erase parts of the previous input.

After computing an \emph{intermediate update value}
$\widetilde{c}^{(t)}$ for the current time step~$t$:
\begin{align*}
  \widetilde{c}^{(t)} &= tanh\left(W_c\cdot \mathbf{x}^{(t)} + U_c
  \cdot \vec{h}^{(t-1)} + \vec{b}_c\right),
\end{align*}
they estimated the \emph{final update}~$\vec{c}^{(t)}$ by taking a
weighted sum of the candidate update vector~$\widetilde{c}^{(t)}$ and
the previous update value~$\vec{c}^{(t-1)}$:
\begin{align*}
  \vec{c}^{(t)} &= \vec{i}^{(t)} \odot \widetilde{c}^{(t)} + \vec{f}^{(t)} \odot \vec{c}^{(t-1)};
\end{align*}
from which, they finally computed the output vector
$\vec{o}^{(t)}$ and the new value of the hidden state $\vec{h}^{(t)}$:
\begin{align*}
  \vec{o}^{(t)} &= \sigma\left(W_o\cdot \mathbf{x}^{(t)} + U_o \cdot \vec{h}^{(t-1)} + V_o \cdot \vec{c}^{(t)} + \vec{b}_o\right),\\
  \vec{h}^{(t)} &= \vec{o}^{(t)} \odot tanh(\vec{c}^{(t)}).
\end{align*}

\textbf{GRU.}\quad{}Despite their enormous popularity and many
successful practical
applications~\cite[cf.][]{Filippova:15,Ghosh:16,Rao:16}, LSTMs have
often been criticized for the high complexity of the recurrent unit.
In order to overcome this problem while still keeping the gradients
within an appropriate range, \citet{Cho:14a} introduced an alternative
architecture called Gated Recurrent Units (GRU).  In this framework,
the authors also used activation and forget gates---$\vec{i}^{(t)}$
and $\vec{f}^{(t)}$---similar to the ones defined
by~\citet{Hochreiter:97}:
\begin{align*}
  \vec{i}^{(t)} &= \sigma\left(W_i\cdot \mathbf{x}^{(t)} + U_i \cdot \vec{h}^{(t-1)} + \vec{b}_i\right),\\
  \vec{f}^{(t)} &= \sigma\left(W_f\cdot \mathbf{x}^{(t)} + U_f \cdot \vec{h}^{(t-1)} + \vec{b}_f\right).
\end{align*}
The candidate activation $\widetilde{c}^{(t)}$ was then estimated as:
\begin{align*}
  \widetilde{c}^{(t)} &= tanh\left(W_c\cdot \mathbf{x}^{(t)} + U_c
  \cdot \left(\vec{f}^{(t)} \odot \vec{h}^{(t-1)}\right)  + \vec{b}_c\right),
\end{align*}
and the final hidden state $\vec{h}^{(t)}$ was computed as follows:
\begin{align*}
  \vec{h}^{(t)} &= \vec{i}^{(t)} \odot \vec{h}^{(t-1)} + \left(\vec{1} -
  \vec{i}^{(t)}\right) \odot \widetilde{c}^{(t)}.
\end{align*}

Since the output vectors of these recurrences---$\vec{o}^{(t)}$ in the
LSTM case, and $\vec{h}^{(t)}$ in the case of GRU---did not strictly
represent probabilities of the labels (as their elements could also be
negative, and their sum not necessarily run up to one), and, moreover,
because the size of the final tagset (four tags: \textsc{SNT},
\textsc{SRC}, \textsc{TRG}, and \textsc{NON}) was unequivocally too
small for the size of the hidden unit, we set the dimensionality of
the intermediate RNN vectors to 100, and applied another linear
transformation matrix $O \in \mathbb{R}^{4 \times 100}$ to the final
output of the recursion loop, computing the softmax of their dot
product:
\begin{align*}
  \vec{p}^{(t)} &= softmax\left(O\cdot\vec{o}^{(t)}\right).
\end{align*}

\textbf{Training.}\quad{}A neat property of both of these approaches
is that the final equation, which is obtained after unrolling the
loop, is differentiable with respect to all of its parameters, and can
therefore be optimized using the standard gradient update techniques.
Since most of these parameters, however, represent high-dimensional
matrices or vectors, finding an optimal learning rate (\ie{} the size
of the update step taken in the direction of the gradient) might pose
considerable difficulties, leading either to prohibitively large
training times (if the steps are too small) or a complete divergence
of the trained model (if the steps are too big).

Several algorithms have been proposed for solving this problem
including the method of momentum~\cite{Rumelhart:88},
AdaGrad~\cite{Duchi:11}, AdaDelta~\cite{Zeiler:12},
RmsProp~\cite{Tieleman:12} etc.  In our RNN experiments, we chose the
last of these options---the RmsProp technique
of~\citet{Tieleman:12}---as this algorithm showed both a faster
convergence and superior classification results.

Another important factor, which could significantly affect the
training results, were the initial values of the models' parameters.
As shown by~\citet{He:15}, an inappropriate initialization of neural
network might lead to a complete stalling of the whole learning
process.  Following the recommended practices~\cite{Saxe:13}, we used
an orthogonal initialization for all linear transformation matrices,
and applied the uniform He sampling~\cite{He:15} for setting the
initial values of the bias vectors.

Finally, due to a high imbalance of the target classes in the training
set (where most of the instances represented objective statements
without any sentiment tags), we ``over-sampled'' opinionated tweets
(\ie{} we randomly repeated some of the training microblogs containing
sentiments until we reached an equal proportion of subjective and
objective messages), and chose the \emph{hinge-loss} as the optimized
objective function $L$:\footnote{Since most of the tokens in the
  over-sampled training set still had the \textsc{NON} tag, the
  easiest way for a classifier to minimize the objective function was
  to always predict this tag with a very high confidence.  We hoped to
  mitigate this effect by using the hinge-loss, since this function
  only penalized incorrectly predicted labels or correct tags whose
  probability was insufficiently high (less than $c$), but did not
  reward any over-confident decisions.}
\begin{align}
  L &= \sum_{i}^{N}\sum_{t=0}^{\lvert\mathbf{x}_i\rvert}\max\left(0, %
  c + \max\limits_{y'\neq y}\vec{p}_{t,y'} - \vec{p}_{t,y}\right) + \alpha \norm{O}^2_2,
\end{align}
where $\vec{p}_{t,y'}$ stands for the probability of the most likely
wrong tag $y'$ at position $t$ in the training instance
$\mathbf{x}_i$, $\vec{p}_{t,y}$ represents the probability of the gold
label, and $\norm{O}^2_2$ stands for the $L2$-norm of the $O$ matrix.

We optimized the scalar hyper-parameters $c$ and $\alpha$ on the
development set, and trained the final model for 256 epochs, choosing
parameter values which maximized the macro-averaged \F-score on the
development data.

\textbf{Inference.}\quad{}Since each of the above approaches (LSTM and
GRU) explicitly defines an output unit, the inference of the most
likely label assignment for an input instance $\mathbf{x}$ is
straightforward and amounts to finding the $\argmax$ value of the
output vector at each time step of the recurrence:
\begin{equation}
  \mathbf{\hat{y}} =
  \argmax{\vec{p}^{(1)}},\argmax{\vec{p}^{(2)}},\ldots,\argmax{\vec{p}^{(|\mathbf{x}|)}}.
\end{equation}

\textbf{Results.}\quad{}To account for the random factors in the
initialization, we repeated each training experiment three times, and
show the mean and standard deviation of these results in
Table~\ref{snt-fgsa:tbl:rnn-res}.

As we can see from the table, the LSTM model shows generally better
scores than the GRU system on both training and test data.  The only
aspect at which it yields slightly worse results than the latter
approach is the precision of the sentiment spans, which, however, is
more than compensated for by a much higher recall.  Moreover, the
overfitting effect is significantly less pronounced than in the CRF
case (where the \F{}-scores on the training and test data differed by
a factor of three).  Nonetheless, both RNN systems achieve lower
results than the linear-chain CRFs, which indicates the fact that
deeply learned features still cannot capture the full extent of the
information that a human expert can encode with manually defined
attributes.

\begin{table*}
  \begin{center}
    \bgroup \setlength\tabcolsep{0.1\tabcolsep}\scriptsize
    \begin{tabular}{p{0.162\columnwidth} % first columm
        *{9}{>{\centering\arraybackslash}p{0.074\columnwidth}} % next nine columns
        *{1}{>{\centering\arraybackslash}p{0.136\columnwidth}}} % last two columns
      \toprule
      \multirow{2}*{\bfseries Data Set} & \multicolumn{3}{c}{\bfseries Sentiment} & %
      \multicolumn{3}{c}{\bfseries Source} & %
      \multicolumn{3}{c}{\bfseries Target} & %
      \multirow{2}{0.136\columnwidth}{\bfseries\centering Macro\newline \F{}}\\\cmidrule(lr){2-4}\cmidrule(lr){5-7}\cmidrule(lr){8-10}
      & Precision & Recall & \F{} & %
      Precision & Recall & \F{} & %
      Precision & Recall & \F{} &\\\midrule

      \multicolumn{11}{c}{\cellcolor{cellcolor}LSTM}\\
      %%  Tag        Precision    Recall F-Measure
      %% O             86.60%    86.63%    86.62%
      %% SENTIMENT     55.50%    74.72%    63.69%
      %% SOURCE        46.27%    68.82%    55.33%
      %% TARGET        43.12%    77.99%    55.53%

      %% Tag        Precision    Recall F-Measure
      %% O             88.76%    67.71%    76.82%
      %% SENTIMENT     30.76%    76.51%    43.88%
      %% SOURCE        38.84%    49.82%    43.65%
      %% TARGET        29.45%    67.18%    40.95%

      %% Tag        Precision    Recall F-Measure
      %% O             86.38%    89.91%    88.11%
      %% SENTIMENT     60.90%    74.87%    67.16%
      %% SOURCE        48.96%    71.18%    58.02%
      %% TARGET        50.61%    74.58%    60.30%

      Training Set & 0.49\stddev{0.16} & 0.75\stddev{0.01} & 0.58\stddev{0.13} & %
      0.45\stddev{0.05} & 0.63\stddev{0.12} & 0.52\stddev{0.08} %
      & 0.41\stddev{0.11} & 0.73\stddev{0.06} & 0.52\stddev{0.11} %
      & 0.54\stddev{0.11}\\

      %% Tag        Precision    Recall F-Measure
      %% O             77.77%    82.91%    80.26%
      %% SENTIMENT     31.69%    28.00%    29.73%
      %% SOURCE        23.05%    31.25%    26.53%
      %% TARGET        21.77%    23.57%    22.63%

      %% Tag        Precision    Recall F-Measure
      %% O             79.83%    71.62%    75.50%
      %% SENTIMENT     26.23%    43.07%    32.60%
      %% SOURCE        26.07%    30.68%    28.19%
      %% TARGET        21.53%    30.16%    25.13%

      %% Tag        Precision    Recall F-Measure
      %% O             77.58%    86.08%    81.61%
      %% SENTIMENT     30.16%    22.60%    25.84%
      %% SOURCE        24.42%    30.60%    27.16%
      %% TARGET        24.99%    21.20%    22.94%

      Test Set & 0.29\stddev{0.03} & \textbf{0.31}\stddev{0.11} & \textbf{0.29}\stddev{0.03} &%
      \textbf{0.25}\stddev{0.02} & \textbf{0.31}\stddev{0.0} & \textbf{0.27}\stddev{0.01} & %
      \textbf{0.23}\stddev{0.02} & \textbf{0.25}\stddev{0.05} & \textbf{0.24}\stddev{0.01} & %
      \textbf{0.27}\stddev{0.02}\\

      \multicolumn{11}{c}{\cellcolor{cellcolor}GRU}\\

      %% Tag        Precision    Recall F-Measure
      %% O             82.40%    89.44%    85.77%
      %% SENTIMENT     60.16%    60.25%    60.21%
      %% SOURCE        44.77%    66.74%    53.59%
      %% TARGET        46.36%    63.22%    53.49%

      %% Tag        Precision    Recall F-Measure
      %% O             87.39%    80.32%    83.71%
      %% SENTIMENT     44.88%    70.75%    54.92%
      %% SOURCE        39.69%    63.31%    48.79%
      %% TARGET        35.37%    73.96%    47.86%

      %% Tag        Precision    Recall F-Measure
      %% O             81.03%    87.80%    84.28%
      %% SENTIMENT     47.95%    66.84%    55.84%
      %% SOURCE        42.49%    56.86%    48.64%
      %% TARGET        58.22%    51.08%    54.42%

      Training Set & 0.51\stddev{0.08} & 0.66\stddev{0.05} & 0.57\stddev{0.03} & %
      0.42\stddev{0.03} & 0.62\stddev{0.05} & 0.5\stddev{0.03} & %
      0.47\stddev{0.11} & 0.63\stddev{0.11} & 0.52\stddev{0.04} & 0.53\stddev{0.03}\\

      %% Tag        Precision    Recall F-Measure
      %% O             76.78%    86.47%    81.34%
      %% SENTIMENT     30.77%    19.68%    24.01%
      %% SOURCE        20.71%    26.44%    23.22%
      %% TARGET        24.20%    20.93%    22.45%

      %% Tag        Precision    Recall F-Measure
      %% O             78.15%    79.14%    78.64%
      %% SENTIMENT     28.61%    30.25%    29.41%
      %% SOURCE        19.71%    29.45%    23.62%
      %% TARGET        21.54%    27.49%    24.15%

      %% Tag        Precision    Recall F-Measure
      %% O             77.52%    86.03%    81.56%
      %% SENTIMENT     30.66%    28.43%    29.50%
      %% SOURCE        24.46%    29.09%    26.58%
      %% TARGET        27.17%    14.15%    18.61%

      Test Set & \textbf{0.3}\stddev{0.01} & 0.26\stddev{0.06} & 0.28\stddev{0.03} & %
      0.22\stddev{0.03} & 0.28\stddev{0.02} & 0.24\stddev{0.02} & %
      0.24\stddev{0.03} & 0.21\stddev{0.07} & 0.22\stddev{0.03} & 0.25\stddev{0.01}\\\bottomrule
    \end{tabular}
    \egroup
    \caption{Results of fine-grained sentiment analysis with recurrent
      neural networks}
    \label{snt-fgsa:tbl:rnn-res}
  \end{center}
\end{table*}

\subsection{Effect of Word Embeddings}

Similarly to the CRF features, we also investigated the impact of the
input embeddings on the final results of the RNN approaches.  For this
purpose, instead of learning task-specific word representations as we
did in the initial experiments, we re-ran the training using two other
available options:
\begin{itemize}
\item\emph{least-squares word vectors}, in which we again learned
  task-specific word representations, but additionally computed an
  optimal transformation matrix $W$ using the method of the ordinary
  least squares:
  \begin{equation}\label{eq:fgsa:least-sq}
    W = \argmin_{W}\lVert V_{TS} - W^T\cdot V_{W2V}\rVert_F,
  \end{equation}
  where $V_{TS}$ stands for the matrix of task-specific word
  representations learned by the system, $V_{W2V}$ represents the
  respective matrix of word2vec embeddings, and
  $\left\lVert\cdot\right\rVert_F$ is the Frobenius norm of the
  resulting difference.  We then used the learned task-specific
  representations for all known words encountered during testing, and
  obtained the best possible task-specific approximations for all
  unknown words by taking a dot product of the matrix $W$ with their
  word2vec vectors;
\item another option was to use the normal \emph{word2vec
    embeddings}~\cite{Mikolov:13}, which we previously trained on the
  German Twitter snapshot~\cite{Scheffler:14} using the default
  options of the original word2vec implementation.  These embeddings
  were kept fixed during the RNN training and did not get updated by
  the optimization procedure.
\end{itemize}

The results for these alternatives are given in
Table~\ref{snt-fgsa:tbl:embeddings}.  As we can see from the scores,
the least-squares method significantly boosts the recall, which, in
turn, leads to much higher macro-averaged \F-measures, outperforming
all other compared approaches. The task-specific variant shows
second-best results, mainly due to a higher precision of the targets
and sources.  Last but not least, the word2vec approach still improves
the prediction of sentiment spans, but otherwise leads to a notable
degradation at literally every other opinion-related aspect.

\begin{table*}
  \begin{center}
    \bgroup \setlength\tabcolsep{0.1\tabcolsep}\scriptsize
    \begin{tabular}{p{0.162\columnwidth} % first columm
        *{9}{>{\centering\arraybackslash}p{0.074\columnwidth}} % next nine columns
        *{1}{>{\centering\arraybackslash}p{0.136\columnwidth}}} % last two columns
      \toprule
      \multirow{2}*{\bfseries RNN} & \multicolumn{3}{c}{\bfseries Sentiment} & %
      \multicolumn{3}{c}{\bfseries Source} & %
      \multicolumn{3}{c}{\bfseries Target} & %
      \multirow{2}{0.136\columnwidth}{\bfseries\centering Macro\newline \F{}}\\\cmidrule(lr){2-4}\cmidrule(lr){5-7}\cmidrule(lr){8-10}
      & Precision & Recall & \F{} & %
      Precision & Recall & \F{} & %
      Precision & Recall & \F{} &\\\midrule

      \multicolumn{11}{c}{\cellcolor{cellcolor}Task-Specific Embeddings}\\

      LSTM & 0.283 & 0.288 & 0.278 & %
       \textbf{0.293} & 0.372 & 0.328 & %
       \textbf{0.254} & 0.27 & \textbf{0.259} & 0.288\\

      GRU & 0.287 & 0.246 & 0.263 & %
       0.287 & 0.405 & \textbf{0.335} & %
       0.252 & 0.205 & 0.216 & 0.271\\

      \multicolumn{11}{c}{\cellcolor{cellcolor}Least-Squares Embeddings}\\

      LSTM & 0.268 & \textbf{0.37} & 0.307 & %
      0.261 & \textbf{0.414} & 0.314 & %
      0.223 & \textbf{0.275} & 0.245 & \textbf{0.289}\\

      GRU & 0.256 & 0.341 & 0.291 & %
       0.267 & 0.395 & 0.318 & %
       0.229 & 0.262 & 0.245 & 0.285\\

      \multicolumn{11}{c}{\cellcolor{cellcolor}word2vec Embeddings}\\

      LSTM & \textbf{0.291} & 0.329 & \textbf{0.309} & %
       0.2 & 0.311 & 0.244 & %
       0.221 & 0.219 & 0.22 & 0.257\\

      GRU & 0.273 & 0.355 & 0.301 & %
       0.207 & 0.353 & 0.257 & %
       0.213 & 0.26 & 0.233 & 0.264\\\bottomrule
    \end{tabular}
    \egroup
    \caption{Results of fine-grained sentiment analysis with different
      word embeddings}
    \label{snt-fgsa:tbl:embeddings}
  \end{center}
\end{table*}

\section{Evaluation}

After estimating the results of the most popular FGSA approaches with
the (mostly) standard settings, we decided to investigate the impact
of different external factors on the net scores of these methods.  For
this purpose, we reran the evaluation, changing one aspect of the
training procedure at a time, and re-estimated the scores on the
development set (in order to keep the test corpus undisclosed).  The
results of these experiments are presented below.

\subsection{Annotation Scheme}

As the first factor which could significantly affect the quality of
the automatic approaches, we considered the annotation scheme that we
used to create the corpus.  As described in
Section~\ref{subsec:snt:ascheme}, we initially asked our experts to
assign the \textsc{sentiment} label to complete syntactic or
discourse-level units which encompassed both the target of an opinion
and its immediate evaluation.  Even though this decision was
linguistically plausible and extremely helpful for determining the
boundaries of opinions, it also posed considerable difficulties for
sequence labeling approaches, since the same tag got assigned not only
to immediately subjective words, but also to objective terms which
resided within the same syntactic constituent as the polar term and
its target.  Since none of the tested methods could explicitly
incorporate this logic, we decided to check whether an alternative
interpretation of the annotation scheme could alleviate their
inference.

In particular, instead of unconditionally labeling all words belonging
to a sentiment span in the original annotation with the \textsc{SNT}
tag as we did previously (which we call a \emph{broad} interpretation
of the annotation scheme), we only assigned this label to the
emotional expressions found in the corpus (which we term a
\emph{narrow} interpretation of the scheme).  The difference between
these two interpretations is shown in
Examples~\ref{snt:fgsa:exmp:wide} and~\ref{snt:fgsa:exmp:narrow}.
\begin{example}[Broad Sentiment
  Interpretation]\label{snt:fgsa:exmp:wide}
  \noindent\sentiment{\target{Francis} makes a \intensifier{very}
    \emoexpression{good} impression on\\ \source{me}!
    \emoexpression{:)}}

  $\rightarrow$

  \noindent Francis/TRG makes/SNT a/SNT very/SNT good/SNT
  impression/SNT on/SNT\\ me/SRC !/SNT :)/SNT
\end{example}

\begin{example}[Narrow Sentiment Interpretation]\label{snt:fgsa:exmp:narrow}
  \noindent\sentiment{\target{Francis} makes a \intensifier{very}
    \emoexpression{good} impression on\\ \source{me}!
    \emoexpression{:)}}

  $\rightarrow$

  \noindent Francis/TRG makes/NON a/NON very/NON good/SNT
  impression/NON on/NON\\ me/SRC !/NON :)/SNT
\end{example}
\noindent In the former (broad) case, we labeled the whole opinionated
sentence with the \textsc{SNT} tag except for the words which denoted
the target and source of the opinion.  In the latter (narrow) case, we
only assigned the \textsc{SNT} tag to the emotional expression
``good'' and the emoticon ``:),'' which, however, were expressive
enough to convey the main evaluative sense of the whole subjective
statement.

\begin{table*}[hbt!]
  \begin{center}
    \bgroup \setlength\tabcolsep{0.1\tabcolsep}\scriptsize
    \begin{tabular}{p{0.162\columnwidth} % first columm
        *{9}{>{\centering\arraybackslash}p{0.074\columnwidth}} % next nine columns
        *{1}{>{\centering\arraybackslash}p{0.136\columnwidth}}} % last two columns
      \toprule
      \multirow{2}*{\bfseries Method} & \multicolumn{3}{c}{\bfseries Sentiment} & %
      \multicolumn{3}{c}{\bfseries Source} & %
      \multicolumn{3}{c}{\bfseries Target} & %
      \multirow{2}{0.136\columnwidth}{\bfseries\centering Macro\newline \F{}}\\\cmidrule(lr){2-4}\cmidrule(lr){5-7}\cmidrule(lr){8-10}
      & Precision & Recall & \F{} & %
      Precision & Recall & \F{} & %
      Precision & Recall & \F{} &\\\midrule

      \multicolumn{11}{c}{\cellcolor{cellcolor}Broad Interpretation}\\

      %% SENTIMENT     37.62%    31.85%    34.49%
      %% SOURCE        29.75%    33.00%    31.29%
      %% TARGET        29.25%    23.06%    25.79%

      CRF & 0.38 & 0.32 & 0.34 & %
      \textbf{0.3} & 0.33 & 0.31 & %
      \textbf{0.29} & 0.23 & \textbf{0.26} & 0.31\\

      % Tag        Precision    Recall F-Measure
      % O             77.89%    70.02%    73.75%
      % SENTIMENT     24.03%    37.51%    29.29%
      % SOURCE        28.52%    37.43%    32.37%
      % TARGET        22.73%    30.54%    26.06%

      % Tag        Precision    Recall F-Measure
      % O             76.44%    86.10%    80.98%
      % SENTIMENT     30.53%    23.22%    26.38%
      % SOURCE        27.74%    36.33%    31.46%
      % TARGET        28.76%    23.59%    25.92%

      % Tag        Precision    Recall F-Measure
      % O             76.87%    83.75%    80.16%
      % SENTIMENT     30.39%    25.69%    27.84%
      % SOURCE        31.59%    37.88%    34.45%
      % TARGET        24.59%    26.94%    25.71%

      % Summary:
      % Tag             Precision    Recall        F1
      % SENTIMENT           28.32     28.81     27.84
      % SOURCE              29.28     37.21     32.76
      % TARGET              25.36     27.02     25.90
      % Macro-F1  28.8311

      LSTM & 0.28 & 0.29 & 0.28 & %
      0.29 & \textbf{0.37} & \textbf{0.33} & %
      0.25 & \textbf{0.27} & \textbf{0.26} & 0.29\\

      % Tag        Precision    Recall F-Measure
      % O             76.64%    86.78%    81.39%
      % SENTIMENT     30.31%    19.73%    23.90%
      % SOURCE        27.54%    39.00%    32.28%
      % TARGET        23.10%    19.12%    20.93%

      % Tag        Precision    Recall F-Measure
      % O             77.05%    78.92%    77.97%
      % SENTIMENT     28.27%    28.35%    28.31%
      % SOURCE        27.30%    43.05%    33.41%
      % TARGET        22.68%    29.29%    25.56%

      % Tag        Precision    Recall F-Measure
      % O             76.82%    85.63%    80.98%
      % SENTIMENT     27.49%    25.78%    26.60%
      % SOURCE        31.34%    39.30%    34.87%
      % TARGET        29.86%    13.15%    18.26%

      % Summary
      % Tag             Precision    Recall        F1
      % SOURCE              28.73     40.45     33.52
      % SENTIMENT           28.69     24.62     26.27
      % TARGET              25.21     20.52     21.58
      % Macro-F1  27.1244

      GRU & 0.29 & 0.25 & 0.26 & %
       0.29 & 0.4 & 0.34 & %
       0.25 & 0.21 & 0.22 & 0.27\\

      \multicolumn{11}{c}{\cellcolor{cellcolor}Narrow Interpretation}\\

      %% Tag        Precision    Recall F-Measure
      %% O             85.93%    85.69%    85.81%
      %% SENTIMENT     58.84%    64.49%    61.54%
      %% SOURCE        26.13%    23.00%    24.47%
      %% TARGET        22.14%    20.14%    21.09%

      CRF & 0.59 & 0.64 & 0.62 & %
      0.26 & 0.23 & 0.24 & %
      0.22 & 0.20 & 0.21 & 0.36\\

      % Tag        Precision    Recall F-Measure
      % O             85.08%    90.61%    87.76%
      % SENTIMENT     57.18%    66.52%    61.50%
      % SOURCE        27.22%    40.25%    32.48%
      % TARGET        25.54%    12.56%    16.84%

      % Tag        Precision    Recall F-Measure
      % O             84.57%    91.93%    88.10%
      % SENTIMENT     69.83%    60.92%    65.07%
      % SOURCE        32.28%    35.55%    33.84%
      % TARGET        26.04%    16.18%    19.96%

      % Tag        Precision    Recall F-Measure
      % O             84.05%    91.82%    87.77%
      % SENTIMENT     58.79%    67.52%    62.85%
      % SOURCE        30.90%    27.85%    29.30%
      % TARGET        27.07%    13.31%    17.85%

      LSTM & \textbf{0.62} & \textbf{0.65} & \textbf{0.63} & %
      \textbf{0.3} & 0.35 & 0.32 & %
      0.26 & 0.14 & 0.18 & \textbf{0.38}\\

      %% Tag        Precision    Recall F-Measure
      %% O             84.78%    87.61%    86.17%
      %% SENTIMENT     59.02%    61.71%    60.34%
      %% SOURCE        27.35%    37.83%    31.75%
      %% TARGET        22.37%    21.18%    21.76%

      %% Tag        Precision    Recall F-Measure
      %% O             84.38%    90.46%    87.31%
      %% SENTIMENT     60.14%    64.76%    62.36%
      %% SOURCE        27.93%    29.53%    28.71%
      %% TARGET        27.40%    20.07%    23.17%

      % Tag        Precision    Recall F-Measure
      % O             84.98%    85.60%    85.29%
      % SENTIMENT     65.71%    61.64%    63.61%
      % SOURCE        29.91%    31.20%    30.54%
      % TARGET        19.18%    31.63%    23.88%

      GRU & \textbf{0.62} & 0.63 & 0.62 & %
      0.28 & 0.33 & 0.3 & %
      0.23 & 0.24 & 0.23 & \textbf{0.38}\\\bottomrule

    \end{tabular}
    \egroup
    \caption{Results of fine-grained analysis with broad and
      narrow sentiment interpretations}
    \label{snt-fgsa:tbl:broad-narrow}
  \end{center}
\end{table*}

The results of the automatic systems with these two approaches are
given in Table~\ref{snt-fgsa:tbl:broad-narrow}.  As we can see from
the table, the broad interpretation generally leads to notably lower
scores for the sentiment spans, but yields much better results for
sources and targets of the opinions.  An opposite situation is
observed with the narrow scheme: even though the \F-values for
sentiments are twice as high as in the broad case, the scores for the
remaining opinion elements are up to seven percent lower~(consider,
for instance, 0.31~\F{} attained by the linear-chain CRFs with the
broad interpretation versus 0.24~\F{} achieved by this model with the
narrow mapping).

An obvious explanation for these results is the expectedly better
amenability of the narrow scheme to the prediction of sentiment
labels: since \textsc{sentiment} tags only get assigned to
unequivocally polar terms, it becomes easier for the models to infer
this class using their state features---especially morphological or
lexical ones---or the learned word embeddings.  However, on the other
hand, such short spans lead to disrupted label chains for the other
opinion-related elements, making sentiment tags be far apart from the
spans of the respective sources and targets.  Consequently, these
latter classes suffer from the lack of context and become heavily
dependent on the state attributes too.  Although, this time, the
effect of the state features is rather negative, since, in contrast to
polar terms, being a holder or target of an opinion is not an inherent
property of a lexical term, but arises solely from the context which
this term appears in.

Consider, for instance, the name ``Silvio Berlusconi'' in
Example~\ref{snt:fgsa:trg-ctxt}, where it appears as a target of an
evaluation in the first sentence, but serves as a normal subject of an
objective clause in the second case.  The decision about the role of
this name depends primarily on the sense of the whole statement rather
than the name itself.  Consequently, state attributes might only
increase our prior belief that certain words would rather appear in a
subjective context, but cannot tell for sure whether they actually do
so or not.\footnote{The negative effect of state features on the
  prediction of sources and targets was actually observed in our
  corpus, where one of the most frequently made mistakes was the
  uncoditional assignment of the TRG tag to the word ``Nordkorea''
  (\emph{North Korea}) regardless of whether its surrounding context
  was polar or not.}  Therefore, the recognition of sources and
targets of opinions becomes much harder, once the context information
is deprived (as in the narrow case).

\begin{example}[Contextual Dependence of Target
  Elements]\label{snt:fgsa:trg-ctxt}
  Hoffentlich ist es nicht \target{Silvio Berlusconi}. \#Papst\\[0.5em]
  \noindent Hopefully, this won't be \target{Silvio Berlusconi}. \#Pope\\[1em]
  Silvio Berlusconi ist ein italienischer Medienmagnat und Politiker.\\[0.5em]
  \noindent Silvio Berlusconi is an Italian media tycoon and politician.\\
\end{example}

\subsection{Graph Structures}

Since the lack of contextual links appeared to have a strong negative
effect on the prediction of sources and targets, we decided to
investigate whether redefining the way these links were established in
the models would improve the results.  For this purpose, we
implemented three possible extensions to the traditional first-order
linear-chain CRFs, which are shown in Figure~\ref{fig:snt:ho-crf}:
\begin{itemize}
  \item higher-order linear-chain CRFs,
  \item first- and higher-order semi-Markov models, and
  \item tree-structured CRFs.
\end{itemize}

\begin{figure*}[thb]
  \centering
  \begin{subfigure}[t]{0.4\textwidth}
    \centering
      {\scriptsize
  \begin{tikzpicture}
    \tikzstyle{xnode}=[circle,draw,fill=gray76,minimum size=2.3em] %
    \tikzstyle{ynode}=[circle,draw,inner sep=1pt] %
    \tikzstyle{factor}=[rectangle,fill=black,midway,inner sep=0pt,%
    minimum size=0.4em] %
    \tikzstyle{ctxt}=[red] %

    \node[ynode] (SNT1) at (2, 2) {SNT};

    \node[ynode] (TRG1) [above=0.4em of SNT1] {TRG};
    \node[ynode] (SRC1) [above=0.4em of TRG1] {SRC};
    \node[ynode] (NON1) [above=0.4em of SRC1] {NON};

    \node[ynode] (SNT0) [left=2.5em of SNT1] {SNT};

    \node[ynode] (TRG0) [above=0.4em of SNT0] {TRG};
    \node[ynode] (SRC0) [above=0.4em of TRG0] {SRC};
    \node[ynode] (NON0) [above=0.4em of SRC0] {NON};

    \node[xnode] (FEAT1) [below left=2em and 1.7em of SNT1] {};
    \node[xnode] (FEAT2) [below=1.2em of SNT1] {};
    \node[xnode] (FEAT3) [below right=2em and 1.7em of SNT1] {};

    \path [-] (FEAT1) edge node [factor] {} (SNT1);
    \path [-] (FEAT2) edge node [factor] {} (SNT1);
    \path [-] (FEAT3) edge node [factor] {} (SNT1);

    \path [-] (SNT0) edge[ctxt] node [factor,ctxt] {} (SNT1);
    \path [-] (TRG0) edge[ctxt] node [factor,ctxt] {} (SNT1);
    \path [-] (SRC0) edge[ctxt] node [factor,ctxt] {} (SNT1);
    \path [-] (NON0) edge[ctxt] node [factor,ctxt] {} (SNT1);
\end{tikzpicture}
}

    \caption{First-order linear-chain CRF}
  \end{subfigure}
  ~
  \begin{subfigure}[t]{0.4\textwidth}
    \centering
      {\scriptsize
  \begin{tikzpicture}
    \tikzstyle{xnode}=[rectangle,draw,fill=gray76,minimum size=2em] %
    \tikzstyle{ynode}=[rounded rectangle,draw,fill=gray76,inner sep=1pt,%
    minimum size=2.3em,minimum width=width("MMM|MMM")] %
    \tikzstyle{znode}=[rounded rectangle,fill=none,inner sep=1pt] %
    \tikzstyle{factor}=[rectangle,fill=black,midway,inner sep=0pt,%
    minimum size=0.4em] %

    \node[ynode] (NON0) at (1, 5) {NON|NON};
    \node[ynode] (DOTS0) at (1, 4) {$\ldots$};
    \node[ynode] (SRC0) at (1, 3) {SRC|SNT};
    \node[ynode] (TRG0) at (1, 2) {TRG|SNT};
    \node[ynode] (SNT0) at (1, 1) {SNT|SNT};
    \hyperNodeXX{NON0}{DOTS0}{SRC0}{TRG0}{SNT0}{w$_1$};

    \node[ynode] (NON1) at (3, 5) {NON|NON};
    \node[ynode] (DOTS1) at (3, 4) {$\ldots$};
    \node[ynode] (SRC1) at (3, 3) {SRC|SNT};
    \node[ynode] (TRG1) at (3, 2) {TRG|SNT};
    \node[ynode] (SNT1) at (3, 1) {SNT|SNT};
    \hyperNodeXX{NON1}{DOTS1}{SRC1}{TRG1}{SNT1}{w$_2$};

    \crfFeaturesXX{0/2, 1/3, 2/4}{1}{0};
    %% \node[xnode] (FEAT1) at (2, 0) {};
    %% \node[xnode] (FEAT2) at (3, 0) {};
    %% \node[xnode] (FEAT3) at (4, 0) {};

    %% \path [-] (FEAT1) edge node [factor] {} (SNT1);
    %% \path [-] (FEAT2) edge node [factor] {} (SNT1);
    %% \path [-] (FEAT3) edge node [factor] {} (SNT1);

    \begin{scope}[on background layer]
      \path [-] (NON0.10) edge[] node [factor] {} (NON1.177);
      \path [-] (NON0.350) edge[] node [factor] {} (NON1.185);
      \path [-] (SRC0.340) edge[] node [factor] {} (SNT1.155);
      \path [-] (SRC0.320) edge[] node [factor] {} (SNT1.160);
      \path [-] (TRG0.350) edge[] node [factor] {} (SNT1.165);
      \path [-] (TRG0.335) edge[] node [factor] {} (SNT1.170);
      \path [-] (SNT0.10) edge[] node [factor] {} (SNT1.177);
      \path [-] (SNT0.350) edge[] node [factor] {} (SNT1.185);
    \end{scope}
  \end{tikzpicture}
}

    \caption{Second-order linear-chain CRF}
  \end{subfigure}\\[1em]
  \begin{subfigure}[t]{0.4\textwidth}
    \centering
      {\scriptsize
  \begin{tikzpicture}
    \tikzstyle{xnode}=[rectangle,draw,fill=gray76,minimum size=2em] %
    \tikzstyle{ynode-el}=[rounded rectangle,draw,fill=gray76,inner sep=1pt,%
        minimum size=2.3em,minimum width={13em}] %
    \tikzstyle{ynode}=[rounded rectangle,draw,fill=gray76,inner sep=1pt,%
    minimum size=2.3em,minimum width=width("MMM")] %
    \tikzstyle{znode}=[rounded rectangle,draw=none,inner sep=1pt,minimum size=2.3em] %
    \tikzstyle{factor}=[rectangle,fill=black,midway,inner sep=0pt,%
    minimum size=0.4em] %

    \node[ynode] (NON0) at (1, 4) {NON};
    \node[ynode] (SRC0) at (1, 3) {SRC};
    \node[ynode] (TRG0) at (1, 2) {TRG};
    \node[ynode] (SNT0) at (1, 1) {SNT};
    \hyperNodeX{NON0}{SRC0}{TRG0}{SNT0}{w$_1$};

    \node[znode] (NON2) at (2.7, 4) {};
    \node[znode] (SRC2) at (2.7, 3) {};
    \node[znode] (TRG2) at (2.7, 2) {};
    \node[znode] (SNT2) at (2.7, 1) {};
    \hyperNodeX{NON2}{SRC2}{TRG2}{SNT2}{w$_2$};

    \node[znode] (NON4) at (5.3, 4) {};
    \node[znode] (SRC4) at (5.3, 3) {};
    \node[znode] (TRG4) at (5.3, 2) {};
    \node[znode] (SNT4) at (5.3, 1) {};
    \hyperNodeX{NON4}{SRC4}{TRG4}{SNT4}{w$_3$};

    \node[ynode-el] (NON1) at (4, 4) {NON};
    \node[ynode-el] (SRC1) at (4, 3) {SRC};
    \node[ynode-el] (TRG1) at (4, 2) {TRG};
    \node[ynode-el] (SNT1) at (4, 1) {SNT};

    \crfFeaturesSemiMarkov{1/2/1.193, 2/2.7/1.195, 3/3.4/1.197}{%
    1/4.6/1.343, 2/5.3/1.345, 3/6/1.347}{0};
    %% \begin{scope}
      %% \node[xnode] (FEAT1) at (2, 0) {};
      %% \node[xnode] (FEAT2) at (2.7, 0) {};
      %% \node[xnode] (FEAT3) at (3.4, 0) {};

      %% \node[xnode] (FEAT4) at (4.6, 0) {};
      %% \node[xnode] (FEAT5) at (5.3, 0) {};
      %% \node[xnode] (FEAT6) at (6., 0) {};
    %% \end{scope}
    %% \path [-] (FEAT1) edge node [factor] {} (SNT1.193);
    %% \path [-] (FEAT2) edge node [factor] {} (SNT1.195);
    %% \path [-] (FEAT3) edge node [factor] {} (SNT1.197);

    %% \path [-] (FEAT4) edge node [factor] {} (SNT1.343);
    %% \path [-] (FEAT5) edge node [factor] {} (SNT1.345);
    %% \path [-] (FEAT6) edge node [factor] {} (SNT1.347);

    \begin{scope}[on background layer]

      \path [-] (NON0) edge[] node [factor] {} (NON1);
      \path [-] (SRC0) edge[] node [factor] {} (NON1.184);
      \path [-] (TRG0) edge[] node [factor] {} (NON1.186);
      \path [-] (SNT0) edge[] node [factor] {} (NON1.188);

      \path [-] (NON0) edge[] node [factor] {} (SRC1.178);
      \path [-] (SRC0) edge[] node [factor] {} (SRC1.180);
      \path [-] (TRG0) edge[] node [factor] {} (SRC1.182);
      \path [-] (SNT0) edge[] node [factor] {} (SRC1.185);

      \path [-] (NON0) edge[] node [factor] {} (TRG1.175);
      \path [-] (SRC0) edge[] node [factor] {} (TRG1.178);
      \path [-] (TRG0) edge[] node [factor] {} (TRG1.180);
      \path [-] (SNT0) edge[] node [factor] {} (TRG1.182);

      \path [-] (NON0) edge[] node [factor] {} (SNT1.172);
      \path [-] (SRC0) edge[] node [factor] {} (SNT1.174);
      \path [-] (TRG0) edge[] node [factor] {} (SNT1.176);
      \path [-] (SNT0) edge[] node [factor] {} (SNT1);
    \end{scope}
  \end{tikzpicture}
}

    \caption{Semi-Markov CRF}
  \end{subfigure}
  ~
  \begin{subfigure}[t]{0.4\textwidth}
    \centering
      {\scriptsize
  \begin{tikzpicture}
    \tikzstyle{xnode}=[circle,draw,fill=gray76,minimum size=2.3em] %
    \tikzstyle{ynode}=[circle,draw,inner sep=1pt] %
    \tikzstyle{factor}=[rectangle,fill=black,midway,inner sep=0pt,%
    minimum size=0.4em] %
    \tikzstyle{ctxt}=[red] %

    \node[ynode] (SNT1) at (1, 4) {SNT};

    \node[ynode] (TRG1) [right=0.4em of SNT1] {TRG};
    \node[ynode] (SRC1) [right=0.4em of TRG1] {SRC};
    \node[ynode] (NON1) [right=0.4em of SRC1] {NON};

    \node[ynode] (SNT0) [below left=8em and 4em of SNT1] {SNT};
    \node[ynode] (TRG0) [right=0.4em of SNT0] {TRG};
    \node[ynode] (SRC0) [right=0.4em of TRG0] {SRC};
    \node[ynode] (NON0) [right=0.4em of SRC0] {NON};

    \node[ynode] (SNT2) [right=1.2em of NON0] {SNT};
    \node[ynode] (TRG2) [right=0.4em of SNT2] {TRG};
    \node[ynode] (SRC2) [right=0.4em of TRG2] {SRC};
    \node[ynode] (NON2) [right=0.4em of SRC2] {NON};

    \node[xnode] (FEAT1) [below right=3em and 0.6em of SNT1] {};
    \node[xnode] (FEAT2) [right=0.4em of FEAT1] {};
    \node[xnode] (FEAT3) [right=0.4em of FEAT2] {};

    \path [-] (FEAT1) edge node [factor] {} (SNT1);
    \path [-] (FEAT2) edge node [factor] {} (SNT1);
    \path [-] (FEAT3) edge node [factor] {} (SNT1);

    \begin{pgfonlayer}{background}
      \path [-] (SNT0) edge[ctxt] node [factor,ctxt] {} (SNT1);
      \path [-] (TRG0) edge[ctxt] node [factor,ctxt] {} (SNT1);
      \path [-] (SRC0) edge[ctxt] node [factor,ctxt] {} (SNT1);
      \path [-] (NON0) edge[ctxt] node [factor,ctxt] {} (SNT1);

      \path [-] (SNT2) edge[ctxt] node [factor,ctxt] {} (SNT1);
      \path [-] (TRG2) edge[ctxt] node [factor,ctxt] {} (SNT1);
      \path [-] (SRC2) edge[ctxt] node [factor,ctxt] {} (SNT1);
      \path [-] (NON2) edge[ctxt] node [factor,ctxt] {} (SNT1);
    \end{pgfonlayer}
\end{tikzpicture}
}

    \caption{Tree-structured CRF}
  \end{subfigure}
  \caption[Factor graphs of different CRF topologies]{Factor graphs of
    different CRF topologies\\{\small (circles represent random
      variables; gray boxes denote observed input; factors [\ie{}
        feature functions] are shown as tiny black
      squares)}\label{fig:snt:ho-crf}}
\end{figure*}

In the first (higher-order) variant, instead of estimating the
likelihood of a single tag~$y$ at the given
position~$i$~(\eg{}~$P(y_i)=$ SNT), we kept a separate track of the
probability of each possible label
sequence~$P(y_{i-n+1},\ldots,y_{i})$, where $n$ denotes the order of
the model.  In compliance with this structure, we established
transition functions only between those pairs of adjacent nodes where
the label suffix of the preceding unobserved state matched the tag
prefix of the successor.  (For example, in the second-order case, we
only connected the node TRG|SNT to the preceding nodes SNT|TRG,
SRC|TRG, TRG|TRG, and NON|TRG via transition factors; since these were
the only unobserved states whose last labels matched the first tag of
the former node.)  Furthermore, instead of considering just one
transition attribute between those states, as we did in the
first-order case (\eg{}
$f(\mathbf{x}_i, TRG, SNT) = 1.\text{ if }y_{i-1}=TRG\text{ and
}y_i=SNT\text{ else }0.$), we took a sum of several transition
features (one for each possible prefix length).  For instance, in the
case of the transition between the states NON|TRG and TRG|SNT, we took
a sum of two factors:
\begin{equation*}
  f_1(\mathbf{x}_i, NON|TRG, SNT) = \begin{cases} 1, &
    \mbox{if } \mathbf{y}_{i-2} = NON\mbox{ and }\mathbf{y}_{i-1} = TRG\mbox{ and }\mathbf{y}_{i} =
    SNT\\ 0, & \mbox{otherwise;}\\
  \end{cases}
\end{equation*}
and
\begin{equation*}
  f_2(\mathbf{x}_i, TRG, SNT) = \begin{cases} 1, &
    \mbox{if } \mathbf{y}_{i-1} = TRG\mbox{ and }\mathbf{y}_{i} =
    SNT\\ 0, & \mbox{otherwise.}
  \end{cases}
\end{equation*}

Since the number of states in this extension increased exponentially
with the order of the model, we applied the heuristic inference
algorithm of~\citet{Nguyen:14} by only considering those label
sequence which were actually observed in the training data and
pre-caching valid prefix and suffix transitions while doing the belief
propagation.

The same optimization was also applied to higher-order semi-Markov
CRFs, which, in contrast to the linear-chain models, operate on whole
chunks of text, simultaneously trying to predict both the most
probably segmentation of the input and the best possible label
assignment to these segments.  In particular, instead of simply
optimizing the conditional probability of the labels, as it is done by
the linear CRFs:
\begin{equation*}
  P(\mathbf{y}|\mathbf{x}) = \frac{\exp\left(\sum_{m=1}^{M}\sum_jw_{j}
      \times f_j(x_{m}, y_{m-1},
      y_{m})\right)}{Z\left(\mathbf{x}\right)},
\end{equation*}
semi-Markov models seek to maximize the conditional likelihood of the
segments and their labels over the training data:
\begin{equation*}
  P(\mathbf{s}|\mathbf{x}) = \frac{\exp\left(\sum_{n=1}^{N}\sum_jw_{j}
      \cdot f_j(s_{n}, y_{s_{n-1}},
      y_{s_n})\right)}{Z\left(\mathbf{x}\right)}.
\end{equation*}
The $\mathbf{s}$ term in the latter formula stands for the total
segmentation of an input instance, $s_{n}$ denotes the $n$-the
segment, and $y_{s_{n-1}}$ and $y_{s_n}$ represent the labels of the
previous and current segments respectively.  The normalization factor
$Z\left(\mathbf{x}\right)$ here is computed over all possible label
and segment assignments, with segments' length ranging from 1 to $K$,
where $K$ is the maximum length of a contiguous tag span observed in
the training data.

Finally, tree-structured CRFs represent another generalization of the
linear-chain model, in which transition functions are established
between syntactic dependents instead of adjacent tokens.  As the
underlying graph structure for this variant of CRFs, we used the
automatically derived dependency trees, getting these analyses from
the state-of-the-art dependency parser~\cite[Mate;][]{Bohnet:09}.
Since these graphs were guaranteed to be acyclic, we could still apply
the normal belief propagation method with an exact inference, getting
the same convergence guarantees as in the linear-chain case.

The results of these systems on the training and development sets are
shown in Table~\ref{fgsa:tbl:crf-topologies}.
\begin{table*}[hbt!]
  \begin{center}
    \bgroup \setlength\tabcolsep{0.1\tabcolsep}\scriptsize
    \begin{tabular}{p{0.12\columnwidth} % first columm
        *{9}{>{\centering\arraybackslash}p{0.094\columnwidth}}} % next nine columns
      \toprule
      \multirow{2}*{\bfseries Element} & %
      \multicolumn{9}{c}{\bfseries Topology}\\\cline{2-10}
      & lcCRF$^1$ & lcCRF$^2$ & lcCRF$^3$ & lcCRF$^4$ & %
      smCRF$^1$ & smCRF$^2$ & smCRF$^3$ & smCRF$^4$ & trCRF$^1$\\\midrule

      \multicolumn{10}{c}{\cellcolor{cellcolor}Training Set}\\

      Sentiment & 0.928 & 0.919 & 0.922 & 0.925 & 0.931 & 0.931 & 0.933 & 0.931 & 0.906\\
      Source & 0.887 & 0.876 & 0.89  & 0.901 & 0.869 & 0.886 & 0.874 & 0.878 & 0.881\\
      Target & 0.898 & 0.811 & 0.816 & 0.827 & 0.813 & 0.827 & 0.815 & 0.817 & 0.876\\

      \multicolumn{10}{c}{\cellcolor{cellcolor}Development Set}\\

      Sentiment & 0.345 & 0.334 & 0.332 & 0.335 & \textbf{0.395} & 0.385 & 0.389 & 0.378 & 0.331\\
      Source & 0.313 & \textbf{0.32} & 0.272 & 0.304 & 0.298 & 0.282 & 0.287 & 0.291 & 0.223\\
      Target & 0.258 & 0.235 & 0.24 & 0.229 & 0.287 & \textbf{0.309} & 0.301 & 0.292 & 0.243\\\bottomrule
    \end{tabular}
    \egroup
    \caption[Results of fine-grained sentiment analysis with different
    CRF topologies]{Results of fine-grained sentiment analysis with
      different CRF topologies\\ {\small lcCRF---linear-chain CRFs,
        smCRF---semi-Markov CRFs, trCRF---tree-structured CRFs;\\1, 2,
        3, and 4 in the superscripts denote the order}}
    \label{fgsa:tbl:crf-topologies}
  \end{center}
\end{table*}

As we can see from the scores, semi-Markov CRFs achieve better results
at predicting sentiments and targets, but show a degradation in
classifying the sources of the opinions.  Moreover, second-order
semi-Markov and linear-chain topologies outperform the first-order
models on classifying targets and sources.  However, further
increasing the order of these structures does not bring about any
further improvements.  Somewhat surprisingly, tree-structured CRFs
show even worse scores than their linear counterparts.

In order to see whether the same tendencies would hold for the deep
learning methods, we also implemented higher-order and tree-structured
extensions to the LSTM and GRU systems.  In the former case, we passed
a concatenation of $n$ preceding $\vec{h}$ vectors (where $n$ was the
order of the model) as input to the reccurrence loop.  In the
tree-structure modification, we followed the approach
of~\citet{Tai:15} and defined the LSTM unit as follows:
\begin{align*}
  \tilde{h}^{(t)} &= \sum_{k \in C\left(t\right)}\vec{h}^{(k)},\\
  \vec{i}^{(t)} &= \sigma\left(W_i\cdot\vec{x}^{(t)} + U_i\cdot\tilde{h}^{(t)} + \vec{b}_i\right),\\
  \vec{o}^{(t)} &= \sigma\left(W_o\cdot\vec{x}^{(t)} + U_o\cdot\tilde{h}^{(t)} + \vec{b}_o\right),\\
  \vec{u}^{(t)} &= \sigma\left(W_o\cdot\vec{x}^{(t)} + U_o\cdot\tilde{h}^{(t)} + \vec{b}_u\right),\\
  \vec{f}^{(t,k)} &= \sigma\left(W_f\cdot\vec{x}^{(t)} + U_f\cdot\vec{h}^{(k)} + \vec{b}_f\right),\\
  \vec{c}^{(t)} &= \vec{i}^{(t)}\odot\vec{u}^{(t)} + \sum_{k \in C(t)}f^{(t,k)}\odot c^{(k)},\\
  \vec{h}^{(t)} &= \vec{o}^{(t)}\odot tanh\left(\vec{c}^{(t)}\right);
\end{align*}
where $C\left(t\right)$ stands for the indices of all child nodes of
the token $t$.

In a similar way, we also redefined the GRU unit to the following
solutions:
\begin{align*}
  \tilde{h}^{(t)} &= \sum_{k \in C\left(t\right)}\vec{h}^{(k)},\\
  \vec{i}^{(t)} &= \sigma\left(W_i\cdot \mathbf{x}^{(t)} + U_i \cdot \tilde{h}^{(t)}\right),\\
  \vec{f}^{(t,k)} &= \sigma\left(W_f\cdot \mathbf{x}^{(t)} + U_f \cdot \vec{h}^{(t,k)}\right),\\
  \widetilde{c}^{(t)} &= tanh\left(W_c\cdot \mathbf{x}^{(t)} + U_c
  \cdot \sum_{k\in C(t)}\left(\vec{f}^{(t,k)} \odot \vec{h}^{(k)}\right)\right),\\
  \vec{h}^{(t)} &= \vec{i}^{(t)} \odot \tilde{h}^{(t)} + \left(\vec{1} -
  \vec{i}^{(t)}\right) \odot \widetilde{c}^{(t)}.
\end{align*}

The results of these modifications are shown in
Table~\ref{fgsa:tbl:nn-topologies}, from which we can see that the
first-order LSTM model still outperforms all higher-order LSTM and GRU
extensions on predicting targets and sources of the opinions.
Moreover, first-order GRU also achieves the best scores on predicting
sentiment spans among all compared models.  This time, again, none of
the tree-structured extensions could outperform the linear-chain
systems, which might be partially explained by the errors produced by
the parser whose original target domain are standard-language news
texts.

\begin{table*}
  \begin{center}
    \bgroup \setlength\tabcolsep{0.1\tabcolsep}\scriptsize
    \begin{tabular}{p{0.12\columnwidth} % first columm
        *{8}{>{\centering\arraybackslash}p{0.10575\columnwidth}}} % next nine columns
      \toprule
      \multirow{2}*{\bfseries Element} & %
      \multicolumn{8}{c}{\bfseries Topology}\\\cline{2-9}
      & lcLSTM$^1$ & lcLSTM$^2$ & lcLSTM$^3$ & lcGRU$^1$ & %
      lcGRU$^2$ & lcGRU$^3$ & trLSTM$^1$ & trGRU$^1$\\\midrule

      \multicolumn{9}{c}{\cellcolor{cellcolor}Training Set}\\

      Sentiment & 0.584 & 0.559 & 0.54 & 0.57 & 0.587 & 0.606 & 0.43 & 0.518\\
      Source & 0.525 & 0.458 & 0.424  & 0.503 & 0.546 & 0.548 & 0.317 & 0.372 \\
      Target & 0.521 & 0.513 & 0.501 & 0.519 & 0.544 & 0.605 & 0.305 & 0.425\\

      \multicolumn{9}{c}{\cellcolor{cellcolor}Development Set}\\

      Sentiment & 0.278 & 0.285 & 0.281 & \textbf{0.335} & 0.252 & 0.253 & 0.314 & 0.292\\
      Source & \textbf{0.328} & 0.314 & 0.303  & 0.263 & 0.298 & 0.306 & 0.256 & 0.262\\
      Target & \textbf{0.259} & 0.218 & 0.222 & 0.216 & 0.219 & 0.188 & 0.205 & 0.193\\\bottomrule
    \end{tabular}
    \egroup
    \caption[Results of fine-grained sentiment analysis with different
    neural network topologies]{Results of fine-grained sentiment
      analysis with different neural network topologies\\ {\small
        lcLSTM---linear-chain LSTM, lcGRU---linear-chain GRU,
        trLSTM---tree-structured LSTM, trGRU---tree-structured
        GRU;\\1, 2, and 3 in the superscripts denote the order}}
    \label{fgsa:tbl:nn-topologies}
  \end{center}
\end{table*}

\subsection{Text Normalization}

Another question which remained open in the foregoing experiments was
whether the input passed to the models actually had to be normalized
or not.  As mentioned in Section~\ref{snt:fgsa:subsec:data}, when
preparing the data, we preprocessed all corpus tweets using the
rule-based normalization procedure of~\citet{Sidarenka:13}.  In this
approach, we
\begin{itemize}
\item replaced syntactically integrated Twitter phenomena (@-mentions,
  hyperlinks, e-mail addresses etc.) with special unified tokens
  representing their semantic classes (\eg{} ``\%Username'' for
  @-mentions, and ``\%URI'' for hyperlinks);
\item removed these phenomena if they were syntactically independent
  and did not play a potential role for the expression of sentiments
  (\eg{} we stripped off all retweet mentions and hyperlinks appearing
  at the very end of the tweets if they were not preceded by a
  preposition);
\item substituted all emoticons with unified placeholders representing
  their polarities (\eg{} \smiley{} $\rightarrow$ ``\%PosSmiley,''
  \frownie{} $\rightarrow$ ``\%NegSmiley,'' \texttt{:-O} $\rightarrow$
  ``\%Smiley'' etc.);
\item stripped off the number sign (\#) from the beginning of hashtags
  (\eg{} ``\#gl\"ucklich'' $\rightarrow$ ``gl\"ucklich'');
\item and, finally, restored some common forms of misspellings (\eg{}
  ``zuguckn'' $\rightarrow$ ``zugucken'' (\emph{to watch}), ``Tach''
  $\rightarrow$ ``Tag'' (\emph{day}) etc.) using a set of
  manually-defined heuristic rules.
\end{itemize}

Even though these transformations were supposed to improve the
grammaticality of the sentences, an opposite consequence of this
normalization could be the loss of (potentially valuable) surface
features.  In order to check, which of these effects had a stronger
influence on the FGSA results, we repeated the evaluation once again,
turning the preprocessing pipeline off this time.

\begin{table*}[bht!]
  \begin{center}
    \bgroup \setlength\tabcolsep{0.1\tabcolsep}\scriptsize
    \begin{tabular}{p{0.162\columnwidth} % first columm
        *{9}{>{\centering\arraybackslash}p{0.074\columnwidth}} % next nine columns
        *{1}{>{\centering\arraybackslash}p{0.136\columnwidth}}} % last two columns
      \toprule
      \multirow{2}*{\bfseries Data Set} & \multicolumn{3}{c}{\bfseries Sentiment} & %
      \multicolumn{3}{c}{\bfseries Source} & %
      \multicolumn{3}{c}{\bfseries Target} & %
      \multirow{2}{0.136\columnwidth}{\bfseries\centering Macro\newline \F{}}\\\cmidrule(lr){2-4}\cmidrule(lr){5-7}\cmidrule(lr){8-10}
      & Precision & Recall & \F{} & %
      Precision & Recall & \F{} & %
      Precision & Recall & \F{} &\\\midrule

      \multicolumn{11}{c}{\cellcolor{cellcolor}w Normalization}\\

      CRF & \textbf{0.376} & \textbf{0.319} & \textbf{0.345} & %
       \textbf{0.298} & 0.33 & 0.313 & %
       \textbf{0.293} & 0.231 & 0.258 & \textbf{0.305}\\

      LSTM & 0.283 & 0.288 & 0.278 & %
       0.293 & 0.372 & 0.328 & %
       0.254 & \textbf{0.27} & \textbf{0.259} & 0.288\\

      GRU & 0.287 & 0.246 & 0.263 & %
       0.287 & \textbf{0.405} & \textbf{0.335} & %
       0.252 & 0.205 & 0.216 & 0.271\\

      \multicolumn{11}{c}{\cellcolor{cellcolor}w/o Normalization}\\

      CRF & 0.301 & 0.278 & 0.289 & %
       0.276 & 0.3  & 0.287 & %
       0.255 & 0.23 & 0.242 & 0.273\\

      LSTM & 0.274 & 0.252 & 0.261 & %
       0.284 & 0.367 & 0.32 & %
       0.237 & 0.241 & 0.237 & 0.273\\

      GRU & 0.266 & 0.245 & 0.252 & %
       0.296 & 0.369 & 0.328 & %
       0.232 & 0.268 & 0.245 & 0.275\\\bottomrule

    \end{tabular}
    \egroup
    \caption{Results of fine-grained sentiment analysis with (w) and
      without (w/o) text normalization}
    \label{snt-fgsa:tbl:normalization}
  \end{center}
\end{table*}

As we can see from the results in
Table~\ref{snt-fgsa:tbl:normalization}, text preprocessing was
unequivocally beneficial to the sentiment classification as all of the
best observed results were achieved exclusively with normalized text.
The only aspect which benefited from keeping the input unchanged was
the precision of the target classification with GRU, which, in turn,
led to a slightly higher (+0.4\%) macro-averaged \F-scores for this
system.  Apart from that, all other aspects and classifiers showed a
notable degradation when the preprocessing module was switched off.

\subsection{Sentiment Lexicon}

Finally, to answer the question pointed out at the beginning of this
chapter (how useful sentiment lexicons were extrinsically), we
scrutinized the effect of the lexicon that we used in the CRF system
\cite[Zurich Polarity List;][]{Clematide:10} in more detail.  For this
purpose, we repeated the CRF experiments after removing all lexicon
features from the input.  In addition to that, in order to see whether
the lexicon information would be beneficial to the RNN methods, we
also re-evaluated the GRU and LSTM systems, appending the lexicon
scores of the input words to their respective embedding vectors.
(This additional part of the embeddings was kept fixed and did not get
updated during training.)

\begin{table*}
  \begin{center}
    \bgroup \setlength\tabcolsep{0.1\tabcolsep}\scriptsize
    \begin{tabular}{p{0.162\columnwidth} % first columm
        *{9}{>{\centering\arraybackslash}p{0.074\columnwidth}} % next nine columns
        *{1}{>{\centering\arraybackslash}p{0.136\columnwidth}}} % last two columns
      \toprule
      \multirow{2}*{\bfseries Data Set} & \multicolumn{3}{c}{\bfseries Sentiment} & %
      \multicolumn{3}{c}{\bfseries Source} & %
      \multicolumn{3}{c}{\bfseries Target} & %
      \multirow{2}{0.136\columnwidth}{\bfseries\centering Macro\newline \F{}}\\\cmidrule(lr){2-4}\cmidrule(lr){5-7}\cmidrule(lr){8-10}
      & Precision & Recall & \F{} & %
      Precision & Recall & \F{} & %
      Precision & Recall & \F{} &\\\midrule

      \multicolumn{11}{c}{\cellcolor{cellcolor}w Lexicon}\\

      CRF & 0.379 & \textbf{0.309} & \textbf{0.34} & %
        0.291 & 0.32 & 0.305 & %
        0.284 & \textbf{0.226} & 0.252 & \textbf{0.299}\\

      LSTM & 0.314 & 0.226 & 0.262 & %
        0.264 & 0.377 & 0.31 & %
        0.256 & 0.225 & 0.239 & 0.27\\

      GRU & 0.268 & 0.235 & 0.249 & %
        0.28 & 0.383 & 0.323 & %
        0.253 & 0.193 & 0.217 & 0.263\\

      \multicolumn{11}{c}{\cellcolor{cellcolor}w/o Lexicon}\\

      CRF & \textbf{0.382} & 0.303 & 0.338 & %
       0.288 & 0.3 & 0.294 & %
       \textbf{0.291} & 0.221 & 0.251 & 0.294\\

      LSTM & 0.283 & 0.288 & 0.278 & %
       \textbf{0.293} & 0.372 & 0.328 & %
       0.254 & 0.27 & \textbf{0.259} & 0.288\\

      GRU & 0.287 & 0.246 & 0.263 & %
       0.287 & \textbf{0.405} & \textbf{0.335} & %
       0.252 & 0.205 & 0.216 & 0.271\\\bottomrule

    \end{tabular}
    \egroup
    \caption{Results of fine-grained sentiment analysis with (w) and
      without (w/o) the sentiment lexicon}
    \label{snt-fgsa:tbl:lexicons}
  \end{center}
\end{table*}

The results of these approaches with both configurations (with and
without the lexicon) are shown in Table~\ref{snt-fgsa:tbl:lexicons}.
Somewhat surprisingly, the removal of the lexicon data did not cause a
severe drop of the CRF results: the biggest degradation here is
observed for the \F-score of the sentiment sources and amounts to only
0.9\%.  Other sentiment-related aspects show even smaller losses, so
that the final macro-averaged \F-measure achieved without the lexicon
is only 0.5\% lower than the original score attained with the full
feature set.

An even more unexpected outcome is observed for the results of the RNN
methods where adding the lexicon values worsened the final
classification scores: The macro-averaged \F-measure of the LSTM
system, for instance, dropped down by almost 2\% from 0.288 to 0.27.
A similar situation is observed for the GRU approach, whose \F-score
decreased from 0.271 to 0.263 after adding the lexicon information.
We can explain this degradation by the following possible reasons:
\begin{itemize}
\item since the utilized polarity list did not provide a continuous
  ranking of its polar terms (all positive entries in this lexicon had
  a uniform score of~+1, and all negative words had a negative value
  of~-1), the decision boundary of the classifier could become more
  rigid, loosing its ability to separate classes which were located
  closely in the feature space;
\item because the number of the input features in the RNN methods was
  much smaller than in the CRF approach (in the former case, the
  number of features was actually the dimension of the word vectors
  (100 + the lexicon score) whereas, in the CRF case, the number of
  state attributes only run up to 15,047), the lexicon scores could
  get an unduly big influence, outweighing all other embedding
  dimensions.
\end{itemize}
We leave, however, a more detailed analysis of these reasons as one of
the directions for future research.

\section{Related Work}

However, before we proceed to another topic, we first would like to
conclude with a short summary of related work on the fine-grained
opinion mining.  At this point we should note that, due to its high
social impact and big economic importance, FGSA has always been an
area of active research in the sentiment analysis community.  One of
the first groundbreaking steps in this direction was made
by~\citet{Nasukawa:03}, who proposed a lexicon-based approach for
determining phrase-level sentiments towards particular pre-defined
objects.  More precisely, in the initial phase of their method, the
authors determined occurrences of a priori known targets in text.
Afterwards, they analyzed the immediate context of these occurrences,
looking for subjective terms from their manually compiled lexicon;
and, finally, determined the overall orientation of these opinions,
judging by the polarity scores of the lexicon terms and a set of
heuristic rules which accounted for the contextual changes.

Similarly to this approach, \citet{Bethard:04} proposed a method for
identifying subjective opinions and their holders, in which they first
determined polar expressions using the lexicon of~\citet{Wiebe:03} and
then used these expressions as features for a set of SVM classifiers.
The authors applied the resulting ensemble to the nodes of syntactic
constituency trees, trying to predict which of these nodes were the
heads of opinionated clauses or sources of sentiments.  In the same
vein, \citet{Kobayashi:07} used a lexicon-based system for finding
subjective expressions pertaining to some pre-determined product
aspects.

One of the first attempts at automatically analyzing subjective
opinions with Conditional Random Fields was made by \citet{Choi:05},
who combined a linear-chain CRF system with the unsupervised pattern
extractor AutoSlog-TS~\cite{Riloff:96} in order to predict holders of
opinions.  This approach showed a significant improvement over
compared baselines (noun phrases pre-selected by heuristic rules).
Later on, \citet{Choi:06} further enhanced their system,
simultaneously trying to identify both sources of sentiments and their
respective polar expressions, also establishing links between these
entities.  For this purpose, they applied two CRF classifiers (one for
each entity type), getting top-10 results from each of these systems.
After obtaining these predictions, they pruned off invalid suggestions
using a set of hard and soft constraints, whose weights were optimized
with a set of integer linear programming rules.  This two-stage
procedure not only improved the automatic recognition of sources and
polar expressions, but also yielded an impressive \F-score of 0.689 on
predicting inter-entity links.

With the release of the MPQA corpus \cite{Wiebe:05}, fine-grained
opinion-mining researchers gradually shifted the focus of their work
to the analysis of newspaper texts: For example, \citet{Breck:07}
addressed the problem of predicting \emph{direct speech events} (DSEs)
and \emph{expressive subjective elements} (ESE), for which they also
adopted the CRF approach, getting a substantial improvement over
dictionary-based baselines.  \citet{Choi:10} attempted to jointly
predict boundaries, polarity, and intensity of subjective elements
(DSEs and ESEs) with a single CRF classifier.  To this end, they
applied the hierarchical parameter sharing technique
of~\citet{Zhao:08}, letting their system differentiate between ten
different sentiment classes (two classes for the presence or absence
of an opinion times three polarity classes times three intensities),
ensuring, however, that the parameters belonging to the same group
(\eg{} positive polarity) were shared among all of the respective
subclasses (\eg{} positive sentiments with high, low, and middle
intensities).

Instead of relying on just one most likely label assignment when
predicting subjective expressions, \citet{Johansson:10a} explored the
possibility of reranking the initial decisions by first obtaining
top-$k$ alternative predictions (with $k = 8$) and then reweighting
these labelings with a different classifier.  In particular, they
experimented with linear and tree-based preference kernels as well as
structured perceptron and passive-aggressive algorithms for doing this
reweighting, finding that the last option---the passive-aggressive
method---performed best on the MPQA dataset, yielding 4\% improvement
over the standard linear-chain CRF baseline.
\citet[cf.][]{Johansson:10b} also used this idea for the joint
extraction of subjective expressions and their holders from newspaper
articles.

\citet{Yang:12} were allegedly the first who used semi-Markov CRFs in
an opinion mining application.  In order to determine the boundaries
of DSEs and ESEs, the authors derived a set of potential segments from
shallow syntactic parses and then let a semi-Markov model determine
the most likely segmentation and label assignment over these
hypotheses.

A veritable milestone in the fine-grained sentiment analysis research
was set by~\citet{Socher:13}, who first introduced a deep learning
method for predicting the polarity of syntactic constituents.  In
their work, the authors computed a vector representation of a given
constituency node ($\vec{w}_c \in \mathbb{R}^n$) by recursively
multiplying the embeddings of its descendants (\eg{}
$\vec{w}_1 \in \mathbb{R}^n$ and $\vec{w}_2 \in \mathbb{R}^n$) with a
compositional tensor $V \in \mathbb{R}^{2n\times2n\times2n}$ and
applying the softmax non-linearity afterwards:
\begin{align*}
  \vec{w}_c = softmax\left(%
  \begin{bmatrix}
  \vec{w}_1\\
  \vec{w}_2
  \end{bmatrix}^T \cdot V^{1:n}\cdot\begin{bmatrix}
  \vec{w}_1\\
  \vec{w}_2
  \end{bmatrix}%
  + W\cdot\begin{bmatrix}
  \vec{w}_1\\
  \vec{w}_2
  \end{bmatrix}%
  \right),
\end{align*}
The $W\in\mathbb{R}^{n\times n}$ term in the above equation stands for
an additional shared compositionality matrix, and the tensor
product for a single element $\vec{v}_i$ is defined as follows:
\begin{align*}
  \vec{v}_i = \begin{bmatrix}
  \vec{w}_1\\
  \vec{w}_2
  \end{bmatrix}^T \cdot V^{i}\cdot\begin{bmatrix}
  \vec{w}_1\\
  \vec{w}_2
  \end{bmatrix}.
\end{align*}
The embeddings of the underlying words as well as the compositionality
matrix $W$ and the tensor $V$ were learned automatically during
training.

A recurrent approach to the FGSA problem was proposed
by~\citet{Irsoy:14a}, who explored deep bidirectional neural networks
for predicting DSEe and ESEs on the MPQA corpus.  In particular, they
estimated the label probabilities $\vec{y}$ for the $t$-th token as:
\begin{align*}
  y_t = softmax\left(U_{\rightarrow}\cdot \vec{h}_t^{(L)} + U_{\leftarrow}\cdot \cev{h}_t^{(L)} + \vec{c}\right),
\end{align*}
where $\vec{h}_t^{(L)}$ and $\cev{h}_t^{(L)}$ were the outputs of the
top-$L$ LSTM-levels computed from left to right and from right to left
respectively; and $U_{\rightarrow}$, $U_{\leftarrow}$, and the bias
term $\vec{c}$ were the other learned parameters, which were used to
reduce the dimensions.  This system could outperform the previous
state of the art set by \citet{Yang:12}, reaching a proportional
\F-score of 66.01 for DSEs and 56.26 for ESEs.

Other notable works on fine-grained opinion mining with deep neural
networks include those of \citet{Irsoy:14c}, who used a tensor
multiplication within a recursive loop; \citet{Tang:16}, who proposed
a memory-based multilevel network for classifying opinions about the
given aspects; and, finally, \citet{Wang:16}, who used the output of a
deep recursive network as input to a CRF system, updating the weights
of the network along with the CRF parameters.

\section{Summary and Conclusions}\label{fgsa:subsec:conclusions}

Summarizing the above findings, we would like to recap that, in this
chapter, we presented two most common approaches to the fine-grained
sentiment analysis: conditional random fields and recurrent neural
networks; applying these methods to the corpus of German tweets
introduced earlier.  Our experiments showed that CRFs with manually
defined features outperform both recurrent neural network
architectures (LSTM and GRU), reaching a macro-averaged \F-score
of~0.287 on three opinion-relevant entity classes (sentiments,
sources, and targets).

Furthermore, a closer look at each of the settings in our experiments
revealed that:
\begin{itemize}
\item Higher sentiment scores can be achieved by narrowing down the
  text spans of this class to the emotional expression only.  This,
  however, might negatively affect the classification of other opinion
  entities (sources and targets);
\item Even though the context seems to play an important role for the
  prediction, redefining the context flow in the models by changing
  their graph topologies (such as increasing the order of the
  dependencies or modifying the underlying graph structure from linear
  chains to trees) does not bring much improvement.  We could,
  however, still outperform the results of the traditional
  linear-chain CRFs with their first- and second-order semi-Markov
  variants;
\item In order to understand why the effect of topology changes was so
  little, we analyzed in more depth the features that were learned by
  the CRF model, coming to the conclusion that the large number state
  attributes massively predominated the final decisions, suppressing
  the influence of the contextual flow;
\item Nevertheless, a closer look at the top-scoring state and
  transition features showed that the CRF system could learn
  meaningful correlations.  However, an additional ablation test of
  the state attributes revealed that different types of features had
  varying effects on different opinion entities: while sentiment
  classification benefited from all provided traits without exception,
  the prediction of sources profited most from lexical and complex
  attributes, whereas the analysis of targets was positively
  influenced by morphological and syntactic features only;
\item Similarly to the review of the CRF features, we also analyzed
  the effect of the different embedding types on the net scores of the
  RNN approaches, finding that task-specific word vectors with a
  least-square fallback to word2vec embeddings yielded best overall
  results;
\item In the penultimate step, we revised the influence of the text
  normalization on the classification scores by rerunning all
  experiments on the original (unnormalized) messages.  This test
  showed that preprocessing is an extremely helpful preparation step,
  which might boost the FGSA results by up to 3\%;
\item Finally, in the concluding part of the evaluation, we estimated
  the utility of the Zurich Polarity List~\cite{Clematide:10}---one of
  the best-performing German sentiment lexicons to date; finding that
  the lexicon features only marginally improved the CRF scores, and
  had an even negative effect on the results of the LSTM and GRU
  approaches.
\end{itemize}

With these observations, we hope to have provided a better insight
into the abilities of modern FGSA approaches, the importance of their
hyper-parameter settings, and the general amenability of the German
Twitter domain to an automatic extraction of fine-grained opinions.
Even though we did our best to make this description and evaluation as
comprehensive as possible, there still remain many potential avenues
to explore and potential ways to improve the presented results: It
would, for instance, be extremely helpful to evaluate other polarity
lists (produced by different SLG methods) in combination with the
described FGSA methods.\footnote{Unfortunately, we were not able to do
  this analysis by ourselves due to prohibitively long training times:
  a single run of an RNN classifier was taking more than 27 hours on a
  very strong machine.}  Another interesting opportunity would be a
modification of the CRF formalism in which the influence from the
context would be automatically equalized with the sum of the state
feature weights.  Finally, an ensemble of the CRF and RNN systems
would be a viable possibility for improving the scores of each of
these systems.  We encourage interested readers to explore these ways
in their future research, whereas we should soon turn our attention to
another big area of opinion mining in which the polarity lists might
unleash their full power: the coarse-grained sentiment analysis.
