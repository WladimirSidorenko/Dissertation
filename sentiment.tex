% FILE: chapter-sentiment.tex  Version 0.01
% AUTHOR: Uladzimir Sidarenka

% This is a modified version of the file main.tex developed by the
% University Duisburg-Essen, Duisburg, AG Prof. Dr. G�nter T�rner
% Verena Gondek, Andy Braune, Henning Kerstan Fachbereich Mathematik
% Lotharstr. 65., 47057 Duisburg entstanden im Rahmen des
% DFG-Projektes DissOnlineTutor in Zusammenarbeit mit der
% Humboldt-Universitaet zu Berlin AG Elektronisches Publizieren Joanna
% Rycko und der DNB - Deutsche Nationalbibliothek

\chapter{Sentiment Analysis}

\section{Introduction to Sentiment Analysis}

Interpersonnel communication is not only a way to share objective
information with other people but also a vibrant channel to convey
one's subjective thoughts, attitudes, and feelings.  It is, in fact,
this latter use which provides a personal touch to our discourses,
making them more grasping, more entertaining, and more living.  And it
is often this use which significantly influences our decisions,
preferences, and choices in everyday life.  Therefore, a high-quality
automatic analysis of the subjective part of information is typically
not less important than the extraction and analysis of objective
facts.

The field of knowledge which deals with the analysis of people's
opinions, sentiments, evaluations, appraisals, attitudes, and emotions
towards particular entities in discourse is called \emph{sentiment
  analysis} (SA) \citep{Liu:12}.  The definition of this discipline,
however, much like the definition of the term \emph{sentiment} itself
is neither complete nor universally accepted.  The main reasons for
these controversies are
\begin{inparaenum}[\itshape i)\upshape]
  \item a frequently unclear delimitation of the subjective and
    objective components of information and
  \item the heteroginity of the language system, to which SA methods
    are applied.
\end{inparaenum}

The former factor, for instance, makes it difficult to define what
kinds of events ans expressions actually fall in the domain of
sentiment analysis and which ones should rather be excluded from it.
A prominent example of such borderline cases are the so-called
subjective facts such as \emph{bomb attack} or \emph{cancer treatment}
which some people perceive as emotionally laden while the other do
not.

The latter factor complicates a precise definition, since different
levels of language have their own concepts of subjectivity which in
turn require different approaches and methods of analysis.  Typically,
depending on the granularity and language level being analyzed,
researchers distinguish the following subdisciplines of SA:
\begin{itemize}
  \item\emph{subsentential} of \emph{fine-grained} sentiment analysis
    whose task is to determine and analyze specific subjective
    opinions and/or evealuations within single clauses,
  \item\emph{sentential} analysis which tries to ascribe a single
    polarity class to each sentence in text, and
  \item\emph{suprasentential} or \emph{document-level} sentiment
    classification which seeks to determine polarity and subjectivity
    classes of complete discourses.
\end{itemize}

Each of these subfields has its own assets and drawbacks.

that is
being analyzed.  Typical subsentential, sentential, and
supersentential SA.  The first sub-field (also calles fine-grained SA)
seeks to determine and analyze specific subjective opinions and/or
evealuations within single clauses.  The last two (coarse-grained
approaches) rely on the assumption that a single polarity or emotion
class can be ascribed to each sentence or text respectively, which
they then try to determine.

Each of these sub-fields has its own strengths and weaknesses.  While
fine-grained SA typically represents the ultimate goal of the opinion
research, achieving this goal often is a challenging endeavor even for
human beings let alone computer programs.  In this regard, the
coarse-grained methods usually show much better results but this
success comes at the costs of loosing important details of
information. So, for example, when one ascribes a single polarity
class to a complete sentence or even text, any contradicting

But despite a relatively long history of this research area with the
first works dating back to the early 1990s \cite{Wiebe:90a,Wiebe:90b},
neither a precise delimitation of this discipline nor a universally
accepted definition of sentiment exists to date.  The reasons for
these difficulties are mostly due to two factors:
\begin{inparaenum}[\itshape a)\upshape]
  \item the mere term \emph{sentiment} is ambiguos and subjective by
    itself (i.e. it is interpreted different by different people)
\end{inparaenum}


More precisely, sentiment analysis or opinion mining is a subfield of
computational linguistics (CL) which deals with TODO: add definition from
Liu\dots.  But like for many other branches of CL, giving a precise complete
definition of SA is a difficult.  These difficulties arise for two reasons:
\begin{inparaenum}
  \item
\end{inparaenum}

\subsection{Sentiment Analysis of Social Media}

\subsection{Overview of This Chapter}

\section{Sentiment Corpus}
\subsection{Selection Criteria}
\subsubsection{Topic Selection}
\subsubsection{Formal Selection}

\subsection{Annotation Scheme}
\subsection{Statistics and Preliminary Results}
\begin{table}[h]
  \centering\small
  \caption[Sentiment corpus statistics]{Statistics on the annotated
    sentiment corpus.\\ POL = corpus part with discussions about
    general politic topics; FE = corpus part describing the federal
    election 2013; PE = corpus part with discussions about the Pope
    election 2013; GEN = part of the corpus containing tweets with no
    particular topic}
  \begin{tabular}{|>{\centering}p{0.15\textwidth}|*{4}{>{\centering}p{\oosixthClmnWidth}|}
      >{\centering\bfseries}p{\oosixthClmnWidth}|*{4}{>{\centering}p{\oosixthClmnWidth}|}
      >{\centering\bfseries}p{\oosixthClmnWidth}|}
    \hline

    \multirow{2}{*}{\parbox{0.13\textwidth}{\centering Markable Type}}
    & \multicolumn{5}{>{\centering}p{7\oosixthClmnWidth}|}{Annotator
      1} &
    \multicolumn{5}{>{\centering}p{7\oosixthClmnWidth}|}{Annotator
      2}\tabularnewline\cline{2-11}

    & POL & FE & PE & GEN & Total & POL & FE & PE & GEN &
    Total\tabularnewline\hline

    Sentiment & 212 & 222 & 163 & 131 & 728 & 317 & 335 & 314 & 305 & 1271
    \tabularnewline\hline

    Source & 101 & 119 & 68 & 73 & 361 & 114 & 109 & 94 & 85 & 402
    \tabularnewline\hline

    Target & 229 & 279 & 184 & 151 & 843 & 342 & 369 & 328 & 324 & 1363
    \tabularnewline\hline

    Emotional Expression & 727 & 689 & 581 & 811 & 2808 & 662 & 669 & 671 & 768 & 2770
    \tabularnewline\hline

    Intensifier & 16 & 32 & 14 & 44 & 106 & 31 & 35 & 31 & 58 & 155
    \tabularnewline\hline

    Diminisher & 2 & 4 & 3 & 2 & 11 & 2 & 9 & 4 & 2 & 17
    \tabularnewline\hline

    Negation & 18 & 15 & 23 & 14 & 70 & 33 & 33 & 31 & 23 & 120
    \tabularnewline\hline
  \end{tabular}
  \label{table:sentiment-agreement-topics}
\end{table}

\subsection{Inter-Annotator Agreement}
\subsubsection{Inter-Annotator Agreement for Topics}
\begin{table}[h]
  \centering\small
  \caption[Inter-annotator agreement for the sentiment corpus across
    topics]{Inter-annotator agreement for the sentiment corpus across
    topics.\\ POL = corpus part with discussions about general politic
    topics; FE = corpus part describing the federal election 2013; PE
    = corpus part with discussions about the Pope election 2013; GEN =
    part of the corpus containing tweets with no particular topic}
  \begin{tabular}{|>{\centering}p{0.15\textwidth}|*{4}{>{\centering}p{\oosixthClmnWidth}|}
      >{\centering\bfseries}p{\oosixthClmnWidth}|*{4}{>{\centering}p{\oosixthClmnWidth}|}
      >{\centering\bfseries}p{\oosixthClmnWidth}|}
    \hline

    \multirow{2}{*}{\parbox{0.13\textwidth}{\centering Markable Type}}
    &
    \multicolumn{5}{>{\centering}p{7\oosixthClmnWidth}|}{$\kappa$-Agreement
      for Binary Overlap} &
    \multicolumn{5}{>{\centering}p{7\oosixthClmnWidth}|}{$\kappa$-Agreement
      for Proportional Overlap}\tabularnewline\cline{2-11}

    & POL & FE & PE & GEN & Total & POL & FE & PE & GEN &
    Total\tabularnewline\hline

    Sentiment & 0.35 & 0.35 & 0.45 & 0.41 & 0.39 & 0.27 & 0.29 & 0.36 & 0.34 & 0.32
    \tabularnewline\hline

    Source & 0.39 & 0.27 & 0.41 & 0.41 & 0.37 & 0.38 & 0.28 & 0.4 & 0.4 & 0.36
    \tabularnewline\hline

    Target & 0.32 & 0.38 & 0.4 & 0.39 & 0.38 & 0.26 & 0.28 & 0.31 & 0.32 & 0.3
    \tabularnewline\hline

    Emotional Expression & 0.64 & 0.57 & 0.68 & 0.66 & 0.64 & 0.6 & 0.54 & 0.65 & 0.63 & 0.61
    \tabularnewline\hline

    Intensifier & 0.46 & 0.48 & 0.21 & 0.62 & 0.52 & 0.46 & 0.48 & 0.21 & 0.6 & 0.51
    \tabularnewline\hline

    Diminisher & 0.67 & 0.44 & 0.0 & 0.4 & 0.37 & 0.67 & 0.44 & 0.0 & 0.4 & 0.37
    \tabularnewline\hline

    Negation & 0.44 & 0.1 & 0.36 & 0.21 & 0.28 & 0.44 & 0.1 & 0.36 & 0.21 & 0.28
    \tabularnewline\hline
  \end{tabular}
  \label{table:sentiment-agreement-topics}
\end{table}

\subsubsection{Inter-Annotator Agreement for Formal Criteria}

\subsection{Analysis of Annotator Mistakes}
\subsection{Related Work}
\subsection{Conclusions}

\section{Identification of Subjective Expressions}
\subsection{Ontology-based Identification of Subjective Elements}
\subsection{Corpus-based Identification of Subjective Elements}
\subsection{Machine Learning Approaches to Identification of Subjective Expressions}
%% \subsection{Hybrid Methods for Polarity Identification}
\subsection{Related Work}
\subsection{Conclusions}

\section{Identification of Targets, Sources, and Sentiments}
\subsection{Rule-based Approaches to Sentiment Tagging}
\subsection{Machine Learning Approaches to Sentiment Tagging}
\subsubsection{Compared Systems}
\subsubsection{Features}
\subsubsection{Ablation Tests}
\subsubsection{Evaluation}
\subsection{Related Work}
\subsection{Conclusions}

\section{Coarse-grained Sentiment Analysis}
\subsection{Sentiment Analysis on the Level of Discussions}
\subsection{Sentiment Analysis on the Level of Messages}
\subsection{Sentiment Identification on the Level of Sentences}
\subsection{Related Work}
\subsection{Conclusions}

\section{Effects of Text Normalization and Domain Adaptation}
\subsection{Text Normalization Impact on Fine-grained Sentiment Analysis}
\subsection{Text Normalization Effect on Coarse-grained Sentiment Analysis}
\subsection{Comparison with other Corpora and other Domains}
\subsection{Related Work}
\subsection{Conclusions}
