% FILE: main.tex  Version 2.1
% AUTHOR:
% Universit�t Duisburg-Essen, Standort Duisburg
% AG Prof. Dr. G�nter T�rner
% Verena Gondek, Andy Braune, Henning Kerstan
% Fachbereich Mathematik
% Lotharstr. 65., 47057 Duisburg
% entstanden im Rahmen des DFG-Projektes DissOnlineTutor
% in Zusammenarbeit mit der
% Humboldt-Universitaet zu Berlin
% AG Elektronisches Publizieren
% Joanna Rycko
% und der
% DNB - Deutsche Nationalbibliothek

\part{Discourse-Augmented Sentiment Analysis}

%% \section{Introduction to Discourse Analysis}
%% \subsection{Definitions and Related Work}
%% \subsection{Peculiarities of Discourse Analysis on Twitter}

%% \section{Corpus of Discourse Relations}
%% \subsection{Annotation Scheme}
%% \subsection{Statistics and Preliminary Results}
%% \subsection{Inter-Annotator Agreement}
%% \subsection{Related Work}
%% \subsection{Conclusions}

\section{Discourse-Augmented Sentiment Analysis}
\subsection{Discourse-Related Features}
\subsection{Analysis of Results}

\subsection{Related Work}

\todo[inline]{\citet{Pang:02}}
Pang et al. (2002) propose three different machine learning methods to
extract the SO of adjectives. Their results are above a
human-generated baseline, but the authors point out that discourse
structure is necessary to detect and exploit the rhetorical devices
used by the review authors

Quote:

Fundamentally, it seems that some form of discourse analysis is
necessary (using more sophisticated techniques than our positional
feature mentioned above), or at least some way of determining the
focus of each sentence, so that one can decide when the author is
talking about the film itself. (Turney (2002) makes a similar point,
noting that for reviews, ``the whole is not necessarily the sum of the
parts''.)


\todo[inline]{\citet{Hu:04}, Section 3.6}

\todo[inline]{}

One of the first discourse-aware approaches to coarse-grained
sentiment analysis was proposed by~\citet{Pang:04}.  In their work,
the authors partitioned the text of a movie review into subjective and
objective sentences using the min-cut method of~\citet{Blum:01}, and
then analyzed the sentences from the former group in order to classify
the overall polarity of the review.

\citet{Somasundaran:08,Somasundaran:08a} introduced the concept of
\emph{opinion frames} which represented a pair of subjective
statements (arguments or sentiments) that were related to each via the
\emph{same} or \emph{alternative} link.  In the former case, both
statements had to refer to the same target, while, in the latter case,
these targets had to represent mutually exclusive alternatives.

\citet{Voll:07} addressed the problem of classifying the polarity of
Epinions reviews.\footnote{\url{www.epinions.com}} For this purpose,
they applied the SO-CAL system (Semantic Orientation CALculator) to
adjectives appearing in the user posts, considering three ways of
pre-filtering this input: In the first (baseline) approach, they
invariably passed all adjectives found in the review to SO-CAL.  In
the second method, they only analyzed those words which appeared in
the top-most nuclei of the sentences.  These nuclei were determined
automatically by an RST parsing system \cite[SPADE; ][]{Soricut:03}.
Finally, in the third experiment, the authors trained a decision-tree
classifier, which predicted whether a particular sentence pertained to
the general topic of its document or not, and applied the sentiment
system only to those sentences which were classified as
topic-relevant.  With this last approach, \citeauthor{Voll:07} were
able to obtain their best results (69\% accuracy), getting their
second-best scores with the system which considered all adjectives.
The suboptimal results of the discourse-based approach were partially
explained by the limited performance of the automatic RST system.

Another joint framework for predicting the polarity of both sentences
and documents was described by \cite{McDonald:07}.  In their work, the
authors designed a unified CRF system, in which the predicted labels
of the sentences formed a linear chain and simultaneously were
connected to the label node of the whole document which had to be
predicted along with sentence orientations.

\todo[inline]{\citet{Somasundaran:09a,Somasundaran:09b}}

\citet{Yessenalina:10} proposed a latent-variable approach, in which
they tried to predict the overall polarity of a document (movie review
or congressional floor debate) by estimating the difference between
the polarities of the most prominent subjective sentences.  Since the
labels of the sentences were unknown at the training and testing time,
the authors considered their polaity scores as latent variables whose
most likely assignments were to be determined with a structural SVM
algorithm~\cite{Yu:09}.  An enhancement of this algorithm was
presented by~\citet{Trivedi:13}, who extended the original training
objective with a set of discourse connectives attributes and polarity
coherence features.

Other ways of incorporating discourse structure into opinion mining
applications were proposed by~\citet{Bhatia:15}.  In their work, the
authors proposed two methods for augmenting a sentiment analysis
system with discourse information:
\begin{inparaenum}[(i)]
\item discourse depth reweighting and
\item rhetorical recursive neural network (R2N2).
\end{inparaenum}
In the former approach, they introduced a special parameter $\lambda_i$:
\begin{equation*}
  \lambda_i = \max\left(0.5, 1 - d_i/6\right),
\end{equation*}
where $d_i$ stands for the depth of an elementary discourse unit in
the discourse tree of the whole document.  With this parameter, they
estimated the overall polarity of a document~$\Psi$, by summing up the
individual linaer predictions for each sentence scaled by~$\lambda_i$:
\begin{equation*}
  \Psi = \sum_i\lambda_i\mathbf{\theta}^T\cdot\mathbf{w}_i,
\end{equation*}
where $\mathbf{\theta}$ represents a vector of polarity scores ($-1$
for negative terms and $+1$ for positive) obtained from the sentiment
lexicon of~\citet{Wilson:05}, and $w_i$ stands for the bag-of-words
representation of the $i$-th sentence.  In the R2N2 method, the
authors adopted the RNN approach of~\citet{Socher:13} and recursively
estimated the polarity of a document from the leaves of the RST tree
up to its root.

Other notable works on discourse-level sentiment analysis include
those of~\citet{Mao:06}, who used isotonic CRFs to predict the
polarity of single sentences and then used a nearest neighbors
classifier to induce the polarity of the whole document from these
predictions, observing an improvement over the traditional
bag-of-words approach.

Other notable works on discourse-augented sentiment analysis include
those of~\citet{Snyder:07}, who proposed a joint Good Grief model to
predict users' evaluations of different aspects of restaurants.  In
the first phase of this approach, they applied a set of
aspect-specific classifiers, each of which was supposed to predict
user's polarity towards one particular criterion.  In the next step,
the output of this ensemble was corrected by an additional linear
classifier, which was supposed to account for the overall agreement on
multiple aspects.

Another work, which apparently comes closest to ours, is that
of~\citet{Somasundaran:09a}, who tried to unite the notions of opinion
frames~\cite{Somasundaran:08} with the \emph{dialogue act
  theory}~\cite{Bunt:12} within a single framework.  In particular,
the authors projected the polarity annotation of the
AMI~corpus~\cite{Carletta:05} onto the dialogue segments, and then
used a two-stage approach similar to the one proposed
by~\citet{Snyder:07}, in which they first separately classified
polarity of each dialogue leaf, and then used an integer linear
programming module to correct the cases which violated the overall
agreement of opinion frames.


TODO: Somasundaran, 2007, 2009b

TODO:
Riloff, 2003

Thomas, 2006

Polanyi and Zaenen, 2006

Sadamitsu, 2008

Taboada and Grieve, 2009

Mukherjee, 2012

Lazaridou, 2013

\subsection{Conclusions}
