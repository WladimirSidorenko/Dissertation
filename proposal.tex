%%% Local Variables:
%%% mode: latex
%%% TeX-master: t
%%% End:

\documentclass{article}

%%%%%%%%%%%%%%%%%%%%%%%%%%%%%%%%%%%%%%%%%%%%%%%%%%%%%%%%%%%%%%%%%%
%% Packages
\usepackage[latin1]{inputenc}
\usepackage[T1]{fontenc}

%%%%%%%%%%%%%%%%%%%%%%%%%%%%%%%%%%%%%%%%%%%%%%%%%%%%%%%%%%%%%%%%%%
%% Preamble
\title{Thesis Proposal}
\author{Wladimir Sidorenko}

\date{\today}

%%%%%%%%%%%%%%%%%%%%%%%%%%%%%%%%%%%%%%%%%%%%%%%%%%%%%%%%%%%%%%%%%%
%% Document
\begin{document}

\maketitle

With the ever growing role of social media in everyday life, the
ability to rapidly analyze people's opinion on the Web becomes an
inevitable prerequisite for any social, political, and economic
endeavor.  Unfortunately, to date, such analysis still raises many
important questions.  First of all, it is yet unclear how well even
human experts can detect, analyze, and agree on other people's
emotions on the Internet.  Secondly, it is yet unanswered how
difficult this task would be for purely automatic systems, what
peculiarities of the social media domain might pose especially big
challenges to them, and which approaches could deal with these
challenges in the most efficient way.  In our thesis, we try to give a
comprehensive answer to these and other related questions on the
example of German Twitter.

For this purpose, we first present PotTS \cite{Sidarenka:16}---a
substantial collection of $\approx$8,000 tweets which were manually
annotated with polar terms, sentiment spans, targets and sources of
opinions as well as important compositional elements (intensifiers,
diminishers, and negations) by two human coders.  Using this data set,
we estimate the mutual agreement of these experts on different types of
elements, also proposing two novel reliability metrics---binary and
proportional $\kappa$---which explicitly account for partial matches
of annotations spanning multiple tokens.  Since this corpus was
compiled from microblogs pertaining to different topics (German
federal elections, papal conclave, casual everyday conversations etc.)
with different formal traits (tweets containing emoticons, messages
with a priory known polar terms and so on), we also investigate
whether these topics or traits by themselves can lead to a greater
number of opinions or significant deviations in the agreement on
annotated elements.

Since automatic sentiment analysis (SA) typically presupposes a
hierarchically organized system rather than a monolith programming
block, we then analyze each of the typical SA system components in
more detail.  In so doing, we first address the problem of an
automatic prediction of polar terms.  To this end, we investigate
which of the currently existing techniques for an automatic generation
of sentiment lexicons (a semi-automatic translation of common English
polarity lists or automatic dictionary- and corpus-based approaches)
can better accommodate the German Twitter domain.  In addition to
that, we also propose several new algorithms that infer sets of polar
terms from distributed vector representations of words---word
embeddings, showing that our suggested methods can substantially
outperform established state-of-the-art corpus-based approaches.

In the next step, we turn to the subsentential analysis of opinions,
whose primary goal is to find textual spans of sentiments, targets,
and sources within a single sentence.  In the course of this study, we
compare which of the two most-popular recent approaches to the
fine-grained SA---supervised conditional random fields or deep
recurrent neural networks---are best suited for recognizing these
elements in microblog texts.  Moreover, for each of these methods, we
estimate the impact of the graphical topologies on the net
classification results by analyzing whether a (possibly higher-order)
linear-chain or (syntactically motivated) tree-like structure forms a
better basis for propagating probability scores of different labels in
a sentence.

Finally, in the last stage, we turn to the message-level analysis of
tweets in which we try to automatically determine the overall polarity
of a microblog.  In this part, we again compare traditional supervised
classification methods with the latest deep learning approaches also
looking into whether systems trained on a smaller amount of
hand-labeled data can outperform methods trained on a bigger amount of
distantly annotated tweets.

With this chapter, we not only make a major contribution to the
ecosystem of publicly available corpora and tools by releasing one of
the biggest manually labeled German sentiment data sets and
open-sourcing all code accompanying our experiments but we also make
for a substantial theoretical advancement by establishing an upper
bound on the performance of automatic systems through the study of
human experts' agreement and exploring different automatic ways of
analyzing people's opinions.

\section*{Text Normalization}
An important question which, however, remains unanswered in the
previous part is that of the inherent difficulty of the Twitter domain
for an automatic analysis.  Traditionally, analyzing text from this
service has been considered a challenging, prohibitively difficult
task due to an inherent noisiness, ungrammaticality, and creativity of
Twitter's users in expressing their opinions and thoughts.  Usually,
this noisiness is addressed with either of the two popular techniques:
domain adaptation or text normalization.  With the first approach,
systems trained on one (usually standard-language) domain are adjusted
to the analyzed target domain by finding statistical correspondences
in features.  With the latter method, text of the target genre is
converted to the standard representation by removing spurious elements
and mapping non-standard spellings to their cannonical form.  Since we
already train all sentiment classifiers on Twitter texts, we should
analyze whether applying the second option---text
normalization---could further improve the results of automatic
sentiment analysis.

For this purpose, we first give a thorough definition of the text
normalization task---an NLP application whose notion, however, was
only vaguely defined in the liguistic literature \cite{Eisenstein:13}.
In particular, we split this task into two major sub-objectives: text
segmentation and the actual normalization.  The former component
splits a contiguous run of text into single sentences and tokens,
while the latter component deletes or replaces non-canonical tokens
with their standard-language equivalents.

In the next step, we present a manually segmented and normalized
corpus of German tweets, on which the difficulty of these objectives
for human coders.  After analyzing the most efficient ways of doing
these tasks automatically, we estimate the impact of these factors on
the net results of subsentential and sentential sentiment analysis.

\section*{Discourse Analysis}

\bibliographystyle{apalike}
\bibliography{bibliography}
\end{document}
