\documentclass[11pt]{article}

%%%%%%%%%%%%%%%%%%%%%%%%%%%%%%%%%%%%%%%%%%%%%%%%%%%%%%%%%%%%%%%%%%
%% Packages
\usepackage{amsmath}
\usepackage{breakcites}
\usepackage[authoryear]{natbib}
\usepackage{paralist}
\usepackage{titling}

%%%%%%%%%%%%%%%%%%%%%%%%%%%%%%%%%%%%%%%%%%%%%%%%%%%%%%%%%%%%%%%%%%
%% Commands
\newcommand{\ienocomma}{\textit{i.e.}}
\newcommand{\ie}{\ienocomma,}
\newcommand{\eg}{\textit{e.g.},}
\newcommand{\F}[0]{$F_1$}
\newcommand{\markable}[1]{\texttt{#1}}
\newcommand{\attribute}[1]{\emph{\texttt{#1}}}
\newcommand{\argmin}{\operatornamewithlimits{argmin}}
\renewcommand{\cite}{\citep}

%%%%%%%%%%%%%%%%%%%%%%%%%%%%%%%%%%%%%%%%%%%%%%%%%%%%%%%%%%%%%%%%%%
%% Lengths
\setlength{\droptitle}{-7em}

%%%%%%%%%%%%%%%%%%%%%%%%%%%%%%%%%%%%%%%%%%%%%%%%%%%%%%%%%%%%%%%%%%
%% Variables
\author{Uladzimir Sidarenka}
\title{  {\large Summary of the Dissertation}\\[0.5em]
  {\Large ``Sentiment Analysis of German Twitter''}}
\date{\vspace{-3ex}}

%%%%%%%%%%%%%%%%%%%%%%%%%%%%%%%%%%%%%%%%%%%%%%%%%%%%%%%%%%%%%%%%%%
%% Document
\begin{document}
\maketitle

Online communication services have not only revolutionized our
everyday life (allowing us to stay in touch with friends, exchange
ideas with colleagues, and get latest news from celebrities and
politicians without leaving the desktop) but also opened the doors to
numerous new dangers, such as cyber-attacks, online bullying, and
deliberate manipulation of public opinion---dangers that can hardly be
averted manually due to the immense amount of data exchanged on online
platforms.  Therefore, if we want to protect our society from these
looming threats, we urgently need more robust, higher-quality natural
language processing (NLP) applications, which can recognize such
menaces automatically, by analyzing uncensored texts.  Unfortunately,
most NLP programs today have been created for standard language, as we
know it from newspapers, or, in the best case, adapted to the
specifics of English social media.

My dissertation reduces the existing deficit by entering the new
frontier of German online communication and addressing one of its most
prolific forms---users' conversations on Twitter.  In particular, it
explores the ways and means by how people express their opinions on
this service, examines current approaches to automatic mining of these
feelings, and proposes new sentiment analysis techniques, which
outperform state-of-the-art methods.

I perform these tasks on three different linguistic levels:
\begin{itemize}
  \item\emph{subsentential}, in which I try to predict the polarity of
    single words, and automatically determine the main components of
    an opinion (its target, holder, and the actual evaluative
    expression) within a sentence;

  \item\emph{sentential}, where I consider the whole microblog as a
    single statement and classify its overall semantic orientation;

  \item and, finally, \emph{suprasentential}, in which I try to
    improve on message-level sentiment classification by inferring the
    overall polarity of a tweet from the semantic orientation of its
    single sentences, taking into account their semantic and pragmatic
    links (discourse relations).
\end{itemize}
At each of these levels, I evaluate the most prominent classes of
existing systems, seek to outperform their scores with my own
solutions, and estimate the effect of different hyper-paramaters and
input properties (such as word embedding types, feature groups, and
text normalization) on the net results of classification.

\section{Research Questions}

By doing so, I hope to find answers to the following questions:
\begin{itemize}
\item Can we apply opinion mining methods devised for standard English
  to German Twitter?
\item Which groups of approaches are best suited for which sentiment
  analysis tasks?
\item How much do word- and discourse-level analyses affect
  message-level sentiment classification?
\item Does text normalization help analyze sentiments?
\item Can we do better than existing systems?
\end{itemize}

\section{Outline of this Work}

To get these answers, I proceed in the following way:

\subsection{Chapter~1: Introduction to Sentiment Analysis}

In Chapter~1, I give a brief introduction to the main goals and
definitions of sentiment analysis, and make a digression into the
history of this field, tracing its roots back to the ancient
philosophy, and following the course if its development through the
main milestones in psychology, narratology, and linguistics of the
20th century.  In the final sections, I elaborate on the recent
advances in automatic opinion mining of social media, and pay special
attention to the current state of the art in sentiment analysis of
Twitter.

\subsection{Chapter~2: Sentiment Corpus}

Afterwards, in Chapter~2, I introduce the Potsdam Twitter Sentiment
(PotTS) corpus---a collection of $7,992$ German microblogs pertaining
to the German federal elections, papal conclave, general political
discussions, and casual everyday conversations.

I obtained messages for the first three groups by tracking the public
Twitter API between March and September 2013, using extensive lists of
hand-picked keywords, which, in my opinion, best described these
topics.  As casual everyday posts, I considered microblogs from the
German Twitter Snapshot~\cite{Scheffler:14}, which comprises
$\approx97\%$ of all German tweets (17M messages) posted in April
2013.

To ensure a representative number of subjective opinions in the
resulting corpus, I grouped all downloaded microblogs pertaining to
the same topic into three formal classes:
\begin{inparaenum}[(i)]
  \item tweets that contained a polar term from the sentiment lexicon
    SentiWS~\cite{Remus:10};
  \item messages that had an emoticon; and, finally,
  \item all remaining microblogs.
\end{inparaenum}
Afterwards, I sampled an equal number of tweets (666) from each of the
four topics, from each of these three formal categories.

In the next step, I defined an annotation scheme---a list of elements
that had to be annotated by human experts.  These were:
\begin{itemize}
  \item\emph{sentiment}s, which were specified as polar subjective
    evaluative opinions about people, entities, or events;

  \item\emph{target}s, which represented entities or events being
    evaluated;

  \item\emph{source}s, which denoted the immediate author(s) or
    holder(s) of an opinion;

  \item\emph{polar term}s, which were words or idioms that had a
    distinguishable evaluative lexical meaning;

  \item\emph{intensifier}s, which represented elements that increased
    the expressivity and subjective sense of polar terms;

  \item\emph{diminisher}s, which vice versa reduced the strength of a
    polar term;

  \item and, finally, \emph{negation}s, which completely reversed the
    semantic orientation of a polar item (\eg{} ``\emph{nicht} gut''
    [\emph{not} good] or ``\emph{kein} schlechtes Beispiel''
    [\emph{not} a bad advice]).
\end{itemize}

The annotation process was performed in three steps: first, two
linguists labeled one half of the data after only minimal training;
then, I automatically highlighted their divergent analyses and asked
them to resolve these differences; finally, the annotators continued
with the analysis of the remaining files.

To estimate the inter-rater reliability, I introduce two modified
versions of the popular $\kappa$-metric~\cite{Cohen:60}---binary and
proportional kappa---which differ in the way how they treat multiple
annotations of the same tokens (\eg{} if one phrase was labeled as
\emph{sentiment} several times by one annotator) and how they deal
with partial matches of different annotations (\eg{} if two
\emph{sentiment}s labeled by different linguists agree on some, but
not all of their tokens).  Using these measures, I estimate the
inter-annotator agreement (IAA) of the experts at different stages of
their work.  This study shows that, initially, both assistants could
hardly agree on the mere notion of targeted opinions, but their
disagreements could be resolved with the help of the adjudication
procedure, which was applied in step two.  Despite a small drop of the
IAA scores in the final stage, all $\kappa$-values still remained at
the level of at least moderate reliability.

Finally, I demonstrate that the initial selection criteria (topics and
formal classes) had a strong impact on the number and agreement of
annotated sentiments and polar terms, with tweets about federal
elections and messages without prefiltered topics being the most
prolific sources of these elements.

\subsection{Chapter~3: Sentiment Lexicons}

In Chapter~3, I turn to the first subsentential sentiment analysis
task, in which I try to predict the polarity (positive, negative, or
neutral) of single words in the text.

For this purpose, I first evaluate existing, manually-curated German
sentiment lexicons---SentiWS \cite{Remus:10}, German Polarity Clues
\cite{Waltinger:10}, and the Zurich Polarity List
\cite{Clematide:10}---by directly comparing their entries with the
annotated polar terms in the PotTS corpus.  This evaluation shows that
the last resource (Zurich Polarity List) outperforms the first two
competitors in terms of both macro- and micro-averaged \F{}-scores,
reaching 0.589 and 0.955 \F{} respectively.

Afterwards, I compare three major groups of automatic
lexicon-generation methods: dictionary-, corpus-, and
word-embedding--based ones.  The first of these groups induces
polarity lists from a manually annotated lexical taxonomy
\cite[\textsc{GermaNet}; ][]{Hamp:97}---a graph whose nodes represent
sets of synonyms (synsets), which are connected to each other by
semantic relations (meronymy-holonymy, hypernymy-hyponymy, antonymy
etc.).  Corpus-based algorithms, on the contrary, infer sentiment
lexicons from collocation statistics computed on raw, unlabeled data,
by taking a set of terms with known semantic orientation (seed terms)
and looking for other words that frequently co-occur with these seeds.
In the same vein, word-embedding--based methods generate new polarity
lists from words whose vector representations (which are automatically
learned by a classifier that predicts the occurrence of other words in
the nearby context) are most similar to the embeddings of seed terms.

Besides evaluating existing methods, I propose four new
word-embedding--based solutions, which induce sentiment lexicons by
clustering word vectors using the methods of
\begin{inparaenum}[(i)]
\item $k$-nearest neighbors,
\item nearest centroids,
\item principal component analysis, and
\item my own linear projection algorithm.
\end{inparaenum}
In the last of these systems, I first determine a line that maximizes
the distance between the projections of seed embeddings with opposite
semantic orientations, and then project the vectors of all remaining
terms on this line, considering the distances of these projections to
the origin as the respective polarity scores.  My algorithm surpasses
all existing alternatives in terms of micro-averaged \F{}-score,
reaching 0.963 micro-\F{}, and yields a very competitive macro-\F{}
(0.462), outperforming all corpus- and word-embedding--based
approaches except for the method of $k$-nearest neighbors.

As confirmed by the evaluation, the linear projection and $k$-NN
algorithms are the only viable lexicon-generation solutions that
dispense with any manually annotated linguistic resources and whose
results are at least close to the scores of manually-curated and
dictionary-based sentiment lexicons.  Furthermore, additional
experiments demonstrate that the quality of almost all automatically
generated polarity lists can be further improved by using larger and
less ambiguous sets of initial seed terms.

\subsection{Chapter~4: Aspect-Based Sentiment Analysis}

Another subsentential sentiment analysis task---automatic prediction
of targets, holders, and text spans of opinions---is addressed in
Chapter~4.  Following the current state of the art, I consider this
objective as a sequence-labeling (SL) problem and approach it with two
most popular SL techniques: conditional random fields~\cite[CRFs;
][]{Lafferty:01} and recurrent neural networks \cite[RNNs;
][]{Hochreiter:97,Cho:14a}.  For the former method, I devise an
extensive set of feature attributes, which reflect different lexical,
mophologic, and syntactic traits of the input.  Using these features,
CRFs achieve 0.287 macro-\F{} for the three classes (sentiments,
sources, and targets) on the PotTS test set.  An additional ablation
test of these attributes shows that complex features, which
simutaneously reflect lexical and syntactic properties, are most
useful for predicting sentiment spans; pure lexical traits are most
helpful when classifying sources; and syntactic attributes work best
when predicting the targets of opinions.

Unfortunately, recurrent neural networks yield lower results than
CRFs, with the best scores (0.27 macro-\F{}) achieved by the
long-short term memory network~\cite[LSTM; ][]{Hochreiter:97}.  Their
accuracy, however, can be improved by initializing the word embeddings
with word2vec vectors~\cite{Mikolov:13} and optimizing these values
along with other parameters during the training.  In addition to this,
I propose a new fallback strategy based on the method of the ordinary
least squares, which allows to approximate task-specific vector
representations of words that have word2vec embeddings, but do not
occur in the training set of the sentiment corpus.

In the concluding sections, I investigate whether using alternative
structure of computational graphs for CRFs and RNNs and providing
unnormalized text to these methods as input can improve their results.
As it turns out, CRFs benefit from increasing the order of their graph
dependencies and yield better scores when classifying spans of the
text (hypernodes) rather than individual tokens (single nodes); RNNs,
on the contrary, achieve their best scores with the standard
first-order linear-chain topology.  Both approaches, however, strongly
profit from text normalization, showing a quality boost by almost 3\%
when this preprocessing step is active.

%% Finally, I compare two different approaches to the definition of
%% \emph{sentiment} spans: \emph{broad} and \emph{narrow}, in which I
%% assign the \emph{sentiment} tag either to minimal complete
%% syntactic unit (typically noun or verb phrase) that simultaneously
%% comprise both the target of an evaluation and the evaluative phrase
%% (\emph{broad}), or to the polar term only.  A comparison of both
%% schemes shows that the former approach leads to better
%% classification scores for \emph{source}s and \emph{target}s,
%% whereas the second take is more amenable to the prediction of
%% \emph{sentiment}s.

\subsection{Chapter~5: Message-Level Sentiment Analysis}

Afterwards, in Chapter~5, I turn to a sentential task---message-level
sentiment classification, whose goal is to predict the polarity
(positive, negative, or neutral) of a single tweet.  In this part, I
again compare the most prominent groups of methods: dictionary-,
mechine-learning--, and deep-learning--based ones; and evaluate them
on two different German corpora: my own PotTS dataset and the SB10k
Twitter corpus~\cite{Cieliebak:17}, which has been explicitly
annotated with message-level semantic orientations.

The presented comparison shows that English dictionary-based methods
are least suitable for being applied to German data, due to the
language specifics of their manually defined rules.  Machine-learning
systems, however, are mostly language-independent and can be equally
well applied to English and German microblogs.  A more diverse
situation is observed for deep-learning classifiers, which perform
relatively poor when trained with randomly initialized word vectors,
but outperform many other methods when used with word2vec embeddings,
especially with the least-square fallback.

For the last group of methods, I also propose my own system---a
bidirectional recurrent neural network with lexicon-based attention
(LBA), which, in addition to the standard LSTM recurrence and
traditional attention mechanism~\cite{Bahdanau:14}, assigns greater
importance to terms that have higher polarity scores in the sentiment
lexicon, as well as their syntactic dependents, which might serve as
contextual modifiers of polar terms.  My algorithm attains 0.69
macro-\F{} and 0.73 micro-\F{} on the PotTS corpus, outperforming all
other competitors on this dataset, and comes very close to the best
overall scores on the SB10k test data.

An extensive evaluation of all methods once again confirms the utility
of text normalization and the linear-projection lexicon, which
outperforms all other polarity lists on the PotTS corpus when used in
combination with the LBA classifier, reaching 0.69~macro-\F and
0.73~micro-\F{} on PotTS and attaining 0.55~macro-\F{} and
0.73~micro-\F{} on the SB10k corpus.  An interesting finding, however,
is that distant supervision (\ie{} training on tweets whose labels are
automatically inferred from the polarity of their smileys) does not
necessarily improve the results of message-level sentiment
classification.

\subsection{Chapter~6: Discourse-Aware Sentiment Analysis}

Finally, in Chapter~6, I improve the accuracy of the proposed LBA
classifier by making it aware of the discourse structure of tweets.
For this purpose, I split all microblogs from the PotTS and SB10k
corpora into elementary discourse units\footnote{An elementary
  discourse unit Typically represents a single proposition.} (EDUs)
using an ML-based discourse segmenter~\cite{Sidarenka:15} and
construct a rhetorical structure tree of these units \cite{Mann:88}
with the English discourse parser DPLP \cite{Ji:14}, which has been
retrained on the Potsdam Commentary Corpus \cite{Stede:14}.

Afterwards, I let the LBA system predict polarity scores of single
EDUs and infer the overall polarity of a tweet from the polarity of
its units.  In particular, I first evaluate three existing
discourse-aware sentiment methods (the systems of
\citeauthor{Wang:13}~[\citeyear{Wang:13}] and two approaches presented
by~\citeauthor{Bhatia:15}~[\citeyear{Bhatia:15}]) on all microblogs
that have more than one discourse segments, and then improve on the
results of these classifiers with three new alternative solutions:
latent and latent-marginalized conditional random fields and a
recursive Dirichlet process.

In the final steps, I estimate the effect of different base sentiment
classifiers (\ie{} downstream systems that are used to predict the
polarity scores of single EDUs) and the utility of different discourse
relations, which are taken into account by discourse-aware systems.
These experiments show that my proposed methods yield consistently
good results with almost all settings, outperming their competitors in
the prevailing major of the cases.  The only exception to this general
tendency are the macro- and micro-averaged \F{}-scores achieved by the
mehods of~\citet{Bhatia:15} on the SB10k corpus in combination with
richer relation sets (0.572 macro-\F{} and 0.719 micro-\F{}), which
surpass the results of my best solutions by 0.008 and 0.002
respectively.

\section{Conclusions}

Based on these experiments, I come to the following conclusions
regarding the initially posed questions:

\begin{itemize}
  \item\textbf{Can we apply opinion mining methods devised for
    standard English to German Twitter?}

    Yes, we can, but the success of these approaches might
    significantly vary depending on the task, the size and the
    reliability of the training data, as well as the evaluation metric
    that we use. For example, dictionary-based lexicon methods
    achieved fairly good results on their objective, but this success
    was mostly due to the high quality of the \textsc{GermaNet}
    annotation.  On the other hand, our manually labeled PotTS corpus
    was evidently too small for aspect-based sentiment systems, which
    failed to generalize to unseen tweets despite their very high
    scores on the training set.  Message-level sentiment approaches,
    vice versa, seemed to be quite happy with the size of the training
    dataset, attaining good results on both corpora (PotTS and SB10k).
    Nevertheless, we again experienced a lack of data while working on
    discourse-aware enhancements, many of which hit the same ceiling
    of the macro-averaged \F{}-scores.

    Apart from these difficulties arising from insufficient data, we
    also noticed a significant degradation of the scores for systems
    whose original tasks and evaluation metrics were different from
    ours.  For example, the lexicon generation method of
    \citet{Esuli:05} was originally designed to assign polarity scores
    to all \emph{synsets} found in the \textsc{WordNet} and not to
    produce a list of polar \emph{words}.  Similarly, the RNTN
    approach of \citet{Socher:13} was trained and evaluated on all
    syntactic subtrees of a document and not only at the top text
    level.  Likewise, the system of~\citet{Yessenalina:11} was devised
    for doing ordinal logistic regression and not polarity
    classification, as in our case.  As a result, all these approaches
    showed lower scores than their competitors in our evaluation, even
    though they are undoubtedly well suited for their original data
    and tasks.

    Due to the high diversity of methods, metrics, and tasks, it is
    difficult to provide a general recipe for transferring existing
    English sentiment systems to German Twitter, but we still would
    like to formulate at least a few rules of thumb, which came up
    during our experiments:
    \begin{itemize}
      \item\textbf{Prefer methods which are closest to your training
        objective} and which were trained under similar conditions
        w.r.t. the amount of data, their class distribution and
        domain;
      \item\textbf{Put every single setting of these methods into
        question}---bear in mind that things which work well in the
        original cases are not guaranteed to work in your
        situation.\footnote{In this respect, it is important to
          realize that every classification task is merely an attempt
          to solve a system of equations, so that methods which are
          good at solving one system might completely fail to solve
          other equations.}  The more options you try, the better will
        be your results;
      \item\textbf{Try using manually labeled resources for your
        target domain}, if they are available, but pay attention at
        the quality of their annotation---it often matters more than
        the corpus size;
      \item If there are manually annotated data, \textbf{prefer
        machine-learning methods to hard-coded rules}--- they will
        penalize their bad components automatically by themselves;
      \item\textbf{Do not use randomly initialized word embeddings for
        deep-learning systems}---initialize them with language-model
        vectors (which are cheap to obtain).  Otherwise, your model
        might get stuck in a very bad local optimum.
    \end{itemize}

  \item\textbf{Which groups of approaches are best suited for which
    sentiment tasks?}

    Based on our evaluation, we answer to this question as follows:
    \begin{itemize}
      \item\emph{Sentiment lexicon generation} is more amenable to
        dictionary-based solutions, provided that there exists a
        sufficiently big, reliably annotated lexical taxonomy for
        these systems.  If there is no such resource, one should
        better resort to word-embedding--based algorithms;

      \item With a limited amount of training data, \emph{aspect-based
        sentiment analysis} can be better addressed with probabilistic
        graphical models such as conditional random fields with
        hand-crafted features;

      \item On the other hand, plain \emph{message-level sentiment
        analysis} can be efficiently tackled with both machine- and
        deep-learning algorithms (such as SVM, logistic regression, or
        RNN);

      \item But probabilistic graphical models strike back at
        \emph{discourse-aware sentiment methods}, where they might
        even outperform pure neural-network solutions, although the
        margin of these improvements is not that large.
    \end{itemize}

    Thus, probabilistic model can still hold their ground when it
    comes to structured prediction, but the difference of these
    algorithms from and their improvements upon neural networks are
    gradually vanishing.

  \item\textbf{How much do word- and discourse-level analyses affect
    message-level sentiment classification?}

    Our evaluation in Section~\ref{cgsa:subsec:eval:lexicons} showed
    that the macro-averaged \F{}-scores of our proposed lexicon-based
    attention system varied by up to 14\% (from 0.64 to 0.69
    macro-\F{} on the PotTS corpus, and from 0.44 to 0.58 on SB10k)
    depending on the lexicon used by this approach.  At the same,
    discourse enhancements could only improve the results of LBA by at
    most 1.5\% percent (from 0.677 to 0.678 on PotTS, and from 0.557
    to 0.572 on SB10k).  Although it appears as if the lexicon
    component were more important to a sentiment system, we would like
    to preclude such incorrect conclusion, because
    \begin{inparaenum}[(a)]
      \item a full-fledged sentiment solution should take into account
        both linguistic levels (words and discourse) and
      \item these relative results might look different if we expand
        the analyzed domain to longer documents or apply
        discourse-aware methods to complete discussion threads.
    \end{inparaenum}

  \item\textbf{Does text normalization help analyze sentiments?}

    Yes, it definitely does.  As we could see in
    Chapters~\ref{chap:fgsa} and \ref{chap:cgsa}, normalization
    significantly improves the quality of aspect-based and
    message-level sentiment analyses, boosting the results on the
    former task by up to 4\% (see
    Table~\ref{snt-fgsa:tbl:normalization}) and improving the
    macro-averaged \F{}-measure of message-level sentiment methods by
    up to 25\% (see Table~\ref{snt-cgsa:tbl:res-no-normalization});

    The only question that remained unanswered in this context is
    which normalization steps exactly improve the scores of sentiment
    systems.  To make up for this omission, we separately deactivated
    each individual step of our text normalization pipeline
    (unification of Twitter phenomena, spelling correction, and
    normalization of slang terms) and rerun our message-level
    classification experiments using the lexicon-based attention
    system.  As we can see from the results in
    Table~\ref{afterword:tbl:lba-normalization-steps}, the
    micro-averaged \F{}-scores on both datasets benefit most from the
    unification of Twitter-specific phenomena, sinking by almost 19\%
    when this component is deactivated.  This step is also most useful
    for the macro-\F{} on the SB10k corpus, whereas the macro-average
    on PotTS mostly capitalizes on the normalization of slang terms.

  \item\textbf{Can we do better than existing approaches?}

    Yes, we can:
    \begin{itemize}
    \item we improved the macro-averaged results of exisitng
      lexicon-generation methods with our proposed linear projection
      algorithms;
    \item we increased the scores of aspect-based analysis by
      redefining the topologies of CRFs;
    \item our lexicon-based attention network outperformed many of
      its competitors on message-level classification;
    \item and, finally,we surpassed the discourse-unware baseline and
      other existing discourse-aware sentiment solutions with the
      proposed latent-marginalized CRFs and Recursive Dirichlet
      Process.
    \end{itemize}
\end{itemize}

% Bibliography
\bibliographystyle{apalike}
\bibliography{../bibliography}

\end{document}
