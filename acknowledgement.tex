% FILE: acknowledgement.tex  Version 2.1
% AUTHOR:
% Universit�t Duisburg-Essen, Standort Duisburg
% AG Prof. Dr. G�nter T�rner
% Verena Gondek, Andy Braune, Henning Kerstan
% Fachbereich Mathematik
% Lotharstr. 65., 47057 Duisburg
% entstanden im Rahmen des DFG-Projektes DissOnlineTutor
% in Zusammenarbeit mit der
% Humboldt-Universitaet zu Berlin
% AG Elektronisches Publizieren
% Joanna Rycko
% und der
% DNB - Deutsche Nationalbibliothek

\textbf{Acknowledgment.}

At the beginning of this thesis, I would like to say a really big
thank you to all people who supported me throughout these difficult
yet immensely enjoyable years of work on my dissertation; to people
who always were there, when I needed them; to people who taught,
advised, and helped me with my research and who, I believe, made me
stronger, more clever, and perhaps also more confident in my
knowledge.  These people are my colleagues, my friends, my teachers,
but, first of all, my family and my advisor.  I am heavily indebted to
all of them and I am absolutely convinced that this work had never
come into existence if I did not have such a great working environment
as I was lucky to have it in Potsdam.

It would certainly take me another dissertation to express all the
great debt of gratitude that I owe each of my supporters, but I am
afraid they would not accept to wait that long untill I finish it.
So, I should rather try to constrain myself in this acknowledgment and
describe all the infinite gratefulness that I feel for my helpers with
just few but still essential words.

First of all, I certainly would like to thank my advisor, Manfred
Stede, who was much more a ``Doktorvater'' than just a supervisor for
me.  I thank Manfred for his courage in employing me, for the numerous
hours that he invested in our discussions, and for his valuable hints,
suggestions, and criticism.  His experience and wisdom laid the
foundation for the success of this work.  And his patience allowed me
to bring this thesis to the end.  I am very thankful to Manfred that
he introduced me to the amazing world of science, a world which does
not require a visa but which does require a sober mind and a
considerable load of knowledge.  And I sincerely hope that he will
open the doors to this unbelievably beautiful world for many other
Ph.D. students that he certainly will have.

Of course, having a good ``Doktorvater'' is an invaluable precondition
for any completed dissertation, but another prerequisite which plays
an equally important role is still one's own family at home.  And I
have to admit that I had a great luck with both of my families: the
scientific and the traditional one.  My wife and my parents have
always been my greatest supporters.  And, unfortunately, they also
were those people who, I think, suffered most from my absence and the
lack of my spare time.  So, taking this rare opportunity to speak to
them from these pages, I would like to simultaneously apologize for my
possibly neglected family duties and to promise that I will make up
for it in future.

My time at the University of Potsdam had certainly never been that
enjoyable if I did not have such great colleagues as Tatjana
Scheffler, Arne Neumann, Yulia Grishina, Andreas Peldszus, and
Jonathan Sonntag.  I thank Tatjana for her help with our research
project ``Discourse Analysis of Social Media'' and for her valuable
advices and corrections of my papers.  I thank Arne for our talks,
cultural excursions, and his nice attitude to people.  And I am, of
course, much indebted to Yulia who was the most reliable annotator in
our numerous conducted annotation experiments.

I am also especially grateful to our colleagues from the chair of
Theoretical Computational Linguistics -- Alexander Koller and Thomas
Hanneforth.  The unsurpassable level of Alexander's professional
competence and the consummate quality of his work set an example for
me that I eagerly tried to follow.  And even if this humble attempt is
still far away from being completed, it would have been even less
successful if I did not have the unique opportunity to visit Thomas'
fanastic courses in automata theory and C++.  But I am also
particularly thankful to Thomas for the invaluable research hints that
he gave me in our discussions.

Last but not least, I should also mention Konstantina Garoufi, Nikos
Engonopoulos, Mart\'{i}n Villalba, and \v{Z}eljko Agi\'{c} for our
many interesting talks and our enjoyable common lunches.
