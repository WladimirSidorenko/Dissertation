% FILE: introduction.tex  Version 0.01
% AUTHOR: Uladzimir Sidarenka

% This is a modified version of the file main.tex developed by the
% University Duisburg-Essen, Duisburg, AG Prof. Dr. Günter Törner
% Verena Gondek, Andy Braune, Henning Kerstan Fachbereich Mathematik
% Lotharstr. 65., 47057 Duisburg entstanden im Rahmen des
% DFG-Projektes DissOnlineTutor in Zusammenarbeit mit der
% Humboldt-Universitaet zu Berlin AG Elektronisches Publizieren Joanna
% Rycko und der DNB - Deutsche Nationalbibliothek

\chapter{Introduction}

\section{Overview}
As social media become more and more ubiquitous nowadays, the need for
automatic analysis of their data rises.  This analysis, however, is greatly
exacerbated by the fact that the language form used on the Web is
fundamentally different from the linguistic norm that is common in newspapers
or scientific articles.  Indeed, sentences like the ones shown in Example
\ref{intro:exmp:tweets:en} (taken from \citet{Eisenstein:13}) are highly
unlikely to appear in editorials or official documents even though their
wording is commonplace on the U.S. English Twitter.
\begin{example}\label{intro:exmp:tweets:en}
Boom! Ya ur website suxx bro {\upshape (Sarah Silverman)}

\dots dats why pluto is pluto it can neva b a star {\upshape (Shaquille o'Neil)}
\end{example}
Similar differences between Web sociolects and established linguistic
standards can be observed in other languages too, and German is not an
exception.

Thus far,

Of course, one would expect that most of the existing NLP tools -- the vast
majority of which was created for and trained only on standard language data
-- would not perform as well on text forms that only vaguely resemble their
originally intended domain.  But thus far, this hypothesis has only been a
conjecture.  And one of the major goals of this thesis is to check the
validity of this conjecture.

But previously to drawing any conclusions about whether the social media genre
is principally less or more amenable to the automatic processing, a number of
methodologically important questions need to be addressed and clarified.

First of all, in this work, we should only concentrate on German Twitter data.
The justification for choosing precisely this genre as typical representative
of the ensemble of online languages is given in Section
\ref{sct:intro:twitter}.  The object of our research will be lingustic
differences of Twitter data from the standard language texts and the impact of
these differences on the performace of NLP applications.

To this end, we will first perform an extensive analysis

As we shall see later in this work, some researchers consider the
discrepancies between the two language variants so overwhelming that they even
suggest using a machine translation approach in order to transform one form
into another \citep[cf.][]{Aw:06,Pennell:11}.

Indeed, messages like the tweets in Example \ref{intro:exmp:tweets:en} are
arguably hard to understand even for humans, let alone computer programs.
But even though almost everyone today agrees that the language of the Web is
different, not to say \emph{bad}, no formal definition exists until today of
what this difference actually is, how it is expressed, and no ultimate answer
is given to the question of what one should do about this difference.


As noted by \cite{Eisenstein:13}, ``the rise of social media has brought
computational linguistics to ever-closer contact with \emph{bad language}:
text that defies our expectations about vocabulary, spelling, and syntax.''
To prove his statement, the author gives the following examples of tweets
posted by american celebrities:

\begin{example}
Boom! Ya ur website suxx bro {\upshape (Sarah Silverman)}

... dats why pluto is pluto it can neva b a star {\upshape (Shaquille o'Neil)}
\end{example}
These messages are arguably hard to understand even for humans, let alone
computer programs.  But while almost everyone today agrees that the language
of the Web is essentially \emph{different} from the standard norm, not to say
\emph{bad}, no one can precisely define what this \emph{difference} or
\emph{badness} actually is, what it is caused by, and how it, in fact, affects
the analysis.


Another question which is closely related with the previous ones is what one
should do about the \emph{badness} of casual online texts.  Should it be
curated or should it be accepted as is, following the famous ``that's not a
bug, that's an undocumented feature!'' approach.  As we should see later, this
question remains unsolved untill today, though both approaches have their
distinct pros and contras.
